%% USPSC-Cap5-ConclusaoTutorial.tex
% ---
% Conclusão
% ---
\chapter{Conclusão}
% ---
% O comando abaixo insere parágrafos aleatórios só para exemplificar
Apresentar as conclusões correspondentes aos objetivos ou hipóteses propostos para o desenvolvimento do trabalho, podendo incluir  sugestões para novas pesquisas.

O Grupo desenvolvedor do Pacote USPSC, atualmente na versão 3.1 composta pela \textbf{Classe USPSC}, pelo \textbf{Modelo para TCC em \LaTeX\ utilizando o Pacote USPSC} e pelo \textbf{Modelo para teses e dissertações em \LaTeX\ utilizando o Pacote USPSC}, acredita que esta ferramenta propiciará o aprimoramento na qualidade dos trabalhos acadêmicos produzidos pelos alunos de pós-graduação das Unidades de Ensino e Pesquisa do Campus USP de São Carlos, garantindo a normalização e padronização estabelecidas.

A perspectiva é que em breve seja possível a customização da Classe USPSC em conformidade com as orientações dadas em \url{https://github.com/abntex/abntex2/wiki/ComoCustomizar}.

A expectativa é que o Pacote USPSC passe a ser adotado por outras Unidades da USP e outras instituições interessadas, sendo que a facilidade de customização fatalmente contribuirá para tanto.

