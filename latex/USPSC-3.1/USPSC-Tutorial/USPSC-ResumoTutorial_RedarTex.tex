%% USPSC-ResumoTutorial.tex
\setlength{\absparsep}{18pt} % ajusta o espa\c{c}amento dos par\'agrafos do resumo		
\begin{resumo}
	\begin{flushleft} 
		\setlength{\absparsep}{0pt} % ajusta o espa\c{c}amento da refer\^encia	
		\SingleSpacing 
		\imprimirautorabr.~~\textbf{\imprimirtituloresumo}.~~\imprimirorientador~~	
		%Substitua p. por f. quando utilizar oneside em \documentclass
		%\pageref{LastPage}f.
		\imprimirlocal: \imprimirinstituicao, \imprimirdata. \pageref{LastPage}p. 
	\end{flushleft}
\OnehalfSpacing 			
 O resumo deve ressaltar o  objetivo, o m\'etodo, os resultados e as conclus\~oes do documento. A ordem e a extens\~ao  destes itens dependem do tipo de resumo (informativo ou indicativo) e do  tratamento que cada item recebe no documento original. O resumo deve ser
 precedido da refer\^encia do documento, com exce\c{c}\~ao do resumo inserido no
 pr\'oprio documento. (\ldots)  Salientamos que algumas Unidades exigem o titulo dos trabalhos acad\^emicos em ingl\^es, tornando necess\'ario a inclus\~ao das refer\^encias nos resumos e abstracts, o que foi adotado no \textbf{Modelo para TCC em \LaTeX\ utilizando o Pacote USPSC} e no \textbf{Modelo para teses e disserta\c{c}\~oes em \LaTeX\ utilizando o Pacote USPSC}. As palavras-chave devem figurar logo abaixo do  resumo, antecedidas da express\~ao Palavras-chave:, separadas entre si por  ponto e finalizadas tamb\'em por ponto \cite{nbr6028}.
 

 \textbf{Palavras-chave}: LaTeX. abnTeX. Classe USPSC. Editora\c{c}\~ao de texto. Normaliza\c{c}\~ao da documenta\c{c}\~ao. Trabalho acad\^emico. Tese. Disserta\c{c}\~ao. Trabalho de conclus\~ao de curso (TCC). Documentos (elabora\c{c}\~ao). Documentos eletr\^onicos. 
\end{resumo}