%% USPSC-Agradecimentos.tex
\begin{agradecimentos}
Quero expressar minha gratid\~ao \`as crian\c{c}as e aos jovens; que passaram pelo WASH; que est\~ao conosco e que ser\~ao o futuro do Programa.
Em especial, agrade\c{c}o a uma “crian\c{c}a sempre viva”, ativa, presente, curiosa e que outrora usou o LOGO, gostou de fazer seu jogo e trouxe essa experi\^encia para o seu mundo adulto de cientista, professor, pesquisador, gestor, amigo e companheiro de luta, h\'a mais de uma d\'ecada. Refiro-me ao Dr. Victor Pellegrini Mammana, que ao vivenciar esse prazeroso experimento, quis legar a outras crian\c{c}as o \^extase das descobertas e comprovou que \'e poss\'{\i}vel somar esfor\c{c}os da sociedade civil, das unidades de pesquisa e de educa\c{c}\~ao para contribuir com os processos de aprendizagens em ci\^encia e tecnologia.
Sou, tamb\'em, grata ao meu orientador, Prof. Dr Paulo S\'ergio de Camargo Filho pela companhia e orienta\c{c}\~ao, ao Grupo de Pesquisa STEM Education; e \`a banca de avalia\c{c}\~ao, Prof. Dra. Luciane Capelo e Profº. Dr Eduardo Damasceno.
N\~ao posso deixar de mencionar a generosidade e disposi\c{c}\~ao dos professores Drs. Ala\'{\i}de Pellegrini Mammana e Carlos Mammana, que me forneceram preciosas informa\c{c}\~oes para o trabalho.
Agrade\c{c}o, carinhosamente, \`a Professora Dra. Afira Viana Ripper, por sua contribui\c{c}\~ao para a educa\c{c}\~ao cientifica no ensino fundamental; e por participar do v\'{\i}deo, que resgata essa trajet\'oria e foi parte integrante da minha pesquisa.
Algumas pessoas, tamb\'em, precisam ser destacadas, pois foram imprescind\'{\i}veis para o Programa WASH, desde as suas origens, e marcaram essa hist\'oria (precisei colocar em ordem alfab\'etica): Adriane Pinheiro da Silva, Adriana Tito, Aldo Rabelo, Alex \^Angelo, Alexandre C\^andido Paulo, Alisson Alexandre de Ara\'ujo, Aloizio Mercadante, Am\'elia Naomi, Ant\^onio Carlos dos Santos (o Tot\'o), Ant\^onio Pestana, Alexandre Motta, Ana Paula Rodrigues, Andrea Saraiva, Andrea Dias Victor, Andrea Napolitano, Angel Luis, Antonio Bezerra de Albuquerque, Benedita Aparecida Rodrigues de Freitas, Carlinhos Almeida, C\'assia Oliveira, Cec\'{\i}lia Baranauskas, Celio Turino, Mirza Maria Pellicciotta, Celso Pansera, Celso Pan, Chico Sim\~oes, C\'{\i}ntia Cinquini, Claudio Romanelli, Cleide Santos, Mariana Moura, Clotilde Diogo, Ana Carolina de Deus Soares, Daniel Sp\'ozito, Everbal de Castro, Denise  Vieira Pereira, Dilma Rousseff,  Fabiana Kitagawa, F\'abio Couto, Delma  Medeiros, Fernanda Gon\c{c}alves, Gisele Fink, N\'adia Abiel, Gl\'aucia Veloso, Guida Calixto,  Ingridy Janaina Alves, Haissa Gabriela Silva, Irma Passoni, Isabela Maria Vieira Pereira Rodrigues, Jacqueline Baumgratrz, Jaciara Rodrigues dos Santos, Jandira Maria Rodrigues de Freitas, Jos\'e Leonardo de Oliveira, Juliana Moralles Louvison, Juliana Rabelo, Kevin Martins, Layla Xavier, Leila Bomfim, Let\'{\i}cia Mizael, Lucas Gabriel Batista da Silva, Lucas Titon, Luciano Rudinik, Fernando Accorsi, Magna Gon\c{c}alves, Malu Alencar, Marcela Moreira, Marcelo Aguirre, Maria Fernandes, Michel Morandi Alencar, Nelcina Tropardi. Pedro Tourinho, Paula Ropelo, Paulo B\'ufalo, Priya Patel, Rafael de Deus Soares, Rafael Gomes da Cruz, Rafael Proc\'opio, Renan Inqu\'erito, Renata Xavier, Roberta Santana,  Sandra Lanza, Saulo Monteiro, Sebastian Marques, Tayssa Santana,  S\'ergio Benassi, S\'{\i}lvio Ant\^onio Damasceno, S\'{\i}lvio Aparecido Spinella, TC ( Antonio Carlos), Thatiane Verni Lopes de Ara\'ujo, Toni Klaus, Valdirene Maria dos Santos, Vitor de Oliveira Pochmann, Wagner Rodrigo Silva, Wil Namen, dentre tantos outros.
A todas as gera\c{c}\~oes do WASH: as que passaram, as que compartilham conosco, nesse ano de 2023, os 10 anos do Programa: s\~ao colegas, bolsistas, educandos, educadores, cientistas, coordenadores, orientadores, Conselhos de classes, Sindicatos, gestores, pesquisadores, vereadores, comunidades, os entes federados, que acreditam na ci\^encia e no papel transformador da educa\c{c}\~ao.
Meus agradecimentos, tamb\'em. \`as institui\c{c}\~oes parceiras: Conselho Nacional de Desenvolvimento Cient\'{\i}fico e Tecnol\'ogico - CNPq , Funda\c{c}\~ao Arauc\'aria, WASH Paran\'a, Cia Bola de Meia, Legislativo Federal, por meio dos deputados: Ivan Valente, Alex e Luiza Canziani, Alexandre Padilha, Vicentinho, Carlos Zaratini, Orlando Silva e Alexandre Cury, que foram sens\'{\i}veis e valorizaram a educa\c{c}\~ao cientifica, atrav\'es do Programa WASH. Aos legislativos de Prado Ferreira e Dr. Camargo por fazerem o WASH leis municipais. N\~ao posso deixar de reconhecer a contribui\c{c}\~ao da AkiPosso , com o apoio dos colegas Kevin Martins, Adriana Tito, Priya Patel, Caroline Gardemann, Nelcina Tropardi e Daniela Napolitano. Por fim, agrade\c{c}o a minha fam\'{\i}lia, a minha filha, Agatha Abayomi Silva Sene, aos meus pais, Maria Imaculada de Oliveira e Silva e Joaquim Roberto da Silva, ao meu irm\~ao Eduardo Roberto da Silva in memoriam - presente! e ao Alessandro, que contribu\'{\i}ram para que as condi\c{c}\~oes necess\'arias para o desenvolvimento dessa pesquisa fossem as mais leves para a execu\c{c}\~ao do meu estudo.
Termino enfatizando os papeis especiais do Vereador Paulo Bufalo, que est\'a nesta caminhada conosco desde os prim\'ordios do programa, e da Dra. Andrea Dias Victor, servidora do CNPq que permanece aceitando, com compromisso p\'ublico, excel\^encia administrativa e acad\^emica, a carga de gest\~ao do programa WASH, representada por centenas de bolsistas semestrais.
% @[pontoinsercaoparagrafoagradecimento]@

\end{agradecimentos}
% ---