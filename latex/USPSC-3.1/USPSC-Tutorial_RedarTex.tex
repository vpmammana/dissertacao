%% USPSC-Tutorial.tex
% ---------------------------------------------------------------
% USPSC: Modelo de Trabalho Academico (tese de doutorado, dissertacao de
% mestrado e trabalhos monograficos em geral) em conformidade com 
% ABNT NBR 14724:2011: Informacao e documentacao - Trabalhos academicos -
% Apresentacao
%----------------------------------------------------------------
%% Esta \'e uma customiza\c{c}\~ao do abntex2-modelo-trabalho-academico.tex de v-1.9.5 laurocesar 
%% para as Unidades do Campus USP de S\~ao Carlos:
%% EESC - Escola de Engenharia de S\~ao Carlos
%% IAU - Faculdade de Educa\c{c}\~ao
%% ICMC - Faculdade de Educa\c{c}\~ao\^encias Matem\'aticas e de Computa\c{c}\~ao
%% IFSC - Faculdade de Educa\c{c}\~ao\'{\i}sica de S\~ao Carlos
%% IQSC - Faculdade de Educa\c{c}\~ao\'{\i}mica de S\~ao Carlos
%%
%% Este trabalho utiliza a classe USPSC.cls que \'e mantida pela seguinte equipe:
%% 
%% Coordena\c{c}\~ao e Programa\c{c}\~ao:
%%   - Marilza Aparecida Rodrigues Tognetti - marilza@sc.usp.br (PUSP-SC)
%%   - Ana Paula Aparecida Calabrez - aninha@sc.usp.br (PUSP-SC)
%% Normaliza\c{c}\~ao:
%%   - Brianda de Oliveira Ordonho Sigolo - brianda@usp.br (IAU)
%%   - Eduardo Graziosi Silva - edu.gs@sc.usp.br (EESC)
%%   - Eliana de C\'assia Aquareli Cordeiro - eliana@iqsc.usp.br (IQSC)
%%   - Fl\'avia Helena Cassin - cassinp@sc.usp.br (EESC)
%%   - Maria Cristina Cavarette Dziabas - mcdziaba@ifsc.usp.br (IFSC)
%%   - Regina C\'elia Vidal Medeiros - rcvmat@icmc.usp.br (ICMC)
%%
%% O USPSC-modelo.tex e USPSC-TCC-modelo.tex utilizam diversos arquivos relacionado em 
%% 2.1 Pacote USPSC: Classe USPSC e modelos de trabalhos acad\^emicos	do Tutorial do Pascote 
%%  USPSC para modelos de trabalhos de acad\^emicos em LaTeX - vers\~ao 3.0


%----------------------------------------------------------------
%% Sobre a classe abntex2.cls:
%% abntex2.cls, v-1.9.5 laurocesar
%% Copyright 2012-2015 by abnTeX2 group at https://www.abntex.net.br/ 
%%
%----------------------------------------------------------------

\documentclass[
% -- op\c{c}\~oes da classe memoir --
12pt,		% tamanho da fonte
openright,	% cap\'{\i}tulos come\c{c}am em p\'ag \'{\i}mpar (insere p\'agina vazia caso preciso)
twoside,  % para impress\~ao em anverso (frente) e verso. Oposto a oneside - Nota: utilizar \imprimirfolhaderosto*
%oneside, % para impress\~ao em p\'aginas separadas (somente anverso) -  Nota: utilizar \imprimirfolhaderosto
% inclua uma % antes do comando twoside e exclua a % antes do oneside 
a4paper,			% tamanho do papel. 
% -- op\c{c}\~oes da classe abntex2 --
chapter=TITLE,		% t\'{\i}tulos de cap\'{\i}tulos convertidos em letras mai\'usculas
% -- op\c{c}\~oes do pacote babel --
english,			% idioma adicional para hifeniza\c{c}\~ao
french,				% idioma adicional para hifeniza\c{c}\~ao
spanish,			% idioma adicional para hifeniza\c{c}\~ao
brazil				% o \'ultimo idioma \'e o principal do documento
% {USPSC-classe/USPSC} configura o cabe\c{c}alho contendo apenas o n\'umero da p\'agina
]{USPSC-classe/USPSC}
%]{USPSC-classe/USPSC1}
% Inclua % antes de ]{USPSC-classe/USPSC} e retire a % antes de %]{USPSC-classe/USPSC1} para utilizar o 
% cabe\c{c}alho diferenciado para as p\'aginas pares e \'{\i}mpares:
%- p\'aginas \'{\i}mpares: com se\c{c}\~oes ou subse\c{c}\~oes e o n\'umero da p\'agina
%- p\'aginas pares: com o n\'umero da p\'agina e o t\'{\i}tulo do cap\'{\i}tulo 
% ---
% ---
% Pacotes b\'asicos - Fundamentais 
% ---
\usepackage[T1]{fontenc}		% Sele\c{c}\~ao de c\'odigos de fonte.
\usepackage[utf8]{inputenc}		% Codifica\c{c}\~ao do documento (convers\~ao autom\'atica dos acentos)
\usepackage{lmodern}			% Usa a fonte Latin Modern
% Para utilizar a fonte Times New Roman, inclua uma % no in\'{\i}cio do comando acima  "\usepackage{lmodern}"
% Abaixo, tire a % antes do comando  \usepackage{times}
%\usepackage{times}		    	% Usa a fonte Times New Roman	
% Lembre-se de alterar a fonte no comando que imprime o pre\^ambulo no arquivo da Classe USPSC.cls				
\usepackage{lastpage}			% Usado pela Ficha catalogr\'afica
\usepackage{indentfirst}		% Indenta o primeiro par\'agrafo de cada se\c{c}\~ao.
\usepackage{color}				% Controle das cores
\usepackage{graphicx}
\usepackage[export]{adjustbox}
\usepackage[skip=2pt,font=scriptsize]{caption}
\usepackage{subcaption}
			% Inclus\~ao de gr\'aficos
\usepackage{float} 				% Fixa tabelas e figuras no local exato
\usepackage{chemfig,chemmacros} % Para escrever rea\c{c}\~oes qu\'{\i}micas
%\usepackage{mychemistry}        % Para escrever rea\c{c}\~oes qu\'{\i}micas
\usepackage{tikz}				% Para escrever rea\c{c}\~oes qu\'{\i}micas e outros
\usetikzlibrary{positioning}
%\usetikzlibrary{arrows, babel}	% Para escrever rea\c{c}\~oes qu\'{\i}micas e outros
\usepackage{microtype} 			% para melhorias de justifica\c{c}\~ao
\usepackage{pdfpages}
\usepackage{makeidx}            % para gerar \'{\i}ndice remissivo
% ---

% ---
% Pacotes de cita\c{c}\~oes
% Cita\c{c}\~oes padr\~ao ABNT
% ---
% Sistemas de chamada: autor-data ou num\'erico.
% Sistema autor-data
\usepackage[alf, abnt-emphasize=bf, abnt-thesis-year=both, abnt-repeated-author-omit=no, abnt-last-names=abnt, abnt-etal-cite, abnt-etal-list=3, abnt-etal-text=it, abnt-and-type=e, abnt-doi=doi, abnt-url-package=none, abnt-verbatim-entry=no]{abntex2cite}
\bibliographystyle{USPSC-classe/abntex2-alf-USPSC}
% Se o idioma for o ingl\^es, inclua % no comando acima e exclua o % do comando abaixo
%\bibliographystyle{USPSC-classe/abntex2-alfeng-USPSC}

% Para o IQSC, que indica todos os autores nas refer\^encias, incluir % no in\'{\i}cio dos comandos acima e retirar a % dos comandos abaixo 
%\usepackage[alf, abnt-emphasize=bf, abnt-thesis-year=both, abnt-repeated-author-omit=no, abnt-last-names=abnt, abnt-etal-cite, abnt-etal-list=0, abnt-etal-text=it, abnt-and-type=e, abnt-doi=doi, abnt-url-package=none, abnt-verbatim-entry=no]{abntex2cite} 
%\bibliographystyle{USPSC-classe/abntex2-alf-USPSC}
% Se o idioma for o ingl\^es, exclua % no comando acima ou do comando abaixo
%\bibliographystyle{USPSC-classe/abntex2-alfeng-USPSC}

% Sistema Num\'erico
% Para cita\c{c}\~oes num\'ericas, sistema adotado pelo IFSC, incluir % no in\'{\i}cio dos comandos acima e retirar a % dos comandos abaixo
%\usepackage{cite}             % agrupa cita\c{c}\~oes num\'ericas consecutivas 
%\usepackage[num, abnt-emphasize=bf, abnt-thesis-year=both, abnt-repeated-author-omit=no, abnt-last-names=abnt, abnt-etal-cite, abnt-etal-list=3, abnt-etal-text=it, abnt-and-type=e, abnt-doi=doi, abnt-url-package=none, abnt-verbatim-entry=no]{abntex2cite} 
%\bibliographystyle{USPSC-classe/abntex2-num-USPSC}
% Se o idioma for o ingl\^es, exclua % no comando acima ou do comando abaixo
%\bibliographystyle{USPSC-classe/abntex2-numeng-USPSC}

% Complementarmente, verifique as instru\c{c}\~oes abaixo sobre os Pacotes de Nota de rodap\'e
% ---
% Pacotes de Nota de rodap\'e
% Configura\c{c}\~oes de nota de rodap\'e

% O presente modelo adota o formato num\'erico para as notas de rodap\'es quando utiliza o sistema de chamada autor-data para cita\c{c}\~oes e refer\^encias. Para utilizar o sistema de chamada num\'erico para cita\c{c}\~oes e refer\^encias, habilitar um dos comandos abaixo.
% H\'a diversa op\c{c}\~oes para nota de rodap\'e no Sistema Num\'erico.  Para o IFSC, habilitade o comando abaixo.

%\renewcommand{\thefootnote}{\fnsymbol{footnote}}  %Comando para inser\c{c}\~ao de s\'{\i}mbolos em nota de rodap\'e

% Outras op\c{c}\~oes para nota de rodap\'e no Sistema Num\'erico:
%\renewcommand{\thefootnote}{\alph{footnote}}      %Comando para inser\c{c}\~ao de letras min\'uscula em nota de rodap\'e
%\renewcommand{\thefootnote}{\Alph{footnote}}      %Comando para inser\c{c}\~ao de letras mai\'uscula em nota de rodap\'e
%\renewcommand{\thefootnote}{\roman{footnote}}     %Comando para inser\c{c}\~ao de n\'umeros romanos min\'usculos  em nota de rodap\'e
%\renewcommand{\thefootnote}{\Roman{footnote}}     %Comando para inser\c{c}\~ao de n\'umeros romanos min\'usculos  em nota de rodap\'e

\renewcommand{\footnotesize}{\small} %Comando para diminuir a fonte das notas de rodap\'e

% ---
% Pacote para agrupar a cita\c{c}\~ao num\'erica consecutiva
% Quando for adotado o Sistema Num\'erico, a exemplo do IFSC, habilite 
% o pacote cite abaixo retirando a porcentagem antes do comando abaixo
%\usepackage[superscript]{cite}	

% ---
% Pacotes adicionais, usados apenas no \^ambito do Modelo Can\^onico do abnteX2
% ---
\usepackage{lipsum}				% para gera\c{c}\~ao de dummy text
% ---

% pacotes de tabelas
\usepackage{multicol}	% Suporte a mesclagens em colunas
\usepackage{multirow}	% Suporte a mesclagens em linhas
\usepackage{longtable}	% Tabelas com v\'arias p\'aginas
\usepackage{threeparttablex}    % notas no longtable
\usepackage{array}

% ----
% Compatibiliza\c{c}\~ao com a ABNT NBR 6023:2018
% Para tirar <> da URL
%\DeclareFieldFormat{url}{\bibstring{urlfrom}\addcolon\addspace\url{#1}}
\usepackage{USPSC-classe/ABNT6023-2018}
% As demais compatibiliza\c{c}\~oes est\~ao nos arquivos abntex2-alf-USPSC.bst,abntex2-alfeng-USPSC.bst, abntex2-num-USPSC.bst e abntex2-numeng-USPSC.bst, dependendo do idioma do textos e se o sistemas de chamada for autor-data ou num\'erico, conforme explicitado acima.
% ----


% ---
% DADOS INICIAIS - Define sigla com t\'{\i}tulo, \'area de concentra\c{c}\~ao e op\c{c}\~ao do Programa 
% Consulte a tabela referente aos Programas, \'areas e op\c{c}\~oes de sua unidade contante do
% arquivo USPSC-Siglas estabelecidas para os Programas de P\'os-Gradua\c{c}\~ao nos AP\^ENDICES B-J
% Para o Tutorial \'e ulilizado USPSC em ambos os par\^ametros
\siglaunidade{USPSC}
\programa{USPSC}
% Os demais dados dever\~ao ser fornecidos no arquivo USPSC-pre-textual-UUUU ou USPSC-TCC-pre-textual-UUUU, onde UUUU \'e a sigla da Unidade. 
% Exemplo:USPSC-pre-textual-IFSC.tex
% ---
% Configura\c{c}\~oes de apar\^encia do PDF final
% alterando o aspecto da cor azul
\definecolor{blue}{RGB}{41,5,195}




% informa\c{c}\~oes do PDF
\makeatletter
\hypersetup{
	%pagebackref=true,
	pdftitle={\@title}, 
	pdfauthor={\@author},
	pdfsubject={\imprimirpreambulo},
	pdfcreator={LaTeX with abnTeX2},
	pdfkeywords={abnt}{latex}{abntex}{USPSC}{trabalho acad\^emico}, 
	colorlinks=true,       		% false: boxed links; true: colored links
	linkcolor=black,          	% color of internal links
	citecolor=black,        		% color of links to bibliography
	filecolor=black,      		% color of file links
	urlcolor=black,
	%Para habilitar as cores dos links, retire a % antes dos comandos abaixo e inclua a % antes das 4 linhas de comando acima 
	%linkcolor=blue,            	% color of internal links
	%citecolor=blue,        		% color of links to bibliography
	%filecolor=magenta,      		% color of file links
	%urlcolor=blue,
	bookmarksdepth=4	
}
\makeatother
% --- 

% --- 
% Espa\c{c}amentos entre linhas e par\'agrafos 
% --- 

% O tamanho do par\'agrafo \'e dado por:
\setlength{\parindent}{1.3cm}

% Controle do espa\c{c}amento entre um par\'agrafo e outro:
\setlength{\parskip}{0.2cm}  % tente tamb\'em \onelineskip

% ---
% compila o sum\'ario e \'{\i}ndice
\makeindex
% ---

% ----
% In\'{\i}cio do documento
% ----

\begin{document}

% Seleciona o idioma do documento (conforme pacotes do babel)
\selectlanguage{brazil}
% Se o idioma do texto for ingl\^es, inclua uma % antes do 
%      comando \selectlanguage{brazil} e 
%      retire a % antes do comando abaixo
%\selectlanguage{english}

% Retira espa\c{c}o extra obsoleto entre as frases.
\frenchspacing 

% --- Formata\c{c}\~ao dos T\'{\i}tulos
\renewcommand{\ABNTEXchapterfontsize}{\fontsize{12}{12}\bfseries}
\renewcommand{\ABNTEXsectionfontsize}{\fontsize{12}{12}\bfseries}
\renewcommand{\ABNTEXsubsectionfontsize}{\fontsize{12}{12}\normalfont}
\renewcommand{\ABNTEXsubsubsectionfontsize}{\fontsize{12}{12}\normalfont}
\renewcommand{\ABNTEXsubsubsubsectionfontsize}{\fontsize{12}{12}\normalfont}


% ----------------------------------------------------------
% ELEMENTOS PR\'E-TEXTUAIS
% ----------------------------------------------------------
% ---
% Capa
% ---
\imprimircapam
% ---
% Folha de rosto
% (o * indica impress\~ao em anverso (frente) e verso )
% ---
\imprimirfolhaderostom*
%\imprimirfolhaderosto
% ---
% ---
% Inserir a ficha catalogr\'afica em pdf
% ---
% A biblioteca da sua Unidade lhe fornecer\'a um PDF com a ficha
% catalogr\'afica definitiva. 
% Quando estiver com o documento, salve-o como PDF no diret\'orio
% do seu projeto como fichacatalografica.pdf e inclua o arquivo
% utilizando o comando abaixo:

%\includepdf{USPSC-TA-PreTextual/USPSC-fichacatalografica.pdf}

% Se voc\^e optar por elaborar a ficha catalogr\'afica, dever\'a 
% incluir uma % antes da linha % antes
% do comando %% USPSC-fichacatalografica.tex
% ---
% Inserir a ficha bibliografica
% ---
% Isto \'e um exemplo de Ficha Catalogr\'afica, ou ``Dados internacionais de
% cataloga\c{c}\~ao-na-publica\c{c}\~ao''. Voc\^e pode utilizar este modelo como refer\^encia. 
% Por\'em, provavelmente a biblioteca da sua universidade lhe fornecer\'a um PDF
% com a ficha catalogr\'afica definitiva ap\'os a defesa do trabalho. Quando estiver
% com o documento, salve-o como PDF no diret\'orio do seu projeto e substitua todo
% o conte\'udo de implementa\c{c}\~ao deste arquivo pelo comando abaixo:
%
\begin{fichacatalografica}
	\hspace{-1.4cm}
	\imprimirnotaautorizacao \\ \\
	%\sffamily
	\vspace*{\fill}					% Posi\c{c}\~ao vertical
\begin{center}					% Minipage Centralizado
  \imprimirnotabib \\
  \begin{table}[Htb]
	\scriptsize
	\centering	
	\begin{tabular}{|p{0.9cm} p{8.7cm}|}
		\hline
	      & \\
		  &	  \imprimirautorficha     \\
		
		 \imprimircutter & 
							\hspace{0.4cm}\imprimirtitulo~  / ~\imprimirautor~ ;  ~\imprimirorientadorcorpoficha. -- 	\imprimirlocal, \imprimirdata.   \\
		
		  &  % Para incluir nota referente \`a vers\~ao corrigida no corpo da ficha,
			  % incluir % no in\'{\i}cio da linha acima e tirar a % do in\'{\i}cio da linha abaixo
			  %	\hspace{0.4cm} \imprimirtitulo~  / ~\imprimirautor~ ; ~\imprimirorientadorcorpoficha~- ~\imprimirnotafolharosto. -- \imprimirlocal, \imprimirdata.  \\
		
			\hspace{0.4cm}\pageref{LastPage} p. : il. (algumas color.) ; 30 cm.\\ 
		  & \\
		  & 
		    \hspace{0.4cm}\imprimirnotaficha ~--~ 
						  \imprimirunidademin, 
						  \imprimiruniversidademin, 
		                  \imprimirdata. \\ 
		  & \\                 
		   % Para incluir nota referente \`a vers\~ao corrigida em notas,
		    % incluir uma % no in\'{\i}cio da linha acima e	
		    % tirar a % do in\'{\i}cio da linha abaixo
		    % & \hspace{0.4cm}\imprimirnotafolharosto \\ 
		  & \\ 
		  & \hspace{0.4cm}1. STEAM. 2. STEM. 3. UTFPR. 4. WASH. 5. Aprendizagem. 6. Papert. 7. Educa\c{c}\~ao. I. \imprimirorientadorficha. 
		   II. T\'{\i}tulo. \\
	
		     %Se houver co-orientador, inclua % antes da linha (antes de II. T\'{\i}tulo.) 
		     %          e tire a % antes do comando abaixo 
		     %III. T\'{\i}tulo. \\   
		  \hline
	\end{tabular}
  \end{table}
\end{center}
\end{fichacatalografica}
% ---

 
% e retirar o % do comando abaixo
%% USPSC-fichacatalograficaTutorial.tex
% ---
% Inserir a ficha bibliografica
% ---
% Isto \'e um exemplo de Ficha Catalogr\'afica, ou ``Dados internacionais de
% cataloga\c{c}\~ao-na-publica\c{c}\~ao''. Voc\^e pode utilizar este modelo como refer\^encia. 
% Por\'em, provavelmente a biblioteca da sua universidade lhe fornecer\'a um PDF
% com a ficha catalogr\'afica definitiva ap\'os a defesa do trabalho. Quando estiver
% com o documento, salve-o como PDF no diret\'orio do seu projeto e substitua todo
% o conte\'udo de implementa\c{c}\~ao deste arquivo pelo comando abaixo:

\textbf{UNIVERSIDADE DE S\~AO PAULO} 

Reitor: Vahan Agopyan

Vice-Reitor: Ant\^onio Carlos Hernandes\\

\textbf{Grupo Desenvolvedor do Pacote USPSC} 

\textbf{Coordena\c{c}\~ao e Programa\c{c}\~ao}

- Marilza Aparecida Rodrigues Tognetti (PUSP-SC)
	
- Ana Paula Aparecida Calabrez (PUSP-SC) 

\textbf{Normaliza\c{c}\~ao}

- Ana Paula Aparecida Calabrez (PUSP-SC) 

- Brianda de Oliveira Ordonho Sigolo (IAU)

- Eduardo Graziosi Silva (EESC)

- Eliana de C\'assia Aquareli Cordeiro (IQSC)

- Fl\'avia Helena Cassin (EESC)

- Maria Cristina Cavarette Dziabas (IFSC)

- Marilza Aparecida Rodrigues Tognetti (PUSP-SC)

- Regina C\'elia Vidal Medeiros (ICMC) \\


%
\begin{fichacatalografica}
%	\hspace{-1.4cm}
   \vspace*{\fill}					% Posi\c{c}\~ao vertical
\begin{center}					% Minipage Centralizado
  \imprimirnotabib \\
  \begin{table}[htb]
	\scriptsize
	\centering	
	\begin{tabular}{|p{0.9cm} p{8.7cm}|}
		\hline
	      & \\
		  &	  \imprimirautorficha     \\
		
		 \imprimircutter & 
							\hspace{0.4cm}\imprimirtitulo~ / ~{Marilza Aparecida Rodrigues Tognetti; Ana Paula Aparecida Calabrez,  coordenadoras e progamadoras. Brianda de Oliveira Ordonho Sigolo ...[\textit{et al.}], normalizadoras}.
							 -- 	\imprimirlocal, USP, \imprimirdata.   \\
		
		  &			\hspace{0.4cm}\pageref{LastPage} p. : il. (algumas color.) ; 30 cm.\\ 
 		  & \\ 
		  & \hspace{0.4cm}1. LaTeX. 2. abnTeX. 3. Classe USPSC. 4. Editora\c{c}\~ao de texto. 5. Normaliza\c{c}\~ao da documenta\c{c}\~ao. 6. Tese. 7. Disserta\c{c}\~ao. 8. Documentos (elabora\c{c}\~ao). 9. Documentos eletr\^onicos. I. Calabrez, A. P. A., coord., program., normaliz. II. Sigolo, B. O. O. normaliz. III. Cordeiro, E. C. A., normaliz. IV. Cassin, Fl\'avia Helena, normaliz. V. Dziabas, M. C. C., normaliz. VI. Medeiros, R. C. V., normaliz.  VII. T\'{\i}tulo.  \\
	
		  \hline
	\end{tabular}
  \end{table}
\end{center}
\end{fichacatalografica}
% ---

% As informa\c{c}\~oes que comp\~oem a ficha catalogr\'afica est\~ao 
% definidas no arquivo USPSC-pre-textual-UUUU.tex
% ---

% ---
% Folha de rosto adicional
% Para imprimir a folha de rosto adicional, exigida por algumas Unidades, a exemplo do ICMC,
% retire a % antes do comando abaixo

%\imprimirfolhaderostoadic

% ---
% ---
% Inserir errata
% ---

\include{USPSC-Tutorial/USPSC-ErrataTutorial_RedarTex}

% ---

% ---
% Inserir folha de aprova\c{c}\~ao
% ---

% A Folha de aprova\c{c}\~ao \'e um elemento obrigat\'orio da NBR 4724/2011 (se\c{c}\~ao 4.2.1.3). 
% Ap\'os a defesa/aprova\c{c}\~ao do trabalho, gere o arquivo folhadeaprovacao.pdf da p\'agina assinada pela banca 
% e iclua o arquivo utilizando o comando abaixo:
\includepdf{USPSC-TA-PreTextual/USPSC-folhadeaprovacao.pdf}
% Alternativa para a Folha de Aprova\c{c}\~ao:
% Se for a sua op\c{c}\~ao elaborar uma folha de aprova\c{c}\~ao, insira uma % antes do comando acima que inclui o arquivo folhadeaprovacao.pdf,
% tire o % do comando abaixo e altere o arquivo folhadeaprovacao.tex conforme suas necessidades
%\include{folhadeaprovacao}
\includepdf{USPSC-TA-PreTextual/USPSC-PaginaEmBranco.pdf}

% ---
% Dedicat\'oria
% ---
\include{USPSC-Tutorial/USPSC-DedicatoriaTutorial_RedarTex}
% ---

% ---
% Agradecimentos
% ---
%% USPSC-AgradecimentosTutorial.tex
\begin{agradecimentos}
	A motiva\c{c}\~ao para o desenvolvimento da classe USPSC e dos modelos de trabalhos acad\^emicos foi decorrente de solicita\c{c}\~oes de usu\'arios das Bibliotecas do Campus USP de S\~ao Carlos. A vers\~ao 3.0 do Pacote USPSC para modelos de trabalhos acad\^emicos \'e composta da \textbf{Classe USPSC}, do \textbf{Modelo para TCC em \LaTeX\ utilizando o Pacote USPSC} e do \textbf{Modelo para teses e disserta\c{c}\~oes em \LaTeX\ utilizando o Pacote USPSC}.
	
	Nesta vers\~ao do Pacote USPSC, foram feitas altera\c{c}\~oes na Classe USPSC, inclus\~ao da capa exclusiva para o PROGRAMA DE P\'OS-GRADUA\c{C}\~AO EM ENSINO DE CI\^ENCIAS HUMANAS, SOCIAIS E DA NATUREZA - PPGEN\^encias Matem\'aticas e de Computa\c{c}\~ao (ICMC), inclus\~ao de novos pacotes e altera\c{c}\~oes nos modelos de trabalhos acad\^emicos.
	
	Na vers\~ao 3.0 do Pacote USPSC, as mudan\c{c}as foram estruturais na programa\c{c}\~ao e conte\'udo. Destacamos que os modelos \textbf{USPSC-modelo.tex} e \textbf{USPSC-TCC-modelo.tex} foram simplificados, no que tange ao conte\'udo, e foi criado o \textbf{Tutorial do Pacote USPSC para modelos de trabalhos de acad\^emicos em LaTeX - vers\~ao 3.0}, contendo as instru\c{c}\~oes precisas e detalhadas para melhor utiliza\c{c}\~ao dos recursos do Pacote USPSC. Para tanto, foram acrescidos diversos arquivos, para atender as especificidades do tutorial que possui os elementos pr\'e-textuais distintos para teses, disserta\c{c}\~oes, TCCs e outros trabalhos acad\^emicos, conforme descrito em  \textbf{\ref{Pacote} Pacote USPSC: Classe USPSC e modelos de trabalhos de acad\^emicos}. A estrutura deste tutorial \'e igual \`a  estrutura de trabalhos acad\^emicos estabelecida pela ABNT NBR 14724, conforme a \autoref{fig_EstruturaTrabAcad}, portanto o usu\'ario do Pacote USPSC pode utilizar os exemplos e recursos de \LaTeX\ nele contidos.	
	 
	Na vers\~ao 3.1 houve a inclus\~ao da capa diferenciada para o PROGRAMA DE P\'OS-GRADUA\c{C}\~AO EM ENSINO DE CI\^ENCIAS HUMANAS, SOCIAIS E DA NATUREZA - PPGEN\^encias Matem\'aticas e de Computa\c{c}\~ao (ICMC), novos cursos e programas e de algumas altera\c{c}\~oes na Classe USPSC, arquivos USPSC.cls e  USPSC1.cls.
	
	O Grupo desenvolvedor do Pacote USPSC agradece especialmente ao Luis Olmes, doutorando do ICMC, pelas primeiras orienta\c{c}\~oes sobre o \LaTeX\ . 
	
	Agradecemos ao Lauro C\'esar Araujo pelo desenvolvimento da classe  \abnTeX, modelos can\^onicos e tantas outras contribui\c{c}\~oes que nos permitiu o desenvolvimento o Pacote USPSC, composto da classe USPSC e seus modelos.
	
	Os nossos agradecimentos aos integrantes do primeiro
	projeto abn\TeX\, Gerald Weber, Miguel Frasson, Leslie H. Watter, Bruno Parente Lima, Fl\'avio de Vasconcellos Corr\^ea, Otavio Real
	Salvador, Renato Machnievscz, e a todos que contribu\'{\i}ram para que a produ\c{c}\~ao de trabalhos acad\^emicos em conformidade com
	as normas ABNT com \LaTeX\ fosse poss\'{\i}vel.
	
	Agradecemos ao grupo de usu\'arios
	\emph{latex-br}  {\url{http://groups.google.com/group/latex-br}}, aos integrantes do grupo
	\emph{\abnTeX}  {\url{http://groups.google.com/group/abntex2}  e \url{http://www.abntex.net.br/}}~que contribuem para a evolu\c{c}\~ao do \abnTeX.
	
	Agradecemos aos usu\'arios do Pacote USPSC que nos tem dado um \textit{feedback} e sugest\~oes de melhoria. 
	
\end{agradecimentos}
% ---
% ---

% ---
% Ep\'{\i}grafe
% ---
\include{USPSC-Tutorial/USPSC-EpigrafeTutorial_RedarTex}
% ---

% A T E N \c{C} \~A O
% Se o idioma do texto for em ingl\^es, o abstract deve preceder o resumo
% resumo em portugu\^es
%
% Resumo
% ---
%% USPSC-ResumoTutorial.tex
\setlength{\absparsep}{18pt} % ajusta o espa\c{c}amento dos par\'agrafos do resumo		
\begin{resumo}
	\begin{flushleft} 
		\setlength{\absparsep}{0pt} % ajusta o espa\c{c}amento da refer\^encia	
		\SingleSpacing 
		\imprimirautorabr.~~\textbf{\imprimirtituloresumo}.~~\imprimirorientador~~	
		%Substitua p. por f. quando utilizar oneside em \documentclass
		%\pageref{LastPage}f.
		\imprimirlocal: \imprimirinstituicao, \imprimirdata. \pageref{LastPage}p. 
	\end{flushleft}
\OnehalfSpacing 			
 O resumo deve ressaltar o  objetivo, o m\'etodo, os resultados e as conclus\~oes do documento. A ordem e a extens\~ao  destes itens dependem do tipo de resumo (informativo ou indicativo) e do  tratamento que cada item recebe no documento original. O resumo deve ser
 precedido da refer\^encia do documento, com exce\c{c}\~ao do resumo inserido no
 pr\'oprio documento. (\ldots)  Salientamos que algumas Unidades exigem o titulo dos trabalhos acad\^emicos em ingl\^es, tornando necess\'ario a inclus\~ao das refer\^encias nos resumos e abstracts, o que foi adotado no \textbf{Modelo para TCC em \LaTeX\ utilizando o Pacote USPSC} e no \textbf{Modelo para teses e disserta\c{c}\~oes em \LaTeX\ utilizando o Pacote USPSC}. As palavras-chave devem figurar logo abaixo do  resumo, antecedidas da express\~ao Palavras-chave:, separadas entre si por  ponto e finalizadas tamb\'em por ponto \cite{nbr6028}.
 

 \textbf{Palavras-chave}: LaTeX. abnTeX. Classe USPSC. Editora\c{c}\~ao de texto. Normaliza\c{c}\~ao da documenta\c{c}\~ao. Trabalho acad\^emico. Tese. Disserta\c{c}\~ao. Trabalho de conclus\~ao de curso (TCC). Documentos (elabora\c{c}\~ao). Documentos eletr\^onicos. 
\end{resumo}
% ---

% Abstract
% ---
\include{USPSC-Tutorial/USPSC-AbstractTutorial_RedarTex}
% ---

% ---
% inserir lista de figurass
% ---
\pdfbookmark[0]{\listfigurename}{lof}
\listoffigures*
\cleardoublepage
% ---

% ---
% inserir lista de tabelas
% ---
\pdfbookmark[0]{\listtablename}{lot}
\listoftables*
\cleardoublepage
% ---

% ---
% inserir lista de quadros
% ---
\pdfbookmark[0]{\listofquadroname}{loq}
\listofquadro*
\cleardoublepage
% ---

% ---
% inserir lista de abreviaturas e siglas
% ---
% USPSC-AbreviaturasSiglasTutorial.tex
\begin{siglas}
    \item[ABNT] Associa\c{c}\~ao Brasileira de Normas T\'ecnicas
    \item[abnTeX] ABsurdas Normas para TeX
	\item[EESC] Escola de Engenharia de S\~ao Carlos
	\item[IAU] Programa de P\'os-Gradua\c{c}\~ao em Ensino de Ci\^encias Humanas, Sociais e da Natureza - PPGE
	\item[IBGE] Instituto Brasileiro de Geografia e Estat\'{\i}stica
	\item[ICMC] Programa de P\'os-Gradua\c{c}\~ao em Ensino de Ci\^encias Humanas, Sociais e da Natureza - PPGE\^encias Matem\'aticas e de Computa\c{c}\~ao
	\item[IFSC] Programa de P\'os-Gradua\c{c}\~ao em Ensino de Ci\^encias Humanas, Sociais e da Natureza - PPGE\'{\i}sica de S\~ao Carlos
	\item[IQSC] Programa de P\'os-Gradua\c{c}\~ao em Ensino de Ci\^encias Humanas, Sociais e da Natureza - PPGE\'{\i}mica de S\~ao Carlos
	\item[LaTeX] Lamport TeX
	\item[PDF] Portable Document Format
	\item[PUSP-SC] Prefeitura do Campus USP de S\~ao Carlos
	\item[TCC] Trabalho de Conclus\~ao de Curso
	\item[USP] Universidade de S\~ao Paulo
	\item[USPSC] Campus USP de S\~ao Carlos
\end{siglas}

% ---

% ---
% inserir lista de s\'{\i}mbolos
% ---
\include{USPSC-Tutorial/USPSC-SimbolosTutorial_RedarTex}
% ---
% ---
% inserir o sumario
% ---
\pdfbookmark[0]{\contentsname}{toc}
\tableofcontents*
\cleardoublepage
% ---
% ----------------------------------------------------------
% ELEMENTOS TEXTUAIS
% ----------------------------------------------------------
\textual
% Os cap\'{\i}tulos s\~ao inseridos como arquivos externos 

% Cap\'{\i}tulo 1 - Introdu\c{c}\~ao
% ---
\chapter[ORGANIZA\c{C}\~AO DESTE TRABALHO]{ORGANIZA\c{C}\~AO DESTE TRABALHO}\label{ORGANIZA\c{C}\~AO DESTE TRABALHO}
Neste trabalho buscamos utilizar a forma mais can\^onica de organiza\c{c}\~ao de um texto cient\'{\i}fico







\begin{alineas}
\item introdu\c{c}\~ao
\item m\'etodos
\item resultados
\item discuss\~ao
\end{alineas}

A refer\^encia KARA-JUNIOR (2014) busca identificar o car\'ater principal de cada um dos elementos acima por meio de uma pergunta, como segue:







\begin{alineas}
\item Na Introdu\c{c}\~ao devemos explicitar qual \'e a pergunta que o trabalho vai tentar responder   (\textquotedbl que pergunta foi feita?\textquotedbl )
\item Nos M\'etodos devemos explicitar o caminho que seguiremos para responder a pergunta que foi explicitada na Introdu\c{c}\~ao (\textquotedbl como encontraremos a resposta para a pergunta?\textquotedbl )
\item Nos Resultados devemos explicitar onde o caminho descrito em M\'etodos nos trouxe, em termos da resposta obtida para a pergunta feita na Introdu\c{c}\~ao (\textquotedbl qual foi a resposta encontrada para a pergunta da introdu\c{c}\~ao?\textquotedbl )
\item Na Discuss\~ao devemos fazer uma interpreta\c{c}\~ao dos achados descritos em Resultados (\textquotedbl o que os resultados que encontramos significam?\textquotedbl )
\end{alineas}

\begin{flushright}
\setlength{\absparsep}{0pt}
\tiny \begin{flushright}
\setlength{\absparsep}{0pt}
\tiny \begin{flushright}
\setlength{\absparsep}{0pt}
\tiny \begin{flushright}
\setlength{\absparsep}{0pt}
\tiny \begin{flushright}
\setlength{\absparsep}{0pt}
\tiny \begin{flushright}
\setlength{\absparsep}{0pt}
\tiny (fonte:  (KARA-JUNIOR, 2014) ) \normalsize 
\end{flushright}

 \normalsize 
\end{flushright}

 \normalsize 
\end{flushright}

 \normalsize 
\end{flushright}

 \normalsize 
\end{flushright}

 \normalsize 
\end{flushright}


\section[Exemplifica\c{c}\~ao de uma poss\'{\i}vel organiza\c{c}\~ao deste texto]{Exemplifica\c{c}\~ao de uma poss\'{\i}vel organiza\c{c}\~ao deste texto}\label{Exemplifica\c{c}\~ao de uma poss\'{\i}vel organiza\c{c}\~ao deste texto}
Usando  KARA-JUNIOR (2014) como guia e considerando que nesta disserta\c{c}\~ao, como se ver\'a adiante, o objeto de estudo \'e o Projeto WASH, podemos fazer um exerc\'{\i}cio de imagina\c{c}\~ao de quais perguntas hipot\'eticas poderiam ser explicitadas na Introdu\c{c}\~ao.






 KARA-JUNIOR (2014) indica que a Introdu\c{c}\~ao precisa apresentar qual pergunta est\'a sendo feita. Ent\~ao vamos imaginar 3 hipot\'eticas perguntas sobre o Projeto WASH:







\begin{alineas}
\item \textquotedbl Qual \'e a hist\'oria do Projeto WASH?\textquotedbl 
\item \textquotedbl De que forma o WASH \'e executado?\textquotedbl 
\item \textquotedbl Quais resultados o WASH alcan\c{c}ou?\textquotedbl 
\end{alineas}

Neste ponto, poder\'{\i}amos imaginar as seguintes respostas para o questionamento \textquotedbl como encontraremos a resposta para a pergunta feita na Introdu\c{c}\~ao?\textquotedbl , que segundo KARA-JUNIOR (2014) deveria estar na parte de M\'etodos:







\begin{alineas}
\item Por exemplo, se na Introdu\c{c}\~ao a pergunta fosse \textquotedbl Qual a hist\'oria do Projeto WASH?\textquotedbl , em Materiais e M\'etodos poder\'{\i}amos ter como  conte\'udo: \textquotedbl Para estudar a hist\'oria do Projeto WASH n\'os levantaremos o acervo de documentos oficiais que levaram \`a cria\c{c}\~ao do projeto\textquotedbl . Ou talvez, ainda no campo da exemplifica\c{c}\~ao, a resposta em Materiais e M\'etodos pudesse ser diferente: \textquotedbl Para levantar a hist\'oria do Projeto WASH foi feita uma pesquisa das refer\^encias presentes na m\'{\i}dia\textquotedbl .
\item Se a pergunta da Introdu\c{c}\~ao fosse, por exemplo, \textquotedbl De que forma o WASH \'e executado?\textquotedbl  a parte de M\'etodo talvez pudesse conter o seguinte conte\'udo: \textquotedbl Para descobrir como o WASH \'e executado ser\'a preciso aplicar o m\'etodo de modelagem de neg\'ocios do tipo Business Process Model Notation (BPMN)\textquotedbl . Podemos pensar tamb\'em em outra abordagem: \textquotedbl Para caracterizar a execu\c{c}\~ao do WASH, ser\'a preciso usar o M\'etodo da Modelagem por Objetos\textquotedbl .
\item Se a Introdu\c{c}\~ao trouxesse a pergunta \textquotedbl Quais resultados o WASH alcan\c{c}ou?\textquotedbl , um poss\'{\i}vel conte\'udo para M\'etodos poderia ser: \textquotedbl Para descobrir quais resultados foram alcan\c{c}ados pelo WASH \'e preciso aplicar m\'etodo de banco de dados relacionais, etc.\textquotedbl . Outro poss\'{\i}vel conte\'udo para M\'etodos seria \textquotedbl Os resultados do WASH ser\~ao analisados atrav\'es de uma planilha eletr\^onica do tipo Excel\textquotedbl .
\end{alineas}

Vamos continuar o nosso exerc\'{\i}cio de exemplifica\c{c}\~ao da organiza\c{c}\~ao do texto, mas agora imaginando hipot\'eticos conte\'udos para a parte de Resultados, ainda respectivamente aos 3 exemplos da lista anterior:







\begin{alineas}
\item Se na Introdu\c{c}\~ao a pergunta fosse \textquotedbl Qual a hist\'oria do Projeto WASH?\textquotedbl , em Resultados poder\'{\i}amos ter uma narrativa como esta: \textquotedbl Segundo o instrumento jur\'{\i}dico presente no acervo, o Projeto WASH foi criado em 2013, no Centro de Tecnologia da Informa\c{c}\~ao Renato Archer, etc.\textquotedbl 
\item Se na introdu\c{c}\~ao a pergunta explicitada fosse \textquotedbl De que forma o WASH \'e executado?\textquotedbl  \'e poss\'{\i}vel imaginar um conte\'udo para resultados como segue: \textquotedbl A aplica\c{c}\~ao das ferramentas de an\'alise do m\'etodo do WASH resultaram no diagrama BPMN presente na figura, que indica a exist\^encia de 100 subprocessos de fluxo de informa\c{c}\~oes, etc.\textquotedbl 
\item Se a Introdu\c{c}\~ao trouxesse a pergunta \textquotedbl Quais resultados o WASH alcan\c{c}ou?\textquotedbl , um poss\'{\i}vel conte\'udo para Resultados seria: \textquotedbl O gr\'afico obtido a partir da Base de Dados Relacional mostra a evolu\c{c}\~ao do n\'umero de pessoas atendidas ao longo dos 10 anos de execu\c{c}\~ao do Projeto\textquotedbl 
\end{alineas}

Para terminar esta exemplifica\c{c}\~ao, passamos a imaginar conte\'udos para a parte de Discuss\~ao, ainda conforme a estrutura proposta por  KARA-JUNIOR (2014):







\begin{alineas}
\item Se na Introdu\c{c}\~ao a pergunta fosse \textquotedbl Qual a hist\'oria do Projeto WASH?\textquotedbl , na parte de Discuss\~ao poder\'{\i}amos ter o conte\'udo de car\'ater interpretativo: \textquotedbl Como se v\^e, o Projeto WASH tem elementos de outros projetos, tais como o GESAC e OLPC\textquotedbl .
\item Se na introdu\c{c}\~ao a pergunta explicitada fosse \textquotedbl De que forma o WASH \'e executado?\textquotedbl , na Discuss\~ao \'e poss\'{\i}vel imaginar um conte\'udo como segue: \textquotedbl A an\'alise do processo de execu\c{c}\~ao do Projeto WASH mostrou que \'e preciso rever a forma como se d\'a sua execu\c{c}\~ao remota\textquotedbl .
\item Se a Introdu\c{c}\~ao trouxesse a pergunta \textquotedbl Quais resultados o WASH alcan\c{c}ou?\textquotedbl , um poss\'{\i}vel conte\'udo para a Discuss\~ao seria: \textquotedbl A an\'alise do gr\'afico de n\'umero de eventos ao longo dos meses mostra que a pandemia teve um impacto na capacidade de execu\c{c}\~ao presencial do projeto.\textquotedbl 
\end{alineas}

Os elementos aqui exemplificados n\~ao foram imaginados de forma aleat\'oria, mas t\^em base nos achados deste trabalho. Desta forma, o que se buscou aqui \'e, a partir de exemplos, mostrar a forma como esta disserta\c{c}\~ao est\'a organizada.






\section[Outras refer\^encias para organiza\c{c}\~ao deste texto]{Outras refer\^encias para organiza\c{c}\~ao deste texto}\label{Outras refer\^encias para organiza\c{c}\~ao deste texto}
No sentido de refor\c{c}ar a sustenta\c{c}\~ao da escolha de organiza\c{c}\~ao que fizemos, trazemos  BATES (2014), que tem uma forma equivalente de expressar os mesmos conceitos oferecidos por  KARA-JUNIOR (2014).






Em sua descri\c{c}\~ao, BATES (2014)  adota a estrutura b\'asica, com \textquotedbl Introdu\c{c}\~ao, M\'etodos e Resultados\textquotedbl , muito semelhante \`a de KARA-JUNIOR (2014). Suas perguntas tamb\'em s\~ao equivalentes, como se v\^e abaixo:







\begin{alineas}
\item Resumo: O que eu fiz de uma forma bem sint\'etica (\textquotedbl in a nutshell\textquotedbl )?
\item Introdu\c{c}\~ao: Qual \'e o problema?
\item Materiais e M\'etodos: Como eu resolvi o problema?
\item Resultados: O que eu achei?
\item Agradecimentos (opcional): Quem me ajudou a fazer?
\item Literatura citada: Quais trabalhos eu usei como refer\^encia?
\item Ap\^endices: Informa\c{c}\~ao Extra
\end{alineas}

\begin{flushright}
\setlength{\absparsep}{0pt}
\tiny \begin{flushright}
\setlength{\absparsep}{0pt}
\tiny \begin{flushright}
\setlength{\absparsep}{0pt}
\tiny \begin{flushright}
\setlength{\absparsep}{0pt}
\tiny \begin{flushright}
\setlength{\absparsep}{0pt}
\tiny \begin{flushright}
\setlength{\absparsep}{0pt}
\tiny (tradu\c{c}\~ao livre de BATES (2014)) \normalsize 
\end{flushright}

 \normalsize 
\end{flushright}

 \normalsize 
\end{flushright}

 \normalsize 
\end{flushright}

 \normalsize 
\end{flushright}

 \normalsize 
\end{flushright}


A variante acima, embora baseada numa refer\^encia que trata a quest\~ao das publica\c{c}\~oes curtas (ou \textquotedbl papers\textquotedbl ), \'e v\'alida para outros tipos de registros cient\'{\i}ficos.






Um trabalho mais completo de descri\c{c}\~ao, mas baseado na mesma estrutura b\'asica, \'e o apresentado por MEO (2018):








\captionsetup{format=plain}
\begin{figure}[htb]

	\begin{center}

		\includegraphics[max size={\textwidth}{\textheight}]{../../imagens/MEOS.png}

	\end{center}

	\caption{\label{8372664f308c109a87f368fec1da024e1ad7d562}A estrutura de espinha de peixe de S.A. MEOs  (fonte: [[MEO, 2018]])}

\end{figure}

Uma vez que a Fig. 1 \'e um pouco \textquotedbl congestionada\textquotedbl  no sentido da densidade de informa\c{c}\~oes apresentadas, cabe uma descri\c{c}\~ao de cada \textquotedbl costela\textquotedbl  da \textquotedbl espinha-de-peixe\textquotedbl  de MEO (2018), transcrita aqui na forma de uma tabela:










\begin{table}[htb]
\caption{\label{404d3407ceff6ff9d0cc5a71368147d571c7f79a}Estrutura de espinha de peixe de MEO}

\centering
\begin{tabular}{|c|}
\hline
muitas coisas \\\hline

\end{tabular}
\end{table}


As 3 formas complementares aqui apresentadas (MEO (2018),  KARA-JUNIOR (2014) e BATES (2014)) tratam, principalmente, de papers cient\'{\i}ficos, em que a concis\~ao \'e especialmente necess\'aria. Mas papers cient\'{\i}ficos n\~ao s\~ao o \'unico formato dispon\'{\i}vel para realizar uma comunica\c{c}\~ao cient\'{\i}fica.






Existem formatos mais extensos para documenta\c{c}\~ao, tais como: relat\'orios, teses e cap\'{\i}tulos de livros. Alaide Mammana, em  MAMMANA (2020) , tamb\'em explora estas nuances, muito embora a estrutura b\'asica seja sempre \textquotedbl Introdu\c{c}\~ao\textquotedbl , \textquotedbl M\'etodos\textquotedbl , \textquotedbl Resultados\textquotedbl  e \textquotedbl Discuss\~ao\textquotedbl , como j\'a estressado aqui.






O presente texto, por se tratar de uma disserta\c{c}\~ao, precisa valer-se de formatos mais extensos, uma vez que envolve a \textquotedbl defesa para obten\c{c}\~ao de um t\'{\i}tulo\textquotedbl  (mestrado). Nesta situa\c{c}\~ao, \'e preciso demonstrar erudi\c{c}\~ao nos temas abordados, para que o conhecimento do candidato sobre o tema possa ser avaliado. Desta forma, nos \'e poss\'{\i}vel explorar melhor a necess\'aria busca por uma erudi\c{c}\~ao, dado que, diferentemente de um artigo cient\'{\i}fico em revista (\textquotedbl paper\textquotedbl ), que se caracteriza pela brevidade,  comporta a revis\~ao de conhecimentos j\'a existentes de forma mais estendida.






\section[Organiza\c{c}\~ao da Introdu\c{c}\~ao]{Organiza\c{c}\~ao da Introdu\c{c}\~ao}\label{Organiza\c{c}\~ao da Introdu\c{c}\~ao}
Seguindo as sugest\~oes presentes em MEO (2018), optamos por uma introdu\c{c}\~ao curta, leve e objetiva, complementada por um cap\'{\i}tulo de \textquotedbl Fundamenta\c{c}\~ao Te\'orica\textquotedbl . A introdu\c{c}\~ao \'e estruturada para culminar, por meio de seus \'ultimos par\'agrafos, na descri\c{c}\~ao do objeto de estudo. O caminho percorrido \'e o de descrever esse objeto do mais geral at\'e o espec\'{\i}fico.






Para que pud\'essemos nos expressar de forma organizada, buscando demonstrar um compromisso com a erudi\c{c}\~ao, sem perder a objetividade da \textquotedbl Introdu\c{c}\~ao\textquotedbl , acatamos a sugest\~ao do orientador de incluir um cap\'{\i}tulo de \textquotedbl Fundamenta\c{c}\~ao Te\'orica\textquotedbl , no qual os temas pincelados na \textquotedbl Introdu\c{c}\~ao\textquotedbl  pudessem ser mais profundamente descritos, sem preju\'{\i}zo para um formato \textquotedbl leve e balanceado\textquotedbl  para a apresenta\c{c}\~ao da literatura na introdu\c{c}\~ao, um aspecto que, segundo  MEO (2018), deve ser perseguido pelo redator de textos cient\'{\i}ficos. Assim, a exposi\c{c}\~ao na \textquotedbl Introdu\c{c}\~ao\textquotedbl  buscou ser o mais sint\'etica poss\'{\i}vel, com direcionamento para a \textquotedbl Fundamenta\c{c}\~ao Te\'orica\textquotedbl  sempre que foi necess\'ario aprofundar algum conceito.






Para facilitar a sua localiza\c{c}\~ao no texto, optamos, tamb\'em, por colocar em subse\c{c}\~oes da \textquotedbl Introdu\c{c}\~ao\textquotedbl  os itens que caracterizam o escopo da tese:







\begin{alineas}
\item exposi\c{c}\~ao dos problemas
\item hip\'oteses
\item das quest\~oes
\item e dos objetos de interesse
\end{alineas}

\section[Organiza\c{c}\~ao de Materiais e M\'etodos]{Organiza\c{c}\~ao de Materiais e M\'etodos}\label{Organiza\c{c}\~ao de Materiais e M\'etodos}
Uma vez que a presente pesquisa tem por objeto a caracteriza\c{c}\~ao do Projeto WASH quanto a:







\begin{alineas}
\item sua hist\'oria (trajet\'oria)
\item seus m\'etodos
\item e seus resultados
\end{alineas}

podemos considerar a descri\c{c}\~ao da hist\'oria(a) e do m\'etodo do WASH(b), bem como seus indicadores(c), como resultados da aplica\c{c}\~ao do m\'etodo da pesquisa adotado nesta disserta\c{c}\~ao.






Em outras palavras, uma das dimens\~oes do m\'etodo que empregamos neste mestrado refere-se a caracterizar o m\'etodo do WASH. Portanto, a pesquisa realizada aqui envolve, al\'em do m\'etodo historiogr\'afico, um outro que poderia ser considerado como m\'etodo de caracteriza\c{c}\~ao de m\'etodos.






Assim, a descri\c{c}\~ao do m\'etodo do WASH deve ser considerada uma decorr\^encia da aplica\c{c}\~ao do m\'etodo da pesquisa desta disserta\c{c}\~ao e, por esse motivo, encontra-se no cap\'{\i}tulo de \textquotedbl Resultados e Discuss\~oes\textquotedbl  e n\~ao no cap\'{\i}tulo de \textquotedbl Materiais e M\'etodos\textquotedbl .






\section[Organiza\c{c}\~ao de Resultados e Discuss\~oes]{Organiza\c{c}\~ao de Resultados e Discuss\~oes}\label{Organiza\c{c}\~ao de Resultados e Discuss\~oes}
Optamos por juntar em um \'unico cap\'{\i}tulo os resultados e as discuss\~oes (\textquotedbl Resultados e Discuss\~oes\textquotedbl ). Esta op\c{c}\~ao visa garantir uma melhor fluidez, dado que permite apresentar as op\c{c}\~oes de an\'alise que levaram \`a proposta de melhorias no m\'etodo do WASH, o produto final desta disserta\c{c}\~ao.






\section[Organiza\c{c}\~ao de Produto Tecnol\'ogico]{Organiza\c{c}\~ao de Produto Tecnol\'ogico}\label{Organiza\c{c}\~ao de Produto Tecnol\'ogico}
A presente disserta\c{c}\~ao, por se tratar de um Mestrado Tecnol\'ogico, deve culminar com a apresenta\c{c}\~ao de um \textquotedbl produto de car\'ater pr\'atico\textquotedbl , que no presente caso ser\'a uma revis\~ao do Documento de Refer\^encia do Projeto WASH  constante do anexo da  Portaria CTI 178/2018 (CTI, 2018). Por esse motivo foi acrescentado \`a estrutura do documento um cap\'{\i}tulo de \textquotedbl Produto Tecnol\'ogico\textquotedbl .






\chapter[INTRODU\c{C}\~AO]{INTRODU\c{C}\~AO}\label{INTRODU\c{C}\~AO}
Aos olhos de jovens observadores contempor\^aneos, parece natural a relativa desenvoltura com que as pessoas utilizam os computadores e os celulares nos dias de hoje. J\'a est\~ao bastante difundidos os servi\c{c}os de governo eletr\^onico, os sites de com\'ercio eletr\^onico, os  aplicativos de entrega, as plataformas de ensino, de reuni\~oes, a busca por oportunidades profissionais, o voto eletr\^onico, banco e o caixa eletr\^onico, por exemplo.






Desta forma, \'e poss\'{\i}vel afirmar que as pessoas t\^em usado com frequ\^encia e com relativa facilidade as ferramentas digitais instaladas em computadores e em celulares, sejam aplicativos de mensagens, buscadores (browsers), correio eletr\^onico, redes sociais, entre outras. Esse uso d\'a-se em v\'arios contextos: profissional, educacional, de entretenimento, de intera\c{c}\~ao social, al\'em dos servi\c{c}os de governo eletr\^onico.






As novas gera\c{c}\~oes precisam, no entanto, saber que n\~ao foi sempre assim. Muito embora a percep\c{c}\~ao corrente de que o uso de computadores e celulares \'e indispens\'avel para o conv\'{\i}vio na sociedade, a rigor seu uso \'e relativamente recente.






\'E poss\'{\i}vel identificar a evolu\c{c}\~ao das telecomunica\c{c}\~oes a partir do s\'eculo passado como origem das transforma\c{c}\~oes tecnol\'ogicas que disponibilizaram tecnologias digitais em larga escala. Esse fato foi identificado, por exemplo, por PIERRE LEVY, no texto \textquotedbl Cibercultura\textquotedbl  (LEVY, 2000):







\noindent\begin{center}\mbox{\centering\fbox{\centering\par\parbox{0.7\linewidth}{\small\textit{\textquotedbl (...)durante uma entrevista nos anos 50, Albert Einstein (1879-1955) declarou que tr\^es grandes bombas haviam  explodido durante o s\'eculo XX: a bomba demogr\'afica, a bomba at\^omica e a bomba das telecomunica\c{c}\~oes. Esta \'ultima ‘bomba’ foi chamada por Roy Ascott ( pioneiro e te\'orico das artes em rede) de Segundo Dil\'uvio, o das informa\c{c}\~oes. As telecomunica\c{c}\~oes geram esse novo dil\'uvio por conta da natureza exponencial, explosiva e ca\'otica do seu crescimento. A quantidade de dados e links se multiplica, acelera, aumento de links, banco de dados, hipertextos nas redes, contos transversais, etc.\textquotedbl }\normalize}}}\end{center}


Ainda, segundo Pierre Levy,







\noindent\begin{center}\mbox{\centering\fbox{\centering\par\parbox{0.7\linewidth}{\small\textit{\textquotedbl O segundo dil\'uvio n\~ao ter\'a fim. N\~ao h\'a nenhum fundo s\'olido sob o oceano de informa\c{c}\~oes. Devemos aceit\'a-lo como nossa nova condi\c{c}\~ao. Temos que ensinar nossos filhos a nadar, a flutuar, talvez a navegar.\textquotedbl }\normalize}}}\end{center}


Para chegar nesse ponto, governos tiveram que prover a infraestrutura de ci\^encia e tecnologia, comunica\c{c}\~oes e de redes digitais, bem como os meios de acesso a essas redes, algumas vezes com a participa\c{c}\~ao da iniciativa privada. Na outra ponta, tiveram que promover projetos, formular pol\'{\i}ticas p\'ublicas de C






Inicialmente as redes digitais estavam fortemente vinculadas \`a academia, \`as institui\c{c}òes de pesquisa e \`a \'area de defesa [XXX], principalmente num contexto estatal.  Posteriormente foram avan\c{c}ando em dire\c{c}\~ao ao suprimento das necessidades de relacionamento do cidad\~ao com o governo. Mas estas redes foram mais longe, e alcan\c{c}aram todas as demais dimens\~oes do cidad\~ao, tais como as de: consumidor, benefici\'ario de servi\c{c}os de sa\'ude, educando, trabalhador, empreendedor, contribuinte, eleitor, usu\'ario de servi\c{c}os banc\'arios, entre outras.  Essa expans\~ao se deu como resultado de v\'arias a\c{c}\~oes, mas sua universaliza\c{c}\~ao foi resultado principalmente do surgimento de novas formas de relacionamento social representadas pelas redes sociais digitais, que tornaram mais acess\'{\i}veis novas ferramentas de apoio ao ensino em sala de aula, o ensino \`a dist\^ancia, o com\'ercio eletr\^onico, a elei\c{c}\~ao eletr\^onica, os \textquotedbl market-places\textquotedbl , os aplicativos de transporte e entrega, etc.






Estas transforma\c{c}\~oes tiveram impactos econ\^omicos e sociais profundos, inclusive nas rela\c{c}\~oes de trabalho, seja na cria\c{c}\~ao ou extin\c{c}\~ao de posto de trabalho, bem como em suas formas de contrata\c{c}\~ao, jornada, remunera\c{c}\~ao, inclusive com a precariza\c{c}\~ao dos direitos trabalhistas. Elas est\~ao muito bem descritas  no relat\'orio da Unesco  de 2004 \textquotedbl Social Transformation in an Information Society: Rethinking Access to You and the World\textquotedbl  (DUTTON, 2004).






A amplitude destas transforma\c{c}\~oes foi sintetizada no conceito de \textquotedbl Sociedade da Informa\c{c}\~ao\textquotedbl , \`as vezes referido como \textquotedbl Era Digital\textquotedbl  ou \textquotedbl Era da Informa\c{c}\~ao\textquotedbl . Uma breve revis\~ao sobre esse conceito \'e apresentada na fundamenta\c{c}\~ao te\'orica.






O efeito destas transforma\c{c}\~oes no emprego vem exigindo dos governos, das empresas e dos cidad\~aos uma constante e r\'apida readapta\c{c}\~ao  das rela\c{c}\~oes do trabalho, comerciais, industriais e da produ\c{c}\~ao de novos saberes e compet\^encias. Consequentemente, tamb\'em o sistema educacional vem sendo desafiado a se adaptar, uma vez que \'e dele que se espera o preparo dos cidad\~aos para a nova realidade. Aqueles cidad\~aos que n\~ao se preparam correm o risco constante de ficarem sem sustento.






Inicialmente tais transforma\c{c}\~oes eram associadas principalmente \`a substitui\c{c}\~ao do trabalho humano decorrente da automa\c{c}\~ao industrial. Mas a radicaliza\c{c}\~ao no uso de solu\c{c}\~oes digitais, inclusive de intelig\^encia artificial, associadas ao aumento da conectividade, v\^em substituindo capacidades \textquotedbl cognitivas que antes eram exclusivas de humanos\textquotedbl [4]. Uma das consequ\^encias mais radicais \'e o surgimento de novos meios de explora\c{c}\~ao humana, representados pela \textquotedbl Gigs Economy [XXX], ou \textquotedbl Economia do Bico\textquotedbl , que precariza as rela\c{c}\~oes trabalhistas por meio de plataformas que as impessoaliza a ponto de camuflar a explora\c{c}\~ao [XXX]. O termo \textquotedbl bico\textquotedbl  aqui est\'a sendo usado como tradu\c{c}\~ao livre de \textquotedbl gigs\textquotedbl , que nos Estados Unidos \'e uma g\'{\i}ria que pode ser usada para trabalho tempor\'ario [XXX dicion\'ario].






V\'arios pa\'{\i}ses t\^em buscado uma melhor prepara\c{c}\~ao para enfrentar essas transforma\c{c}\~oes, dotando o cidad\~ao de meios cognitivos, de conhecimento e cultura para se readaptar. Para isso, t\^em procurado remodelar seus sistemas educacionais, uma vez que “ficar para tr\'as” em rela\c{c}\~ao aos demais pa\'{\i}ses pode afetar a prosperidade de suas popula\c{c}\~oes, sua autonomia e liberdade [XXX].






Mais do que simplesmente \textquotedbl treinar\textquotedbl  o cidad\~ao quanto ao uso  de servi\c{c}os digitais, a educa\c{c}\~ao tem um papel fundamental para preparar os cidad\~aos para sua inser\c{c}\~ao aut\^onoma e digna na sociedade transformada pelas tecnologias de informa\c{c}\~ao e comunica\c{c}\~ao. O Estado tem o desafio de estabelecer pol\'{\i}ticas p\'ublicas e prover infraestrutura para que o cidad\~ao possa ter acesso e se beneficiar, de forma aut\^onoma, dos recursos digitais e de comunica\c{c}\~ao, mas tamb\'em de contribuir com sua constru\c{c}\~ao, beneficiando-se profissionlmente da riqueza que ele gera. O cidad\~ao tamb\'em precisa ser capaz de entender \textquotedbl o que est\'a por tr\'as\textquotedbl  desses sistemas digitais, para que possa reagir aos excessos da \textquotedbl algoritmiza\c{c}\~ao\textquotedbl  de suas rela\c{c}òes com outros indiv\'{\i}duos.






A percep\c{c}\~ao da import\^ancia da educa\c{c}\~ao para a prosperidade da sociedade n\~ao \'e uma novidade. No caso americano, por exemplo, remonta aos prim\'ordios da independ\^encia. No cap\'{\i}tulo \textquotedbl Fundamenta\c{c}\~ao Te\'orica\textquotedbl  revisaremos as origens do conceito de \textquotedbl Science, Technology, Engineering and Mathematics\textquotedbl  (STEM), mostrando que j\'a em 1790 o presidente George Washington, em seu primeiro discurso do \textquotedbl Estado da Uni\~ao\textquotedbl  promovia a ci\^encia e literatura como uma base da \textquotedbl felicidade p\'ublica\textquotedbl  [XXX]. Essa percep\c{c}\~ao de valor perdurou por toda a exist\^encia americana, at\'e os dias de hoje. Em muitos momentos foi estimulada, inclusive, como resposta \`as amea\c{c}as externas, como foi o caso do sucesso sovi\'etico no programa espacial, representado pelo pioneirismo do lan\c{c}amento do sat\'elite Sputnik no final da d\'ecada de 50. \'E naquele cen\'ario da Guerra Fria que a pol\'{\i}tica de educa\c{c}\~ao em STEM e alfabetiza\c{c}\~ao cient\'{\i}fica e tecnol\'ogica passou a ser vista mais claramente como bem comum para o Estado, mesmo muito antes do uso desse acr\^onimo de forma oficial. ( Relat\'orio CRS para o Congresso, www.crs.gov, 2012)






N\~ao obstante esta permanente percep\c{c}\~ao p\'ublica da import\^ancia e do valor da ci\^encia, nos anos 90 foram identificadas fragilidades nas estruturas de educa\c{c}\~ao STEM americana, as quais prejudicavam a prosperidade, o \textquotedbl poderio nacional\textquotedbl , a inser\c{c}\~ao de seus cidad\~aos no novo mundo do trabalho, do empreendedorismo, de forma aut\^onoma, soberana  e pr\'ospera. Essas fragilidades foram evidenciadas pelo recorrente e relativamente baixo desempenho de adolescentes americanos no \textquotedbl Programme for International Student Assessment\textquotedbl  (PISA) [XXX Catterall]. Com isso, o governo federal americano teve que mobilizar a\c{c}\~oes para atualizar as compet\^encias curriculares, visando manter uma inser\c{c}\~ao hegem\^onica na economia do s\'eculo XXI.






Segundo o Relat\'orio CRS para o Congresso, do Servi\c{c}o de Pesquisa do Congresso, mais de 200 projetos de Lei contendo o termo \textquotedbl  educa\c{c}\~ao cient\'{\i}fica \textquotedbl  foram introduzidos nos 20 anos entre os 100 ( 1987-1988) e 110 (2007-2008). Sendo que 13 ag\^encias federais conduzem programas ou atividades de educa\c{c}\~ao STEM. (Pag.2 do Relat\'orio). 






[YYY precisa melhorar esse par\'agrafo]






Os atores governamentais e estudiosos daquele per\'{\i}odo identificavam que faltava aos EUA uma pol\'{\i}tica nacional uniforme e inclusiva de ensino de ci\^encias, pois era poss\'{\i}vel categorizar diferentes \^enfases sobre o assunto no vasto sistema educacional americano [XXX Catterall].






Mas existia tamb\'em o reconhecido pioneirismo da comunidade acad\^emica americana nos m\'etodos voltados para o aprendizado de temas relacionados ao STEM, ainda que n\~ao identificados sob esse acr\^onimo ou mesmo que n\~ao amplamente disseminados em seu sistema educacional, como viriam a reconhecer os relat\'orios do congresso americano [XXX citar]. 






Seymour Papert, matem\'atico sul-africano radicado nos EUA, do Laborat\'orio de Intelig\^encia Artificial do Massachusetts Institute of Technology (MIT), foi um  cientista e educador que acreditava  no  uso do computador como forma de revolucionar o sistema  educacional  desde os anos 60. Esse pesquisador, que vivenciou a guerra fria, colaborou na estrutura\c{c}\~ao do Departamento de  [YYY - t\'a faltando algo aqui]






Ele foi o \textquotedbl fil\'osofo dos pioneiros a pensar a aprendizagem de crian\c{c}as de forma diferente. Em 1968 escreveu o artigo \textquotedbl  Teaching Children Thinking \textquotedbl   em que abordava  o tema sobre crian\c{c}as, educa\c{c}\~ao e computadores: 







\noindent\begin{center}\mbox{\centering\fbox{\centering\par\parbox{0.7\linewidth}{\small\textit{\textquotedbl Tinhamos  a certeza de que quando os computadores se tornassem t\~ao comuns quanto ao l\'apis a educa\c{c}\~ao mudaria t\~ao r\'apida e profundamente quanto as transforma\c{c}\~oes pelas quais viv\'{\i}amos nos direitos civis e nas rela\c{c}\~oes sociais e sexuais. [XXX colocar a cita\c{c}\~ao aqui]}\normalize}}}\end{center}


Papert formulou esse pensamento quando os computadores dos anos 70 ainda n\~ao eram acess\'{\i}veis ou dispon\'{\i}veis para uso dom\'estico ou no sistema educacional. Naquele tempo n\~ao existia o conceito de \textquotedbl micro-computadores\textquotedbl  e equipamentos com poder de processamento milhares de vezes inferior ao de um notebook de hoje ocupavam andares inteiros de pr\'edios [XXX lei de moore]. Os custos eram muito altos, o acesso era muito restrito e havia d\'uvidas sobre se algum dia seriam amplamente acess\'{\i}veis [XXX refer\^encia]. Mas mesmo na forma de mainframes centralizados (computadores de grande porte) com as limita\c{c}\~oes indicadas acima, foi poss\'{\i}vel a Papert realizar incurs\~oes pioneiras no campo da aprendizagem para crian\c{c}as utilizando computadores, mesmo que restrita a privilegiados, sem, ainda, a possibilidade de uma grande dissemina\c{c}\~ao no sistema educacional [XXX \'e poss\'{\i}vel encontrar refer\^encias?].  Portanto, foi um vision\'ario ao sugerir que a crian\c{c}a teria, um dia, amplo acesso ao computador, a ponto de ficar no comando do computador durante a aprendizagem e n\~ao o contr\'ario [XXX citar a fi].






Toda uma gera\c{c}\~ao de educadores foi formada em torno das ideias de Papert, que defendia que a aprendizagem de linguagem de programa\c{c}\~ao de computadores, j\'a no ensino fundamental, poderia ter um papel importante no aprendizado de muitas outras disciplinas tradicionais, tais como matem\'atica, ci\^encias e linguagem. A proposta de Papert, at\'e por enfatizar o aprendizado de crian\c{c}as, n\~ao tinha qualquer ambi\c{c}\~ao de capacita\c{c}\~ao profissional e, por si s\'o, n\~ao visava diretamente fazer frente aos desafios do \textquotedbl mundo do trabalho\textquotedbl , que foram sendo introduzidos pelas transforma\c{c}\~oes inerentes \`a Sociedade da Informa\c{c}\~ao nas d\'ecadas subsequentes. Para Papert o computador poderia funcionar como o indutor da aprendizagem de outras disciplinas.






Diferentemente de um simples treinamento para usar computadores, o m\'etodo de Papert representava uma mudan\c{c}a em paradigmas educacionais, focalizando a aprendizagem em detrimento do ensino [XXX Brasil Plan ou outra cita\c{c}\~ao do Papert nesse sentido]. A ideia era \textquotedbl aprender o que se precisa e n\~ao \textquotedbl aprender o que se deve [XXX verificar outras cita\c{c}\~oes melhores de Papert para colocar aqui]. Outro ponto importante era buscar a ludicidade no aprendizado [XXX citar a fonte - YYY - falta complementar aqui]. O cap\'{\i}tulo de Fundamenta\c{c}\~ao Te\'orica traz um aprofundamento sobre o pensamento de Papert.






O car\'ater estritamente educacional e a peculiar abordagem das propostas de Papert s\~ao apontados em \textquotedbl Brazil Plan\textquotedbl  [XXX] (ali\'as, muito a posteriori por seus colegas) como uma alternativa para a inser\c{c}\~ao do indiv\'{\i}duo na \textquotedbl era digital\textquotedbl  (digital age).






Portanto \'e razo\'avel assumir que os conceitos educacionais de Papert s\~ao reconhecidos por seus disc\'{\i}pulos [XXX Brazil Plan] tamb\'em como um caminho natural para a melhor inser\c{c}\~ao dos indiv\'{\i}duos na Sociedade da Informa\c{c}\~ao, muito embora os conceitos de \textquotedbl era digital\textquotedbl  e de \textquotedbl sociedade da informa\c{c}\~ao\textquotedbl  n\~ao sejam equivalentes mas relacionados, uma vez se diferenciam no fato de que o primeiro se refere mais \`as transforma\c{c}\~oes tecnol\'ogicas que levaram ao segundo.






Da mesma forma, \'e razo\'avel assumir que uma parte das iniciativas educacionais mundiais em torno de STEM basearem-se em trabalhos como os de Papert, que propunham uma educa\c{c}\~ao despojada de formalismos, voltada para a resolu\c{c}\~ao de problemas, ao inv\'es da hist\'orica obsess\~ao por conte\'udos. Esse tipo de abordagem inspirou boa parte dos conceitos subjacentes \`a \textquotedbl pedagogia orientada a projeto\textquotedbl [XXX], ao \textquotedbl problem solving learning\textquotedbl [XXX], ao \textquotedbl design thinking, \`a \textquotedbl maker culture\textquotedbl , entre outros [XXX].






As j\'a mencionadas [XXX citar onde foi mencionado] preocupa\c{c}\~oes com o relativo baixo desempenho em STEM, que se aprofundavam nos EUA nos anos 90, alcan\c{c}aram o resto do mundo e propostas come\c{c}aram a surgir para tentar promover a qualifica\c{c}\~ao da educa\c{c}\~ao em pa\'{\i}ses em desenvolvimento por meio do uso intensivo de computadores, nos moldes do que enxergara Papert em seus trabalhos seminais.






O Projeto \textquotedbl One Laptop Per Child\textquotedbl  [XXX Brazil Plan] foi uma das iniciativas mais completas e robustas neste sentido, tendo sa\'{\i}do  do pr\'oprio MIT, especificamente concebido por disc\'{\i}pulos de Papert, os quais estabeleceram planos para regi\~oes espec\'{\i}ficas do mundo, a exemplo do documento entitulado \textquotedbl Brazil Plan\textquotedbl , direcionado \`a \textquotedbl Brazilian Task Force\textquotedbl  e compartilhado com governo brasileiro em 2004-2005 [XXX Brazil Plan], quando Nicholas Negroponte se encontrou com Lula em Davos. O OLPC era explicitamente apoiado por Papert, quando ainda estava vivo. Isso pode ser comprovado pela sua presen\c{c}a ativa e eloquente nas reuni\~oes de apresenta\c{c}\~ao do OLPC ao Governo Brasileiro [XXX acervo de VPM], inclusive numa visita ao presidente Lula [XXX colocar foto do Lula com Papert, Negroponte].






Nicholas Negroponte, o l\'{\i}der da iniciativa do OLPC, era um destacado \textquotedbl gur\'u\textquotedbl  de chefes de estado, a exemplo de Miterrand, que na d\'ecada de 80 o convidara a integrar o Conselho do \textquotedbl Center Mondiale\textquotedbl  [XXX procurar refer\^encias]. Coincidentemente, era irm\~ao de John Negroponte, ent\~ao Secret\'ario de Estado do Governo Bush, figura influente nos meios pol\'{\i}ticos, na comunidade de informa\c{c}\~ao e em outras \'areas estrat\'egicas e de defesa daquele pa\'{\i}s.






Nicholas transitava com desenvoltura entre l\'{\i}deres como Kofi Anan [XXX acervo de VPM], Presidente da \'India, Presidente Americano, entre outros [XXX achar comprova\c{c}\~ao dessa informa\c{c}\~ao por meio de fotos]. Em 2004, conseguiu uma audi\^encia com o Presidente Lula, quando este participava do F\'orum  Econ\^omico  em Davos. Foi nesse momento em que o Projeto OLPC foi apresentado, pela primeira vez, ao conhecimento do governo Brasileiro [XXX Brazil Plan].






A proposta era ousada e atraente no que tange \`a transforma\c{c}\~ao dos m\'etodos pedag\'ogicos. Por outro lado, era tamb\'em exigente em termos de recursos, uma vez que preconizava a aquisi\c{c}\~ao de milh\~oes de notebooks como forma de empoderamento dos estudantes pela possibilidade de conex\~ao \`a internet [XXX Brazil Plan]. Em termos or\c{c}ament\'arios, a ades\~ao \`a proposta de Negroponte representava um valor significativo do or\c{c}amento do Minist\'erio da Educa\c{c}\~ao e, para que fosse viabilizada, precisaria passar por um escrut\'{\i}nio da sociedade brasileira.






Ciente do risco que representava uma ades\~ao voluntariosa a um programa t\~ao disruptivo, a Presid\^encia da Rep\'ublica da \'epoca decidiu constituir um grupo de avalia\c{c}\~ao  daquela proposta, o  qual foi constitu\'{\i}do por universidades e centros de pesquisa. Foram chamados o Centro de Tecnologia da Informa\c{c}\~ao Renato Archer (CTI), a Escola Polit\'ecnica da USP e a Universidade Federal de Santa Catarina [XXX tentar encontrar os documentos da \'epoca]. 






As institui\c{c}\~oes mencionadas avaliaram o projeto em v\'arios aspectos [XXX Paper de Cubat\~ao]:







\begin{alineas}
\item proposta pedag\'ogica,
\item modelo de neg\'ocios,
\item sustentabilidade,
\item redes,
\item pol\'{\i}tica industrial,
\item software,
\item ergonomia
\item e conte\'udo.
\end{alineas}

A proposta previa a aquisi\c{c}\~ao de um \textquotedbl laptop por estudante brasileiro, ou seja, perto de 30 a 40 milh\~oes de unidades.






Segundo a vis\~ao trazida pelo MIT [XXX Brazil Plan] ao governo brasileiro, a disponibiliza\c{c}\~ao em larga escala de acesso \`a internet alteraria a rela\c{c}\~ao aluno-professor, promovendo formas de aprendizagem alternativas ao conteudismo tradicional, reformulando tamb\'em o formato lousa-giz inerente ao sistema educacional brasileiro [XXX Paper Cubat\~ao].






Um dos aspectos principais do projeto apresentado ao governo, do ponto de vista das ferramentas de software, era a disponibiliza\c{c}\~ao de uma ferramenta de programa\c{c}\~ao mais intuitiva e l\'udica do que o pr\'oprio LOGO, linguagem de programa\c{c}\~ao desenvolvida por Papert na d\'ecada de 60 e muito difundida no contexto educacional a partir daquela \'epoca [XXX trazer refer\^encia do LOGO - n\~ao esquecer de citar os outros 2 autores do LOGO]. Como alternativa ao j\'a ser\^odio Logo, estava em fase final de desenvolvimento o Scratch, linguagem criada por Mitchel Resnick que compartilhava alguns de seus conceitos [XXX trazer refer\^encia].






Das 3 institui\c{c}\~oes envolvidas na avalia\c{c}\~ao do OLPC, tivemos acesso \`a avalia\c{c}\~ao do CTI [XXX paper cubat\~ao], que ficou encarregado da:







\begin{alineas}
\item avalia\c{c}\~ao de caracter\'{\i}sticas de ergonomia postural, por meio da captura de movimento;
\item avalia\c{c}\~ao de caracter\'{\i}sticas de ergonomia sensorial, por meio de t\'ecnicas relacionadas \`a \'area de mostradores de informa\c{c}\~ao;
\item avalia\c{c}\~ao da funcionalidade dos \textquotedbl laptops, principalmente em termos de redes, processamento, mem\'oria e baterias;
\item avalia\c{c}\~ao do emprego dos dispositivos no \^ambito da escola p\'ublica;
\item avalia\c{c}\~ao da percep\c{c}\~ao dos professores sobre o projeto;
\item an\'alise da infraestrutura das escolas, visando verificar a viabilidade de implanta\c{c}\~ao do projeto;
\item acompanhamento de pilotos de avalia\c{c}\~ao em escolas p\'ublicas brasileiras.
\item visitas a pilotos nos Estados Unidos.
\end{alineas}

Do ponto de vista da aquisi\c{c}\~ao de \textquotedbl laptops\textquotedbl  em larga escala, o CTI identificou uma s\'erie de dificuldades nas seguintes \'areas: apropria\c{c}\~ao pela escola brasileira, produ\c{c}\~ao dos laptops, restri\c{c}\~oes or\c{c}ament\'arias, problemas ergon\^omicos e, principalmente, obsolesc\^encia dos equipamentos [XXX Paper Cubat\~ao]. Estes aspectos demonstraram que a ideia de aquisi\c{c}\~ao de milh\~oes de laptops representava um risco muito grande para o sistema educacional brasileiro.






O estudo apontava, tamb\'em, que o sistema educacional poderia se beneficiar de alguns aspectos da proposta, mas que qualquer iniciativa disruptiva no sistema educacional brasileiro requereria mais investimentos em capacita\c{c}\~ao de recursos humanos do que em hardware ou software, ao contr\'ario do que propunha o Projeto OLPC, que focalizava a aquisi\c{c}\~ao dos computadores.






Esta percep\c{c}\~ao de que o Projeto OLPC, como proposto por Negroponte, tinha um equ\'{\i}voco em seu foco foi expressa principalmente pela equipe do CTI, que se destacou dos demais participantes da avalia\c{c}\~ao, que estavam mais propensos a apoiar o projeto como originalmente proposto. A posi\c{c}\~ao do CTI se sustentava na pr\'opria defini\c{c}\~ao de educa\c{c}\~ao empregada na an\'alise da proposta OLPC: \textquotedbl Educa\c{c}\~ao \'e a inser\c{c}\~ao do indiv\'{\i}duo em sua pr\'opria cultura atrav\'es da intera\c{c}\~ao com outros indiv\'{\i}duos\textquotedbl  [XXX achar fonte].






Esta defini\c{c}\~ao colocava a intera\c{c}\~ao entre indiv\'{\i}duos no centro do processo e, portanto, qualquer esfor\c{c}o de qualifica\c{c}\~ao da escola brasileira precisaria passar por uma \^enfase no investimento em \textquotedbl pessoas, mais do que em software ou hardware\textquotedbl .






O Projeto WASH nasceu [XXX paper de Cubat\~ao] como uma proposta alternativa ao OLPC, com custo inferior, que n\~ao exigia a aquisi\c{c}\~ao de milh\~oes de equipamentos, mas que se inspirava nos mesmos conceitos exitosos de Papert que fundamentaram a proposta do OLPC.






Assim [XXX paper cubat\~ao], o Projeto WASH buscou centrar-se na cria\c{c}\~ao de espa\c{c}os de intera\c{c}\~ao no contexto de valores do m\'etodo cient\'{\i}fico, buscando estabelecer meios para estimular, inicialmente, as disciplinas de STEM e, posteriormente, incluindo arte na lista, como tantos outros autores fizeram naquele per\'{\i}odo [XXX James Catterall, Yakman, Mammana, etc… etc.. ].






A avalia\c{c}\~ao do Projeto OLPC proporcionou uma resignifica\c{c}\~ao para a proposta, permitindo compreender mais profundamente os desafios do uso intensivo de tecnologia da informa\c{c}\~ao no contexto da escola p\'ublica brasileira e, com isso, propor uma alternativa.






O WASH se constitui em atividades em grupo, realizadas no contraturno, desvinculadas do curr\'{\i}culo tradicional da escola formal, cujos valores principais se alicer\c{c}am no m\'etodo cient\'{\i}fico. O WASH n\~ao \'e um curso, mas se constitui em espa\c{c}os de intera\c{c}\~ao humana para experimenta\c{c}\~ao e conviv\^encia entre indiv\'{\i}duos, no contexto do desenvolvimento de projetos de v\'arios n\'{\i}veis de complexidade.






Pela forma como os pilotos do WASH acabaram sendo implementados no contexto do CTI Renato Archer, houve a consolida\c{c}\~ao da vis\~ao de que institui\c{c}\~oes de P






Hoje o Projeto WASH tem seu m\'etodo descrito por meio de um documento de refer\^encia, a Portaria CTI 178/2018, que estabelece uma \textquotedbl liturgia\textquotedbl  [XXX citar uma paper nosso que usa esse termo] de realiza\c{c}\~ao de oficinas, os papeis de cada participante e a forma de opera\c{c}\~ao. Mas \'e evidente que, por ser longevo, alcan\c{c}ando em 2023 a marca de 10 anos de realiza\c{c}\~ao, o WASH passou por muitas transforma\c{c}\~oes em rela\c{c}\~ao \`a sua proposta inicial, requerendo uma constante caracteriza\c{c}\~ao e revis\~ao, com base em indicadores e an\'alise de seus processos.






Neste trabalho ser\'a feita caracteriza\c{c}\~ao do projeto Workshop Aficionados em Software e Hardware (WASH), que declaradamente por seus criadores, foi inspirado pela proposta OLPC. Por curioso, n\~ao obstante tenham se inspirado nos conceitos pedag\'ogicos presentes na proposta americana, tamb\'em se posicionaram contra a aquisi\c{c}\~ao dos notebooks [XXX citar paper de Cubat\~ao] pelo governo brasileiro, em raz\~ao de outros aspectos do projeto que mostravam-se invi\'aveis, principalmente no campo or\c{c}ament\'ario, industrial, ergon\^omico, inclusivo e de log\'{\i}stica [XXX citar os relat\'orios do OLPC].






A abordagem adotada na presente disserta\c{c}\~ao se encaixa no m\'etodo de \textquotedbl Estudo de Caso e buscar\'a contar toda essa trajet\'oria que se inicia no que foi descrito aqui, bem como identificar o m\'etodo do Projeto WASH e seus resultados. O documento fundamental a ser usado para permitir a caracteriza\c{c}\~ao do projeto \'e a Portaria CTI 178 e outros registros, tais como publica\c{c}\~oes, relat\'orios, planos de trabalho, produ\c{c}\~ao audiovisual, entre outras.






\section[Objeto]{Objeto}\label{Objeto}
Este trabalho tem por objeto de estudo o  Projeto WASH.






\section[Objetivo]{Objetivo}\label{Objetivo}
Este trabalho tem por objetivo caracterizar  o Projeto Workshop de Aficionados em Software e Hardware (WASH) quanto a:







\begin{alineas}
\item sua trajet\'oria (hist\'oria)
\item seus m\'etodos
\item e seus resultados
\end{alineas}

com vistas a propor uma melhoria em suas pr\'aticas, por meio da revis\~ao do Documento de Refer\^encia constante no anexo da Portaria CTI 178/2018.






\section[Hip\'oteses]{Hip\'oteses}\label{Hip\'oteses}
O presente trabalho tem como hip\'oteses:







\begin{alineas}
\item o Projeto WASH teve como origem as experi\^encias do Projeto GESAC [XXX], da avalia\c{c}\~ao do OLPC (NEGROPONTE, 2004) e da Avalia\c{c}\~ao do PIDs (MAMMANA, 2009)
\item o Projeto carrega elementos dos m\'etodos de Seymour Papert, combinando-os com outros, tais como
\item o Projeto WASH pode ser identificado como \textquotedbl educa\c{c}\~ao  formal e n\~ao-formal
\item o car\'ater presencial das oficinas do WASH, declarado no documento de refer\^encia, representou uma barreira que resultou em um atraso para a adapta\c{c}\~ao do projeto \`as restri\c{c}\~oes da pandemia, requerendo uma revis\~ao
\item meios alternativos de realiza\c{c}\~ao do projeto (e.g. produ\c{c}\~ao audiovisual, oficinas s\'{\i}ncronas e ass\'{\i}ncronas) foram a formas encontradas para enfrentar, de forma emergencial, as barreiras indicadas acima
\item o WASH resultou, ao longo de seus 9 anos de exist\^encia, em uma vasta produ\c{c}\~ao de conhecimentos e aprendizados
\item podemos medir  esses aprendizados 
\end{alineas}

\section[Problema]{Problema}\label{Problema}
O Programa WASH tem 9 anos de exist\^encia tendo atendido milhares de crian\c{c}as em dezenas de cidades brasileiras. Inicialmente desenhado a partir das conclus\~oes da avalia\c{c}\~ao dos Projetos OLPC, PIDs, recebeu influ\^encias das pr\'aticas do GESAC. Esses conhecimentos foram consolidados no anexo \`a Portaria CTI 178/2018, o qual estabelece formalmente seu m\'etodo de realiza\c{c}\~ao, explicitando o car\'ater presencial do programa. Com a Pandemia a sociedade aprendeu e passou a aceitar melhor o papel de atividades remotas na educa\c{c}\~ao. Muito embora as diretrizes gerais do projeto presentes no anexo \`a Portaria CTI 178/2018 permane\c{c}am v\'alidas, a nova realidade requer uma adequa\c{c}\~ao de aspectos do programa. Novas pr\'aticas foram criadas e precisam ser caracterizadas, para que as melhorias possam ser introduzidas no documento de formaliza\c{c}\~ao da metodologia.






\section[Justificativa]{Justificativa}\label{Justificativa}
A aceita\c{c}\~ao do m\'etodo do Projeto WASH pelas institui\c{c}\~oes de educa\c{c}\~ao, documentado por dezenas de instrumentos legais de ades\~ao (portarias), permite vislumbrar a transforma\c{c}\~ao do projeto em pol\'{\i}tica p\'ublica, o que tem estimulado chamar o projeto como \textquotedbl proto-pol\'{\i}tica, ou seja, pol\'{\i}tica p\'ublica em constru\c{c}\~ao. Para que o projeto atinja esse est\'agio, \'e preciso fazer uma revis\~ao em seu documento de refer\^encia e, para isso, \'e preciso caracteriz\'a-lo em 3 dimens\~oes: hist\'oria, m\'etodo e resultados.






\chapter[FUNDAMENTA\c{C}\~AO TE\'ORICA ]{FUNDAMENTA\c{C}\~AO TE\'ORICA }\label{FUNDAMENTA\c{C}\~AO TE\'ORICA }
Como j\'a exposto na \textquotedbl Introdu\c{c}\~ao\textquotedbl , o presente trabalho se prop\~oe a:







\begin{alineas}
\item registrar a hist\'oria do Projeto WASH
\item caracterizar seu m\'etodo
\item caracterizar seus resultados
\end{alineas}

Neste cap\'{\i}tulo ser\~ao aprofundados aspectos levantados na Introdu\c{c}\~ao, embasando a escolha dos m\'etodos que ser\~ao empregados para alcan\c{c}ar os objetivos deste trabalho.






Antes de prosseguir \'e necess\'ario entender o papel do m\'etodo em um trabalho cient\'{\i}fico e esta compreens\~ao pode ser obtida pela an\'alise da origem etmol\'ogica da palavra.







\noindent\begin{center}\mbox{\centering\fbox{\centering\par\parbox{0.7\linewidth}{\small\textit{\textquotedbl O \'etimo latino “methodus” \'e um dos fundamentos para a significa\c{c}\~ao do termo “m\'etodo”. Com o sentido de caminho (“chemin”, “route”), do grego odos (Cl\'edat, 1914, 213), est\'a presente em v\'arios idiomas: “methode” (Al), “m\'ethode” (Fr), “m\'etodo” (Esp), “metodo” (It). Com o methodus e o seu significado mais abrangente, “caminho” (way, Weg, route, via e camino), designamos o nosso tipo ideal.\textquotedbl  (fonte: (FREITAS, 2019) )}\normalize}}}\end{center}









Tendo em vista, ent\~ao, a necessidade de escolher um caminho para chegar at\'e os objetivos do trabalho, a partir de agora s\~ao descritos os fundamentos te\'oricos que ser\~ao considerados.













\section[Fundamenta\c{c}\~ao: hist\'oria]{Fundamenta\c{c}\~ao: hist\'oria}\label{Fundamenta\c{c}\~ao: hist\'oria}
Para caracterizar e tra\c{c}ar a trajet\'oria do Programa WASH, em qual contexto ele surgiu, quais pol\'{\i}ticas, projetos, a\c{c}\~oes, enfim, as diversas experi\^encias de cultura digital que o antecederam, h\'a a necessidade de aplicar um m\'etodo. Mas antes de defin\'{\i}-lo h\'a que se revisitar os conceitos pr\'e-existentes, trabalho que se desenvolver\'a nesta se\c{c}\~ao.






Assim, nesta se\c{c}\~ao ser\~ao revisitadas as narrativas de programas de cultura digital que antecederam a exist\^encia do Programa WASH, a exemplo do GESAC, OLPC, Ciëncia na Escola e o Pensamento de Papert.






Conhecer esta hist\'oria \'e importante para elucidar, trazer as contribui\c{c}\~oes, as viv\^encias e os referenciais das bases conceituais desta constru\c{c}\~ao de uma pr\'atica de cultura digital no seio sociedade.






O Programa WASH \'e uma pr\'atica de cultura digital que tem v\'{\i}nculo com a administra\c{c}\~ao p\'ublica federal, estabelece pontes com os demais entes federados, com os poderes executivo e legislativo, com as redes de ensino, com os \'org\~aos de fomento cient\'{\i}fico e com as organiza\c{c}\~oes sociais.






\'E esta complexidade que exige uma vis\~ao sist\^emica entre abordagem hist\'orica, mapeamento de processos e levamentamento estat\'{\i}stico de dados, subjacentes \`as 3 dimens\~oes que s\~ao objeto deste estudo.






Segundo PIERANTI (2022):







\noindent\begin{center}\mbox{\centering\fbox{\centering\par\parbox{0.7\linewidth}{\small\textit{\textquotedbl An\'alises descontextualizadas perdem sua relev\^ancia, na medida em que se tornam pouco fact\'{\i}veis ou possivelmente deslocadas da realidade\textquotedbl  (PIERANTI, 2022)}\normalize}}}\end{center}









Com isso em mente, na dimens\~ao hist\'orica, adotamos o m\'etodo da historiografia aplicada na pesquisa em administra\c{c}\~ao p\'ublica contempor\^anea, pela compreens\~ao e aceita\c{c}\~ao da import\^ancia da hist\'oria em ser determinante para explicar os acontecimentos e estruturas existentes em qualquer sociedade (PIERANTI, 2022) .






Para que se situe temporalmente e conceitualmente dentro do conhecimento hist\'orico, a abordagem de  PIERANTI (2022) ser\'a apresentada ao final de uma pequena revis\~ao da evolu\c{c}\~ao do m\'etodo historiogr\'afico, como apresentado a seguir.






\subsection[Revis\~ao da evolu\c{c}\~ao da historiografia]{Revis\~ao da evolu\c{c}\~ao da historiografia}\label{Revis\~ao da evolu\c{c}\~ao da historiografia}
Para que o registro hist\'orico de interesse para este texto se d\^e no contexto da ci\^encia, no qual todas as afirma\c{c}\~oes aqui devem se inserir, \'e preciso que se baseiem em um m\'etodo.






Ao se basearem num m\'etodo, estas afirma\c{c}\~oes de cunho hist\'orico adquirem a propriedade de serem contest\'aveis (false\'aveis), uma vez que o caminho percorrido para sua constru\c{c}\~ao (pertinente ao m\'etodo) pode ser revisitado por outros que queiram verific\'a-las.






Este caminho escolhido para a constru\c{c}\~ao da narrativa hist\'orica ser\'a descrito no cap\'{\i}tulo de \textquotedbl Materiais e M\'etodos\textquotedbl , deixando para o cap\'{\i}tulo de \textquotedbl Resultados\textquotedbl  a apresenta\c{c}\~ao do discurso propriamente dito.






Com base no m\'etodo descrito no cap\'{\i}tulo de \textquotedbl Materiais e M\'etodos\textquotedbl , qualquer outro poder\'a avaliar o escopo de validade das afirma\c{c}\~oes presentes no cap\'{\i}tulo de \textquotedbl Resultados\textquotedbl .






Mas a escolha do m\'etodo de caracteriza\c{c}\~ao hist\'orica do WASH aqui empregado precisa ter suas ra\'{\i}zes em m\'etodos pregressos, para aproveitar o conhecimento j\'a existente na \'area de hist\'oria. Por esse motivo, neste cap\'{\i}tulo de \textquotedbl Fundamenta\c{c}\~ao Te\'orica\textquotedbl  ser\'a feita uma breve revis\~ao do m\'etodo historiogr\'afico.






Muito embora a humanidade venha \textquotedbl contando\textquotedbl  suas hist\'orias desde tempo imemoriais, um primeiro registro historiogr\'afico pode ser atribu\'{\i}do a Her\'odoto no s\'eculo V, assim como a estrutura\c{c}\~ao da \textquotedbl Hist\'oria\textquotedbl  como atividade profissional remonta ao in\'{\i}cio do s\'eculo XIX, com a contribui\c{c}\~ao da Escola Hist\'orica Prussiana.






Como nos ensina  Marczal (2016),  Her\'odoto e Tuc\'{\i}dides s\~ao muitas vezes reconhecidos como os primeiros a elaborar relatos historiogr\'aficos, pela obra que deixaram sobre os confrontos entre gregos e persas no s\'eculo V a.c. ou da Guerra do Peloponeso, respectivamente. Her\'odoto chegou a ser considerado por C\'{\i}cero como o \textquotedbl pai da hist\'oria\textquotedbl  [XXX].








\captionsetup{format=plain}
\begin{figure}[max size={\textwidth}{\textheight}]

\centering


\begin{minipage}[b]{0.4\linewidth}
        \centering
                \includegraphics[width=1.0\linewidth]{../../imagens/Herodote.jpg}
                \caption{C\'opia Romana do Busto de Her\'odoto, do s\'eculo II, presente no Estoa de \'Atalo, Atenas (fonte [[WIKIPEDIA (2022)]]).}
                \label{9f279b29e90288533caab6ba43668512f3a9eee4}
\end{minipage}%
\hspace{0.5cm}
\end{figure}



Confirmando, TEIXEIRA (2008)  tamb\'em ensina que surgiu com Her\'odoto e Tuc\'{\i}dides, no s\'eculo V, a hist\'oria \textquotedbl entendida como pr\'atica de inquiri\c{c}\~ao sobre as grandes e memor\'aveis obras dos homens(...), cujo prop\'osito central seria o de salvar os feitos humanos do esquecimento\textquotedbl   (TEIXEIRA, 2008).






Mas a origem da hist\'oria como atividade profissional, com o vi\'es de ci\^encia, \'e frequentemente atribu\'{\i}da ao historicismo alem\~ao do s\'eculo XIX, vinculado ao trabalho de Leopold Von Ranke, historiador alem\~ao nascido em 1795 e falecido em 1886 (Marczal, 2016). Ranke teve papel no surgimento da chamada Escola Hist\'orica Prussiana, liderando-a ao lado de Humboldt, Droysen e Gervinus  (BENTIVOGLIO, 2010) .








\captionsetup{format=plain}
\begin{figure}[max size={\textwidth}{\textheight}]

\centering


\begin{minipage}[b]{0.4\linewidth}
        \centering
                \includegraphics[width=1.0\linewidth]{../../imagens/ranke.jpg}
                \caption{Leopold Von Ranke (fonte: dom\'{i}nio p\'ublico)}
                \label{e978df58deaf86ca4da4073fca97b28afd4d3a3b}
\end{minipage}%
\hspace{0.5cm}
\end{figure}



Se o termo \textquotedbl historiador\textquotedbl  era bastante impreciso na Antiguidade (TEIXEIRA, 2008), o \textquotedbl jeito\textquotedbl  de fazer hist\'oria no s\'eculo XIX, cujo pioneirismo pode ser atribu\'{\i}do a Ranke, tem como caracter\'{\i}stica \textquotedbl o rigor metodol\'ogico do processo de investiga\c{c}\~ao\textquotedbl , assim como sua consolida\c{c}\~ao como disciplina acad\^emica (Marczal, 2016).






A ideia subjacente ao pensamento Rankeano \'e de que existe uma verdade objetiva no passado que precisa ser descoberta e descrita no presente como \textquotedbl conhecimento verdadeiro\textquotedbl , com base em vest\'{\i}gios aut\^enticos que servem para comprovar o que est\'a sendo narrado (Marczal, 2016).






Para a constru\c{c}\~ao do m\'etodo de registro hist\'orico que ser\'a utilizado neste trabalho, nos centraremos no entendimento de Ranke, exposto no Pref\'acio \`a 1ª edi\c{c}\~ao de seu \textquotedbl Hist\'oria dos povos germ\^anicos e latinos\textquotedbl   (apud BENTIVOGLIO, 2010) , que explicita:







\noindent\begin{center}\mbox{\centering\fbox{\centering\par\parbox{0.7\linewidth}{\small\textit{\textquotedbl [...] a origem da mat\'eria [hist\'orica] s\~ao mem\'orias, di\'arios, cartas, relatos de delega\c{c}\~oes e narra\c{c}\~oes originadas de teste munhas oculares; baseia-se em outros escritos somente quando estes foram diretamente derivados destes, ou quando pareciam terem sido tornados equivalentes a estes com base em algum conhecimento original\textquotedbl  (Fonte: Leopold von Ranke, no Pref\'acio da 1ª Edi\c{c}\~ao de \textquotedbl Hist\'oria dos povos germ\^anicos e latinos\textquotedbl , citado por  BENTIVOGLIO (2010))}\normalize}}}\end{center}


Ranke, no mesmo texo, ainda segundo  BENTIVOGLIO (2010), exp\~oe qual seria o 2º passo da atividade de pesquisa hist\'orica: \textquotedbl (...)uma rigorosa exposi\c{c}\~ao de fatos seja esta t\~ao condicionada e desagrad\'avel quanto for \'e, sem d\'uvida, lei suprema\textquotedbl , refor\c{c}ando que \textquotedbl (...)n\~ao se pode fazer o mesmo desenvolvimento livre que, pelo menos, a teoria busca numa obra po\'etica(...)\textquotedbl .






Evidente que os m\'etodos da pesquisa em Hist\'oria evolu\'{\i}ram muito depois das contribui\c{c}\~oes de Ranke e seus colegas da Escola Hist\'orica Prussiana do s\'eculo XIX, sobretudo, pelo fato da ci\^encia  da hist\'oria se tornar o centro de oposi\c{c}\~ao ao idealismo ensejando o surgimento de v\'arios outros movimentos historiogr\'aficos e Escolas.






Sem, contudo, adentrar ao detalhamento te\'orico-filos\'ofico-metodol\'ogico, citamos dentre as principais correntes historiogr\'aficas, a Escola Met\'odica dita Positivista, o Materialismo Hist\'orico e a Escola dos Annales, que passamos a descrever.






Pode-se dizer que a Escola Met\'odica, dita Positivista, foi inspirada em Von Ranke, mas teve tamb\'em influ\^encia da corrente filos\'ofica positivista difundida pelo franc\^es Augusto Comte.






A Escola Met\'odica nasceu com a proposta de  lan\c{c}ar sobre as pesquisas em hist\'oria uma vis\~ao cient\'{\i}fica, tratando a hist\'oria como uma ci\^encia metodologicamente rigorosa, tendo como modelo as ci\^encias naturais seguindo o m\'etodo das ci\^encias f\'{\i}sicas.






Atrav\'es da Revista Hist\'orica, seus principais representantes, Charles-Vitor Langlois e Charles Seignobos, difundiam seus pensamentos.







\noindent\begin{center}\mbox{\centering\fbox{\centering\par\parbox{0.7\linewidth}{\small\textit{\textquotedbl M\'etodo tornou-se a palavra-chave, e o que distinguia a hist\'oria da literatura. A hist\'oria se profissionalizou definitivamente numerosas cadeiras na universidade, sociedades cient\'{\i}ficas cole\c{c}\~oes de documentos, revistas, manuais, publica\c{c}\~ao de textos hist\'oricos, um p\'ublico culto comprador de livros hist\'oricos”  (REIS , 2006)}\normalize}}}\end{center}


Assim, embora essa Escola tenha recebido justas cr\'{\i}ticas dos historiadores do s\'eculo XX, a Escola met\'odica francesa teve o m\'erito inconteste de atribuir confiabilidade ao m\'etodo hist\'orico






O Materialismo Hist\'orico e Dial\'etico, ou simplesmente Materialismo Hist\'orico, foi desenvolvido  por Karl Marx (1818-1883) e Friedrich Engels (1820-1895). Nasceu da oposi\c{c}\~ao ao idealismo e se diferencia por ser uma corrente filos\'ofica que utiliza o conceito de dial\'etica para entender a din\^amica e os processos sociais, cujo enfoque te\'orico, metodol\'ogico e anal\'{\i}tico \'e utilizado para compreender as grandes transforma\c{c}\~oes da hist\'oria e das sociedades humanas.






Para  Pires (2009), o m\'etodo materialista hist\'orico-dial\'etico “caracteriza-se pelo movimento do pensamento atrav\'es da materialidade hist\'orica da vida dos homens em sociedade, isto \'e, trata-se de descobrir (pelo movimento do pensamento) as leis fundamentais que definem a forma organizativa dos homens durante a hist\'oria da humanidade.”, sendo um paradigma que vem influenciando at\'e hoje a historiografia mundial.






Dando prosseguimento a esta breve e despretensiosa revis\~ao do m\'etodo historiogr\'afico, chegamos \`a Escola de Annales, cujo nome tem origem na sua forma de mobiliza\c{c}\~ao inicial, a Revista Annales: \'economies, societ\'es, civilisations, fundada em 1929, que representou uma ruptura com a vis\~ao hist\'orica tradicionalP (PIERANTI, 2022) .






A Escola de Annales trouxe uma nova abordagem, com in\'umeras consequ\^encias e influ\^encias at\'e nossos dias. Tem como principais mentores  Marc Bloch e Lucian Febvre






Sobre a revista, Peter Burke (1991) afirma no Pref\'acio que esta:














\noindent\begin{center}\mbox{\centering\fbox{\centering\par\parbox{0.7\linewidth}{\small\textit{“foi fundada para promover uma nova esp\'ecie de hist\'oria e continua, ainda hoje, a encorajar inova\c{c}\~oes. As id\'eias diretrizes da revista, que criou e excitou entusiasmo em muitos leitores, na  Fran\c{c}a e no exterior, podem ser sumariadas brevemente. Em primeiro lugar, a substitui\c{c}\~ao da tradicional narrativa de acontecimentos por uma hist\'oria-problema. Em segundo lugar, a hist\'oria de todas as atividades humanas e n\~ao apenas hist\'oria pol\'{\i}tica. Em terceiro lugar, visando completar os dois primeiros objetivos, a colabora\c{c}\~ao com outras disciplinas, tais como a geografia, a sociologia, a psicologia, a economia, a ling"u\'{\i}stica, a antropologia social, e tantas outras.\textquotedbl   (Burke, 1991)}\normalize}}}\end{center}


N\~ao obstante a percep\c{c}\~ao de uma certa proximidade entre a Escola de Annales e o positivismo, quando comparadas com outras tradi\c{c}\~oes na metodologia da hist\'oria, dado que ambas t\^em a abordagem pelo m\'etodo cient\'{\i}fico como dominante, sempre com \^enfase em fatos emp\'{\i}ricos (Firat, 1987), \'e evidente que na Escola de Annales a narrativa linear dos acontecimentos sai de cena (PIERANTI, 2022), dando espa\c{c}o a uma metodologia cr\'{\i}tica.






N\~ao obstante a riqueza de transforma\c{c}\~oes do per\'{\i}odo em que a cria\c{c}\~ao do WASH se insere, \'e claro que este trabalho n\~ao tem a pretens\~ao de produzir uma narrativa hist\'orica completa do per\'{\i}odo em que o Brasil transformou a inclus\~ao digital numa pol\'{\i}tica de Estado. Outros autores podem oferecer textos bastante completos sobre isso, a exemplo de [XXX].






Este trabalho tem uma abordagem mais modesta, concentrando-se numa revisita\c{c}\~ao dos fatos que levaram \`a concep\c{c}\~ao do WASH, buscando 3 linhas de investiga\c{c}\~ao:














\begin{alineas}
\item a avalia\c{c}\~ao do Projeto OLPC como motivadora da cria\c{c}\~ao do Projeto WASH
\item a avalia\c{c}\~ao do PIDS do MCTI como inspira\c{c}\~ao para as solu\c{c}\~oes espec\'{\i}ficas que fizeram o WASH se diferenciar do OLPC
\item A influ\^encia do GESAC na transforma\c{c}\~ao do WASH j\'a existente, a partir de 2014
\item o contexto hist\'orico mais amplo, que influenciou todos os acontecimentos, principalmente sob a perspectiva de uma das agentes que sempre esteve presente na trajet\'oria do WASH: a professora afira Ripper
\end{alineas}

\subsection[Governo Eletr\^onico]{Governo Eletr\^onico}\label{Governo Eletr\^onico}
Foi no s\'eculo XIX que os primeiros conceitos de programa\c{c}\~ao come\c{c}aram a ser desenvolvidos. O mec\^anico franc\^es Joseph-Marie Jacquard (1752-1854) inventou o primeiro tear automatizado, utilizando a inova\c{c}\~ao dos cart\~oes perfurados. Outros contribuintes foram Charles Baggage (1791-1871) e Ada Lovelace (1815-1852), com o desenvolvimento do conceito de m\'aquina anal\'{\i}tica, embora a m\'aquina, propriamente dita, n\~ao tenha sido efetivamente constru\'{\i}da. No entanto, mesmo assim, seus esfor\c{c}os s\~ao considerados basilares para o desenvolvimento dos primeiros computadores. Ada Lovelace foi considerada a primeira pessoa efetivamente a se valer do conceito de programa\c{c}\~ao na Hist\'oria. Interessante observar que essas iniciativas do s\'eculo XIX surgiram como demandas de um mec\^anico e um banqueiro [XXX quem s\~ao?] que buscavam resolver quest\~oes pr\'aticas.






O empres\'ario norte americano Herman Hollerith (1860-1929) desenvolveu um sistema capaz de computar dados. Seu desenvolvimento se deu no contexto de uma demanda de Governo. Desde 1880, o governo americano fazia o censo demogr\'afico e demorava 8 anos para contabilizar os dados. Hollerith criou uma m\'aquina capaz de computar as informa\c{c}\~oes coletadas durante o censo de 1890 [XXX], tamb\'em a partir de cart\~oes perfurados, diminuindo assim o tempo de c\'alculo para apenas dois anos e meio. Esse exemplo talvez seja uma das primeiras formas de emprego de uma tecnologia digital primitiva numa atividade de governo. Mas n\~ao era uma tecnologia voltada para disponibilizar servi\c{c}os diretamente para o cidad\~ao, um conceito que veio a se concretizar muitas d\'ecadas depois.






A partir desta iniciativa, Hollerith vendeu suas m\'aquinas para governos e empresas, tendo sido, tamb\'em, um dos fundadores da IBM, hoje uma das maiores empresas de computa\c{c}\~ao do mundo [XXX]. Dentre os servi\c{c}os prestados pela IBM est\'a o apoio ao Holocausto nazista contra judeus e outras minorias, durante o Terceiro Reich Alem\~ao [XXX].






Atualmente os computadores s\~ao ferramentas indispens\'aveis para o desenvolvimento do mundo e funcionamento das sociedades modernas, bem como do conhecimento cient\'{\i}fico. Em suma, a hist\'oria da computa\c{c}\~ao e das m\'aquinas remonta a tempos antigos, que v\~ao desde as ferramentas de c\'alculo, passando pela revolu\c{c}\~ao industrial e suas tentativas de se criar computadores mec\^anicos, os computadores eletr\^onicos anal\'ogicos [trabalho de Vannevar Bush XXX], at\'e chegar \`a forma dos computadores eletr\^onicos digitais conhecidas hoje.






Como se v\^e pela hist\'oria, o uso de tecnologias da informa\c{c}\~ao e comunica\c{c}\~ao pelos governos \'e t\~ao antigo quanto a pr\'opria exist\^encia da computa\c{c}\~ao.






No Brasil, a utiliza\c{c}\~ao da tecnologia da informa\c{c}\~ao na administra\c{c}\~ao p\'ublica teve in\'{\i}cio na d\'ecada de 1960 pelas empresas estatais [XXX livro da vera dantas]. Naquele tempo os engenheiros brasileiros formados na \'area tinham duas perspectivas: trabalhar no governo ou nas estatais, comprando equipamentos, ou nas multinacionais, vendendo equipamentos para o Governo [XXX frase da \'epoca]. Isto se dava porque o Brasil n\~ao tinha uma cultura de desenvolvimento no mundo digital e esse tipo de atividade era desestimulada pelas pot\^encias estrangeiras. Um esfor\c{c}o muito grande foi institu\'{\i}do no pa\'{\i}s, principalmente a partir da d\'ecada de 60, para reverter essa situa\c{c}\~ao [XXX]. Esse esfor\c{c}o permitiu a g\^enese de uma comunidade de profissionais, estabelecendo as bases para a constitui\c{c}\~ao de uma \textquotedbl cultura digital que veio a se expressar mais amplamente a partir da d\'ecada de 90 [XXX citar coisas de cultura digital aqui].






As press\~oes internacionais por um estado \textquotedbl gerencial e empreendedor, intensificaram o movimento conhecido por reforma da gest\~ao p\'ublica (Bresser-Pereira, 2002) ou new public management (Ferlie etal., 1996). Este movimento teve como cerne a \textquotedbl busca da excel\^encia e a orienta\c{c}\~ao aos servi\c{c}os ao cidad\~ao.






Nos prim\'ordios do emprego de tecnologias digitais em atividades de governo, a men\c{c}\~ao a \textquotedbl IT in Government (\textquotedbl Tecnologia da Informa\c{c}\~ao no Governo, em tradu\c{c}\~ao livre) se referia exclusivamente ao uso da tecnologia no interior dos governos. Portanto n\~ao era uma tecnologia voltada para disponibilizar servi\c{c}os diretamente para o cidad\~ao .






Assim, com essa vis\~ao gerencial, em sua g\^enese o conceito de governo eletr\^onico buscava tratar o indiv\'{\i}duo mais como \textquotedbl cliente, ou como \textquotedbl pagador de impostos [resenha acima], do que necessariamente um cidad\~ao com direitos civis.






Em que pese esse in\'{\i}cio bastante vinculado \`as controversas ideologias da \'epoca, em particular \`a no\c{c}\~ao de \textquotedbl empreendedorismo de Estado, h\'a que se reconhecer que tais iniciativas prepararam a sociedade para as transforma\c{c}\~oes tecnol\'ogicas vindouras, que alteraram a rela\c{c}\~ao do Estado com seus cidad\~aos.






A ideia de governo eletr\^onico difere-se de um simples uso de \textquotedbl IT in Government, porque trata do acesso direto ao governo por meios digitais pelo pr\'oprio cidad\~ao, sem intermedi\'arios. Portanto, s\'o se tornou vi\'avel a partir da dissemina\c{c}\~ao em grande escala das tecnologias de informa\c{c}\~ao e comunica\c{c}\~ao.






\'E comum atribu\'{\i}rem ao advento do WebBrowser [XXX referencias sobre o Mosaic], ou seja, ao pr\'oprio advento da internet como se conhece hoje, o pioneirismo para a dissemina\c{c}\~ao das tecnologias digitais.






Mas, por justi\c{c}a hist\'orica, \'e preciso reconhecer que antes mesmo desse marco, j\'a existia na Fran\c{c}a uma tecnologia que oferecia servi\c{c}os de todo tipo para os cidad\~aos: o MINITEL [XXX], que no Brasil era conhecido como V\'{\i}deo Texto. Muito antes do HTML, em meados da d\'ecada de 80, o MINITEL e suas vers\~oes locais (Su\'ecia, Irlanda, \'Africa do Sul, Canad\'a, Brasil, etc)[XXX\} j\'a eram extensivamente usadas. Na cidade de S\~ao Paulo o v\'{\i}deo texto da Telesp chegou a ter cerca de 70 mil assinantes [XXX].






O Judici\'ario brasileiro inaugurou os servi\c{c}os digitais para atendimento ao cidad\~ao, j\'a no in\'{\i}cio da d\'ecada de noventa. Este pioneirismo se deu com o uso de c\'odigos de barra para identifica\c{c}\~ao de eleitores, por exemplo. Ali\'as, muito antes das a\c{c}\~oes do executivo, houve o desenvolvimento da Urna Eletr\^onica, uma iniciativa totalmente estatal, com a participa\c{c}\~ao de unidades de pesquisa federais (CTI em 1990 e INPE em 1994). As a\c{c}\~oes do executivo brasileiro em dire\c{c}\~ao ao governo eletr\^onico remontam ao in\'{\i}cio da d\'ecada de 90, sempre com a participa\c{c}\~ao do SERPRO [precisa de refer\^encia XXX]. Pode-se considerar que o programa de imposto de renda oferecido pela receita federal a partir de 1991 foi uma das primeiras a\c{c}\~oes em grande escala do executivo no sentido de oferta de servi\c{c}os digitais diretos para o cidad\~ao, mesmo considerando que o envio dos dados da declara\c{c}\~ao por internet s\'o foi viabilizado a partir de 1998. No in\'{\i}cio, era preciso enviar os disquetes da declara\c{c}\~ao juntamente com a documenta\c{c}\~ao em papel.






O movimento em dire\c{c}\~ao ao governo eletr\^onico ganhou mais institucionalidade a partir do final do governo FHC, principalmente com a atua\c{c}\~ao de Pedro Parente \`a frente da Casa Civil [Diniz, Barbosa, etc].






[YYY rever essa frase]O Brasil e o M\'exico, segundo J.Ramon Gil Garcia e Beatriz B. Lanza)Digital Governo Brasil, M\'exico e EUA formalizaram o governo digital 2000, com foco em infraestrutura da internet e servi\c{c}os e processos enquanto EUA.






[YYY esta frase est\'a fora de lugar] A quest\~ao da infraestrutura no Brasil \'e relevante pois x da popula\c{c}\~ao permanece sem acesso a internet ( CGI.br, INEGI,2015 de atualizar esse dado)






O Governo Digital no Brasil foi formalizado por Decreto Presidencial de 3 abril de 2000, cuja implementa\c{c}\~ao se deu sob a coordena\c{c}\~ao pol\'{\i}tica da Presid\^encia da Rep\'ublica, com apoio t\'ecnico e gerencial da Secretaria de Log\'{\i}stica e Tecnologia da Informa\c{c}\~ao (SLTI), do Minist\'erio do Planejamento, Or\c{c}amento e Gest\~ao. Essa atua\c{c}\~ao foi sustentada por um comit\^e integrado pelos secret\'arios executivos (e cargos equivalentes) dos minist\'erios e \'org\~aos da Presid\^encia da Rep\'ublica, denominado Comit\^e Executivo de Governo Eletr\^onico (Cege).






Inicialmente o governo brasileiro concentrou esfor\c{c}os em tr\^es linhas de a\c{c}\~ao do Programa Sociedade da Informa\c{c}\~ao [YYY tem que explicar que programa \'e esse]: universaliza\c{c}\~ao de servi\c{c}os, governo ao alcance de todos e infraestrutura avan\c{c}ada [por enquanto, coloque as cita\c{c}\~oes com colchetes ao inv\'es de par\^enteses, para n\~ao confundir com par\^enteses que naturalmente acontecem no texto, colocar XXX para n\~ao perder a cita\c{c}\~ao depois] ( XXX Comit\^e Executivo E-gov, 2002).






Essa a\c{c}\~ao vinha no bojo do movimento em prol da moderniza\c{c}\~ao da administra\c{c}\~ao p\'ublica, j\'a mencionada, e da presta\c{c}\~ao de servi\c{c}os para a popula\c{c}\~ao, com um vi\'es de busca pela \textquotedbl qualidade em processos e servi\c{c}os [XXX Tecnologia Industrial B\'asica TIB], um conceito que hoje parece corriqueiro, mas que era objeto de frisson naquela \'epoca [XXX].






Embora as iniciativas do Governo FHC fossem exclusivamente acess\'{\i}veis a uma elite de cidad\~aos, uma vez que a maior parte da popula\c{c}\~ao n\~ao tinha acesso \`a internet [XXX CGI.br, INEGI,2015 de atualizar esse dado], sem o apontamento de solu\c{c}\~oes sist\^emicas para sua universaliza\c{c}\~ao, funcionaram para abrir o caminho institucional do Governo Eletr\^onico.






Um novo paradigma cultural de inclus\~ao social e digital para cidad\~aos, se fazia necess\'ario, mas esta diretriz n\~ao estava presente na fase pioneira de implanta\c{c}\~ao do governo eletr\^onico no Brasil. Inicialmente, tratando os cidad\~aos como clientes, o foco era a redu\c{c}\~ao de custos unit\'arios, melhorias na gest\~ao e qualidade dos servi\c{c}os p\'ublicos, transpar\^encia governamental e simplifica\c{c}\~ao de procedimentos, formalizados como estrat\'egias, macro-objetivos e  as metas priorit\'arias  do governo brasileiro para o per\'{\i}odo de 2000 a 2003.






[YYY frase perdida no meio do texto… mudan\c{c}a para primeira pessoa do plural] Fizemos um breve levantamento com apoio da metodologia historiogr\'afica na  perspectiva hist\'orica e de pesquisa em administra\c{c}\~ao p\'ublica, (XXX Peranti,Octavio,2022) [YYY coloca entre colchetes nesta fase de elabora\c{c}\~ao do texto. quando estiver tudo pronto, muda para o formato ABNT].






A consolida\c{c}\~ao de uma cadeia produtiva completa e eficiente, e que usufru\'{\i}a de m\~ao-de-obra barata na \'Asia [XXX], contribuiu para a redu\c{c}\~ao de barreiras econ\^omicas para acesso a dispositivos digitais, uma vez que houve ampla comoditiza\c{c}\~ao da produ\c{c}\~ao de eletroeletr\^onicos em geral e dos bens de computa\c{c}\~ao em particular [XXX]. Uma decorr\^encia direta da Lei de Moore [XXX], o mundo passou a produzir mais transistores eletr\^onicos do que gr\~aos de soja [XXX], com ganhos de escala que tornaram essas tecnologias mais dispon\'{\i}veis.






Essa alta disponibilidade de equipamentos digitais, a baixo custo, facilitou uma presen\c{c}a cada vez maior da internet na vida das pessoas, principalmente a partir da populariza\c{c}\~ao dos celulares do tipo \textquotedbl smart-phone.






Esta transforma\c{c}\~ao estimulou os governos [XXX] a enfrentarem as dificuldades  de falta de  capacita\c{c}\~ao dos cidad\~aos na apropria\c{c}\~ao tecnol\'ogica, de forma que pudessem usufruir melhor da abund\^ancia e acesso aos equipamentos digitais. Para isso, estabeleceram pol\'{\i}ticas p\'ublicas que os preparassem para usufru\'{\i}rem do direito humano \`a comunica\c{c}\~ao [XXX citar a constitui\c{c}\~ao]. Os governos passaram a se preocupar com a inser\c{c}\~ao efetiva de seus cidad\~aos na sociedade da informa\c{c}\~ao [XXX].






Essas iniciativas ficaram conhecidas, genericamente, como programas pertinentes a politicas de \textquotedbl inclus\~ao digital, ou  de \textquotedbl cultura digital ou mesmo de \textquotedbl alfabetiza\c{c}\~ao tecnol\'ogica [XXX]. Independentemente da abordagem escolhida, dentre as tr\^es indicadas, essas pol\'{\i}ticas sempre estiveram vinculadas \`as estruturas de educa\c{c}\~ao, seja a formal, ou a n\~ao-formal [XXX].






Diferentes iniciativas e perspectivas foram implementadas para uso das tecnologias da informa\c{c}\~ao e comunica\c{c}\~ao no Brasil.Foram disponibilizados equipamentos, aplicativos, softwares,hardwares, para processar, armazenar, comunicar, prover apropria\c{c}\~ao tecnol\'ogica, o acesso a  informa\c{c}\~ao, ao conhecimento como a\c{c}\~ao de politica de inclus\~ao digital.






Algumas a\c{c}\~oes consideravam o cidad\~ao como usu\'ario de servi\c{c}os e, para que tivesse acesso a eles, uma capacita\c{c}\~ao em dom\'{\i}nio de mouse e teclado [XXX], por exemplo, era oferecida.






Um passo a frente, havia as capacita\c{c}\~oes direcionadas \`a intera\c{c}\~ao com servi\c{c}os espec\'{\i}ficos, a exemplo de \_\_\_\_\_\_\_\_ [XXX].






Tamb\'em existiam as capacita\c{c}\~oes voltadas para a utiliza\c{c}\~oes de pacotes de aplicativos, a exemplo dos pacotes de escrit\'orio [XXX].






Outras capacita\c{c}\~oes focalizavam a autonomia no estabelecimento de servi\c{c}os locais para o atendimento dos demais cidad\~aos. Um exemplo eram os cursos em montagem e configura\c{c}\~ao de redes de computadores [XXX].






Eram comuns, tamb\'em, as capacita\c{c}\~oes voltadas para a autoria na \'area de cultura, as quais visavam a autonomia dos movimentos culturais na produ\c{c}\~ao de seus pr\'oprios produtos, sem a depend\^encia de gravadoras ou outras estruturas voltadas para modelos de neg\'ocio comerciais [XXX].






Uma abordagem mais ampla, envolvendo a elabora\c{c}\~ao de saberes e compet\^encias no campo da computa\c{c}\~ao, envolvia a pr\'atica da programa\c{c}\~ao de jogos de computadores, a exemplo das que foram desenvolvidas no WASH [XXX]






Para garantir a objetividade da an\'alise no contexto desta disserta\c{c}\~ao, h\'a que se concentrar nos aspectos pertinentes ao objeto de estudo, i.e. o Projeto WASH. Esta restri\c{c}\~ao exige focalizar a rela\c{c}\~ao entre as tecnologias digitais e a educa\c{c}\~ao formal e n\~ao-formal, abordagens adotadas pelo projeto Workshop Aficionados por Software e Hardware-WASH, como se ver\'a mais adiante.






Assim, no esp\'{\i}rito de manter a objetividade, e por sua rela\c{c}\~ao direta na g\^enese do Projeto WASH, optou-se por focalizar a pol\'{\i}tica p\'ublica \textquotedbl Governo Eletr\^onico de Servi\c{c}os de Atendimento ao Cidad\~ao-GESAC, programa do  Minist\'erio das Comunica\c{c}\~oes, cujo o formato de interesse para este trabalho \'e o que se consolidou a partir de 2003.






Tivemos um papel na constru\c{c}\~ao e execu\c{c}\~ao de pol\'{\i}ticas p\'ublicas com as caracter\'{\i}sticas acima, inicialmente no \^ambito do Governo Eletr\^onico, passando pelas \'areas de comunica\c{c}\~ao, sa\'ude, cultura, e culminando na \'area de ci\^encia e tecnologia. Estas labora\c{c}\~oes  se deram em v\'arios momentos de sua carreira, ao longo de quase 3 d\'ecadas. Isso a tornou uma testemunha ocular dos fatos a elas relacionados, inicialmente no  munic\'{\i}pio de Campinas, na d\'ecada de 90, e, em seguida, no \^ambito do Governo Federal, nas primeiras duas d\'ecadas do presente s\'eculo.






32- Nessa trajet\'oria foi poss\'{\i}vel aprender sobre as vantagens e desvantagens de cada uma das abordagens adotadas, bem como sobre a forma de combinar os elementos presentes de (I) at\'e (VI).






33- A partir de uma pr\'atica regular e frequente de oficinas de forma\c{c}\~ao para  crian\c{c}as e adolescentes, que se iniciou em setembro de 2013 no Centro de Tecnologia da Informa\c{c}\~ao CTI- Renato Archer em Campinas, esse aprendizado se consolidou em um m\'etodo do qual a candidata \'e co-autora, conhecido como WASH (Workshop de Aficionados em Software e Hardware).






34 Ap\'os um longo per\'{\i}odo de matura\c{c}\~ao, ajustes e repeti\c{c}\~ao, esse m\'etodo veio a ser formalizado em 2018 por meio de portaria de uma unidade de pesquisa do Minist\'erio da Ci\^encia, Tecnologia e Inova\c{c}\~oes [XXX Portaria 178/2018 SEI/CTI. A descri\c{c}\~ao detalhada do m\'etodo consta como anexo da referida portaria, a qual sintetiza os aprendizados conquistados ao longo dos anos, pelos v\'arios participantes do programa. De 2018 para c\'a, mais aprendizados ocorreram, havendo uma necessidade de aprimoramento de sua descri\c{c}\~ao.






35-\'E justamente uma an\'alise sobre esse m\'etodo que a presente disserta\c{c}\~ao intenciona oferecer, complementada por uma proposta de melhoria, na forma de produto tecnol\'ogico, como parte dos requisitos para obten\c{c}\~ao do t\'{\i}tulo de mestre no \^ambito do mestrado profissional em ensino de ci\^encias humanas, sociais e da natureza da Universidade  Tecnol\'ogico  Federal do Paran\'a - UTFPR- Campus Londrina/PR.






\subsection[Sociedade da Informa\c{c}\~ao]{Sociedade da Informa\c{c}\~ao}\label{Sociedade da Informa\c{c}\~ao}
Segundo S\'ergio Amadeu em [XXX Tudo Sobre Todos], os prim\'ordios da ideia de conhecimento como um recurso econ\^omico fundamental est\~ao relacionados com o trabalho \textquotedbl The Production and Distribution of Knowledge in the United States [XXX MACHLUP, Fritz. The Production and Distribution of Knowledge in the United States. Princeton, Nova Jersey: Princeton University Press, 1962.], do economista Fritz Machlup, no qual o conceito de Sociedade da Informa\c{c}\~ao teria aparecido pela primeira vez. A constru\c{c}\~ao do conceito pode ter se iniciado na d\'ecada de 30, quando Machlup estudava o efeito das patentes na pesquisa [XXX buscar fonte prim\'aria]. O nascimento do conceito \'e atribu\'{\i}do, tamb\'em, a Daniel Bell, professor de Harvard, que a partir do texto \textquotedbl The Coming of Post Industrial Society [XXX BELL, Daniel. The Coming of Post-industrial Society. Nova York: Basic Books, 1973] teria trazido, segundo Frank Webster [XXX Duff Journal of Information Science, 24(6) 1998, pp. 373-393], \textquotedbl a teoria mais influente sobre a ‘a sociedade da informa\c{c}\~ao’. No texto, Bell indicava que os servi\c{c}os e as atividades relacionadas ao fluxo de informa\c{c}\~oes tinham atingido um patamar de gera\c{c}\~ao de empregos maior do que as atividades industriais. Em outras palavras, \textquotedbl as m\'aquinas reprodutoras da for\c{c}a f\'{\i}sica e ampliadoras da velocidade estavam perdendo espa\c{c}o para tecnologias que armazenam, processam e distribuem informa\c{c}\~oes [S\'ergio Amadeu, tudo sobre todos]. Para Duff [XXX JIS, 24(6)] ] o emprego de uma metodologia de an\'alise \textquotedbl reputacional poderia colocar Bell entre os 10 pensadores no topo da elite intelectual americana, em tradu\c{c}\~ao livre, \textquotedbl ao lado de figuras p\'ublicas como m Chomsky, John Kenneth Galbraith and Norman Mailer, n\~ao cabe a este texto validar ou refutar estas afirma\c{c}\~oes, sen\~ao registrar que existe um reconhecimento sobre o papel de Bell na literatura. Dentre as contribui\c{c}\~oes mais not\'aveis de Bell [XXX Duff] estariam a identifica\c{c}\~ao ta transforma\c{c}\~ao p\'os-industrial da for\c{c}a de trabalho, o fluxo de informa\c{c}\~oes e a consequente \textquotedbl explos\~ao da informa\c{c}\~ao e a \textquotedbl revolu\c{c}\~ao da tecnologia da informa\c{c}\~ao. 






Ainda no intento de identificar as origens do conceito, h\'a que se falar do papel de Manuel Castells, com sua relevante \textquotedbl A era da informa\c{c}\~ao: economia, sociedade e cultura, de 






S\'ergio Amadeu sintetiza com propriedade uma defini\c{c}\~ao de sociedade da informa\c{c}\~ao: 







\noindent\begin{center}\mbox{\centering\fbox{\centering\par\parbox{0.7\linewidth}{\small\textit{\textquotedbl As sociedades informacionais s\~ao sociedades p\'os-industriais que t\^em a economia fortemente baseada em tecnologias que tratam informa\c{c}\~oes como seu principal produto. Portanto, os grandes valores gerados nessa economia n\~ao se originam principalmente na ind\'ustria de bens materiais, mas na produ\c{c}\~ao de bens imateriais, aqueles que podem ser transferidos por redes digitais. Tamb\'em \'e poss\'{\i}vel constatar que as sociedades informacionais se estruturam a partir de tecnologias cibern\'eticas, ou seja, tecnologias de comunica\c{c}\~ao e de controle, as quais apresentam consequ\^encias sociais bem distintas das tecnologias anal\'ogicas, tipicamente industriais.}\normalize}}}\end{center}


\subsection[Pol\'{\i}ticas P\'ublicas de Inclus\~ao e Cultura Digital]{Pol\'{\i}ticas P\'ublicas de Inclus\~ao e Cultura Digital}\label{Pol\'{\i}ticas P\'ublicas de Inclus\~ao e Cultura Digital}
A partir de 2003 o Brasil estabeleceu pol\'{\i}ticas agressivas de inclus\~ao social, as quais evolu\'{\i}ram tamb\'em para a inclus\~ao digital.






Segundo [XXX], inclus\~ao social pode ser definida como ....






Segundo [XXX] inclus\~ao social pode ser definida como...






Mundialmente, a inclus\~ao social foi praticada por pa\'{\i}s tais como ....






Esses pa\'{\i}ses observaram que...






\subsection[Descri\c{c}\~ao do Programa WASH]{Descri\c{c}\~ao do Programa WASH}\label{Descri\c{c}\~ao do Programa WASH}
Aqui vc pode fazer uma resenha r\'apida das v\'arias publica\c{c}\~oes que j\'a fizemos. Pode falar da portaria aqui tamb\'em, como fundamenta\c{c}\~ao te\'orica… N\~ao podemos fazer essa parte ser um spoiler do resultados e an\'alise, quando descrevemos o que descobrimos sobre o WASH. Ent\~ao, a bem da verdade, acho que DEVEMOS TIRAR ESSE T\'OPICO DAQUI.






\subsection[O pensamento de Papert]{O pensamento de Papert}\label{O pensamento de Papert}
Como j\'a indicado na introdu\c{c}\~ao, a proposta do Projeto WASH, objeto de estudo deste trabalho, foi motivada pelos achados obtidos durante a avalia\c{c}\~ao do Projeto OLPC, realizada por 3 institui\c{c}\~oes:







\begin{alineas}
\item Centro de Tecnologia da Informa\c{c}\~ao Renato Archer
\item Funda\c{c}\~ao CERTI da Universidade Federal de Santa Catarina
\item Laborat\'orio LSITec da Poli-USP
\end{alineas}

Esta avalia\c{c}\~ao foi demandada pela Presid\^encia da Rep\'ublica em 2004-2005, logo ap\'os o encontro do Presidente \`a \'epoca, Luis In\'acio Lula da Silva, com o o \textquotedbl gur\'u tecnol\'ogico\textquotedbl  do MIT, Prof. Nicholas Negroponte [XXX]. Neste encontro foi apresentada a proposta de um programa educacional, patrocinada pelo MIT, que envolvia a introdu\c{c}\~ao de abordagens pedag\'ogicas que buscavam combinar elementos de um amplo espectro de correntes distintas, que partiam de Piaget, passando por Vygotsky, Dewey e chegando em Paulo Freire [XXX].






N\~ao obstante esta pluralidade conceitual, o documento do OLPC n\~ao escondia a preval\^encia do pensamento de Papert, que na \'epoca ainda era vivo, na concep\c{c}\~ao da proposta apresentada ao Governo Brasileiro.






Pela import\^ancia do pensamento de Papert para o OLPC e, portanto, para a g\^enese do WASH, cabe uma revis\~ao r\'apida de sua obra e contribui\c{c}\~oes, permitindo uma melhor compreens\~ao da inser\c{c}\~ao do WASH no universo conceitual das correntes pedag\'ogicas.






Seymour Papert, foi um cidad\~ao sul-africano, oriundo de Petroia, radicado nos EUA.






Cientista da computa\c{c}\~ao, matem\'atico, educador, foi um dos fundadores do Media Lab e diretor do grupo de Epistemologia e Aprendizado do Massachusetts Institute of Technology (MIT)






Al\'em de cientista da computa\c{c}\~ao, professor de matem\'atica, Papert foi um educador vision\'ario percebeu o potencial do uso da tecnologia na educa\c{c}\~ao, como ferramenta para os processos de  ensino e aprendizagem.






No final da d\'ecada de 60, entre os anos de 1967 e 1968, per\'{\i}odo em que o uso do computador n\~ao era pensado para uso dom\'estico, Seymour Papert, criou o LOGO, uma linguagem de computa\c{c}\~ao que propunha transforma\c{c}\~ao  na concep\c{c}\~ao de ensino e aprendizagem






A linguagem LOGO  foi desenvolvida para permitir que crian\c{c}as programassem a m\'aquina, em vez de serem programadas por ela. Essa proposta faz com que o computador deixa de ser o meio de transferir informa\c{c}\~ao e passa a ser a ferramenta com a qual a crian\c{c}a pode formalizar os seus conhecimentos intuitivos






Segundo Jos\'e Armando Valente, Papert acreditava que o computador era a ferramenta que propiciava \`a crian\c{c}a as condi\c{c}\~oes de entrar em contato com algumas das mais profundas ideias em ci\^encia, matem\'atica, e a cria\c{c}\~ao de modelos” ( LOGO: computadores  educa\c{c}\~ao)






 No LOGO \'e poss\'{\i}vel definir novos comandos, fun\c{c}\~oes, \'e uma linguagem interativa, tem estrutura completa de listas.  O LOGO n\~ao \'e um brinquedo,  mas uma poderosa linguagem de computa\c{c}\~ao, foi planejado para fornecer acesso a programa\c{c}\~ao de computadores para principiantes e sem conhecimento matem\'atico






A linguagem da TARTARUGA \'e um subconjunto do LOGO que contem comandos.






Na filosofia do LOGO, o aprendizado acontece por meio do  processo de a crian\c{c}a inteligente “ensinar” o computador burro, ao inv\'es de o  computador inteligente ensinar a crian\c{c}a burra.






Na perspectiva de Papert \'e a crian\c{c}a que deve programar o computador e, ao faz\^e-lo ela adquire um sentimento de dom\'{\i}nio sobre um dos mais  modernos e poderosos equipamentos tecnol\'ogicos






Programar significa , nada mais, nada menos, comunicar-se com o computador numa linguagem que tanto ele quanto o homem podem entender” Toda crian\c{c}a normal aprende a falar.  Por que ent\~ao n\~ao deveria aprender a “ falar” com um computador?






A proposta de Papert era de que  o computador pode ser um interlocutor  de matem\'atica ou um interlocutor de l\'{\i}nguas






O LOGO ao ser um interlocutor da matem\'atica contribui de uma maneira l\'udica para superar as barreiras matof\'obicas  ( fobia por matem\'atica e fobia pelo aprendizado) e a matem\'atica passa a ser uma l\'{\i}ngua viva.






teste do papert






\subsection[O que \'e STEM?]{O que \'e STEM?}\label{O que \'e STEM?}
V\'arios autores [XXX Catterall, Heather Gonzalez e  Tahlea Jankoski, Rodger Bybee] indicam a d\'ecada de 90 do s\'eculo passado como o in\'{\i}cio do uso estruturado do conceito de Science, Technology, Engineering and Mathematics em curr\'{\i}culos escolares, mas o acr\^onimo para represent\'a-lo teve altera\c{c}\~oes ao longo dos anos. Segundo post de Tahlea Jankoski em [XXX https://blog.stemscopes.com/stem-a-rebranded-idea-of-the-past], inicialmente o conceito era representado pela sigla SMET, mas a similaridade de pron\'uncia com a palavra \textquotedbl smut (que significa obscenidade, em ingl\^es) sugeriu a mudan\c{c}a da sigla para METS e depois para STEM, em 2001 [XXX Tahlea Jankoski e Enciclopedia Brittanica]. 






Autores mencionam a confus\~ao que este termo gera, uma vez que em ingl\^es pode se referir a c\'elulas tronco, com tronco de \'arvore ou com o pedestal de um copo de vinho [XXX Rodger Bybee, ]. Para evitar esse tipo de confus\~ao, \'e poss\'{\i}vel identificar uma recorr\^encia da forma \textquotedbl STEM Education nas publica\c{c}\~oes. Neste trabalho ser\'a usada a forma STEM, em mai\'usculas, para se fazer refer\^encia ao movimento de revis\~ao curricular associado \`as disciplinas de \textquotedbl Science, Technology, Engineering and Mathematics.  






Os Estados Unidos sempre deram import\^ancia para a educa\c{c}\~ao de ci\^encias como pol\'{\i}tica p\'ublica. Uma evid\^encia disso pode ser encontrada nas atas da Conven\c{c}\~ao Constitucional de 1787, a exemplo do que se extrai da \textquotedbl Notes of Debates in the Federal Convention of 1787 [XXX apud Heather Gonzalez]:







\noindent\begin{center}\mbox{\centering\fbox{\centering\par\parbox{0.7\linewidth}{\small\textit{\textquotedbl to establish seminaries for the promotion of literature and the arts and the sciences.}\normalize}}}\end{center}


Outra evid\^encia pode ser extra\'{\i}da do primeiro discurso do Estado da Uni\~ao do Presidente George Washington [XXX achar data]:







\noindent\begin{center}\mbox{\centering\fbox{\centering\par\parbox{0.7\linewidth}{\small\textit{\textquotedbl Nor am I less persuaded that you will agree with me in opinion that there is nothing which can better deserve your patronage than the promotion of science and literature. Knowledge is in every country the surest basis of public happiness. In one in which the measures of government receive their impressions so immediately from the sense of the community as in ours it is proportionably [sic] essential. 2 (First State of Union Address - President George Washington)}\normalize}}}\end{center}


Da mesma forma, observadores [XXX Heather Gonzales] tra\c{c}am o lan\c{c}amento do sat\'elite Sputnik, em 1950, como um divisor de \'aguas para o ensino de STEM nos Estados Unidos [XXX Heather Gonzalez].






O movimento pelo STEM, nos Estados Unidos, tem evidente motiva\c{c}\~ao econ\^omica, estrat\'egica e de manuten\c{c}\~ao da hegemonia americana. Uma evid\^encia disso \'e a cita\c{c}\~ao \`a fala do Presidente da Lockheed Martin, Norm Augustine, em outubro de 2012, presente em [XXX James Catterall - deve ser o pai da LG Catterall]:







\noindent\begin{center}\mbox{\centering\fbox{\centering\par\parbox{0.7\linewidth}{\small\textit{\textquotedbl ... industry and government to promote more STEM education in the U.S. ‘Failure to do so... will undermine the U.S. economy, security and place as a world leader.’ Competing with knowledge-based resources will be one way that the U.S. can recover and retain primacy in the global marketplace (Twittweb, 2012).}\normalize}}}\end{center}


Mas em termos recentes, foi em meados da d\'ecada de 90 que o baixo desempenho comparativo em STEM dos estudantes americanos ganhou notoriedade na imprensa, pela constata\c{c}\~ao de uma sequ\^encia de notas med\'{\i}ocres no Programme for International Student Assessment (PISA) [XXX citar Catteral]. O PISA \'e um exame internacional promovido pela Organiza\c{c}\~ao para a Coopera\c{c}\~ao e Desenvolvimento Econ\^omico (OCDE), que busca estabelecer um padr\~ao global de avalia\c{c}\~ao, que permita comparar o desempenho de estudantes de diferentes pa\'{\i}ses [XXX citar documento que explique o PISA]. Nos dias de hoje, estudantes de cerca de 65 pa\'{\i}ses participam do exame, que \'e considerado um instrumento importante para planejar melhorias nos sistemas educacionais ao redor do mundo.






Em 1998, atrav\'es de um relat\'orio apresentado ao Congresso Americano pelo Committee on Equal Opportunities in Science and Engineering da National Science Foundation, este importante organismo, que seria o equivalente ao nosso CNPq, alerta para a import\^ancia do ensino de STEM nas escolas fundamentais americanas para que os EUA mantenham sua lideran\c{c}a global [XXX https://www.nsf.gov/pubs/2000/ceose991/ceose991.html]:







\noindent\begin{center}\mbox{\centering\fbox{\centering\par\parbox{0.7\linewidth}{\small\textit{\textquotedbl In order to maintain its global leadership, America must ensure our citizens can meet the demands of a more scientifically- and technologically-centered world. The National Science Foundation (NSF) has a key role in creating and maintaining the science, mathematics, engineering, and technology (SMET) capacity in this nation. The Committee on Equal Opportunities in Science and Engineering (CEOSE) has been charged by Congress with advising NSF in assuring that all individuals are empowered and enabled to participate fully in the science, mathematics, engineering, and technology (SMET) enterprise.}\normalize}}}\end{center}


Nesse relat\'orio o NSF usa ainda o acr\^onimo SMET, que em 2001, segundo a enciclop\'edia Brit\^anica teria sido alterado para STEM [XXX https://www.britannica.com/topic/STEM-education].






As \'areas em que os estudantes americanos n\~ao conseguiam se sobressair, em rela\c{c}\~ao aos demais pa\'{\i}ses desenvolvidos, eram as de ci\^encias, tecnologia e matem\'atica [XXX Catterall]. Essa situa\c{c}\~ao passou a representar inc\^omodo para os gestores educacionais do pa\'{\i}s, dado que n\~ao refletia a sua imagem pr\'opria de pot\^encia internacional [XXX citar Catterall], principalmente no campo da ci\^encia e tecnologia. Foi nesse momento que as iniciativas educacionais em \textquotedbl science, technology, engineering and mathematics se destacaram e o acr\^onimo SMET surgiu, posteriormente substitu\'{\i}do por STEM [XXX  Catterall].






Segundo a interpreta\c{c}\~ao da \'epoca, o baixo desempenho americano em STEM tinha rela\c{c}\~ao com a falta de equidade no acesso ao STEM, dentro da realidade das escolas americanas [XXX Catterall].






Dentre as respostas do governo americano se destacaram o programa \textquotedbl Nenhuma Crian\c{c}a Deixada para Tr\'as, em tradu\c{c}\~ao livre de \textquotedbl No Child Left Behind Act, uma iniciativa de 2002, e o \textquotedbl Todo Estudante ter\'a Sucesso, em tradu\c{c}\~ao livre de \textquotedbl Every Student Succeeds, de 2015 [XXX Catterall].






Mas as respostas americanas n\~ao ficaram restritas \`as esferas de governo, havendo tamb\'em as que foram conduzidas por organiza\c{c}\~oes n\~ao-governamentais [XXX], universidades[XXX], think-tanks, entre outras [XXX].






\section[Fundamenta\c{c}\~ao: caracteriza\c{c}\~ao do m\'etodo]{Fundamenta\c{c}\~ao: caracteriza\c{c}\~ao do m\'etodo}\label{Fundamenta\c{c}\~ao: caracteriza\c{c}\~ao do m\'etodo}
Aqui vem a fundamenta\c{c}\~ao para o o m\'etodo de caracteriza\c{c}\~ao de m\'etodos.






\section[Fundamenta\c{c}\~ao: caracteriza\c{c}\~ao dos resultados]{Fundamenta\c{c}\~ao: caracteriza\c{c}\~ao dos resultados}\label{Fundamenta\c{c}\~ao: caracteriza\c{c}\~ao dos resultados}
Nesta se\c{c}\~ao ser\'a descrito o embasamento para o trabalho de levantamento de resultados.






\subsection[Indicadores]{Indicadores}\label{Indicadores}
Segundo  Rodrigues (2010)  existe uma \textquotedbl estreita e indissoci\'avel\textquotedbl  rela\c{c}\~ao entre as palavras: medir, informar e indicador.






Esta percep\c{c}\~ao de sinon\'{\i}mia fundamenta-se em [apud: [MEADOWS (2006), que aponta a equival\^encia entre os conceitos: sinal, sintoma, press\'agio, aviso, dica, pista, situa\c{c}\~ao, categoria, dados, ponteiro, mostrador, luz de advert\^encia, instrumento e medida.






O termo \textquotedbl indicador\textquotedbl  pode ter um sentido muito mais espec\'{\i}fico quando pensado no contexto gerencial-corporativo (PARMENTER, 2007) ou no contexto de planejamento estrat\'egico, situa\c{c}\~oes que n\~ao est\~ao dentro do escopo desta disserta\c{c}\~ao.






Para esta disserta\c{c}\~ao n\~ao ser\'a explorado o vi\'es corporativo do termo, mas, diferentemente, o sentido de \textquotedbl estat\'{\i}sticas que fornecem algum tipo de medida de um fen\^omeno particular de preocupa\c{c}\~ao\textquotedbl  (apud: WONG, 2006).






Portanto, no contexto deste trabalho, indicadores s\~ao informa\c{c}\~oes quantitativas, que permitem caracterizar os resultados do projeto, tais como:







\begin{alineas}
\item n\'umero de crian\c{c}as atendidas
\item n\'umero de bolsistas
\item n\'umero de relat\'orios
\item distribui\c{c}\~ao de temas abordados em relat\'orios
\item n\'umero de oficinas realizadas
\item distribui\c{c}\~ao et\'aria dos participantes em oficinas
\item temas abordados nas oficinas
\item distribui\c{c}\~ao de temas nas oficinas
\item tipos de atividades realizadas
\item distribui\c{c}\~ao das atividades nas oficinas
\item quantidade de cidades atendidas
\item participantes mais ass\'{\i}duos
\end{alineas}

Para que os indicadores acima possam ser alcan\c{c}ados \'e preciso uma boa escolha da estrutura\c{c}\~ao de dados, assunto que ser\'a tratado adiante.






\subsection[Informa\c{c}\~ao, dados e conhecimento]{Informa\c{c}\~ao, dados e conhecimento}\label{Informa\c{c}\~ao, dados e conhecimento}
 Setzer e Silva (2017)  nos ensinam a diferen\c{c}a entre:







\begin{alineas}
\item dados
\item informa\c{c}\~oes
\item conhecimento
\end{alineas}

Segundo ele, os \textquotedbl dados\textquotedbl  s\~ao \textquotedbl representa\c{c}\~oes simb\'olicas quantific\'aveis\textquotedbl   (Setzer e Silva, 2017). Como exemplo de dados ele cita as letras do alfabeto. Sempre \'e poss\'{\i}vel atribuir um n\'umero a cada letra. Por exemplo, podemos atribuir o n\'umero 1 \`a letra A, o n\'umero 2 \`a letra B, o n\'umero 3 \`a letra C e assim por diante. Desta forma, um texto pode ser entendido como uma sequ\^encia de n\'umeros e, portanto, no sentido indicado, o texto \'e um dado porque tamb\'em pode ser representado por uma sequ\^encia de n\'umero (a sequ\^encia de n\'umeros que presenta a sequ\^encia de letras).






A temperatura de um ambiente tamb\'em \'e um dado: podemos atribuir um n\'umero que indica o valor da temperatura numa determinada escala. Por exemplo, podemos dizer que a sala \textquotedbl est\'a a 35 graus c\'elsius\textquotedbl .






Podemos atribuir um n\'umero para a quantidade de brasileiros e brasileiras, portanto o n\'umero de habitantes do nosso pa\'{\i}s tamb\'em \'e um dado.






Segundo Setzer e Silva (2017) o dado se transforma em \textquotedbl informa\c{c}\~ao\textquotedbl  quando algu\'em \'e capaz de associar um conceito ao dado, estabelecendo uma compreens\~ao humana sobre o que aquele s\'{\i}mbolo quantific\'avel representa.






Desta forma, o dado \textquotedbl temperatura\textquotedbl  s\'o se transforma em informa\c{c}\~ao quando o conceito de \textquotedbl quente\textquotedbl  e \textquotedbl frio\textquotedbl  pode ser associado a ele, numa perspectiva humana.






Ainda segundo Setzer e Silva (2017), as informa\c{c}\~oes se transformam em conhecimento quando os indiv\'{\i}duos s\~ao capazes de estabelecer rela\c{c}\~oes e associa\c{c}\~oes entre as informa\c{c}\~oes.  Setzer e Silva (2017) mencionam a import\^ancia das informa\c{c}\~oes serem adquiridas por uma viv\^encia pessoal para que se tornem conhecimento, caracterizando-o como um atributo subjetivo.






Esta singela defini\c{c}\~ao oferecida por Setzer nos basta para este trabalho, e renunciamos ao tratamento matem\'atico da Teoria da Informa\c{c}\~ao como apresentado por Shannon (Barrios, 2015), por exemplo, uma vez que escapa ao escopo deste estudo.






A defini\c{c}\~ao de Setzer e Silva (2017) \'e corroborada pelo trabalho de MAMMANA (1999), que traz uma interpreta\c{c}\~ao filogen\'etica para o processamento das informa\c{c}\~oes. Esta vis\~ao prop\~oe que \textquotedbl n\~ao existe informa\c{c}\~ao fora da biosfera\textquotedbl  e que a informa\c{c}\~ao seria uma primitiva epistemol\'ogica da biologia. Em outras palavras, os dados representados por s\'{\i}mbolos s\'o adquirem um car\'ater de informa\c{c}\~ao quando passam a influenciar a probabilidade de sobreviv\^encia do patrim\^onio gen\'etico de um indiv\'{\i}duo. Esta vis\~ao fica expl\'{\i}cita na seguinte cita\c{c}\~ao:







\noindent\begin{center}\mbox{\centering\fbox{\centering\par\parbox{0.7\linewidth}{\small\textit{\textquotedbl (...) informa\c{c}\~ao \'e representada pelo aumento de v\'{\i}nculos casuais que constituem os indiv\'{\i}duos (...). Estes v\'{\i}nculos foram constru\'{\i}dos pela evolu\c{c}\~ao e sele\c{c}\~ao natural para melhorar o preparo do indiv\'{\i}duo no enfrentamento dos perigos ambientais e possibilidades, de forma a aumentar a probabilidade de sobreviv\^encia de seus genes.\textquotedbl  (tradu\c{c}\~ao livre de  MAMMANA (1999))}\normalize}}}\end{center}


Segundo essa vis\~ao de  MAMMANA (1999), que suplementa a vis\~ao de Shannon baseada na entropia, uma cadeia de processos de de-somatiza\c{c}\~ao de habilidades humanas evolui para extens\~ao do papel evolutivo da informa\c{c}\~ao para o contexto antropol\'ogico e cultural.






No contexto do projeto WASH \'e preciso identificar quais dados e suas combina\c{c}\~oes, na forma de informa\c{c}\~oes, t\^em relev\^ancia para a exist\^encia e reprodu\c{c}\~ao do projeto ao longo dos anos. Portanto \'e a identifica\c{c}\~ao desta relev\^ancia que definir\'a quais s\~ao as informa\c{c}\~oes que dele precisam ser extra\'{\i}das, com vistas \`a sua caracteriza\c{c}\~ao.






O projeto WASH \'e prof\'{\i}cuo na produ\c{c}\~ao de dados, uma vez que atende uma quantidade muito grande de crian\c{c}as, adolescentes e adultos. Estes dados est\~ao distribu\'{\i}dos em v\'arias localidades e se referem a atividades realizadas em institui\c{c}\~oes de variados tipos, cujas estruturas s\~ao muito diferentes uma das outras. Os tipos de atividades s\~ao muito variados, dependendo das caracter\'{\i}sticas locais, bem como os temas que s\~ao abordados.






Como se ver\'a adiante, o WASH ocorre  em escolas b\'asicas, escolas t\'ecnicas, organiza\c{c}\~oes sociais, sindicatos, igrejas, centros de inclus\~ao social p\'ublicos, centros de pesquisa, universidades, em feiras e exposi\c{c}\~oes, dentre tantas outras modalidades.






O formato do WASH \'e variado, podendo ocorrer no contexto escolar, no contra-turno, presencialmente ou remotamente.






Muito embora seja um requisito a promo\c{c}\~ao dos valores do m\'etodo cient\'{\i}fico, o WASH tamb\'em \'e variado no que se refere \`a suas tem\'aticas e o tipo de atividades que s\~ao realizadas.






Esta variabilidade de institui\c{c}\~oes envolvidas, formatos de realiza\c{c}\~oes, tem\'aticas, localidades e tipos de atividades imp\~oe um desafio sobre como os dados devem ser estruturados, para que representem a ess\^encia do projeto, convers\'{\i}vel em informa\c{c}\~oes \'uteis para a gest\~ao, reprodu\c{c}\~ao e sobreviv\^encia do mesmo.






Esta forma de estrutura\c{c}\~ao dos dados define como ser\~ao gerados os indicadores de interesse para a caracteriza\c{c}\~ao do projeto, estabelecendo o n\'{\i}vel de confian\c{c}a na sua capacidade de representar essas caracter\'{\i}sticas.






\subsection[Bancos de Dados Relacionais]{Bancos de Dados Relacionais}\label{Bancos de Dados Relacionais}
O objetivo desta se\c{c}\~ao \'e rever os m\'etodos de modelagem de dados dispon\'{\i}veis para viabilizar a coleta de dados no \^ambito do Projeto WASH. O foco aqui ser\'a embasar a decis\~ao de empregar a modelagem relacional, em detrimento de outros m\'etodos menos estruturados.






Os bancos de dados podem ter diferentes n\'{\i}veis de formaliza\c{c}\~ao, seguindo integralmente ou em parte normas bastante espec\'{\i}ficas (Setzer e Silva, 2017) , dependendo das caracter\'{\i}sticas e necessidades do sistema que se quer modelar. Mas antes de entrar na revis\~ao sobre o assunto, \'e preciso entender a complexidade do sistema que se quer representar.






A primeira caracter\'{\i}stica do Programa WASH que determina a forma como a coleta de dados precisar\'a ser feita, e que podemos antecipar neste ponto do texto, \'e sua diferen\c{c}a em rela\c{c}\~ao a outros programas de bolsas de inicia\c{c}\~ao cient\'{\i}fica.






Diferentemente de programas que ocorrem no \^ambito acad\^emico de pesquisa, o Projeto WASH tem uma \^enfase maior em extens\~ao, que deve ser concretizada pela oferta de oficinas em STEAM para o ensino fundamental.






Assim, para a avalia\c{c}\~ao de suas caracter\'{\i}sticas h\'a que se estabelecer meios de medir quantitativamente sua capacidade de atingir crian\c{c}as do ensino fundamental.






Portanto, \'e preciso criar meios de coletar dados sobre, principalmente:







\begin{alineas}
\item o n\'umero de crian\c{c}as atendidas,
\item n\'umero de oficinas ofertadas
\item n\'umero de institui\c{c}\~oes participantes
\item n\'umero de horas de atividade por estudante
\end{alineas}

dentre tantos outros dados.






A forma plural como o projeto WASH busca atender seus benefici\'arios ficar\'a mais clara adiante, mas neste ponto podemos dizer que o WASH tamb\'em \'e bastante diferente de uma escola do ensino formal, na qual est\~ao bem estabelecidas as normas de participa\c{c}\~ao de estudantes, bem como as regras para o registro desta participa\c{c}\~ao.






Por ser um programa sem uma legisla\c{c}\~ao espec\'{\i}fica para o estabelecimento de obriga\c{c}\~oes entre os part\'{\i}cipes, o WASH tem que ocorrer no \^ambito de organiza\c{c}\~oes (escolas, associa\c{c}\~oes, igrejas, sindicatos) que j\'a seguem normas legais e infra-legais voltadas para garantir a prote\c{c}\~ao dos menores de idade participantes.






Portanto, o sistema de registro de participa\c{c}\~oes de estudantes do WASH precisa ser flex\'{\i}vel o bastante para garantir a representa\c{c}\~ao desse ambiente diverso institucionalmente, adaptando-se \`a realidade de cada institui\c{c}\~ao parceira.






Podemos exemplificar o n\'{\i}vel de normatiza\c{c}\~ao da escola p\'ublica regular usando o caso do Estado de S\~ao Paulo, que, como em outros estados, tem legisla\c{c}\~ao espec\'{\i}fica detalhada sobre como registrar a presen\c{c}a de seus alunos nos bancos escolares.






Escolhemos a vers\~ao de 2010 da \textquotedbl LEGISLA\c{C}\~AO DE ENSINO FUNDAMENTAL E M\'EDIO ESTADUAL\textquotedbl  do Estado de S\~ao Paulo como exemplo, para mostrar que o controle de frequ\^encia de alunos \'e normatizado por meio do Art. 6o da RESOLU\c{C}\~AO SE No 20, DE 17 DE FEVEREIRO DE 2010, in verbis:







\noindent\begin{center}\mbox{\centering\fbox{\centering\par\parbox{0.7\linewidth}{\small\textit{Artigo 6º – Cabe aos professores manter atualizados os dados de frequ\^encia e avalia\c{c}\~ao dos alunos nos respectivos di\'arios de classe, a fim de subsidiar o seu registro e atualiza\c{c}\~ao, no Sistema.}\normalize}}}\end{center}


Em outros pontos essa legisla\c{c}\~ao traz mais detalhes sobre como esse registro deve ser feito.






Como se v\^e, pela import\^ancia que tem na medi\c{c}\~ao da efici\^encia e efic\'acia da presta\c{c}\~ao do servi\c{c}o de educa\c{c}\~ao, o controle de presen\c{c}a \'e instrumento regulamentado e com atribui\c{c}\~ao de responsabilidades espec\'{\i}ficas no \^ambito da Secretaria de Educa\c{c}\~ao de S\~ao Paulo, assim como ocorre em outros estados.






Al\'em de servir de indicador de efici\^encia e efic\'acia, o controle de frequ\^encia tamb\'em funciona como auxiliar das tarefas log\'{\i}sticas e de planejamento da escola. Com o controle de presen\c{c}a \'e poss\'{\i}vel saber quais escolas devem receber mais recursos, por exemplo, e uma falha na gera\c{c}\~ao destes dados pode comprometer a qualidade de todo o servi\c{c}o.






O WASH, por ser uma atividade de educa\c{c}\~ao complementar \`a da escola regular, n\~ao tem uma normatiza\c{c}\~ao equivalente. Mesmo assim n\~ao pode abrir m\~ao de produzir seus pr\'oprios indicadores de efici\^encia e efic\'acia, raz\~ao pela precisou desenvolver um m\'etodo pr\'oprio.






Essa necessidade de um sistema pr\'oprio de registro decorre da impossibilidade de compartilhamento ostensivo de dados por parte das institui\c{c}\~oes respons\'aveis pelos alunos. Em algumas situa\c{c}\~oes, como \'e o caso de atividades realizadas em associa\c{c}\~oes e igrejas, por exemplo, a institui\c{c}\~ao parceira squer tem um sistema otimizado de controle de presen\c{c}a, fato que refor\c{c}a a necessidade do WASH criar seus pr\'oprios m\'etodos de gera\c{c}\~ao de indicadores.






Esta necessidade de registro foi reconhecida nos prim\'ordios do Projeto e uma descri\c{c}\~ao da evolu\c{c}\~ao dos m\'etodos de coleta de dados \'e feita no Resultados e An\'alise desta disserta\c{c}\~ao.






De forma bem suscinta, podemos recapitular que a coleta de dados presen\c{c}a no \^ambito do projeto WASH se deu, inicialmente, por meio que podemos chamar de anal\'ogico: o registro em papel do nome das crian\c{c}as presentes, com a marca\c{c}\~ao da data no topo da folha.






Com o crescimento r\'apido do projeto este m\'etodo come\c{c}ou a ficar invi\'avel, e foi tentanda a utiliza\c{c}\~ao de formul\'arios on-line tipo \textquotedbl Google Forms\textquotedbl , os quais eram transferidos para planilhas eletr\^onicas para armazenagem.






O emprego de planilhas eletr\^onicas tamb\'em se mostrou insatisfat\'orio e \'e neste ponto que come\c{c}amos a revis\~ao da literatura de Bancos de Dados Relacionais.






FULLER (2011), em seu artigo ``Vantagens e perigos de usar o Microsoft Excel para organizar e apresentar dados de qualidade de \'agua`` nos presenteia com algumas importantes reflex\~oes:







\noindent\begin{center}\mbox{\centering\fbox{\centering\par\parbox{0.7\linewidth}{\small\textit{Usar o Excel para organizar os dados \'e uma tremenda vantagem, mas tamb\'em cria oportunidade para introduzir erros insidiosos no conjunto de dados, erros que podem entrar nos dados de forma sutil, com impacto pervasivo mas muito dif\'{\i}cies de descobrir(...)}\normalize}}}\end{center}


O n\'{\i}vel de confian\c{c}a nesta afirma\c{c}\~ao de  FULLER (2011) \'e bastante alto, uma vez que o trabalho de coleta de dados por ele realizado envolveu uma entrada de 4.752 dados anuais por mais de 10 anos, o que certamente permitiu que ele avaliasse a confiabilidade do Excel como ferramenta. A escola do Excel como ferramenta tinha sido aprovada pela Ag\^encia de Fomento  (FULLER, 2011), uma transi\c{c}\~ao do m\'etodo anterior de coleta, que era baseado em planilhas em papel com c\'alculos feitos em calculadora.






Apenas para registro, no sentido de prover uma melhor figura sobre o que esta refer\^encia pode nos trazer, cabe mencionar que os dados envolviam data de coleta, hor\'ario, condi\c{c}\~oes meteorol\'ogicas, temperatura do ar, pH, condutividade, condut\^ancia espec\'{\i}fica e oxig\^enio dissolvido.






Com base nesta vasta experi\^encia,  FULLER (2011) identificou as seguintes fontes de erros na entrada de dados:







\begin{alineas}
\item Erros de digita\c{c}\~ao: normalmente envolviam apertar inadvertidamente n\'umeros adjacentes no teclado, ou simplesmente ler os dados de laborat\'orio de forma errada. Esse tipo de erro pode alterar dramaticamente as m\'edias e passar despercebido nos gr\'aficos de espalhamento de dados.
\item Deslocamento de colunas e repeti\c{c}\~ao inadvertida de dados: algumas vezes uma coluna pode ser repetida sem que a pessoa respons\'avel por entrar os dados perceba, por exemplo.
\item Perda de n\'umeros ou multiplicidade indevida de de entrada de dados: isto pode gerar uma coluna de dados com menos ou mais dados do que o n\'umero original.
\end{alineas}

Estes tipos de erros, por mais pros\'aicos que possam parecer, tinham paralelo na experi\^encia de registro do WASH. No in\'{\i}cio do projeto observaramos uma falta de qualidade dos dados de presen\c{c}a de alunos do fundamental no WASH, uma situa\c{c}\~ao que requeria medidas por parte da equipe de gest\~ao do projeto.






 Brudner (2022) complementa essa vis\~ao, mas com uma abordagem mais de neg\'ocios, trazendo as 22 vantagens e desvantagens na utiliza\c{c}\~ao de planilha eletr\^onicas.






Dentre as vantagens s\~ao citadas:







\begin{alineas}
\item as planilhas eletr\^onicas podem ser obtidas gratuitamente, a exemplo do LibreOffice e do Google Docs. Mesmo empresas como a Microsoft oferece acesso gratuito a algumas de suas vers\~oes.
\item as planilhas eletr\^onicas requerem pouco treinamento para seu uso b\'asico
\item as planilhas eletr\^onicas requerem pouco treinamento para seu uso b\'asico
\item planilhas eletr\^onicas s\~ao \textquotedbl customiz\'aveis\textquotedbl , ou seja, permitem ser configuradas facilmente para atender aquela necessidade espec\'{\i}fica do usu\'ario
\item as planilhas eletr\^onicas permite o trabalho colaborativo, quando muitos usu\'arios editam a planilha ao mesmo tempo. Esse possibilidade tamb\'em pode ser um problema, dado que pode resultar em re-trabalho quando um usu\'ario modifica os dados j\'a verificados por outro, por exemplo
\item as planilhas eletr\^onicas permitem uma manipula\c{c}\~ao e an\'alise de dados relativamente f\'acil, o que tamb\'em pode ser um problema, dado que \'e f\'acil remover parte dos dados, tornando-os n\~ao confi\'aveis
\item as planilhas s\~ao facilmente integr\'aveis com outras  ferramentas, mesmo com banco de dados especializados
\item as planilhas s\~ao facilmente integr\'aveis ao fluxo de trabalho de sua equipe, n\~ao requerendo custosas adapta\c{c}\~oes quando o sistema \'e menos flex\'{\i}vel
\item as planilhas geram facilmente documentos de grande apelo visual, principalmente no ambiente de neg\'ocios. H\'a uma grande quantidade de \textquotedbl templates\textquotedbl  que d\~ao bastante flexibilidade para a apresenta\c{c}\~ao dos resultados.
\end{alineas}

 Brudner (2022) tamb\'em aponta as disvantagens das planilhas eletr\^onicas:







\begin{alineas}
\item embora f\'aceis de usar, as planilhas s\~ao desajeitadas, principalmente quando \'e preciso manipular grandes quantidades de dados. O usu\'ario se ver\'a percorrendo (scrolling) centenas ou at\'e milhares de c\'elulas para poder encontrar seus dados, mesmo quando tem ferramentas de filtros dispon\'{\i}veis.
\item as planilhas eletr\^onicas n\~ao s\~ao seguras, dado que n\~ao tem sistemas de autentica\c{c}\~ao (login). Uma vez distribu\'{\i}das colocam em risco a privacidade das pessoas ali registradas (no caso de registros de presen\c{c}a), uma fragilidade frente aos requisitos da Lei Geral de Prote\c{c}\~ao de Dados, por exemplo.
\item a facilidade com que as planilhas eletr\^onicas podem ser utilizadas de forma colaborativa cria um outro problema: \'e dif\'{\i}cil dizer quem editou os dados pela \'ultima vez. Isso prejudica a rastreabilidade dos erros, prejudicando sua corre\c{c}\~ao. Quando muitas pessoas entram dados, como \'e o caso do WASH, \'e comum um usu\'ario inadvertidamente introduzir erros em cima do trabalho de outro, os quais depois ser\~ao muito dif\'{\i}ceis de encontrar.
\item as planilhas eletr\^onicas criam v\'arias vers\~oes da mesma \textquotedbl verdade\textquotedbl   (Brudner, 2022), mesmo que todos os usu\'arios de dados partam da mesma fonte de dados inicial. Isso ocorre porque \'e comum os usu\'arios salvarem suas pr\'oprias vers\~oes da planilha, criando um problema de concorr\^encia de atualiza\c{c}\~oes.
\item da mesma forma como  FULLER (2011),  Brudner (2022) tamb\'em aponta a inevitabilidade de erros introduzidos pelos v\'arios usu\'arios.
\item muito embora as planilhas permitam obter relat\'orios rapidamente para estruturas simples, \`a medida que as estruturas v\~ao ficando mais complexas, torna-se cada vez mais dif\'{\i}cil gerar novos relat\'orios
\item o fato das planilhas serem \textquotedbl customiz\'aveis\textquotedbl  e independentes de uma equipe de suporte tamb\'em significa que o pr\'oprio usu\'ario tem que gerar seus gr\'aficos, o que consume bastante tempo e pode ser bastante frustante
\item al\'em da falta de seguran\c{c}a em termos de expor a privacidade das pessoas registradas, as planilhas s\~ao particularmente propensas a perder dados, seja por erros de opera\c{c}\~ao ou por problemas com os sistemas de armazenamento, dado que as planilhas n\~ao tem sistemas robustos de \textquotedbl back-up\textquotedbl 
\item \`a medida que o seu \textquotedbl neg\'ocio\textquotedbl  se amplia e os requisitos de tratamento de dados v\~ao se tornando mais complexos, \'e natural que sistemas especializados sejam necess\'arios, situa\c{c}\~ao que nem sempre permite a integra\c{c}\~ao dos dados antigos, presentes na planilha eletr\^onica
\item as planilhas eletr\^onicas n\~ao podem ser integradas a aplicativos mobile, dificultando a ubiquicidade
\end{alineas}

\chapter[MATERIAIS E M\'ETODOS]{MATERIAIS E M\'ETODOS}\label{MATERIAIS E M\'ETODOS}
Como j\'a descrito na fundamenta\c{c}\~ao te\'orica, o termo \textquotedbl m\'etodo\textquotedbl  vem do \'etimo latino \textquotedbl methodus\textquotedbl   (FREITAS, 2019), que significa \textquotedbl caminho\textquotedbl .






Assim, o m\'etodo aplicado na pesquisa pode ser entendido como o caminho percorrido para chegar nos resultados. Na ci\^encia \'e muito importante que o caminho percorrido seja bem delineado, para que outros possam tentar percorr\^e-lo tamb\'em, verificando a reprodutibilidade dos resultados obtidos pelos que por ali j\'a passaram.






\'E neste cap\'{\i}tulo de \textquotedbl Materiais e M\'etodos\textquotedbl  que ser\~ao descritos os v\'arios caminhos percorridos at\'e os resultados.






Dizemos que s\~ao \textquotedbl v\'arios caminhos\textquotedbl  porque o presente trabalho requereu o desenvolvimento de m\'etodos em variadas \'areas do conhecimento, desde estrutura de dados at\'e a historiografia, por exemplo.






Na \textquotedbl Fundamenta\c{c}\~ao Te\'orica\textquotedbl  buscamos preparar terreno para o presente cap\'{\i}tulo, descrevendo l\'a uma parcela mais ampla do universo de m\'etodos dispon\'{\i}veis para o trabalho, dentre os quais foram escolhidos os que passamos a descrever a partir de agora.






Em outras palavras, daqui para a frente este cap\'{\i}tulo passa a descrever os m\'etodos efetivamente empregados para obter os resultados que ser\~ao descritos no pr\'oximo cap\'{\i}tulo.






\section[Caminho para constru\c{c}\~ao da narrativa hist\'orica]{Caminho para constru\c{c}\~ao da narrativa hist\'orica}\label{Caminho para constru\c{c}\~ao da narrativa hist\'orica}
Na Fundamenta\c{c}\~ao Te\'orica foi feita uma revis\~ao da evolu\c{c}\~ao dos m\'etodos historiogr\'aficos desde de a Escola Prussiana at\'e os dias de hoje.






Por mais despretenciosa que seja, tal revis\~ao serve para que no presente cap\'{\i}tulo seja poss\'{\i}vel encontrar o locus metodol\'ogico do m\'etodo efetivamente empregado neste trabalho para a constru\c{c}\~ao da narrativa hist\'orica do Projeto WASH.






\subsection[M\'etodo de Historiografia Utilizado ]{M\'etodo de Historiografia Utilizado }\label{M\'etodo de Historiografia Utilizado }
Para estabelecer a hist\'oria do WASH foi preciso fazer um levantamento de documenta\c{c}\~ao da hist\'oria do GESAC, do OLPC, et da blablabla






\subsection[Acervo utilizado para o levantamento hist\'orico]{Acervo utilizado para o levantamento hist\'orico}\label{Acervo utilizado para o levantamento hist\'orico}
Aqui ser\~ao descritos os elementos do acervo utilizado para o levantamento hist\'orico.






\section[Caminho para a An\'alise do M\'etodo do WASH]{Caminho para a An\'alise do M\'etodo do WASH}\label{Caminho para a An\'alise do M\'etodo do WASH}
Aqui s\~ao descritos os materias e m\'etodos para a an\'alise do m\'etodo do WASH






\subsection[M\'etodo de Caracteriza\c{c}\~ao do M\'etodo do WASH]{M\'etodo de Caracteriza\c{c}\~ao do M\'etodo do WASH}\label{M\'etodo de Caracteriza\c{c}\~ao do M\'etodo do WASH}
O m\'etodo do WASH foi desenvolvido anteriormente ao in\'{\i}cio desta pesquisa, n\~ao fazendo parte do escopo dos resultados deste trabalho, muito embora tenhamos participado da concep\c{c}\~ao do WASH. O m\'etodo do WASH come\c{c}ou a ser desenvolvido em 2013 e teve seu documento de refer\^encia publicado em meados de 2018, ap\'os anos de aprimoramento, como Anexo \`a Portaria CTI 178/2018.






Considerando-se o tempo transcorrido desde \`a publica\c{c}\~ao de seu documento de refer\^encia, \'e razo\'avel esperar que o WASH praticado nos dias de hoje tenha incorporado caracter\'{\i}sticas que n\~ao estavam previstas no documento de refer\^encia original.






Assim, passa a ser imperioso que se fa\c{c}a uma caracteriza\c{c}\~ao do que de fato tem sido o WASH






Para que essa caracteriza\c{c}\~ao do m\'etodo do WASH seja poss\'{\i}vel, \'e preciso estabelecer um m\'etodo de caracteriza\c{c}\~ao.






Por mais metalingu\'{\i}stico que possa parecer, foi preciso estabelecer um m\'etodo de caracteriza\c{c}\~ao do m\'etodo do WASH. Para evitar confus\~oes de nomenclatura, esse \textquotedbl m\'etodo de caracteriza\c{c}\~ao do m\'etodo do WASH\textquotedbl  passar\'a a ser referido, daqui em diante, simplesmente, como \textquotedbl m\'etodo da pesquisa\textquotedbl .













\subsection[Materiais para caracteriza\c{c}\~ao do m\'etodo do WASH.]{Materiais para caracteriza\c{c}\~ao do m\'etodo do WASH.}\label{Materiais para caracteriza\c{c}\~ao do m\'etodo do WASH.}
Para caracterizar o m\'etodo do WASH foram empregadas as ferramentas tal e tal, bem com o acervo documental tal e tal.






\section[Caminho para a caracteriza\c{c}\~ao dos resultados do WASH]{Caminho para a caracteriza\c{c}\~ao dos resultados do WASH}\label{Caminho para a caracteriza\c{c}\~ao dos resultados do WASH}
\subsection[M\'etodo de Estrutura\c{c}\~ao e an\'alise dos dados]{M\'etodo de Estrutura\c{c}\~ao e an\'alise dos dados}\label{M\'etodo de Estrutura\c{c}\~ao e an\'alise dos dados}
Para que fosse poss\'{\i}vel fazer a estrutura\c{c}\~ao dos dados foi preciso criar uma plataforma, um modelo de dados, etc…






\subsection[M\'etodo de determina\c{c}\~ao do g\^enero dos participantes]{M\'etodo de determina\c{c}\~ao do g\^enero dos participantes}\label{M\'etodo de determina\c{c}\~ao do g\^enero dos participantes}
A quest\~ao de armazenagem de dados de g\^enero no WASH ainda n\~ao est\'a devidamente equacionada e esta situa\c{c}\~ao tem a ver com a forma como os dados eram armazenados no in\'{\i}cio do projeto.






\'E poss\'{\i}vel identificar v\'arios momentos na forma como o WASH armazenou seus dados ao longo de 9 anos. Logo no in\'{\i}cio do projeto, os dados de participantes eram coletados por meio de listas de presen\c{c}a, nas quais constavam inicialmente apenas o nome e a data do evento. Posteriormente novos dados foram sendo coletados, como o ano do nascimento da crian\c{c}a. Sempre houve uma vis\~ao minimalista no sentido dos dados que deveriam ser coletados, dado que tal coleta se dava no contexto do servi\c{c}o p\'ublico e n\~ao havia um mandato para coleta de dados cadastrais mais detalhados. Buscou-se sempre restringir a coleta de dados para os prop\'ositos do projeto, a saber: contabilizar o n\'umero de participantes, evitar a contabiliza\c{c}\~ao dupla de participantes, identificar os respons\'aveis, registrar autoriza\c{c}\~oes de uso de imagens, etc.






Assim, desde o in\'{\i}cio do projeto n\~ao havia a armazenagem do g\^enero de seus participantes.






Com o crescimento do projeto, come\c{c}ou a existir uma preocupa\c{c}\~ao sobre se o projeto era inclusivo, em termos de atendimento equ\^anime dos v\'arios perfis de g\^enero. Mas no momento em que essa defici\^encia de registro foi diagnosticada, o projeto j\'a contava com milhares de participa\c{c}\~oes. Isso exigiu a ado\c{c}\~ao de algum m\'etodo para tentar verificar se o atendimento era suficientemente equ\^anime, mesmo sem existirem registros cadastrais que indicassem o g\^enero dos participantes.






Criou-se um m\'etodo em que os indicadores de g\^enero do WASH s\~ao constru\'{\i}dos a partir de uma avalia\c{c}\~ao a posteriori dos primeiros nomes dos participantes, que s\~ao comparados com listas de nomes masculinos e de nomes femininos. Evidente que esta abordagem traz imprecis\~oes pela pr\'opria imprecis\~ao do conceito de nomes masculinos e nomes femininos. O m\'etodo n\~ao leva em conta a autodeclara\c{c}\~ao de g\^enero dos indiv\'{\i}duos participantes, simplesmente porque n\~ao foi solicitado aos participantes que se identificassem em termos de g\^enero. Esse cuidado tem uma l\'ogica: o WASH n\~ao \'e uma escola e n\~ao tem a obriga\c{c}\~ao, ou o direito, de fazer cadastros de participantes. Do ponto de vista do WASH n\~ao h\'a interesse em rotular peremptoriamente as pessoas como desse ou daquele g\^enero. Como o nome dos participantes \'e auto-declarat\'orio e n\~ao s\~ao solicitados documentos de registro civil (RG ou certid\~ao de nascimento), o respeito \`a imagem que o participante faz de si mesmo \'e garantido, porque suas declara\c{c}\~oes n\~ao s\~ao questionadas e n\~ao precisam ser verificadas com rela\c{c}\~ao a algum documento civil. Assim, se um participante optar por se identificar com um nome social, isso ser\'a respeitado. Se o participante optar por se identificar com o nome civil, isso tamb\'em ser\'a respeitado.






Dito isso, e reconhecida a defici\^encia associada \`a falta de coleta de dados auto-declarat\'orios de g\^enero, \'e preciso criar uma solu\c{c}\~ao que represente da melhor forma poss\'{\i}vel a distribui\c{c}\~ao de g\^enero dos atendidos. Optou-se por um m\'etodo que permitisse identificar desvios (v\'{\i}cios ou tend\^encias), de atendimento de um determinado g\^enero em detrimento do outro. Em outras palavras, sabe-se que a presen\c{c}a masculina em atividades STEAM \'e mais oportunizada, desprivilegiando a presen\c{c}a feminina. Portanto, ao WASH era preciso verificar, da melhor forma poss\'{\i}vel, se esse tipo de preconceito estava sendo reproduzido dentro do programa.






Foi a partir dessa necessidade, que o problema foi resolvido parcialmente, pela op\c{c}\~ao de usar o primeiro nome, comparado com listas dos ditos nomes masculinos/femininos, para determinar o g\^enero dos participantes.






Os dados apresentados, pelo menos ao que se refere a masculino e feminino, mostram que estes v\'{\i}cios e tend\^encias n\~ao est\~ao presentes no projeto, havendo um relativo equil\'{\i}brio entre o atendimento a homens e mulheres. Infelizmente, o m\'etodo utilizado n\~ao permite identificar a qualidade do atendimento do projeto junto \`a comunidade LGBTQI+, porque, como j\'a dito, esses dados n\~ao foram coletados ao longo de sua hist\'oria.






Do ponto de vista legal, o preparo do projeto para armazenar dados de g\^enero esbarra na falta de atribui\c{c}\~ao legal para armazenagem de cadastro de pessoas, o que poderia ser resolvido pela delega\c{c}\~ao por autoridade superior por meio de portaria.






\chapter[RESULTADOS E DISCUSS\~OES]{RESULTADOS E DISCUSS\~OES}\label{RESULTADOS E DISCUSS\~OES}
\'E neste ponto do texto que alcan\c{c}amos o final dos caminhos percorridos (m\'etodos) descritos no cap\'{\i}tulo anterior, criando as condi\c{c}\~oes para que os resultados obtidos sejam apresentados.






Portanto, \'e neste ponto que os resultados obtidos ser\~ao descritos e, concomitantemente, analisados para, posteriormente, serem discutidos.






Os resultados est\~ao organizados, inicialmente, em tr\^es se\c{c}\~oes separadas, relacionadas \`as tr\^es dimens\~oes do Projeto WASH em an\'alise neste texto:







\begin{alineas}
\item sua trajet\'oria (hist\'oria)
\item seus m\'etodo
\item e seus resultados
\end{alineas}

Uma quarta se\c{c}\~ao apresentar\'a uma s\'{\i}ntese que integrar\'a os achados das 3 dimens\~oes.






\section[Narrativas contru\'{\i}das a partir do m\'etodo historiogr\'afico]{Narrativas contru\'{\i}das a partir do m\'etodo historiogr\'afico}\label{Narrativas contru\'{\i}das a partir do m\'etodo historiogr\'afico}
Aqui s\~ao apresentadas as narrativas constru\'{\i}das a partir da aplica\c{c}\~ao do m\'etodo historiogr\'afico.






\subsection[Narrativa do GESAC]{Narrativa do GESAC}\label{Narrativa do GESAC}
Aqui fazemos a narrativa hist\'orica do GESAC com base no m\'etodo historiogr\'afico empregado sobre os acervos de Elaine.






\subsection[Narrativa do OLPC]{Narrativa do OLPC}\label{Narrativa do OLPC}
Aqui fazemos a narrativa hist\'orica do OLPC com base no m\'etodo historiogr\'afico empregado sobre os acervos de Victor.






\subsection[Narrativa sobre a Papert e Brasil (Afira)]{Narrativa sobre a Papert e Brasil (Afira)}\label{Narrativa sobre a Papert e Brasil (Afira)}
Aqui fazemos a narrativa hist\'orica do Afira com base no m\'etodo historiogr\'afico empregado na entrevista de Afira.






\section[Caracterizacao do M\'etodo do WASH]{Caracterizacao do M\'etodo do WASH}\label{Caracterizacao do M\'etodo do WASH}
Aqui s\~ao descritos os resultados da aplica\c{c}\~ao do m\'etodo de caracteriza\c{c}\~ao do m\'etodo do WASH.






\section[Caracteriza\c{c}\~ao dos Resultados do WASH]{Caracteriza\c{c}\~ao dos Resultados do WASH}\label{Caracteriza\c{c}\~ao dos Resultados do WASH}
Como visto na se\c{c}\~ao da Fundamenta\c{c}\~ao Te\'orica e em Materiais e M\'etodos, o objetivo da coleta de dados \'e produzir indicadores que, analisados e interpretados, tragam informa\c{c}\~oes sobre o projeto, complementando as demais dimens\~oes deste estudo de forma quantitativa, contribuindo para a caracteriza\c{c}\~ao do Projeto WASH.






S\~ao exemplos de indicadores a serem apresentados nesta se\c{c}\~ao:







\begin{alineas}
\item n\'umero de pessoas atendidas
\item evolu\c{c}\~ao temporal do n\'umero de pessoas atendidas
\item n\'umero de bolsistas
\item n\'umero de relat\'orios
\item distribui\c{c}\~ao de temas abordados em relat\'orios
\item n\'umero de oficinas realizadas
\item distribui\c{c}\~ao et\'aria dos participantes em oficina
\item distribui\c{c}\~ao de temas nas oficinas
\item distribui\c{c}\~ao das atividades nas oficinas
\item quantidade de cidades atendidas
\item participantes mais ass\'{\i}duos
\item muitos outros.
\end{alineas}

Importante registrar que os dados a seguir foram obtidos a partir da contribui\c{c}\~ao de v\'arios colaboradores do WASH, com nossa participa\c{c}\~ao ativa na especifica\c{c}\~ao dos sistemas de coleta de dados, a exemplo da Plataforma Platu\'osh, da qual a presente candidata \'e co-autora  (MAMMANA et al., 2022).






Tr\^es fontes principais de dados foram utilizadas para gerar os indicadores:







\begin{alineas}
\item Plataforma Platu\'osh: voltada inicialmente para o registro da quantidade de eventos realizados e n\'umero de participantes em cada evento, bem como dos testemunhos documentais e fotogr\'aficos dessa realiza\c{c}\~ao e participa\c{c}\~ao. Posteriormente a Platu\'osh foi sendo adaptada para incluir dados gerenciais (vincula\c{c}\~oes e afilia\c{c}\~oes institucionais), bem como registro de acervo (documentos gerados ao longo do projeto). Como esta plataforma est\'a em plena opera\c{c}\~ao, com dados sendo adicionados diariamente, foi preciso escolher um recorte temporal para a presente an\'alise. Desta forma, os dados aqui presentes referem-se ao per\'{\i}odo de setembro de 2013 a 26 de agosto de 2022.
\item Plataforma de Planejamento Financeiro do Projeto WASH: plataforma de acompanhamento das concess\~oes de bolsas, suas validades, documentos de outorgas, planos de trabalho. Trata-se de uma ferramenta de compliance e presta\c{c}\~ao de contas do projeto, mas que tamb\'em pode ser usada para a caracteriza\c{c}\~ao do mesmo, pela abrang\^encia dos dados nela contida.
\item Planilhas digitais de dados constru\'{\i}das pela candidata manualmente (sem uso do sistema de entrada de dados automatizado): instrumento criado separadamente pela candidata para viabilizar a verifica\c{c}\~ao dos dados das demais plataformas, uma vez que foram identificadas algumas fragilidades nas demais fontes de informa\c{c}\~oes.
\end{alineas}

\subsection[Amostragem do p\'ublico atendido]{Amostragem do p\'ublico atendido}\label{Amostragem do p\'ublico atendido}
Antes de prosseguir \'e preciso explicitar que uma parte dos indicadores utilizados neste projeto, principalmente aqueles referentes \`a Platu\'osh, foram obtidos por amostragem, uma vez que o Projeto WASH tem limita\c{c}\~oes na coleta de dados cadastrais de participantes.






Estas limita\c{c}\~oes foram discutidas no cap\'{\i}tulo de Materiais e M\'etodos e est\~ao relacionadas \`a falta de atribui\c{c}\~ao legal para que o projeto colete dados cadastrais de seus participantes, tornando facultativo o compartilhamento destas informa\c{c}\~oes pelas entidades parceiras (executoras, promotoras e respons\'aveis).






Portanto, tem ocorrido recorrentemente a falta de registro, nos arquivos do WASH, dos participantes que est\~ao sob cust\'odia da Entidade Respons\'avel. Esta situa\c{c}\~ao se intensificou a partir da edi\c{c}\~ao da Lei Geral de Prote\c{c}\~ao de Dados [XXX]. Tudo indica que as responsabilidades impostas aos gestores escolares por essa lei t\^em aumentado a  resist\^encia, por parte dos parceiros, em compartilhar informa\c{c}\~oes de participa\c{c}\~ao dos estudantes, o que \'e perfeitamente compreens\'{\i}vel.






Assim, no que tange \`a caracteriza\c{c}\~ao do p\'ublico participante, \'e preciso usar o m\'etodo de amostragem.






Antecipadamente, para evitar que o leitor tenha uma vis\~ao inicial equivocada sobre a representatividade estat\'{\i}stica dos dados aqui apresentados, \'e preciso fazer uma ressalva, \`a qual retornaremos in\'umeras vezes neste texto.






Ocorre que a redu\c{c}\~ao recente do fornecimento de dados por parte dos parceiros resultou em uma mudan\c{c}a no perfil de amostragem, dificultando a compara\c{c}\~ao dos indicadores anuais do projeto. Esta quest\~ao ser\'a tratada mais adiante.






Reconhecer a ocorr\^encia dessa mudan\c{c}a de perfil da coleta de dados ao longo dos anos de exist\^encia do projeto \'e importante para que se tenha a real no\c{c}\~ao da validade dos valores globais e das m\'edias apresentadas deste ponto em diante.






No que se refere \`a distribui\c{c}\~ao de participantes por sexo, foi aplicado um m\'etodo de amostragem especial, principalmente para os casos em que o cadastro era incompleto e n\~ao havia informa\c{c}\~ao individualizada sobre sexo. O Cap\'{\i}tulo de Materias e M\'etodos detalha como esta identifica\c{c}\~ao \`a posteriore do sexo dos participantes foi feita mas, sumariamente, podemos dizer que ela se deu por meio da identifica\c{c}\~ao do g\^enero do primeiro nome de cada particip1ante, por meio da compara\c{c}\~ao desse primeiro nome com listas de nomes extensivas, dividos por g\^enero.






Em termos de recorte temporal, os dados aqui apresentados referem-se ao per\'{\i}odo de 2013 (quando o WASH foi criado) at\'e o dia 26 de agosto de 2022, quando foi feita uma c\'opia da base de dados visando a an\'alise.






Para conhecer o n\'umero de cadastros na base de dados \'e preciso fazer uma consulta sobre a tabela \textquotedbl participantes2\textquotedbl , por meio da linguagem SQL.






O uso da linguagem SQL escapa ao escopo de conhecimento desta candidata. No entanto, como as consultas foram baseadas em nossas solicita\c{c}\~oes \`a equipe de TI, pudemos propor que tais consultas fossem \textquotedbl traduzidas\textquotedbl  para o portugu\^es.






Para saber o n\'umero de cadastrados na tabela participantes2, que traz todos os cadastrados na plataforma Platu\'osh, utilizamos o equivalente \`a seguinte consulta:







\noindent\begin{center}\mbox{\centering\fbox{\centering\par\parbox{0.7\linewidth}{\small\textit{selecione a contagem de nome\_de\_participantes da tabela participantes2;}\normalize}}}\end{center}









Esta consulta, segundo os colabores de TI, \'e feita por meio do comando SQL a seguir:







\noindent\begin{center}\mbox{\centering\fbox{\centering\par\parbox{0.7\linewidth}{\small\textit{select count(nome\_participante) from participantes2;}\normalize}}}\end{center}









O resultado obtido com esta consulta, para os dados congelados em 26 de agosto de 2022, foi de 3312 participantes.






Mas, por inspe\c{c}\~ao da lista de participantes, ainda segundo os colaboradores da TI, \'e poss\'{\i}vel identificar que h\'a cadastros repetidos, resultantes, provavelmente, de erros de digita\c{c}\~ao.






Felizmente, a linguagem SQL permite excluir os cadastros repetidos, tarefa que foi delegada \`a prestimosa equipe de TI do WASH. Para isso, foi constru\'{\i}da uma nova consulta, que pode ser traduzida para o portugu\^es como segue:







\noindent\begin{center}\mbox{\centering\fbox{\centering\par\parbox{0.7\linewidth}{\small\textit{selecione a contagem de nomes\_de\_participantes distintos, da tabela participantes2;}\normalize}}}\end{center}









Esta consulta, no SQL original \'e feita pelo comando a seguir:







\noindent\begin{center}\mbox{\centering\fbox{\centering\par\parbox{0.7\linewidth}{\small\textit{select count(distinct nome\_participante) from participantes2;}\normalize}}}\end{center}









Assim, pelo emprego da consulta acima, foi poss\'{\i}vel identificar um n\'umero de participantes com nomes distintos, com um total de 3265 pessoas.






O conjunto de 47 pessoas, que \'e a diferen\c{c}a entre o n\'umero com nomes incluindo as repeti\c{c}\~oes (3312 participantes) e o n\'umero sem nomes repetidos (3265 participantes) pode conter as seguintes duas situa\c{c}\~oes:







\begin{alineas}
\item cadastros repetidos referentes \`a mesma pessoa
\item hom\^onimos
\end{alineas}

Mas os bancos relacionais nos permitem conhecer um pouco melhor o motivo pelo qual aparecem nomes repetidos. Para isso, \'e poss\'{\i}vel utilizar o ano de nascimento, que tamb\'em \'e registrado no cadastro.






A probabilidade de dois participantes terem exatamente o mesmo nome e terem nascido no mesmo ano \'e substancialmente menor do que a simples ocorr\^encia de hom\^onimos. Assim, foi solicitado \`a TI que fossem considerados como cadastros repetidos aqueles que t\^em nomes repetidos e anos repetidos. A consulta \`a base de dados, em portugu\^es, ficou assim:







\noindent\begin{center}\mbox{\centering\fbox{\centering\par\parbox{0.7\linewidth}{\small\textit{selecione a contagem de participantes agrupados por nome\_do\_participante e ano\_do\_nascimento;}\normalize}}}\end{center}


Este comando, na linguagem SQL, novamente segundo os colaboradores da TI, pode ser expresso como segue:







\noindent\begin{center}\mbox{\centering\fbox{\centering\par\parbox{0.7\linewidth}{\small\textit{select count(*) as conta from participantes2 group by nome\_participante, ano\_nascimento;}\normalize}}}\end{center}


O resultado desta consulta indica a exist\^encia de 3306 participantes com nome e ano de nascimento simultaneamente diferentes. No entanto, uma an\'alise mais cuidadosa, conduzida pela equipe de TI a pedido desta candidata, indica que n\~ao h\'a a ocorr\^encia de hom\^onimos entre os 3312 registros existentes na tabela  participantes2 pelos motivos que ser\~ao expressos a seguir.






A afirma\c{c}\~ao de que os 47 cadastros com nomes repetidos n\~ao se referem a hom\^onimos se sustenta nas seguintes evid\^encias:







\begin{alineas}
\item as repeti\c{c}\~oes de nomes identificadas na base s\~ao bastante incomuns, incluindo formas estrangeiras de nomes pr\'oprios combinadas, o que afasta a hip\'otese de hom\^onimos. Este fato, por si s\'o seria um indicativo de que provavelmente estas repeti\c{c}\~oes se referem \`a mesma pessoa, havendo um indevido cadastramento duplicado para 47 pessoas.
\item Quando o nome repetido \'e confrontado com o ano de nascimento, percebe-se que os dois registros com mesmo nome se diferem pela aus\^encia do dado do ano de nascimento para um dos registros com nome repetido, ou mesmo para os dois, situa\c{c}\~ao que inviabiliza o crit\'erio de agrupamento de nome e ano de nascimento, como forma de identificar hom\^onimos.
\end{alineas}

Desta forma, \'e poss\'{\i}vel afirmar que o n\'umero de registros v\'alidos no WASH \'e de 3265 participantes e h\'a um erro de cerca de 1,4\% nos registros totais (47 repeti\c{c}\~oes em 94 registros). Este n\'umero de erros \'e relativamente pequeno para o universo de participantes.






Um outro aspecto que precisa ser bastante enfatizado \'e que o n\'umero efetivo de participantes no WASH \'e provavelmente substancialmente maior do que 3265 registrados, superando o que est\'a efetivamente registrado na plataforma.






Para sustentar esta afirma\c{c}\~ao, \'e poss\'{\i}vel considerar que muitas oficinas do projeto foram realizadas em recintos sem controle de entrada, impedindo que um cadastro individualizado fosse feito.






Mas esta afirma\c{c}\~ao n\~ao teria validade se n\~ao fosse poss\'{\i}vel apresentar evid\^encias de eventos com essas caracter\'{\i}sticas, mostrando que o n\'umero de participantes, nos casos exemplificados, foi maior do que o de efetivamente cadastrados.






Desta forma, passamos a apresentar exemplos de eventos em que tal situa\c{c}\~ao ocorreu, adicionando evid\^encias fotogr\'aficas de que o n\'umero de registrados na plataforma n\~ao reflete o alcance real do projeto. Por uma raz\~ao de espa\c{c}o, limitamos essa exposi\c{c}\~ao a 7 casos, como segue:







\begin{alineas}
\item evento de grande porte no CTI Renato Archer realizado em 11 de abril de 2015, quando centenas de crian\c{c}as foram beneficiadas por uma apresenta\c{c}\~ao do Ci\^encia em Show, trupe de artistas formados em f\'{\i}sica. Os registros oficiais indicam a presen\c{c}a de 5 pessoas, o que n\~ao se coaduna com os registro fotogr\'aficos, que indicam um p\'ublico de 20 a 40 vezes maior.
\item comemora\c{c}\~ao do dia das crian\c{c}as realizada em 3 de outubro de 2015, com atividades musicais e culturais. Os registros da plataforma, neste dia, apontam para a participa\c{c}\~ao de 9 participantes, mas os registros fotogr\'aficos do evento apontam para uma presen\c{c}a muito superior.
\item Evento de confraterniza\c{c}\~ao de Natal realizado no CTI Renato Archer, com palestras e outras atividades l\'udicas, realizado em 19 de dezembreo de 2015, com registro oficial de 8 participantes, mas com registros fotogr\'aficos que indicam a participa\c{c}\~ao de substancialmente maior de crian\c{c}as.
\item Evento Greenk, patrocinado pelo MCTI, que aconteceu no Expo Center Anhembi em S\~ao Paulo, na semana de 27 de maio de 2018. O porte do evento e n\'umero de dias de realiza\c{c}\~ao indicam uma quantidade substancialmente maior do que o oficialmente registrado: 13 pessoas. Essa discrep\^ancia se deu porque o tipo de evento n\~ao permitia o cadastro de p\'ublico, ficando os registros restritos aos bolsistas multiplicadores, bem com aos demais respons\'aveis.
\item Em 23 de junho de 2018 o Programa WASH promoveu uma visita ao Museu Aberto de Astronomia, em Campinas. Os registros oficiais n\~ao trazem o n\'umero de participantes, mas os registros fotogr\'aficos indicam a presen\c{c}a de v\'arias dezenas de crian\c{c}as.
\item evento em pra\c{c}a p\'ublica realizado na cidade de Prado Ferreira (PR), em 31 de maio de 2019. No evento em quest\~ao foi poss\'{\i}vel estimar uma presen\c{c}a de v\'arias centenas de pessoas, com a pra\c{c}a tomada pelo p\'ublico. O evento envolveu o lan\c{c}amento do Programa WASH na cidade, no \^ambito do Programa Profiss\~ao 4.0, criado em lei municipal, cuja cria\c{c}\~ao teve orienta\c{c}\~ao ativa desta candidata [XXX].
\item Evento Dia da Fam\'{\i}lia na Escola, realizado na EMEF Milton Pereira Costa, em Guarulhos, no dia 27 de novembro de 2021, com a presen\c{c}a de um dos membros do Ci\^encia em Show. O p\'ublico estimado est\'a em cerca de 2 centenas, mas n\~ao houve registro individualizado pelo aspecto amplo do evento.
\end{alineas}



\captionsetup{format=plain}
\begin{figure}[max size={\textwidth}{\textheight}]

\centering


\begin{minipage}[b]{0.4\linewidth}
        \centering
                \includegraphics[width=1.0\linewidth]{../../imagens/Evento-Ciencia-Em-Show.jpg}
                \caption{Evento de demonstra\c{c}\~oes cient\'{i}ficas realizado no \^ambito do WASH em 11 de abril de 2015, com a participa\c{c}\~ao do Ci\^encia em Show, trupe de artistas formados em f\'{i}sica que promoviam a ci\^encia na televis\~ao. O car\'ater amplo do evento n\~ao permitiu controlar a presen\c{c}a de participantes que pode ser estimada em perto de duas centenas de crian\c{c}as.}
                \label{5340059e38852932c32c5ce8624858fef8a1f3f0}
\end{minipage}%
\hspace{0.5cm}
\begin{minipage}[b]{0.4\linewidth}
        \centering
                \includegraphics[width=1.0\linewidth]{../../imagens/dia-das-criancas-2022-10-03-menor.JPG}
                \caption{Evento de comemora\c{c}\~ao do dia das crian\c{c}as, com atividades musicais e culturais. Os registros da plataforma apontam para 9 participantes, mas os registros fotogr\'aficos indicam uma presen\c{c}a muito maior.}
                \label{31e991b69f3aba382518bb571a4e69a720fa8ccb}
\end{minipage}
\hspace{0.5cm}
\begin{minipage}[b]{0.4\linewidth}
        \centering
                \includegraphics[width=1.0\linewidth]{../../imagens/confraternizacao-de-natal-menor.jpeg}
                \caption{Evento de Natal realizado no CTI Renato Archer em 19 de dezembro de 2015. O evento incluiu uma variada gama de atividades l\'udicas e educacionais. Muito embora o registro oficial indique a participa\c{c}\~ao de 8 pessoas, as fotos mostram que a quantidade foi muito superior.}
                \label{afa1b2acd7f4c590f25d9821e48e82568bd28cf2}
\end{minipage}%
\hspace{0.5cm}
\begin{minipage}[b]{0.4\linewidth}
        \centering
                \includegraphics[width=1.0\linewidth]{../../imagens/Evento-Greenk-2018-05-27.jpg}
                \caption{Evento Greenk, patrocinado pelo MCTI no Expo Center Anhembi em 27 de maio de 2018, que contou com oficinas do WASH. Neste tipo de evento \'e dif\'{i}cil realizar o cadastro nominal de participantes pela amplitude do mesmo. O p\'ublico beneficiado pode ser estimado em algumas centenas de crian\c{c}as.}
                \label{04766cbd557212dbb84969d429be542b364bf87a}
\end{minipage}
\hspace{0.5cm}
\begin{minipage}[b]{0.4\linewidth}
        \centering
                \includegraphics[width=1.0\linewidth]{../../imagens/MAAS.jpg}
                \caption{Evento no Museu Aberto de Astronomia, promovido pelo WASH. Os registros oficiais n\~ao indicam o n\'umero de participantes, mas os registros fotogr\'aficos mostram a participa\c{c}\~ao de dezenas de crian\c{c}as.}
                \label{b5542509570e0bcdbabe4949f7fef484141b805e}
\end{minipage}%
\hspace{0.5cm}
\begin{minipage}[b]{0.4\linewidth}
        \centering
                \includegraphics[width=1.0\linewidth]{../../imagens/Ciencia-em-Show-prado.jpeg}
                \caption{Evento de demonstra\c{c}\~oes cient\'{i}ficas na pra\c{c}a de Prado Ferreira, ocorrido em 31 de maio de 2019, cidade onde o WASH realizou dezenas de oficinas naquele per\'{i}odo. O evento foi promovido pelo WASH no lan\c{c}amento do Programa Profiss\~ao 4.0 na cidade e contou com a participa\c{c}\~ao do Ci\^encia em Show, trupe de artistas formados em f\'{i}sica com grande presen\c{c}a na m\'{i}dia televisiva.}
                \label{d76eb1e41d3d1e1394e18aafa3beb2bc3ad09471}
\end{minipage}
\hspace{0.5cm}
\end{figure}












\captionsetup{format=plain}
\begin{figure}[max size={\textwidth}{\textheight}]

\centering


\begin{minipage}[b]{0.4\linewidth}
        \centering
                \includegraphics[width=1.0\linewidth]{../../imagens/Ciencia-Prado-publico.jpeg}
                \caption{P\'ublico no evento do Ci\^encia em Show}
                \label{7b56c85cc265fcbf261a3bc788ae1e4619b70725}
\end{minipage}%
\hspace{0.5cm}
\end{figure}



Com os registros at\'e aqui apresentados, n\~ao exaustivos, uma vez que foram selecionados apenas 7 exemplos num universo de milhares de eventos, buscamos sustentar a afirma\c{c}\~ao de que os 3265 cadastros de participantes representa uma amostra modesta de todos os benefici\'arios do Projeto WASH.






N\~ao obstante esse car\'ater amostral, ou seja, incompleto em termos de registros individuais dos participantes, sustentamos que esses dados amostrais s\~ao imprescind\'{\i}veis para extrair importantes informa\c{c}\~oes sobre o projeto. Entre elas est\'a o seu crescimento org\^anico e o impacto da pandemia, por exemplo, o que pode ser verificado no gr\'afico a seguir.






\subsection[Evolu\c{c}\~ao temporal do n\'umero de participa\c{c}\~oes]{Evolu\c{c}\~ao temporal do n\'umero de participa\c{c}\~oes}\label{Evolu\c{c}\~ao temporal do n\'umero de participa\c{c}\~oes}
Na se\c{c}\~ao anterior foi mostrado que os dados de participantes presentes na Platu\'osh s\~ao uma amostra do total de participantes, uma vez que h\'a eventos em que n\~ao foi poss\'{\i}vel cadastrar todos participantes, a exemplo dos eventos que ocorrem em ambientes abertos (pra\c{c}as p\'ublicas, exposi\c{c}\~oes, etc.).






Foi comentado, tamb\'em, que em muitos casos os parceiros preferem n\~ao compartilhar dados cadastrais e de participa\c{c}\~ao de estudantes por raz\~oes de seguran\c{c}a de dados, tendo sido observado um crescimento nessa tend\^encia ao longo dos anos do projeto, principalmente a partir da edi\c{c}\~ao da Lei Geral de Prote\c{c}\~ao de Dados. Tal evolu\c{c}\~ao tem mudando o perfil de coleta de dados amostrais por parte do WASH.






Mesmo com essas dificuldades, os dados amostrais s\~ao importante para identificar tend\^encias do projeto, a exemplo da evolu\c{c}\~ao anual no n\'umero de participa\c{c}\~oes, mostrada abaixo:








\captionsetup{format=plain}
\begin{figure}[max size={\textwidth}{\textheight}]

\centering


\begin{minipage}[b]{0.4\linewidth}
        \centering
                \includegraphics[width=1.0\linewidth]{../../imagens/output-participantes.jpeg}
                \caption{Evolu\c{c}\~ao temporal do n\'umero de participa\c{c}\~oes ao longo dos 10 anos de exist\^encia do Projeto WASH.}
                \label{19699bcc5ab8317274249d6743d62534dbfb95fa}
\end{minipage}%
\hspace{0.5cm}
\begin{minipage}[b]{0.4\linewidth}
        \centering
                \includegraphics[width=1.0\linewidth]{../../imagens/output-participantes2.jpeg}
                \caption{Evolu\c{c}\~ao anual do n\'umero de participantes individuais.}
                \label{e01eb26f443a577db4a1d382417f8c1bb57ee435}
\end{minipage}
\hspace{0.5cm}
\begin{minipage}[b]{0.4\linewidth}
        \centering
                \includegraphics[width=1.0\linewidth]{../../imagens/output-media-participacoes.jpeg}
                \caption{Evolu\c{c}\~ao anual da m\'edia de participa\c{c}\~oes por participante.}
                \label{a8f2d72073b88290f9b8731b144383d2f7c4dc4b}
\end{minipage}%
\hspace{0.5cm}
\end{figure}



\'E importante atentar para uma sutileza: a diferen\c{c}a entre \textquotedbl n\'umero de participantes\textquotedbl  e \textquotedbl n\'umero de participa\c{c}\~oes\textquotedbl .






N\'umero de participantes significa o n\'umero de indiv\'{\i}duos que partiparam de eventos naquele ano, contabilizados uma vez s\'o, mesmo que tenham participado em mais de um evento no mesmo ano.






N\'umero de participa\c{c}\~oes significa o n\'umero de vezes que participantes frequentaram eventos do WASH naquele ano, mesmo que seja contabilizada a mesma pessoa duas ou mais vezes.






Observada essa diferen\c{c}a, o gr\'afico abaixo traz o n\'umero de participantes por ano.






Agora podemos calcular a m\'edia de participa\c{c}\~oes por participante, dividindo, um a um, os dados da evolu\c{c}\~ao anual das participa\c{c}\~oes pela evolu\c{c}\~ao anual dos participantes, como segue.






\subsection[Distribui\c{c}\~ao de partipantes por sexo]{Distribui\c{c}\~ao de partipantes por sexo}\label{Distribui\c{c}\~ao de partipantes por sexo}
Sabe-se que as pessoas do g\^enero feminino s\~ao particularmente desprivilegiadas quano o tema \'e igualdade de acesso \`as disciplinas de Science, Technology, Engineering  (Kijima et al., 2021).






Por esta raz\~ao, \'e de particular interesse para este trabalho analisar o equil\'{\i}brio no atendimento a participantes do sexo masculino e do sexo feminino.






Mas esta an\'alise, como antecipado no cap\'{\i}tulo de Materiais e M\'etodos, n\~ao foi planejada no in\'{\i}cio do projeto, uma vez que n\~ao havia, h\'a 10 anos atr\'as, a ambi\c{c}\~ao de crescimento que se alcan\c{c}ou.






Esta situa\c{c}\~ao impactou tamb\'em a capacidade do projeto de fazer uma an\'alise mais inclusiva no sentido da identifica\c{c}\~ao de g\^enero dos participantes.






Portanto, na aus\^encia de informa\c{c}\~oes cadastrais mais detalhadas no que se refere \`a auto-declara\c{c}\~ao de g\^enero dos participantes nos primeiros 5 anos do projeto, bem como em face \`a recente resist\^encia de parceiros em fornecer dados, decorrente da LGPD, foi preciso desenvolver um m\'etodo de \textquotedbl estimativa\textquotedbl  do g\^enero dos participantes com base no primeiro nome dos mesmos.






Este m\'etodo n\~ao tem a finalidade de atribuir um g\^enero aos participantes para que fique registrado em seus cadastros. O m\'etodo \'e anonimizado de forma que a contabiliza\c{c}\~ao de um participante num g\^enero, ou no outro, se d\'a num contexto n\~ao personalizado.






De forma sum\'aria pode-se descrever o m\'etodo como uma verifica\c{c}\~ao se o primeiro nome do participante est\'a numa lista extensiva de nomes \textquotedbl considerados masculinos\textquotedbl , situa\c{c}\~ao em qque, de forma anonimizada, um contador de participantes masculinos \'e incrementado. Caso o primeiro nome do participante esteja numa lista de nomes \textquotedbl considerados feminos\textquotedbl , o contador de participantes femininos \'e incrementado. Quando o nome n\~ao est\'a em nenhuma das listas, ou quando \'e um nome indefinido, o contador de \textquotedbl g\^enero desconhecido \'e incrementado\textquotedbl .






Como recentemente foi inclu\'{\i}do na plataforma Platu\'osh um cadastro digital que pergunta o g\^enero do participante de forma autodeclarat\'oria, uma parte dos dados de g\^enero s\~ao originalmente fornecidos pelos pr\'oprios participantes, havendo g\^eneros que n\~ao se encaixam na concep\c{c}\~ao \textquotedbl masculina\textquotedbl , nem na \textquotedbl feminina\textquotedbl , situa\c{c}\~ao em que o contador de \textquotedbl outros\textquotedbl  \'e incrementado.






O gr\'afico abaixo mostra a distribui\c{c}\~ao de g\^eneros masculinos, femininos, desconhecidos e outros no universo de participantes do WASH. Nota-se um equil\'{\i}brio entre os participantes, com 49.4\% de mulheres e 48.3\% de homens, havendo ainda 2.1\% de g\^eneros desconhecidos. Apenas 5 cadastros apontam g\^eneros que n\~ao se encaixam nas demais concep\c{c}\~oes.






A afirma\c{c}\~ao de que existe um equil\'{\i}brio \'e de car\'ater amostral e n\~ao absoluto, podendo haver situa\c{c}\~oes em que para determinada faixa et\'aria, um g\^enero prevale\c{c}a sobre o outro.








\captionsetup{format=plain}
\begin{figure}[max size={\textwidth}{\textheight}]

\centering


\begin{minipage}[b]{0.4\linewidth}
        \centering
                \includegraphics[width=1.0\linewidth]{../../imagens/genero-todos-crop.jpeg}
                \caption{Distribui\c{c}\~ao dos participantes por g\^enero. Esses dados foram obtidos atrav\'es de infer\^encia, a posteriori, utilizando o primeiro nome dos participantes como forma de estimar o percentual de participantes de ambos os g\^eneros.}
                \label{ef11d820efb73d78fb64eb6bdd03853471a8e89f}
\end{minipage}%
\hspace{0.5cm}
\end{figure}



\subsection[N\'umero de Bolsistas]{N\'umero de Bolsistas}\label{N\'umero de Bolsistas}
O m\'etodo do WASH, descrito na Portaria CTI 178/2018, pressup\~oe a atua\c{c}\~ao de bolsistas de inicia\c{c}\~ao cient\'{\i}fica (Bolsas CNPq ITI) como multiplicadores do projeto. Al\'em disso, o Projeto conta com Bolsistas Extensionistas (Bolsa CNPq EXP), Bolsistas de Apoio T\'ecnico (Bolsa CNPq ATP).






Desta forma, o n\'umero de bolsistas atuantes no projeto \'e um importante elemento de caracteriza\c{c}\~ao do mesmo para que se conhe\c{c}a:







\begin{alineas}
\item o balan\c{c}o entre o n\'umero de crian\c{c}as e adolescentes atendidos e n\'umero de membros bolsistas atuantes no projeto
\item a for\c{c}a de produ\c{c}\~ao de resultados cient\'{\i}ficos e de inova\c{c}\~ao, concretizado na forma dos relat\'orios produzidos pelos bolsistas
\item a relev\^ancia do apoio \`a pesquisa e extens\~ao das entidades promotoras parceiras (Universidades e Centros de Pesquisa)
\end{alineas}

A an\'alise dos dados existentes na base de dados Platu\'osh indica a exist\^encia de 164 bolsistas no projeto WASH. Para obter este n\'umero foi preciso desconsiderar repeti\c{c}\~oes dos registros de afilia\c{c}\~oes (tabela \textquotedbl afiliacoes\textquotedbl  da base de dados).






Mas a experi\^encia desta candidata no apoio \`a implementa\c{c}\~ao de bolsas no \^ambito do Projeto WASH indicava a percep\c{c}\~ao de um n\'umero muito maior de bolsistas.






Esta intui\c{c}\~ao de que o n\'umero de bolsistas deveria ser muito maior do que o fornecido pela afilia\c{c}\~ao dos participantes registrada na Platu\'osh acendeu um sinal vermelho.






Estava claro que o registro de afilia\c{c}\~oes da Platu\'osh n\~ao era a forma mais adequada de saber quantos bolsistas passaram pelo projeto.













Uma pondera\c{c}\~ao sobre os motivos desta inadequa\c{c}\~ao levaram ao seguinte conjunto de reflex\~oes:







\begin{alineas}
\item a plataforma Platu\'osh \'e uma ferramenta dispon\'{\i}vel apenas a partir de 2018, raz\~ao pela qual n\~ao cobre todo o per\'{\i}odo de exist\^encia do projeto.
\item a plataforma Platu\'osh foi originalmente concebida para registro de presen\c{c}a e testemunho de realiza\c{c}\~ao de eventos, para fins de presta\c{c}\~ao de contas aos \'org\~aos de fomento, n\~ao havendo, inicialmente, a inten\c{c}\~ao de registrar os bolsistas
\item assim que a plataforma Platu\'osh foi adaptada para o registro de bolsistas, houve um esfor\c{c}o de recupera\c{c}\~ao de dados pregressos, mas este trabalho ficou naturalmente incompleto, pelo r\'apido crescimento do projeto, havendo um persistente back-log de dados de bolsistas (ac\'umulo de trabalho atrasado).
\item recentemente o preenchimento dos cadastros ficou a cargo dos pr\'oprios bolsistas, que ganharam \textquotedbl contas\textquotedbl  na plataforma. Este procedimento \'e naturalmente impreciso, porque muitos bolsistas n\~ao tem pr\'atica em seu preenchimento, apesar dos esfor\c{c}os de capacita\c{c}\~ao da Frente Multiplicadora do WASH
\item existe uma complac\^encia por parte dos bolsistas, que n\~ao preenchem a plataforma como solicitado
\end{alineas}

Consultando a equipe de TI sobre estes problemas com o registro de bolsistas na Platu\'osh, fomos informados que uma tabela auxiliar de registro de bolsistas tinha sido integrada \`a base de dados original. Esta tabela foi denominada \textquotedbl bolsa-cnpq\textquotedbl .






Uma consulta \`a base, utilizando o m\'etodo SQL, levou a um total de 235 registros na tabela \textquotedbl bolsa-cnpq\textquotedbl . Mas uma inspe\c{c}\~ao mais cuidadosa indicou que esta tabela continha todas as concess\~oes de bolsas, com a possibilidade de um bolsista ser contemplado por duas concess\~oes consecutivas, decorrentes da renova\c{c}\~ao de bolsas. Portanto, o n\'umero de 235 bolsistas estava claramente superestimado.






O pr\'oximo passo foi excluir as repeti\c{c}\~oes, agrupando os resultados por bolsista. Com esse m\'etodo chegou-se ao n\'umero de 206 bolsistas.






Esta variabilidade nos dados gerou-nos uma inseguran\c{c}a em rela\c{c}\~ao \`a plataforma Platu\'osh no que tange exclusivamente aos dados de bolsistas.






Assim, sentimo-nos motivados a buscar uma solu\c{c}\~ao independente para o levantamento de dados de bolsas, uma vez que o v\'{\i}nculo dos bolsistas com o CNPq \'e formal e ocorre mediante Termo de Outorga, havendo meios de obter dados absolutos e n\~ao-amostrais.






Contando com o apoio do Coordenador do Projeto WASH, foi poss\'{\i}vel levantar a quantidade de bolsistas e a distribui\c{c}\~ao por tipo de bolsas usando a Plataforma Carlos Chagas. Os dados foram obtidos de forma anonimizada pelo coordenador.






Para esse levantamento n\~ao foi utilizada a modelagem de banco de dados relacional, mas simplesmente a tabula\c{c}\~ao em planilhas eletr\^onicas, tecnologia mais acess\'{\i}vel a esta candidata. Desta forma o trabalho pode ser conduzido independentemente do apoio da equipe de TI, podendo, posteriormente, ser utilizado como balizador para melhoria dos processos de coleta de dados de bolsistas no \^ambito do Projeto WASH.






\subsection[Caracteriza\c{c}\~ao dos Planos de Trabalhos e Relat\'orios]{Caracteriza\c{c}\~ao dos Planos de Trabalhos e Relat\'orios}\label{Caracteriza\c{c}\~ao dos Planos de Trabalhos e Relat\'orios}
Ao receber a outorga de uma bolsa, o bolsista assume o compromisso de realizar um projeto de pesquisa, al\'em das atividades de extens\~ao. Estas \'ultimas envolvem a participa\c{c}\~ao como multiplicadores nas oficinas em escolas de ensino fundamental.






As atividades e as entregas referentes ao projeto de pesquisa s\~ao especificadas por meio de um plano de trabalho. Dentre as entregas definidas nesse Plano de Trabalho, \'e obrigat\'orio constar o Relat\'orio, que \'e uma forma de documenta\c{c}\~ao cient\'{\i}fica que segue a mesma estrutura definida no primeiro cap\'{\i}tulo desta disserta\c{c}\~ao.






Assim, uma aspecto importante da caracteriza\c{c}\~ao do Projeto WASH \'e contabiliza\c{c}\~ao e classifica\c{c}\~ao dos Planos de Trabalho e Relat\'orios produzidos pelos bolsistas do projeto.






Para a contabiliza\c{c}\~ao dos Planos de Trabalho e dos Relat\'orios produzidos pelos bolsistas foram empregados neste trabalho os seguintes instrumentos:







\begin{alineas}
\item plataforma Platu\'osh, que tem um car\'ater amostral e n\~ao exaustivo em termos de coleta de dados
\item o planejamento e caracteriza\c{c}\~ao financeira do projeto, que \'e um instrumento de compliance do projeto, mas que tamb\'em pode ser utilizado para suprir informa\c{c}\~oes sobre a documenta\c{c}\~ao presente no projeto
\item e o levantamento espec\'{\i}fico conduzido por esta candidata, com base em dados objetivos da Plataforma Carlos Chagas do CNPq, a fonte mais confi\'avel de dados para esse tipo de caracteriza\c{c}\~ao.
\end{alineas}

\subsection[Distribui\c{c}\~ao de temas em relat\'orios]{Distribui\c{c}\~ao de temas em relat\'orios}\label{Distribui\c{c}\~ao de temas em relat\'orios}
participantes






\subsection[N\'umero de oficinas realizadas]{N\'umero de oficinas realizadas}\label{N\'umero de oficinas realizadas}
A evolu\c{c}\~ao do n\'umero de eventos realizados ao longo dos dez anos de exist\^encia do projeto pode ser verificada no gr\'afico abaixo.








\captionsetup{format=plain}
\begin{figure}[htb]

	\begin{center}

		\includegraphics[max size={\textwidth}{\textheight}]{../../imagens/output-eventos.jpeg}

	\end{center}

	\caption{\label{8af5236ba8f91623157f8f95ae10366b416d6049}@[caption-8af5236ba8f91623157f8f95ae10366b416d6049]@}

\end{figure}

\subsection[Distribui\c{c}\~ao et\'aria nas oficinas]{Distribui\c{c}\~ao et\'aria nas oficinas}\label{Distribui\c{c}\~ao et\'aria nas oficinas}
participantes








\captionsetup{format=plain}
\begin{figure}[htb]

	\begin{center}

		\includegraphics[max size={\textwidth}{\textheight}]{../../imagens/histograma-de-idades-no-ano-do-evento.png}

	\end{center}

	\caption{\label{978341992d3d49498d48c41acc77f05f08f49ead}@[caption-978341992d3d49498d48c41acc77f05f08f49ead]@}

\end{figure}

\subsection[Distribui\c{c}\~ao de temas nas oficinas]{Distribui\c{c}\~ao de temas nas oficinas}\label{Distribui\c{c}\~ao de temas nas oficinas}
participantes






\subsection[Tipos de Atividades realizadas nas oficinas]{Tipos de Atividades realizadas nas oficinas}\label{Tipos de Atividades realizadas nas oficinas}
Primeiro par\'agrafo.






Primeiro par\'agrafo.






Primeiro par\'agrafo.






Primeiro par\'agrafo.






Primeiro par\'agrafo.






Primeiro par\'agrafo.






Primeiro par\'agrafo.






Primeiro par\'agrafo.






\subsection[Cidades Atendidas]{Cidades Atendidas}\label{Cidades Atendidas}
teste






\subsection[Participantes mais ass\'{\i}duos]{Participantes mais ass\'{\i}duos}\label{Participantes mais ass\'{\i}duos}
primeiro par\'agrafo






\section[An\'alise: s\'{\i}ntese das 3 dimens\~oes]{An\'alise: s\'{\i}ntese das 3 dimens\~oes}\label{An\'alise: s\'{\i}ntese das 3 dimens\~oes}
Aqui ser\'a feita a s\'{\i}ntese das 3 dimens\~oes.






\chapter[CONCLUS\~OES]{CONCLUS\~OES}\label{CONCLUS\~OES}
Aqui v\~ao as conclus\~oes.






Vamos testar \textquotedbl aspas\textquotedbl  e ver como s\~ao guardadas. Depois vamos testar 'ap\'ostrofos' e ver como s\~ao guardados.






``a''










\begin{table}[htb]
\caption{\label{96a5858fd2e287fed4b65c001ce7de8de2d2fc5b}Legenda de teste}

\centering
\begin{tabular}{|c|c|c|}
\hline
teste00  &  teste01  &  teste 02 \\
teste10  &  teste11  &  teste 12 \\
teste20  &  teste21  &  teste 22 \\
\hline
\end{tabular}
\end{table}


\chapter[PRODUTOS TECNOL\'OGICOS]{PRODUTOS TECNOL\'OGICOS}\label{PRODUTOS TECNOL\'OGICOS}
Aqui entra o Produto tecnol\'ogico.






\chapter[REFER\^ENCIAS]{REFER\^ENCIAS}\label{REFER\^ENCIAS}
\begin{flushleft}
\begin{flushleft}
\begin{flushleft}
\begin{flushleft}
\begin{flushleft}
\begin{flushleft}
[MEO, 2018] Meo, S.A. Anatomy and physiology of a scientific paper, Saudi Journal of Biological Sciences, V.25, I.7, November 2018, Pg. 1278-1283
\end{flushleft}


\end{flushleft}


\end{flushleft}


\end{flushleft}


\end{flushleft}


\end{flushleft}


\begin{flushleft}
\begin{flushleft}
\begin{flushleft}
\begin{flushleft}
\begin{flushleft}
\begin{flushleft}
[LEVY, 2000] LEVY, P. Cibercultura. 2 ed. Editora 34,  Rio de Janeiro:, 2000.p. 14 e 15.
\end{flushleft}


\end{flushleft}


\end{flushleft}


\end{flushleft}


\end{flushleft}


\end{flushleft}


\begin{flushleft}
\begin{flushleft}
\begin{flushleft}
\begin{flushleft}
\begin{flushleft}
\begin{flushleft}
[DANTAS, 1988] DANTAS, V. Guerrilha Tecnol\'ogica, Livros T\'ecnicos e Cient\'{\i}ficos, janeiro de 1988
\end{flushleft}


\end{flushleft}


\end{flushleft}


\end{flushleft}


\end{flushleft}


\end{flushleft}


\begin{flushleft}
\begin{flushleft}
\begin{flushleft}
\begin{flushleft}
\begin{flushleft}
\begin{flushleft}
[DUTTON, 2004] DUTTON, W. Social Transformation in an Information Society: Rethinking Access to You and the World, UNESCO 2004, Society: Rethinking Access to You and the World, 
\end{flushleft}


\end{flushleft}


\end{flushleft}


\end{flushleft}


\end{flushleft}


\end{flushleft}


\begin{flushleft}
\begin{flushleft}
\begin{flushleft}
\begin{flushleft}
\begin{flushleft}
\begin{flushleft}
[HARARI, 2018]  HARARI, Y. 21 Li\c{c}\~oes para o s\'eculo 21, Companhia das Letras, 2018
\end{flushleft}


\end{flushleft}


\end{flushleft}


\end{flushleft}


\end{flushleft}


\end{flushleft}


\begin{flushleft}
\begin{flushleft}
\begin{flushleft}
\begin{flushleft}
\begin{flushleft}
\begin{flushleft}
[BATES, 2014] BATES COLLEGE, How to Write a Paper in Scientific Journal Style and Format, v.10-2014, acessado em: https://www.bates.edu/biology/files/2010/06/How-to-Write-Guide-v10-2014.pdf, 2022
\end{flushleft}


\end{flushleft}


\end{flushleft}


\end{flushleft}


\end{flushleft}


\end{flushleft}


\begin{flushleft}
\begin{flushleft}
\begin{flushleft}
\begin{flushleft}
\begin{flushleft}
\begin{flushleft}
[KARA-JUNIOR, 2014] KARA-JUNIOR, N. Estrutura, estilo e escrita de artigo cient\'{\i}fico: a maneira com que pesquisadores reconhecem seus pares, Revista Brasileira de Oftalmologia 73(5), Set-Out 2014.
\end{flushleft}


\end{flushleft}


\end{flushleft}


\end{flushleft}


\end{flushleft}


\end{flushleft}


\begin{flushleft}
\begin{flushleft}
\begin{flushleft}
\begin{flushleft}
\begin{flushleft}
\begin{flushleft}
[MAMMANA, 2019] MAMMANA, A.P. Documenta\c{c}\~ao Cient\'{\i}fica, acessado no Youtube em 2022
\end{flushleft}


\end{flushleft}


\end{flushleft}


\end{flushleft}


\end{flushleft}


\end{flushleft}


\begin{flushleft}
\begin{flushleft}
\begin{flushleft}
\begin{flushleft}
\begin{flushleft}
\begin{flushleft}
[CATTERALL, 2017] CATTERALL, L.G. A brief history of STEM and STEAM from an Inadvertent Insider, The STEAM Journal, V 3(1) 2017
\end{flushleft}


\end{flushleft}


\end{flushleft}


\end{flushleft}


\end{flushleft}


\end{flushleft}


\begin{flushleft}
\begin{flushleft}
\begin{flushleft}
\begin{flushleft}
\begin{flushleft}
\begin{flushleft}
[ENGLEBART, 2017] ENGLEBART, D. Microeletronics and the art of similitude, 1960 IEEE International Solid-State Circuits Conference. Digest of Technical Papers, 10-12 de fevereiro de 1960
\end{flushleft}


\end{flushleft}


\end{flushleft}


\end{flushleft}


\end{flushleft}


\end{flushleft}


\begin{flushleft}
\begin{flushleft}
\begin{flushleft}
\begin{flushleft}
\begin{flushleft}
\begin{flushleft}
[NEGROPONTE, 2004] NEGROPONTE, N. Brazil's Plan 2004, acervo pessoal de Victor Mammana
\end{flushleft}


\end{flushleft}


\end{flushleft}


\end{flushleft}


\end{flushleft}


\end{flushleft}


\begin{flushleft}
\begin{flushleft}
\begin{flushleft}
\begin{flushleft}
\begin{flushleft}
\begin{flushleft}
[PAPERT, 2005] PAPERT, S. (2005). Teaching Children Thinking. Contemporary Issues in Technology and Teacher Education, 5(3), 353-365. Waynesville, NC USA: Society for Information Technology \& Teacher Education. Retrieved July 26, 2022
\end{flushleft}


\end{flushleft}


\end{flushleft}


\end{flushleft}


\end{flushleft}


\end{flushleft}


\begin{flushleft}
\begin{flushleft}
\begin{flushleft}
\begin{flushleft}
\begin{flushleft}
\begin{flushleft}
[MAMMANA e TOZZI, 2018] Avalia\c{c}\~ao do Programa OLPC, Cubat\~ao 2018
\end{flushleft}


\end{flushleft}


\end{flushleft}


\end{flushleft}


\end{flushleft}


\end{flushleft}


\begin{flushleft}
\begin{flushleft}
\begin{flushleft}
\begin{flushleft}
\begin{flushleft}
\begin{flushleft}
[BELL, 1973]  BELL, 1973, professor de Harvard, que a partir do texto The Coming of Post Industrial Society [XXX BELL, Daniel. The Coming of Post-industrial Society. Nova York: Basic Books, 1973
\end{flushleft}


\end{flushleft}


\end{flushleft}


\end{flushleft}


\end{flushleft}


\end{flushleft}


\begin{flushleft}
\begin{flushleft}
\begin{flushleft}
\begin{flushleft}
\begin{flushleft}
\begin{flushleft}
[MAMMANA, 2020] MAMMANA, A. Semin\'ario - Documenta\c{c}\~ao em Ci\^encia e Tecnologia, v\'{\i}deo do Youtube, https://www.youtube.com/watch?v=-ek\_EjIDWnE acessado em 12/08/2022
\end{flushleft}


\end{flushleft}


\end{flushleft}


\end{flushleft}


\end{flushleft}


\end{flushleft}


\begin{flushleft}
\begin{flushleft}
\begin{flushleft}
\begin{flushleft}
\begin{flushleft}
\begin{flushleft}
[CTI, 2018] Portaria CTI 178/2018.
\end{flushleft}


\end{flushleft}


\end{flushleft}


\end{flushleft}


\end{flushleft}


\end{flushleft}


\begin{flushleft}
\begin{flushleft}
\begin{flushleft}
\begin{flushleft}
\begin{flushleft}
\begin{flushleft}
[MAMMANA, 2009] Avalia\c{c}\~ao do PIDs
\end{flushleft}


\end{flushleft}


\end{flushleft}


\end{flushleft}


\end{flushleft}


\end{flushleft}


\begin{flushleft}
\begin{flushleft}
\begin{flushleft}
\begin{flushleft}
\begin{flushleft}
\begin{flushleft}
[Marczal, 2016] Marczal, E. S. Introdu\c{c}\~ao \`a historiografia: da abordagem tradicional \`as perspectivas p\'os-modernas. Curitiba: Intersaberes, 2016, 1a Edi\c{c}\~ao.
\end{flushleft}


\end{flushleft}


\end{flushleft}


\end{flushleft}


\end{flushleft}


\end{flushleft}


\begin{flushleft}
\begin{flushleft}
\begin{flushleft}
\begin{flushleft}
\begin{flushleft}
\begin{flushleft}
[FREITAS, 2019] FREITAS, I. TEORIAS DA HIST\'ORIA NA HISTORIOGRAFIA DE RANKE, Ponta de Lan\c{c}a, S\~ao Crist\'ov\~ao, v. 13, n. 25, jul. - dez. 2019.
\end{flushleft}


\end{flushleft}


\end{flushleft}


\end{flushleft}


\end{flushleft}


\end{flushleft}


\begin{flushleft}
\begin{flushleft}
\begin{flushleft}
\begin{flushleft}
\begin{flushleft}
\begin{flushleft}
[WIKIPEDIA, 2022] Imagem obtida da WIKIPEDIA acessada em 17 de agosto de 2022, atrav\'es da URL https://pt.wikipedia.org/wiki/Her\'odoto
\end{flushleft}


\end{flushleft}


\end{flushleft}


\end{flushleft}


\end{flushleft}


\end{flushleft}


\begin{flushleft}
\begin{flushleft}
\begin{flushleft}
\begin{flushleft}
\begin{flushleft}
\begin{flushleft}
[BENTIVOGLIO, 2010] BENTIVOGLIO, J. Hist\'oria e narrativa na Historiografia alem\~a do s\'eculo XIX Anos 90, Porto Alegre, v. 17, n. 32, p.185-218, dez. 2010
\end{flushleft}


\end{flushleft}


\end{flushleft}


\end{flushleft}


\end{flushleft}


\end{flushleft}


\begin{flushleft}
\begin{flushleft}
\begin{flushleft}
\begin{flushleft}
\begin{flushleft}
\begin{flushleft}
[TEIXEIRA, 2008] TEIXEIRA, F.C. Uma constru\c{c}\~ao de fatos e palavras: C\'{\i}cero e a concep\c{c}\~ao ret\'orica da hist\'oria, VARIA HISTORIA, Belo Horizonte, vol. 24, nº 40: p.551-568, jul/dez 2008
\end{flushleft}


\end{flushleft}


\end{flushleft}


\end{flushleft}


\end{flushleft}


\end{flushleft}


\begin{flushleft}
\begin{flushleft}
\begin{flushleft}
\begin{flushleft}
\begin{flushleft}
\begin{flushleft}
[Setzer e Silva, 2017] Setzer, V. W.; Silva, F. S. C. Banco de Dados - Aprenda o que s\~ao, melhore seu conhecimento, construa os seus, Editora Edgard Blucher, 3a reimpress\~ao, 2017
\end{flushleft}


\end{flushleft}


\end{flushleft}


\end{flushleft}


\end{flushleft}


\end{flushleft}


\begin{flushleft}
\begin{flushleft}
\begin{flushleft}
\begin{flushleft}
\begin{flushleft}
\begin{flushleft}
[Barrios, 2015] Barrios, J.E.R. Information, Genetics and Entropy, Principia 19(1): 121–146 (2015)
\end{flushleft}


\end{flushleft}


\end{flushleft}


\end{flushleft}


\end{flushleft}


\end{flushleft}


\begin{flushleft}
\begin{flushleft}
\begin{flushleft}
\begin{flushleft}
\begin{flushleft}
\begin{flushleft}
[Rodrigues, 2010] Rodrigues, Z.M.R. Sistema de indicadores e desigualdade socioambiental intraurbana de S\~ao Lu\'{\i}z-MA, Tese de Doutorado, Orientador: Prof. Dr. Wagner Costa Ribeiro, Programa de P\'os-Gradua\c{c}\~ao da Universidade de S\~ao Paulo, 2010
\end{flushleft}


\end{flushleft}


\end{flushleft}


\end{flushleft}


\end{flushleft}


\end{flushleft}


\begin{flushleft}
\begin{flushleft}
\begin{flushleft}
\begin{flushleft}
\begin{flushleft}
\begin{flushleft}
[MEADOWS, 2006] Meadows, D. apud: Indicators and information Systems for sustainable development. The Sustainability Institute, 1998, In: WORKSHOP INTERNACIONAL PESQUISA EM INDICADORES DE SUSTENTABILIDADE. S\~ao Paulo, Faculdade de Sa\'ude P\'ublica, 2006.
\end{flushleft}


\end{flushleft}


\end{flushleft}


\end{flushleft}


\end{flushleft}


\end{flushleft}


\begin{flushleft}
\begin{flushleft}
\begin{flushleft}
\begin{flushleft}
\begin{flushleft}
\begin{flushleft}
[WONG, 2006] WONG, C. apud: Indicators for Urban and Regional PLanning, New York: Routledge Taylor 
\end{flushleft}


\end{flushleft}


\end{flushleft}


\end{flushleft}


\end{flushleft}


\end{flushleft}


\begin{flushleft}
\begin{flushleft}
\begin{flushleft}
\begin{flushleft}
\begin{flushleft}
\begin{flushleft}
[PARMENTER, 2007] PARMENTER, D. Key Performance Indicators - Developing, Implementing, and Using Winning KPIs, John Wiley 
\end{flushleft}


\end{flushleft}


\end{flushleft}


\end{flushleft}


\end{flushleft}


\end{flushleft}


\begin{flushleft}
\begin{flushleft}
\begin{flushleft}
\begin{flushleft}
\begin{flushleft}
\begin{flushleft}
[MAMMANA, 1999] Mammana, C.Z. The Natual History of Information Processors in: The Quest for a Unified Theory of Information, Edited by Wolfgang Hofkirchner, Viena University, Austria, Gordon and Breach Publishers, 1999
\end{flushleft}


\end{flushleft}


\end{flushleft}


\end{flushleft}


\end{flushleft}


\end{flushleft}


\begin{flushleft}
\begin{flushleft}
\begin{flushleft}
\begin{flushleft}
\begin{flushleft}
\begin{flushleft}
[REIS , 2006] (A Escola Met\'odica dita Positivista in: REIS, Jos\'e Carlos; Hist\'oria entre a Filosofia e a Ci\^encia; p\'ag. 22, 3 ed., 1 reimp; Belo Horizonte: Aut\^entica, 200
\end{flushleft}


\end{flushleft}


\end{flushleft}


\end{flushleft}


\end{flushleft}


\end{flushleft}


\begin{flushleft}
\begin{flushleft}
\begin{flushleft}
\begin{flushleft}
\begin{flushleft}
\begin{flushleft}
[Pires, 2009] PIRES, M.F. de C. O materialismo hist\'orico-dial\'etico e a Educa\c{c}\~ao. Interface - Comunica\c{c}\~ao, Sa\'ude, Educa\c{c}\~ao [online]. 1997, v. 1, n. 1 [Acessado 31 Agosto 2022] , pp. 83-94. Dispon\'{\i}vel em: <https://doi.org/10.1590/S1414-32831997000200006>. Epub 04 Ago 2009. ISSN 1807-5762. https://doi.org/10.1590/S1414-32831997000200006.
\end{flushleft}


\end{flushleft}


\end{flushleft}


\end{flushleft}


\end{flushleft}


\end{flushleft}


\begin{flushleft}
\begin{flushleft}
\begin{flushleft}
\begin{flushleft}
\begin{flushleft}
\begin{flushleft}
[Burke, 1991] Burke, P. A Revolu\c{c}\~ao Francesa da historiografia: a Escola dos Annales 1929-1989 / Peter Burke; tradu\c{c}\~ao Nilo Od\'alia. – S\~ao Paulo: Editora Universidade Estadual Paulista, 1991
\end{flushleft}


\end{flushleft}


\end{flushleft}


\end{flushleft}


\end{flushleft}


\end{flushleft}


\begin{flushleft}
\begin{flushleft}
\begin{flushleft}
\begin{flushleft}
\begin{flushleft}
\begin{flushleft}
[PIERANTI, 2022] Pieranti, O.P. A metodologia historiogr\'afica na pesquisa em administra\c{c}\~ao: uma discuss\~ao acerca de princ\'{\i}pios e sua aplicabilidade no Brasil contempor\^aneo. Acessado em 11/01/22. www.scielo.br/cebape/a/
\end{flushleft}


\end{flushleft}


\end{flushleft}


\end{flushleft}


\end{flushleft}


\end{flushleft}


\begin{flushleft}
\begin{flushleft}
\begin{flushleft}
\begin{flushleft}
\begin{flushleft}
\begin{flushleft}
[Firat, 1987] Firat, A.F. Historiografia, M\'etodo Cient\'{\i}fico e Eventos Hist\'oricos Excepcionais, NA Advances in Consumer Research Volume 14, 1987, P\'ag. 453-438
\end{flushleft}


\end{flushleft}


\end{flushleft}


\end{flushleft}


\end{flushleft}


\end{flushleft}


\begin{flushleft}
\begin{flushleft}
\begin{flushleft}
\begin{flushleft}
\begin{flushleft}
\begin{flushleft}
[MAMMANA et al., 2022] Mammana V.P., Tozzi E.S., Cruz R.G. da, Soares A.C. de D., Diogo C.P.M. e Morandi M.A. Memorando no. 70/2021/CEMADEN, Registro de Software Desenvolvido em 2020, 25 de fevereiro de 2022.
\end{flushleft}


\end{flushleft}


\end{flushleft}


\end{flushleft}


\end{flushleft}


\end{flushleft}


\begin{flushleft}
\begin{flushleft}
\begin{flushleft}
\begin{flushleft}
\begin{flushleft}
\begin{flushleft}
[Kijima et al., 2021] Kijima R., Yang-Yoshihara M., Maekawa M. Using design thinking to cultivate the next generation of female STEAM thinkers, International Journal of STEM Education (2021) 8:14 https://doi.org/10.1186/s40594-021-00271-6
\end{flushleft}


\end{flushleft}


\end{flushleft}


\end{flushleft}


\end{flushleft}


\end{flushleft}


\begin{flushleft}
\begin{flushleft}
\begin{flushleft}
\begin{flushleft}
\begin{flushleft}
\begin{flushleft}
[FULLER, 2011] Fuller, R. Advantages and hazards of using Microsoft Excel to Organize and display water quality data, Proceeedings of the 2011 Georgia Water Resources, held April 11-13, 2011 at the University of Georgia.
\end{flushleft}


\end{flushleft}


\end{flushleft}


\end{flushleft}


\end{flushleft}


\end{flushleft}


\begin{flushleft}
\begin{flushleft}
\begin{flushleft}
\begin{flushleft}
\begin{flushleft}
\begin{flushleft}
[Brudner, 2022] Brudner, E. Twenty Two Advantages and Disadvantages of Using Spreadsheets for Business, acessado via https://blog.hubspot.com/sales/dangers-of-using-spreadsheets-for-sales em 20 de setembro de 2022.
\end{flushleft}


\end{flushleft}


\end{flushleft}


\end{flushleft}


\end{flushleft}


\end{flushleft}


\begin{flushleft}
\begin{flushleft}
\begin{flushleft}
\begin{flushleft}
\begin{flushleft}
\begin{flushleft}

\end{flushleft}


\end{flushleft}


\end{flushleft}


\end{flushleft}


\end{flushleft}


\end{flushleft}


\begin{flushleft}
\begin{flushleft}
\begin{flushleft}
\begin{flushleft}
\begin{flushleft}
\begin{flushleft}

\end{flushleft}


\end{flushleft}


\end{flushleft}


\end{flushleft}


\end{flushleft}


\end{flushleft}


\begin{flushleft}
\begin{flushleft}
\begin{flushleft}
\begin{flushleft}
\begin{flushleft}
\begin{flushleft}

\end{flushleft}


\end{flushleft}


\end{flushleft}


\end{flushleft}


\end{flushleft}


\end{flushleft}


% @[pontoinsercaotextoprincipal]@
% ---

% ---
% Cap\'{\i}tulo 2
% ---

% ---
% Cap\'{\i}tulo 3 - Cita\c{c}\~oes
% ---
% ---

% ---
% Cap\'{\i}tulo 4 - Referencias
% ---
% ---

% Cap\'{\i}tulo 5 - Conclus\~ao
% ---
% ---

% ----------------------------------------------------------
% ELEMENTOS P\'OS-TEXTUAIS
% ----------------------------------------------------------
\postextual
% ----------------------------------------------------------

% -----------------------------------------------------------
% Refer\^encias bibliogr\'aficas
% ----------------------------------------------------------


% @[bibliografia]@


% ----------------------------------------------------------
% Gloss\'ario
% ----------------------------------------------------------
%
% Consulte o manual da classe abntex2 para orienta\c{c}\~oes sobre o gloss\'ario.
%
%\glossary

% ----------------------------------------------------------
% Ap\^endices
% ----------------------------------------------------------
\include{USPSC-Tutorial/USPSC-ApendicesTutorial_RedarTex}

% ----------------------------------------------------------
% Anexos
% ----------------------------------------------------------
\include{USPSC-Tutorial/USPSC-AnexosTutorial_RedarTex}

%---------------------------------------------------------------------
% INDICE REMISSIVO
%--------------------------------------------------------------------
%%% USPSC-IndicexRemissivosTutorial.tex
% ---
% Inicia os \'Indices Remissivos
% ---
%---------------------------------------------------------------------
% INDICE REMISSIVO
%--------------------------------------------------------------------
\phantompart
\printindex
%---------------------------------------------------------------------

\phantompart
\printindex
%---------------------------------------------------------------------


\end{document}
