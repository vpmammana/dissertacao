%%%% RESUMO
%%
%% Apresenta\c{c}\~ao concisa dos pontos relevantes de um texto, fornecendo uma vis\~ao r\'apida e clara do conte\'udo e das conclus\~oes do trabalho.

\begin{resumoutfpr}%% Ambiente resumoutfpr

O  Programa Workshop de Aficionados por Software e Hardware (WASH), de educação em Ciência, Tecnologia, Engenharia, Artes e Matemática (STEAM) é executado desde 2013 em dezenas de municípios brasileiros e com milhares de crianças atendidas. Após anos de prática, as características principais foram agrupadas no Documento de Referência publicado em 2018, anexado à Portaria CTI 178/2018. Esta pesquisa é dividida em 2 eixos: método historiográfico (eixo 1) e o emprego de consultas estruturadas a uma base de dados especialmente desenvolvida para produzir os indicadores (eixo 2). O trabalho buscou comparar, a partir das definições do Documento de Referência, "o que o WASH gostaria de ter sido" com "o que o WASH conseguiu ser", comparação que se sustenta nos achados do capítulo de Resultados e Análise desta dissertação. Para objetivar essa comparação, foram formuladas seis hipóteses, a partir do Documento de Referência, que ao final do trabalho foram submetidas a uma validação.  A análise dos sucessos e insucessos dessa validação permitiu produzir uma revisão do Documento de Referência, a qual é o principal produto educacional desta dissertação. Agrega-se a esse produto educacional a entrevista com a Profa. Dra Afira Vianna Ripper, um dos elementos usados para a análise no eixo 1 e, também, um testemunho ocular bastante raro sobre a vinda de Seymour Papert ao Brasil no final do século passado.
% @[pontoinsercaoparagraforesumo]@

\end{resumoutfpr}
