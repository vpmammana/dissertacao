%% USPSC-fichacatalografica.tex
% ---
% Inserir a ficha bibliografica
% ---
% Isto \'e um exemplo de Ficha Catalogr\'afica, ou ``Dados internacionais de
% cataloga\c{c}\~ao-na-publica\c{c}\~ao''. Voc\^e pode utilizar este modelo como refer\^encia. 
% Por\'em, provavelmente a biblioteca da sua universidade lhe fornecer\'a um PDF
% com a ficha catalogr\'afica definitiva ap\'os a defesa do trabalho. Quando estiver
% com o documento, salve-o como PDF no diret\'orio do seu projeto e substitua todo
% o conte\'udo de implementa\c{c}\~ao deste arquivo pelo comando abaixo:
%
\begin{fichacatalografica}
	\hspace{-1.4cm}
	\imprimirnotaautorizacao \\ \\
	%\sffamily
	\vspace*{\fill}					% Posi\c{c}\~ao vertical
\begin{center}					% Minipage Centralizado
  \imprimirnotabib \\
  \begin{table}[htb]
	\scriptsize
	\centering	
	\begin{tabular}{|p{0.9cm} p{8.7cm}|}
		\hline
	      & \\
		  &	  \imprimirautorficha     \\
		
		 \imprimircutter & 
							\hspace{0.4cm}\imprimirtitulo~  / ~\imprimirautor~ ;  ~\imprimirorientadorcorpoficha. -- 	\imprimirlocal, \imprimirdata.   \\
		
		  &  % Para incluir nota referente \`a vers\~ao corrigida no corpo da ficha,
			  % incluir % no in\'{\i}cio da linha acima e tirar a % do in\'{\i}cio da linha abaixo
			  %	\hspace{0.4cm} \imprimirtitulo~  / ~\imprimirautor~ ; ~\imprimirorientadorcorpoficha~- ~\imprimirnotafolharosto. -- \imprimirlocal, \imprimirdata.  \\
		
			\hspace{0.4cm}\pageref{LastPage} p. : il. (algumas color.) ; 30 cm.\\ 
		  & \\
		  & 
		    \hspace{0.4cm}\imprimirnotaficha ~--~ 
						  \imprimirunidademin, 
						  \imprimiruniversidademin, 
		                  \imprimirdata. \\ 
		  & \\                 
		   % Para incluir nota referente \`a vers\~ao corrigida em notas,
		    % incluir uma % no in\'{\i}cio da linha acima e	
		    % tirar a % do in\'{\i}cio da linha abaixo
		    % & \hspace{0.4cm}\imprimirnotafolharosto \\ 
		  & \\ 
		  & \hspace{0.4cm}1. LaTeX. 2. abnTeX. 3. Classe USPSC. 4. Editora\c{c}\~ao de texto. 5. Normaliza\c{c}\~ao da documenta\c{c}\~ao. 6. Tese. 7. Disserta\c{c}\~ao. 8. Documentos (elabora\c{c}\~ao). 9. Documentos eletr\^onicos. I. \imprimirorientadorficha. 
		   II. T\'{\i}tulo. \\
	
		     %Se houver co-orientador, inclua % antes da linha (antes de II. T\'{\i}tulo.) 
		     %          e tire a % antes do comando abaixo 
		     %III. T\'{\i}tulo. \\   
		  \hline
	\end{tabular}
  \end{table}
\end{center}
\end{fichacatalografica}
% ---

