%%%% EP\'IGRAFE
%%
%% Texto em que o autor apresenta uma cita\c{c}\~ao, seguida de indica\c{c}\~ao de autoria, relacionada com a mat\'eria tratada no corpo do
%% trabalho.

\begin{epigrafe}%% Ambiente epigrafe
Brasil Mostra Tua Cara - Brasil... Meu Brasil ande pra frente. Venha com a gente pra Avenida desfilar . É chegada a hora da verdade. Não é preciso mais você se disfarçar. Levante os panos, mostra tua cara. E assuma essa cara que você tem. Brasil Terra dos Ianomâmis. Essas matas são de Oxossi. Deixa na terra as riquezas de Oxum. Devolva pro povo o que é do povo. Bote os malditos pra fora. E vamos refazer essa nação. Pois, o país que  é o olho d’água do mundo não pode ver sofrer. Não pode ver chorar um povo que trabalha, canta e é feliz. Chega de tanta injustiça, chega de corrupção. Vamos arrumar a casa, vamos dividir o nosso chão. E chega de sofrer e chega de chorar. Oh pátria amada idolatrada. Salve-se Brasil! Antonio Carlos (TC) Santos Silva e Aluízio Jeremias (Samba Enredo, 1988)
\end{epigrafe}
