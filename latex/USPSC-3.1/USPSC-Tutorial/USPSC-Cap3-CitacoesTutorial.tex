% ---
%% USPSC-Cap3-CitacoesTutorial.tex
% --
% Este cap\'{\i}tulo traz os exemplos de cita\c{c}\~oes das "Diretrizes para apresenta\c{c}\~ao de disserta\c{c}\~oes e teses da USP: documento eletr\^onico e impresso - Parte I (ABNT)" dispon\'{\i}lvel em: http://biblioteca.puspsc.usp.br/pdfFiles_Caderno_Estudos_9_PT_1.pdf


% --- 
\chapter{Cita\c{c}\~oes}
\label{Cita\c{c}\~oes}
% --- 
Cita\c{c}\~ao \'e a men\c{c}\~ao no texto de informa\c{c}\~oes extra\'{\i}das de uma fonte documental que tem o prop\'osito de esclarecer ou fundamentar as ideias do autor. A fonte de onde foi extra\'{\i}da a informa\c{c}\~ao deve ser citada obrigatoriamente, respeitando-se os direitos autorais, conforme ABNT NBR 10520 \cite{nbr10520}.

As cita\c{c}\~oes mencionadas no texto devem, obrigatoriamente, seguir a mesma forma de entrada utilizada nas Refer\^encias, no final do trabalho e/ou em Notas de Rodap\'e.

Todos os documentos relacionados nas Refer\^encias devem ser citados no texto, assim como todas as cita\c{c}\~oes do texto devem constar nas Refer\^encias. 

Os textos que constam desse manual e os exemplos de cita\c{c}\~oes e refer\^encias foram elaborados com base nas \textbf{Diretrizes para apresenta\c{c}\~ao de disserta\c{c}\~oes e teses da USP}: documento eletr\^onico e impresso - Parte I (ABNT) \cite{sibi2016}.

Para elaborar as cita\c{c}\~oes utilizando a Classe USPSC \'e necess\'ario a instala\c{c}\~ao do pacote: 

\begin{alineas}
	\item \textbf{usepackage[num]abntex2cite:} para gerar cita\c{c}\~oes e refer\^encias em estilo num\'erico;
	\item \textbf{usepackage[alf]abntex2cite:} para gerar cita\c{c}\~oes e refer\^encias em estilo alfab\'etico.
\end{alineas}

As explica\c{c}\~oes para utiliza\c{c}\~ao do pacote abntex2cite e exemplos de como elaborar cita\c{c}\~oes e refer\^encias de acordo com as normas da ABNT est\'a presente nos manuais: \textbf{O pacote abntex2cite}: estilos bibliogr\'aficos compat\'{\i}veis com a ABNT NBR 6023 \cite{abnetxcite} e  \textbf{O pacote abntex2cite}: t\'opicos espec\'{\i}ficos da ABNT NBR 10520:2002 e o estilo bibliogr\'afico alfab\'etico (sistema autor-data) \cite{abnetxcitealf}.

Abaixo seguem alguns exemplos de cita\c{c}\~oes, mas se o exemplo que voc\^e precisa n\~ao estiver contemplado aqui, acesse o manual \textbf{O pacote abntex2cite} que possui aproximadamente 240 modelos de refer\^encias.

Em todo esse documento e especificamente nos exemplos abaixo, foi utilizado o ponto final ap\'os o comando \verb+\cite{}+, em conformidade com sistema autor-data. Para o sistema num\'erico \'e necess\'ario utilizar o ponto final antes do comando \verb+\cite{}+. 

Alertamos que se este documento for alterado para sistema num\'erico a pontua\c{c}\~ao final ficar\'a incorreta. 

\section{Cita\c{c}\~ao direta}

\'E a transcri\c{c}\~ao (reprodu\c{c}\~ao integral) de parte da obra consultada, conservando-se a grafia, pontua\c{c}\~ao, idioma etc.

A reprodu\c{c}\~ao de um texto de \textbf{at\'e tr\^es linhas} deve ser incorporada ao par\'agrafo entre aspas duplas. As aspas simples s\~ao utilizadas para indicar cita\c{c}\~ao no interior da cita\c{c}\~ao.

\textbf{Nota:} nas cita\c{c}\~oes diretas \'e obrigat\'oria a indica\c{c}\~ao da p\'agina.

\textbf{Exemplos: }

\begin{alineas} 
\item 

Segundo \verb+\citeonline[p.~89]{madigan2010}+ “As ves\'{\i}culas de g\'as s\~ao estruturas fusiformes, preenchidas por g\'as e constitu\'{\i}das de prote\'{\i}nas; elas s\~ao ocas, por\'em r\'{\i}gidas, variando quanto ao comprimento e di\^ametro”.

Que corresponde: \\
Segundo \citeonline[p.~89]{madigan2010} “As ves\'{\i}culas de g\'as s\~ao estruturas
fusiformes, preenchidas por g\'as e constitu\'{\i}das de prote\'{\i}nas; elas s\~ao ocas, por\'em
r\'{\i}gidas, variando quanto ao comprimento e di\^ametro”.

\item 

“A compara\c{c}\~ao \'e a t\'ecnica cient\'{\i}fica aplic\'avel sempre que houver dois ou
mais termos com as mesmas propriedades gerais ou caracter\'{\i}sticas particulares”  \verb+\cite[p.~32]{cervo2007}.+

Que corresponde: \\
“A compara\c{c}\~ao \'e a t\'ecnica cient\'{\i}fica aplic\'avel sempre que houver dois ou
mais termos com as mesmas propriedades gerais ou caracter\'{\i}sticas particulares” \cite[p.~32]{cervo2007}.

\end{alineas}

As transcri\c{c}\~oes com \textbf{mais de tr\^es linhas} devem figurar abaixo do texto, com recuo de 4 cm da margem esquerda, com letra menor que a do texto utilizado e sem aspas. 

Utilize o ambiente cita\c{c}\~ao para incluir cita\c{c}\~oes diretas com mais de tr\^es linhas.

Use o ambiente assim: \\

\verb+\begin{cita\c{c}\~ao}+

Texto texto texto texto texto texto texto texto texto.

\verb+\end{cita\c{c}\~ao}+

O ambiente cita\c{c}\~ao pode receber como par\^ametro opcional um nome de idioma previamente carregado nas op\c{c}\~oes da classe. Nesse caso, o texto da cita\c{c}\~ao \'e automaticamente escrito em it\'alico e a hifeniza\c{c}\~ao \'e ajustada para o idioma selecionado na op\c{c}\~ao do ambiente.\\
  Por exemplo:
 
\verb+\begin{citacao}[english]+
 
 Text in English language in italic with correct hyphenation.
 
\verb+\end{citacao}+
 
Tem como resultado:
\begin{citacao}[english]
Text in English language in italic with correct hyphenation. \\
\end{citacao}

\textbf{Exemplos:} 

\begin{alineas} 

\item 
De acordo com \verb+\citeonline[p.~35]{cervo2007}+

\verb+\begin{citacao}+

A an\'alise e a s\'{\i}ntese racionais s\'o podem ser feitas mentalmente. Empregam-se principalmente na filosofia e na matem\'atica. A an\'alise \'e uma esp\'ecie de indu\c{c}\~ao; parte-se do particular, do complexo, para o princ\'{\i}pio geral e mais simples. A s\'{\i}ntese \'e uma esp\'ecie de dedu\c{c}\~ao; vai do mais simples ao mais complexo.

\verb+\end{citacao}+

Que corresponde: 

De acordo com \citeonline[p.~35]{cervo2007}

\begin{citacao}
A an\'alise e a s\'{\i}ntese racionais s\'o podem ser feitas mentalmente. Empregam-se principalmente na filosofia e na matem\'atica. A an\'alise \'e uma esp\'ecie de indu\c{c}\~ao; parte-se do particular, do complexo, para o princ\'{\i}pio geral e mais simples. A s\'{\i}ntese \'e uma esp\'ecie de dedu\c{c}\~ao; vai do mais simples ao mais complexo.
\end{citacao}

\item
De acordo com \verb+\citeonline[p.~S4]{Hood1999}+

\verb+\begin{citacao}[english]+

Text in English. Text in English. Text in English. Text in
English. Text in English. Text in English. Text in English. 
Text in English. Text in English. Text in English. Text in
English. Text in English.

\verb+\end{citacao}+

Que corresponde: \\

 De acordo com \citeonline[p.~S4]{Hood1999}
\begin{citacao}[english]
	Text in English. Text in English. Text in English. Text in English. Text in English. Text in English. Text in English. Text in English. Text in English. Text in English Text in English. Text in English.
\end{citacao}

\end{alineas}

\section{Cita\c{c}\~ao indireta}

\'E o texto criado com base na obra de autor consultado, em que se reproduz o conte\'udo e ideias do documento original; dispensa o uso de aspas duplas.

\textbf{Exemplos:}

A hipertemia em bovinos Jersey foi constatada quando a temperatura do ambiente
alcan\c{c}ava 2.5o \verb+\cite{reick1948}+

Que corresponde:

A hipertemia em bovinos Jersey foi constatada quando a temperatura do ambiente
alcan\c{c}ava 2.5o \cite{reick1948}


\section{Cita\c{c}\~ao de cita\c{c}\~ao}

\'E a cita\c{c}\~ao direta ou indireta de um texto que se refere ao documento original, que n\~ao se teve acesso.

Indicar no texto o sobrenome do(s) autor(es) do documento n\~ao consultado, seguido da data, da express\~ao latina apud (citado por) e do sobrenome do(s) autor(es) do documento consultado, data e p\'agina. 

Para elabora\c{c}\~ao de cita\c{c}\~ao de cita\c{c}\~ao s\~ao disponibilizados os seguintes comandos: \verb+\apud e \apudonline+.

\textbf{Exemplos:}

\begin{alineas}

\item
Incluir a cita\c{c}\~ao da obra consultada nas refer\^encias. 

\citetext{Reis1956}

\item
Mencionar, em nota de rodap\'e, a refer\^encia do trabalho n\~ao consultado

\newpage

\textbf{No texto:}

Segundo \apudonline{Segatto1995}{Vianna1986}, “[...] o vi\'es organicista da burocracia estatal e o antiliberalismo da cultura politica de 1937, preservado de modo encapu\c{c}ado na Carta de 1046”.

-------------------

\textsuperscript{1}\citetext{Vianna1986}

\textbf{Nas Refer\^encias:}

\citetext{Segatto1995}

\end{alineas}
\textbf{Nota:}

Este tipo de cita\c{c}\~ao s\'o deve ser utilizada nos casos em que o documento original n\~ao foi recuperado (documentos muito antigos, dados insuficientes para a localiza\c{c}\~ao do material etc.).

Ressaltamos que os comandos \verb+\apud e \apudonline+ est\~ao em conformidade com ABNT NBR 10520 e para elaborar a cita\c{c}\~ao de cita\c{c}\~ao conforme as Diretrizes da USP, que sugere a inclus\~ao da cita\c{c}\~ao da obra consultada nas refer\^encias e mencionar, em nota de rodap\'e, a refer\^encia do trabalho n\~ao consultado, \'e necess\'ario criar a cita\c{c}\~ao conforme abaixo, esse recurso deve ser utilizado para cita\c{c}\~oes com sistema num\'erico, j\'a que as notas de rodap\'e est\~ao configuradas com s\'{\i}mbolos. 



\begin{alineas}
\item 
\begin{verbatim}
Segundo Vianna\footnote{VIANNA, S. B. \textbf{ A politica econ\^omica 
no segundo Governo Vargas:} 1951-1954. Rio de Janeiro: BNDES, 1986}
(1986, p. 172 apud  \citeauthor{Segatto1995}, 1995, p. 214-215) 
“[...] o vi\'es organicista da burocracia estatal e o antiliberalismo 
da cultura politica de 1937, preservado de modo encapu\c{c}ado na Carta 
de 1046”.
\end{verbatim}
\end{alineas}


Que Corresponde: \\

Segundo Vianna\footnote{VIANNA, S. B.\textbf{ A politica econ\^omica no segundo Governo Vargas:} 1951-1954. Rio de Janeiro: BNDES, 1986} (1986, p. 172 apud \citeauthor{Segatto1995}, 1995, p. 214-215) “[...] o vi\'es organicista da burocracia estatal e o antiliberalismo da cultura politica de 1937, preservado de modo encapu\c{c}ado na Carta de 1046”.

\newpage

\textbf{Observa\c{c}\~ao:}

Tamb\'em \'e poss\'{\i}vel escolher dentre os dois comandos: \verb+\footciteref{}+ e o comando \verb+\footnote{\citetext{}}+ para inserir refer\^encias em notas de rodap\'es, mas ao utilizar esses comandos a refer\^encia \'e automaticamente inserida na lista final de refer\^encias, constando tanto das notas de rodap\'es quanto da lista de refer\^encias.

\section{Cita\c{c}\~ao de fontes informais}

\textbf{Informa\c{c}\~ao Verbal}

Quando obtidas atrav\'es de comunica\c{c}\~oes pessoais, anota\c{c}\~oes de aulas, trabalhos de eventos n\~ao publicados (confer\^encias, palestras, semin\'arios, congressos, simp\'osios etc.), indicar entre par\^enteses a express\~ao (informa\c{c}\~ao verbal), mencionando os dados dispon\'{\i}veis somente em nota de rodap\'e.

\textbf{Exemplo:}

\begin{alineas}
	\item
	\begin{verbatim}
	Ferreira (2014)\footnote{ Informa\c{c}\~ao fornecida por Ferreira durante 
	o XVIII Semin\'ario Nacional de Bibliotecas Universit\'arias, Belo 
	Horizonte, 2014.} afirma que as bibliotecas universit\'arias passam 
	por transforma\c{c}\~oes decorrentes das tecnologias de informa\c{c}\~ao e 
	comunica\c{c}\~ao (informa\c{c}\~ao verbal).
	\end{verbatim}
\end{alineas}

Que corresponde:

Ferreira (2014)\footnote{ Informa\c{c}\~ao fornecida por Ferreira durante 
o XVIII Semin\'ario Nacional de Bibliotecas Universit\'arias, Belo Horizonte, 
2014.} afirma que as bibliotecas universit\'arias passam por transforma\c{c}\~oes decorrentes das tecnologias de informa\c{c}\~ao e comunica\c{c}\~ao (informa\c{c}\~ao verbal).


\textbf{Informa\c{c}\~ao Pessoal}

Indicar, entre par\^enteses, a express\~ao (informa\c{c}\~ao pessoal) para dados obtidos de comunica\c{c}\~oes pessoais, correspond\^encias pessoais (postal ou \emph{e-mail}), mencionando-se os dados dispon\'{\i}veis em nota de rodap\'e.

\textbf{Exemplo:}


\begin{alineas}
\item
\begin{verbatim}
Pestana menciona que 20% das bibliotecas [\ldots] (informa\c{c}\~ao 
pessoal).\footnote{ PESTANA, F. O. Bibliotecas de ONGs. 
Mensagem recebida porvmbc@terra.com.br em 13 de abr. 2014.}
\end{verbatim}
\end{alineas}


Que corresponde:

Pestana menciona que 20\% das bibliotecas [\ldots] (informa\c{c}\~ao pessoal).\footnote{ PESTANA, F. O. \textbf{Bibliotecas de ONGs}. Mensagem recebida porvmbc@terra.com.br em 13 de abr. 2014.}\\


\textbf{Em fase de impress\~ao}

Trabalhos em fase de impress\~ao devem ser mencionados nas Refer\^encias.

\textbf{Exemplo:}

\begin{alineas}
	\item
PAULA, F. C. E. \textit{et al.} Incinerador de res\'{\i}duos l\'{\i}quidos e pastosos. \textbf{Revista de
Engenharia e Ci\^encias Aplicadas}, S\~ao Paulo, v. 5, 2001. No prelo.
\end{alineas}


\section{Cita\c{c}\~ao de website}

O endere\c{c}o eletr\^onico \'e indicado nas Refer\^encias. No texto, a cita\c{c}\~ao \'e referente ao autor ou ao t\'{\i}tulo do trabalho. 

\textbf{Exemplo:}

\begin{alineas}
\item
\textbf{No texto}
\begin{verbatim}
“[...] a manifesta\c{c}\~ao da CCP dever\'a ser submetida \`a delibera\c{c}\~ao da
CPG.”\cite{USP2013}.
\end{verbatim}
Que corresponde:\\
“[...] a manifesta\c{c}\~ao da CCP dever\'a ser submetida \`a delibera\c{c}\~ao da
CPG.” \cite{USP2013}. \\

\item 
\textbf{Nas refer\^encias}\\

UNIVERSIDADE DE S\~AO PAULO. Resolu\c{c}\~ao nº 6542, de 18 de abril de 2013.
Baixa o Regimento de P\'os-Gradua\c{c}\~ao da Universidade de S\~ao Paulo. \textbf{Di\'ario
Oficial [do] Estado de S\~ao Paulo}, 20 abr. 2013. Dispon\'{\i}vel em: http://www.
leginf.usp.br/?resolucao=resolucao-no-6542-de-18-de-abril. Acesso em: 08 jun.
2015.
\end{alineas}

\section{Destaque e supress\~oes no texto}

Utilizar os comandos abaixo durante a reda\c{c}\~ao das cita\c{c}\~oes com destaques e supress\~oes.

\verb+\underline{}+: para grifar.

\verb+\textbf{}+: para colocar em negrito.

\verb+\textit{}+: para colocar em it\'alico.

\verb+[\ldots]+: para supress\~oes [...]. \\

\textbf{Exemplos:}

\begin{alineas}
\item

\textbf{Destaques}

Usar \underline{grifo} ou \textbf{negrito} ou \textit{it\'alico} para \^enfases ou destaques. Na cita\c{c}\~ao, indicar (grifo nosso ou negrito nosso ou it\'alico nosso) entre par\^enteses, logo ap\'os a data.

\begin{verbatim}
``Se existe algu\'em de quem n\~ao aceitamos um `n\~ao', \'e porque, na 
verdade,\underline{entregamos o controle de nossa vida a essa 
pessoa}.'' \cite[~p.129, grifo nosso]{Cloud1999} \\
\end{verbatim}	

Que corresponde: \\

``Se existe algu\'em de quem n\~ao aceitamos um `n\~ao', \'e porque, na verdade,
\underline{entregamos o controle de nossa vida a essa pessoa}.'' \cite[~p.129, grifo nosso]{Cloud1999} \\

Usar a express\~ao “grifo do autor” “negrito do autor” ou "it\'alico do autor", caso o destaque seja do autor consultado.

\begin{verbatim}
“A palavra \textit{intui\c{c}\~ao} vem do latim \textit{intuire}, que 
significa \textit{ver por dentro}. O conceito varia conforme a
corrente de pensamento” \cite[~p.47, it\'alico do autor]{cervo2007}
\end{verbatim}

Que corresponde: \\

“A palavra \textit{intui\c{c}\~ao} vem do latim \textit{intuire}, que 
significa \textit{ver por dentro}. O conceito varia conforme a
corrente de pensamento” \cite[~p.47, it\'alico do autor]{cervo2007}\\

\item

\textbf{Supress\~oes}

Indicar as \textbf{supress\~oes} por retic\^encias dentro de colchetes, estejam elas no in\'{\i}cio, no meio ou no fim do par\'agrafo e/ou frase.

\begin{verbatim}
Segundo \citeonline[~p.72]{Bottomore1987}  assinala "[\ldots]  
a Sociologia, embora n\~ao pretenda ser mais a ci\^encia capaz de 
incluir toda a sociedade [\ldots] pretende ser sin\'optica"
\end{verbatim}

Que corresponde:\\

Segundo \citeonline[~p.72]{Bottomore1987}  assinala "[\ldots]  a Sociologia, embora n\~ao pretenda ser mais a ci\^encia capaz de incluir toda a sociedade [\ldots] pretende ser sin\'optica".\\ 

\item

\textbf{Interpola\c{c}\~oes}

Indicar as \textbf{interpola\c{c}\~oes}, coment\'arios, acr\'escimos e explica\c{c}\~oes dentro de colchetes, estejam elas no meio ou no fim do par\'agrafo e/ou frase.

\begin{verbatim}
"n\~ao se mova [como se isso fosse poss\'{\i}vel] fa\c{c}a de conta que 
est\'a morta" \cite[~p.72]{Clarac1985}.
\end{verbatim}

Que corresponde:\\

"n\~ao se mova [como se isso fosse poss\'{\i}vel] fa\c{c}a de conta que 
est\'a morta" \cite[~p.72]{Clarac1985}. \\

\item

\textbf{Tradu\c{c}\~ao feita pelo autor}

Quando a cita\c{c}\~ao incluir uma \textbf{tradu\c{c}\~ao feita pelo autor}, acrescentar a chamada da cita\c{c}\~ao seguida da express\~ao “tradu\c{c}\~ao nossa”, tudo entre par\^enteses.

\begin{verbatim}
"A epilepsia pode ocorrer em muitas doen\c{c}as infecciosas, como 
as causadas por v\'{\i}rus, bact\'erias e parasitas." \cite[~p.102,
tradu\c{c}\~ao nossa]{Brito2003}.
\end{verbatim}

Que corresponde:\\

"A epilepsia pode ocorrer em muitas doen\c{c}as infecciosas, como 
as causadas por v\'{\i}rus, bact\'erias e parasitas." . \cite[~p.102, tradu\c{c}\~ao nossa]{Brito2003}.\\
\end{alineas}

\section{Notas de rodap\'e}
As notas de rodap\'e s\~ao observa\c{c}\~oes ou esclarecimentos, cujas inclus\~oes no texto s\~ao feitas pelo autor do trabalho. Inclui dados obtidos por fontes informais tais como: informa\c{c}\~ao verbal, pessoal, trabalhos em fase de elabora\c{c}\~ao ou n\~ao consultados diretamente.

\newpage

Classificam-se em:\\
\begin{alineas}
\item
\textbf{Notas explicativas} constituem-se em coment\'arios, complementa\c{c}\~oes ou tradu\c{c}\~oes que interromperiam a sequ\^encia l\'ogica se colocadas no texto. \cite{Soares2002}

\item
\textbf{Notas de refer\^encias} indicam documentos consultados ou remetem a outras partes do texto onde o assunto em quest\~ao foi abordado. \\
\end{alineas}

Devem ser digitadas em fontes menores, dentro das margens, ficando separadas do texto por um espa\c{c}o simples de entrelinhas e por filete de aproximadamente 5 cm, a partir da margem esquerda.

As notas de rodap\'e podem ser indicadas por numera\c{c}\~ao consecutiva, com n\'umeros sobrescritos dentro do cap\'{\i}tulo ou da parte (n\~ao se inicia a numera\c{c}\~ao a cada folha).\\

\textbf{Exemplo}

\begin{alineas}

\item

\textbf{No texto:}

Compet\^encia: \'e “uma capacidade espec\'{\i}fica de executar a a\c{c}\~ao em um n\'{\i}vel de habilidade que seja suficiente para alcan\c{c}ar o efeito desejado” (RHINESMITH\textsuperscript{1}, 1993 apud VERGARA, 2000, p. 38).
Segundo Vergara (2000) mentalidade n\~ao \'e compet\^encia. A compet\^encia se estabelece a partir de uma mentalidade transformada em comportamento, assim como caracter\'{\i}stica n\~ao \'e compet\^encia.
Para Rhinesmith\textsuperscript{2} (1993 apud VERGARA, 2000, p. 38) as compet\^encias a seguir complementam as mencionadas anteriormente:

\textbf{Em nota de rodap\'e:}

---------------------

\textsuperscript{1}RHINESMITH, S. Guia gerencial para globaliza\c{c}\~ao. Rio de Janeiro: Berkeley, 1993.

\textsuperscript{2}Ibid, p. 38-39.


\end{alineas}

\textbf{Notas}

Os exemplos de inser\c{c}\~ao de notas de rodap\'e j\'a foram expostos nos itens 3.3 e 3.4.

Se a op\c{c}\~ao for pelo sistema de chamada num\'erico, a indica\c{c}\~ao da nota de rodap\'e dever\'a ser por s\'{\i}mbolos (ex.: asterisco etc.). 
Este modelo est\'a com o sistema num\'erico para nota de rodap\'es para mudar para simb\'olico \'e necess\'ario ativar o comando \verb+\renewcommand{\thefootnote}{\fnsymbol{footnote}}+

\section{Express\~oes Latinas}

As express\~oes latinas podem ser usadas para evitar repeti\c{c}\~oes
constantes de fontes citadas anteriormente. A primeira cita\c{c}\~ao de uma obra
deve apresentar sua refer\^encia completa e as subsequentes podem aparecer
sob forma abreviada (Quadro 1).
N\~ao usar destaque tipogr\'afico quando utilizar express\~oes latinas.
As express\~oes latinas n\~ao devem ser usadas no texto, apenas em nota
de rodap\'e, exceto a express\~ao apud.
A presen\c{c}a da refer\^encia em nota de rodap\'e n\~ao dispensa sua inclus\~ao
nas Refer\^encias, no final do trabalho.

As express\~oes idem, ibidem, opus citatum, passim, s\'o podem ser usadas na mesma p\'agina ou folha da cita\c{c}\~ao a que se referem.

Para n\~ao prejudicar a leitura \'e recomendado evitar o emprego de
express\~oes latinas.


\section{Apresenta\c{c}\~ao de Autores no Texto}

As cita\c{c}\~oes devem ser indicadas no texto por um dos sistemas de chamada: autor-data ou num\'erico.

Qualquer que seja o sistema adotado deve ser seguido ao longo de todo o trabalho. 

Para a cita\c{c}\~ao, consideram-se como elementos identificadores: autoria pessoal, institucional ou entrada pela primeira palavra do t\'{\i}tulo em caso de autoria desconhecida e ano da publica\c{c}\~ao referida.

A forma da entrada do nome do autor pessoal ou institucional na cita\c{c}\~ao deve ser a mesma utilizada nas Refer\^encias ou em notas de rodap\'e.

Para a cita\c{c}\~ao direta \'e obrigat\'orio incluir o n\'umero da p\'agina.

Nas cita\c{c}\~oes as chamadas pelo sobrenome do autor, pela institui\c{c}\~ao respons\'avel ou pelo t\'{\i}tulo inclu\'{\i}do na senten\c{c}a ou entre par\^enteses devem estar em letras mai\'usculas e min\'usculas.

\subsection{Alternativas de formata\c{c}\~ao}
Nesse sistema, a indica\c{c}\~ao da fonte \'e feita da seguinte forma:

\begin{alineas}
	\item
	no caso de cita\c{c}\~ao direta, para obras com indica\c{c}\~ao de autoria ou responsabilidade. Pelo sobrenome de cada autor ou pelo nome da entidade respons\'avel, at\'e o primeiro sinal de pontua\c{c}\~ao, seguido da data de publica\c{c}\~ao do documento e da p\'agina de cita\c{c}\~ao, separados por v\'{\i}rgula e entre par\^enteses. Para as cita\c{c}\~oes indiretas o n\'umero das p\'aginas \'e opcional;
	\item
	no caso de cita\c{c}\~ao direta, para obras sem indica\c{c}\~ao de autoria ou responsabilidade. Pela primeira palavra do t\'{\i}tulo, seguida de retic\^encias, da data de publica\c{c}\~ao do documento e da(s) p\'agina(s) da cita\c{c}\~ao direta, separados por v\'{\i}rgula e entre par\^enteses. Para as cita\c{c}\~oes indiretas o n\'umero das p\'aginas \'e opcional;
	\item
	se o t\'{\i}tulo iniciar por artigo (definido ou indefinido), ou monoss\'{\i}labo, este deve ser inclu\'{\i}do na indica\c{c}\~ao da fonte.
	
\end{alineas}
	
\section{Exemplos de cita\c{c}\~oes}

Nesta se\c{c}\~ao s\~ao apresentados diversos exemplos de cita\c{c}\~oes diferenciando os elementos identificadores. 

\subsection{Um autor}

Pelo sobrenome\\

[\ldots] duas camadas t\^em ainda morfologia e fun\c{c}\~oes diferentes \cite{Pereira2013}

ou

\citeonline{Pereira2013} mostrou que duas camadas t\^em ainda morfologia e fun\c{c}\~oes diferentes.\\


\subsection{Dois autores}

Os sobrenomes dos autores entre par\^enteses devem ser separados por ponto e v\'{\i}rgula. Quando citados fora de par\^enteses devem ser separados pela letra “e”\\

[\ldots] \cite{Ramos2014} e de acordo com os resultados obtidos na investiga\c{c}\~ao [\ldots] 

ou 

\citeonline{Ramos2014} obtiveram os resultados de sua investiga\c{c}\~ao [\ldots] \\

\subsection{Tr\^es autores}

Os sobrenomes dos autores citados entre par\^enteses devem ser separados por ponto e v\'{\i}rgula. Quando citados fora de par\^enteses, os autores devem ser separados por v\'{\i}rgula sendo o \'ultimo separado pela letra “e”.\\

[\ldots] o acesso ao prot\'otipo \cite{Oliveira2013}

ou

Conforme \citeonline{Oliveira2013} o prot\'otipo [\ldots]\\

\subsection{Quatro ou mais autores}

Indicar o sobrenome do primeiro autor seguido da express\~ao latina \textit{et al.}\\

[\ldots]  com o grupo de jovens \cite{Sena2012}

ou

\citeonline{Sena2012} pesquisando um grupo de jovens [\ldots]\\

\subsection{Cita\c{c}\~oes consecutivas em Sistema Num\'erico}

Para agrupar a cita\c{c}\~ao num\'erica quando for consecutiva:

Adicionar o pacote “cite” junto aos demais pacotes listados inicialmente:

\verb+\usepackage{cite}+ \\

Ao citar a refer\^encia:

Para 2 refer\^encias consecutivas: 

\verb+\cite{bibtexkey}-\cite{bibtexkey}+ \\

Para 3 ou mais: 

\verb+~\cite{bibtexkey}+ \\

\subsection{Documentos de mesmo autor publicado no mesmo ano}

Quando houver coincid\^encia de trabalhos do mesmo autor publicados
no mesmo ano para identificar o trabalho citado acrescentar letras min\'usculas ap\'os o ano, sem espa\c{c}o.\\

[\ldots] \cite{Garcia2013b}   \textbf{\underline{outra obra}}   [\ldots] \cite{Garcia2013a} \\

ou\\

\citeonline{Garcia2013b}  \textbf{\underline{outra obra}}   \citeonline{Garcia2013a}

\subsection{Coincid\^encia de sobrenome e ano}

Quando houver coincid\^encia de sobrenome de autores com trabalhos
publicados no mesmo ano acrescentar as iniciais dos prenomes dos autores
para estabelecer diferen\c{c}as.\\

[\ldots] (CASTRO FILHO, C., \citeyear{CastroC2012}) \textbf{\underline{outra obra}}   [\ldots] (CASTRO FILHO, M., \citeyear{CastroC2012}) \\

ou\\

Castro Filho, C. (\citeyear{CastroC2012}) \textbf{\underline{outra obra}}    Castro Filho, M. (\citeyear{CastroC2012})

\subsection{Coincid\^encia de sobrenome, inicial do prenome e ano}

Usar os prenomes completos para estabelecer diferen\c{c}as. \\

 (SOUZA FILHO, Alberto \citeyear{Souza2015}) \textbf{\underline{outra obra}}   [\ldots] (SOUZA FILHO, Amauri, \citeyear{Souza2015}) \\


ou\\


Souza Filho, Alberto (\citeyear{Souza2015}) \textbf{\underline{outra obra}}   [\ldots] Souza Filho, Amauri, (\citeyear{Souza2015}) \\


\subsection{Autoria desconhecida}

Quando o documento n\~ao trouxer autoria expl\'{\i}cita citar pela primeira palavra do t\'{\i}tulo, seguida de retic\^encias e do ano de publica\c{c}\~ao.\\

[\ldots] \cite{Controle2015}\\

ou \\

De acordo com a publica\c{c}\~ao Controle [\ldots] (\citeyear{Controle2015}) estima-se em [\ldots]\\


\subsection{Entidades coletivas}

Citar pela forma em que aparece na refer\^encia.\\
\newpage

[\ldots] \cite{Sergipe2010}

ou 

A \citeonline{Sergipe2010} [\ldots] \\


[\ldots] \cite{Food2005}

ou 

A \citeonline{Food2005} [\ldots] \\


\subsection{Patentes}

Citar pela forma em que aparece na refer\^encia.\\

[\ldots] \cite{Bagnato2018}

ou 

Para \citeonline{Bagnato2018} [\ldots] \\


[\ldots] \cite{Rocha2017}

ou 

Para \citeonline{Rocha2017} [\ldots] \\


\subsection{Eventos}

Mencionar o nome completo do evento, seguido do ano de publica\c{c}\~ao.\\

\cite{reuniao1985}\\

ou\\

Os trabalhos apresentados na \citeonline{reuniao1985} [\ldots]\\

\subsection{V\'arios trabalhos da mesma autoria}

Ao citar v\'arios trabalhos de uma mesma autoria, publicados em anos distintos e mencionados simultaneamente, seguir a ordem cronol\'ogica, separando-os com v\'{\i}rgula.\\

[\ldots] (SMITH, \citeyear{Smith1990}, \citeyear{Smith1999}, \citeyear{Smith2002}) \\

ou\\

[\ldots] conforme afirmou Smith (\citeyear{Smith1990}, \citeyear{Smith1999}, \citeyear{Smith2002})\\


\subsection{V\'arios trabalhos de autorias diferentes}

Ao citar v\'arios trabalhos simultaneamente, de autorias diferentes, indicar
em ordem cronol\'ogica. Quando entre par\^enteses separados por ponto e
v\'{\i}rgula (;) e quando citados fora de par\^enteses, separados por v\'{\i}rgula (,) e pela
part\'{\i}cula “e”.\\

\citeonline{Ando1990,Ferreira1989,SilvaRibeiro2001}  estudaram [\ldots]\\
	
ou\\

[\ldots] \cite{Ando1990,Ferreira1989,SilvaRibeiro2001}  \\


\section{Comandos em \LaTeX\ para cita\c{c}\~oes}


No texto voc\^e deve inserir as cita\c{c}\~oes com os comandos relacionados abaixo:

\begin{alineas}
\item
\begin{verbatim}
\cite
\end{verbatim}

Utilizado para inserir o sobrenome do autor dentro de par\^enteses seguido da informa\c{c}\~ao do ano.

\textbf{Exemplos} 

\begin{verbatim}
\cite{Paula2001}
\end{verbatim}
\cite{Paula2001}

\begin{verbatim}
\cite{Demakopoulou2000}
\end{verbatim}
\cite{Demakopoulou2000}

\begin{verbatim}
\cite{PhillipiJunior2000}
\end{verbatim}
\cite{PhillipiJunior2000}

\begin{verbatim}
\cite{resprin1997}
\end{verbatim}
\cite{resprin1997}

\begin{verbatim}
\cite{saopaulo1963}
\end{verbatim}
\cite{saopaulo1963}

\begin{verbatim}
\cite{resolucao1991}
\end{verbatim}
\cite{resolucao1991}

\begin{verbatim}
\cite{codigo1985}
\end{verbatim}
\cite{codigo1985}

\begin{verbatim}
\cite{constituicao1988}
\end{verbatim}
\cite{constituicao1988}

\begin{verbatim}
\cite{buscopan2013}
\end{verbatim}
\cite{buscopan2013}

\begin{verbatim}
\cite{Pasquarelli1987}
\end{verbatim}
\cite{Pasquarelli1987}\\

\item
\begin{verbatim}
\citeonline
\end{verbatim}

\'E utilizado quando voc\^e menciona explicitamente o autor da refer\^encia na senten\c{c}a.

\textbf{Exemplos}

\begin{verbatim}
\citeonline{Novak1967}
\end{verbatim}
\citeonline{Novak1967}

\begin{verbatim}
\citeonline{Dood2002}
\end{verbatim}
\citeonline{Dood2002}

\begin{verbatim}
\citeonline{biblioteca1985}
\end{verbatim}
\citeonline{biblioteca1985}

\begin{verbatim}
\citeonline{usp2001}
\end{verbatim}
\citeonline{usp2001}

\begin{verbatim}
\citeonline{educacao2005}
\end{verbatim}
\citeonline{educacao2005}

\begin{verbatim}
\citeonline{brasil1981}
\end{verbatim}
\citeonline{brasil1981}

\begin{verbatim}
\citeonline{brasil1986}
\end{verbatim}
\citeonline{brasil1986}

\begin{verbatim}
\citeonline{Gomes1980}
\end{verbatim}
\citeonline{Gomes1980}\\

\item
\begin{verbatim}
\citeyear
\end{verbatim}

Apenas o \textbf{ano} da obra constar\'a do texto, suprimindo-se os outros dados presentes na cita\c{c}\~ao e os dados bibliogr\'aficos continuar\~ao constando da lista de refer\^encias. 

\textbf{Exemplos}

\begin{verbatim}
\citeyear{law1967}
\end{verbatim}
\citeyear{law1967}

\begin{verbatim}
\citeyear{Agencia2003}
\end{verbatim}
\citeyear{Agencia2003}

\begin{verbatim}
\citeyear{Dorlands2000}
\end{verbatim}
\citeyear{Dorlands2000}

\begin{verbatim}
\citeyear{abetter2004}
\end{verbatim}
\citeyear{abetter2004}

\begin{verbatim}
\citeyear{abetter2004}
\end{verbatim}
\citeyear{council2001}

\begin{verbatim}
\citeyear{Thome1999}
\end{verbatim}
\citeyear{Thome1999}


\begin{verbatim}
\citeyear{Brennan2006}
\end{verbatim}
\citeyear{Brennan2006}

\begin{verbatim}
\citeyear{microsoft1995}
\end{verbatim}
\citeyear{microsoft1995}\\

\item
\begin{verbatim}
\citeauthor
\end{verbatim}

Apenas o \textbf{sobrenome do autor} da obra constar\'a do texto em letras mai\'usculas, suprimindo-se os outros dados presentes na cita\c{c}\~ao e os dados bibliogr\'aficos continuar\~ao constando da lista de refer\^encias. 

{\tiny {\tiny }}\begin{verbatim}
\citeauthor{Piccini1996} 
\end{verbatim}
\citeauthor{Piccini1996} 

\begin{verbatim}
\citeauthor{Wendel1992}
\end{verbatim}
\citeauthor{Wendel1992}

\begin{verbatim}
\citeauthor{Elewa2006}
\end{verbatim}
\citeauthor{Elewa2006}

\begin{verbatim}
\citeauthor{Hofling1993}
\end{verbatim}
\citeauthor{Hofling1993}

%\begin{verbatim}
%\citeauthor{bule18}
%\end{verbatim}
%\cite{bule18}\\

\item
\begin{verbatim}
\citeauthoronline
\end{verbatim}

Apenas o \textbf{sobrenome do autor} da obra constar\'a do texto, suprimindo-se os outros dados presentes na cita\c{c}\~ao e os dados bibliogr\'aficos continuar\~ao constando da lista de refer\^encias.

\textbf{Exemplos}

\begin{verbatim}
\citeauthoronline{Fonseca2000}
\end{verbatim}
\citeauthoronline{Fonseca2000}

\begin{verbatim}
\citeauthoronline{bibliotecanacional2000}
\end{verbatim}
\citeauthoronline{bibliotecanacional2000}

\begin{verbatim}
\citeauthoronline{Demakopoulou2000}
\end{verbatim}
\citeauthoronline{Demakopoulou2000}

\begin{verbatim}
\citeauthoronline{GlasscockIII1987}
\end{verbatim}
\citeauthoronline{GlasscockIII1987}

\begin{verbatim}
\citeauthoronline{delvecchio1995}
\end{verbatim}
\citeauthoronline{delvecchio1995}

\begin{verbatim}
\citeauthoronline{brasil1990}
\end{verbatim}
\citeauthoronline{brasil1990}

\begin{verbatim}
\citeauthoronline{Herbrick1989}
\end{verbatim}
\citeauthoronline{Herbrick1989}

\begin{verbatim}
\citeauthoronline{Mostafavi2014}
\end{verbatim}
\citeauthoronline{Mostafavi2014}\\

\item
\begin{verbatim}
\citetext
\end{verbatim}

Imprime o conte\'udo da refer\^encia de uma cita\c{c}\~ao dentro do texto e tamb\'em na lista de refer\^encias. Ao utilizar a macro  \verb+\citetext+ ser\'a transcrito o conte\'udo da refer\^encia com a formata\c{c}\~ao padr\~ao do documento, ou seja com espa\c{c}amento entre as linhas de 1,5 cm e na lista de refer\^encias com espa\c{c}amento simples.

\textbf{Exemplos}

\begin{verbatim}
\citetext{Lacasse2005}
\end{verbatim}

\citetext{Lacasse2005} \\

Para alterar o espa\c{c}amento entre linhas da refer\^encia para simples dentro do documento \'e necess\'ario inserir o comando de formata\c{c}\~ao para espa\c{c}os simples \verb+\SingleSpacing+ conforme abaixo:

\begin{verbatim}
\begin{SingleSpace} 
\citetext{Lacasse2005}
\end{SingleSpace}
\end{verbatim}

\begin{SingleSpace} 
	\citetext{Lacasse2005}
\end{SingleSpace}

Os exemplos abaixo est\~ao formatados com espa\c{c}amento simples.

\begin{verbatim}
\begin{SingleSpace} 
\citetext{Palagachev2006}
\end{SingleSpace}
\end{verbatim}

\begin{SingleSpace} 
	\citetext{Palagachev2006}
\end{SingleSpace}

\begin{verbatim}
\begin{SingleSpace} 
\citetext{Zelen2000}
\end{SingleSpace}
\end{verbatim}

\begin{SingleSpace} 
	\citetext{Zelen2000}
\end{SingleSpace}

\begin{verbatim}
\begin{SingleSpace} 
\citetext{Boyd1993}
\end{SingleSpace}
\end{verbatim}

\begin{SingleSpace} 
	\citetext{Boyd1993}
\end{SingleSpace} 

\begin{verbatim}
\begin{SingleSpace} 
\citetext{Cochrane1998}
\end{SingleSpace}
\end{verbatim}

\begin{SingleSpace} 
	\citetext{Cochrane1998}
\end{SingleSpace} 

\begin{verbatim}
\begin{SingleSpace} 
\citetext{Oliveira2006}
\end{SingleSpace}
\end{verbatim}

\begin{SingleSpace} 
	\citetext{Oliveira2006}
\end{SingleSpace}

\begin{verbatim}
\begin{SingleSpace} 
\citetext{Harrison2001}
\end{SingleSpace}
\end{verbatim}

\begin{SingleSpace} 
	\citetext{Harrison2001}
\end{SingleSpace}

\begin{verbatim}
\begin{SingleSpace} 
\citetext{usp2006}
\end{SingleSpace}
\end{verbatim}

\begin{SingleSpace} 
	\citetext{usp2006}
\end{SingleSpace} 

\quad

\item
\begin{verbatim}
\Idem comando espec\'{\i}fico para mesmo autor
\Ibidem comando espec\'{\i}fico para mesma obra
\opcit comando espec\'{\i}fico para obra citada
\passim comando espec\'{\i}fico para aqui e al\'{\i}
\loccit comando espec\'{\i}fico para no lugar citado
\cfcite comando espec\'{\i}fico para confira
\etseq comando espec\'{\i}fico para e sequencia 
\end{verbatim} 

As express\~oes latinas podem ser usadas para evitar repeti\c{c}\~oes constantes de fontes citadas anteriormente. A primeira cita\c{c}\~ao de uma obra deve apresentar sua refer\^encia completa e as subsequentes podem aparecer sob forma abreviada. N\~ao usar destaque tipogr\'afico quando utilizar express\~oes latinas. As express\~oes latinas n\~ao devem ser usadas no texto, apenas em nota de rodap\'e, exceto apud. A presen\c{c}a da refer\^encia em nota de rodap\'e n\~ao dispensa sua inclus\~ao nas Refer\^encias, no final do trabalho. As express\~oes idem, ibidem, opus citatum, passim, loco citato, cf. e et seq. s\'o podem ser usadas na mesma p\'agina ou folha da cita\c{c}\~ao a que se referem. Para n\~ao prejudicar a leitura \'e recomendado evitar o emprego de express\~oes latinas.\\

\textbf{Exemplos}

\begin{verbatim}
\Idem[p.~491]{Abend2002}
\end{verbatim}
\Idem[p.~491]{Abend2002}

\begin{verbatim}
\Idem[p.~15]{tratados1999}
\end{verbatim}
\Idem[p.~15]{tratados1999}

\begin{verbatim}
\Idem[p.~18]{central1998}
\end{verbatim}
\Idem[p.~18]{central1998}

\begin{verbatim}
\Ibidem[p.~1]{Emenda1995}
\end{verbatim}
\Ibidem[p.~1]{Emenda1995}

\begin{verbatim}
\Ibidem[p.~15]{Paciornick1978}
\end{verbatim}
\Ibidem[p.~15]{Paciornick1978}

\begin{verbatim}
\Ibidem[p.~15]{atlas1981}
\end{verbatim}
\Ibidem[p.~35]{atlas1981}

\begin{verbatim}
\opcit[p.~23]{Denver1974}
\end{verbatim}
\opcit[p.~23]{Denver1974}

\begin{verbatim}
\opcit[p.~2]{Almeida1995}
\end{verbatim}
\opcit[p.~2]{Almeida1995}

\begin{verbatim}
\opcit[p.~3]{bionline}
\end{verbatim}
\opcit[p.~3]{bionline}

\begin{verbatim}
\passim{Villa-Lobos1916}
\end{verbatim}
\passim{Villa-Lobos1916}

\begin{verbatim}
\passim{Ramos1999}
\end{verbatim}
\passim{Ramos1999}

\begin{verbatim}
\passim{atlas2001}
\end{verbatim}
\passim{atlas2001}

\begin{verbatim}
\loccit{Wu1999}
\end{verbatim}
\loccit{Wu1999}

\begin{verbatim}
\loccit{Costa2002}
\end{verbatim}
\loccit{Costa2002}

\begin{verbatim}
\loccit{Geografico1986}
\end{verbatim}
\loccit{Geografico1986}

\begin{verbatim}
\cfcite[p.~2]{BRAYNER1994}
\end{verbatim}
\cfcite[p.~2]{BRAYNER1994}

\begin{verbatim}
\cfcite[p.~2]{Sabroza1998}
\end{verbatim}
\cfcite[p.~2]{Sabroza1998}

\begin{verbatim}
\cfcite[p.~46]{Oliva1900}
\end{verbatim}
\cfcite[p.~46]{Oliva1900}

\begin{verbatim}
\etseq[p.~2]{Montgomery1992}
\end{verbatim}
\etseq[p.~2]{Montgomery1992}

\begin{verbatim}
\etseq[p.~2]{Dudek2006}
\end{verbatim}
\etseq[p.~2]{Dudek2006}

\begin{verbatim}
\etseq[p.~2]{brasil1990b}
\end{verbatim}
\etseq[p.~2]{brasil1990b}

\end{alineas}


