%% USPSC-IntroducaoTutorial.tex

% ----------------------------------------------------------
% Introdu\c{c}\~ao (exemplo de cap\'{\i}tulo sem numera\c{c}\~ao, mas presente no Sum\'ario)
% ----------------------------------------------------------
\chapter[Introdu\c{c}\~ao]{Introdu\c{c}\~ao}
\label{Introdu\c{c}\~ao}

Parte inicial do texto, que cont\'em a delimita\c{c}\~ao do assunto tratado, objetivos da pesquisa e outros elementos necess\'arios para apresentar o tema do trabalho \cite{aguia2020}.

A equipe de desenvolvimento e manuten\c{c}\~ao do Pacote USPSC, atualmente na vers\~ao 3.1, contendo a Classes USPSC, tutorial e modelos para trabalhos acad\^emicos em \LaTeX\ utilizando a classe USPSC, foi estabelecida em abril de 2015. \'E integralmente composta por pessoas vinculadas \`as Bibliotecas das Unidades de ensino e pesquisa do Campus USP de S\~ao Carlos, incluindo a Biblioteca da Prefeitura do Campus USP de S\~ao Carlos (PUSP-SC), para garantir a sustentabilidade deste produto, tendo autonomia para implementar novos recursos, efetuar compatibiliza\c{c}\~oes necess\'arias em decorr\^encia de altera\c{c}\~oes de normas da ABNT e/ou normas e padr\~oes estabelecidos pelas comiss\~oes de p\'os-gradua\c{c}\~ao das Unidades, incluir novos programas de p\'os-gradua\c{c}\~ao das Unidades, dentre outras raz\~oes.

O Grupo Desenvolvedor do Pacote USPSC optou por manter os exemplos apresentados nas vers\~oes anteriores do tutorial, que s\~ao os relacionados no cap\'{\i}tulo \textbf{\ref{Refer\^encias} MODELOS DE REFER\^ENCIAS} do documento \textbf{Diretrizes para apresenta\c{c}\~ao de disserta\c{c}\~oes e teses da USP}: documento eletr\^onico e impresso - Parte I (ABNT), 3ª edi\c{c}\~ao de 2016, por\'em em conformidade com ABNT NBR 6023:2018. 

Atualmente a USP em S\~ao Carlos possui a Prefeitura do Campus USP de S\~ao Carlos (PUSP-SC), o Centro de Divulga\c{c}\~ao Cient\'{\i}fica e Cultural (CDCC) e as seguintes Unidades de ensino e pesquisa: Escola de Engenharia de S\~ao Carlos (EESC), Faculdade de Educa\c{c}\~ao, Faculdade de Educa\c{c}\~ao, Faculdade de Educa\c{c}\~ao.

Na vers\~ao 2.0 o Pacote USPSC passou a ser composto pela \textbf{Classe USPSC}, o \textbf{Modelo para TCC em \LaTeX\ utilizando a classe USPSC} e o \textbf{Modelo para teses e disserta\c{c}\~oes em \LaTeX\ utilizando a classe USPSC} para a EESC.

Na vers\~ao 3.0 do Pacote USPSC os modelos de trabalhos acad\^emicos \textbf{USPSC-modelo.tex} e \textbf{USPSC-TCC-modelo.tex} foram simplificados, no que tange ao conte\'udo, e foi criado o \textbf{Tutorial do Pacote USPSC para modelos de trabalhos de acad\^emicos em LaTeX - vers\~ao 3.0}, contendo as instru\c{c}\~oes precisas e detalhadas para melhor utiliza\c{c}\~ao dos recursos do Pacote USPSC. Para tanto, foram acrescidos diversos arquivos, para atender as especificidades do tutorial que possui os elementos pr\'e-textuais distintos para teses, disserta\c{c}\~oes, TCCs e outros trabalhos acad\^emicos, conforme descrito em  \textbf{\ref{Pacote} Pacote USPSC: Classe USPSC e modelos de trabalhos de acad\^emicos}. A estrutura deste tutorial \'e igual \`a  estrutura de trabalhos acad\^emicos estabelecida pela ABNT NBR 14724, conforme a \autoref{fig_EstruturaTrabAcad}.		

A vers\~ao 3.0 do Pacote USPSC traz ainda as seguintes altera\c{c}\~oes e implementa\c{c}\~oes:

\begin{alineas}	 
	\item foi alterada a estrutura da pasta para distribuir mais didaticamente os diversos arquivos que comp\~oem o referido pacote, conforme descrito em \ref{Pacote}; 
	\item foram criados os seguintes arquivos de elementos pr\'e-textuais: USPSC-Errata.tex, USPSC-Dedicatoria.tex, USPSC-Agradecimentos.tex, USPSC-Epigrafe.tex,\\
	USPSC-Resumo.tex, USPSC-Abstract.tex, USPSC-AbreviaturasSiglas.tex e USPSC-Simbolos.tex. 
	Tais informa\c{c}\~oes constavam diretamente dos arquivos \textbf{USPSC-modelo.tex} e  \textbf{USPSC-TCC-modelo.tex} e nesta vers\~ao passaram a ser inclu\'{\i}das atrav\'es do comando \verb+\include{nome do arquivo tex}+;
	\item implementa\c{c}\~ao do Modelo para TCC para o ICMC e IQSC, conforme descrito em \ref{Pacote};
	\item altera\c{c}\~ao do pacote utilizado para estruturas, rea\c{c}\~oes e mecanismos de rea\c{c}\~oes qu\'{\i}micas, conforme descrito em \ref{Reaquimica};
	\item altera\c{c}\~oes na Classe USPSC (USPSC.cls e USPSC1.cls):
		\begin{subalineas}
			\item foi adicionado o comando \verb+\ABNTEXcaptiondelim+ e alterado o separador de \textbf{captions} para \textbf{long dash} visando a compatibiliza\c{c}\~ao com a norma  ABNT NBR 14724:2011 e a conformidade com a classe \abnTeX\ v1.9.6;
			\item para incluir novos comandos e par\^ametros para possibilitar a impress\~ao de p\'agina de rosto adicional, atualmente adotada apenas pelo ICMC;
		\end{subalineas}
	\item altera\c{c}\~oes no arquivo USPSC-pre-textual-EESC.tex em decorr\^encia das altera\c{c}\~oes nos programas de p\'os-gradua\c{c}\~ao;
	\item altera\c{c}\~oes no arquivo USPSC-pre-textual-IFSC.tex para incluir op\c{c}\~oes de programas em ingl\^es;
	\item altera\c{c}\~oes no arquivo USPSC-pre-textual-ICMC.tex para incluir os comandos e par\^ametros referentes \`a p\'agina de rosto adicional;
	\item altera\c{c}\~oes no arquivo USPSC-Unidades.tex para incluir os comandos relativos aos novos Modelos de TCC;
	\item cria\c{c}\~ao dos arquivos USPSC-TCC-pre-textual-ICMC.tex e USPSC-TCC-pre-textual-IQSC.tex, necess\'arios para implementar o Modelo para TCC para o ICMC e IQSC;
	\item altera\c{c}\~ao no cap\'{\i}tulo \textbf{\ref{Refer\^encias} MODELOS DE REFER\^ENCIAS}, mantendo os exemplos contidos nas \textbf{Diretrizes para apresenta\c{c}\~ao de disserta\c{c}\~oes e teses da USP}: documento eletr\^onico e impresso - Parte I (ABNT), por\'em em conformidade com ABNT NBR 6023:2018; 
	\item inclus\~ao da alternativa de cores para os links nos arquivos \textbf{USPSC-modelo.tex} e \textbf{USPSC-TCC-modelo.tex}, conforme descrito em \ref{coreslinks} 
	\item altera\c{c}\~oes no arquivo \textbf{USPSC-modelo.tex} e nos demais arquivos \textbf{.tex} em conformidade com as altera\c{c}\~oes e implementa\c{c}\~oes efetuadas.	\\
\end{alineas}

	A vers\~ao 3.1 traz as altera\c{c}\~oes na Classe USPSC (USPSC.cls e USPSC1.cls) espec\'{\i}ficas para incluir novos par\^ametros para capa e tipo de publica\c{c}\~ao em ingl\^es.

	O Grupo Desenvolvedor do Pacote USPSC est\'a assim constitu\'{\i}do:

\textbf{Coordena\c{c}\~ao e Programa\c{c}\~ao}

- Marilza Aparecida Rodrigues Tognetti - marilza@sc.usp.br (PUSP-SC)	

- Ana Paula Aparecida Calabrez - aninha@sc.usp.br (PUSP-SC) 

\textbf{Normaliza\c{c}\~ao e Padroniza\c{c}\~ao}

- Ana Paula Aparecida Calabrez - aninha@sc.usp.br (PUSP-SC)

- Brianda de Oliveira Ordonho Sigolo - brianda@usp.br (IAU)

- Eduardo Graziosi Silva - edu.gs@sc.usp.br (EESC)

- Eliana de C\'assia Aquareli Cordeiro - eliana@iqsc.usp.br (IQSC)

- Fl\'avia Helena Cassin - cassinp@sc.usp.br (EESC)	

- Maria Cristina Cavarette Dziabas - mcdziaba@ifsc.usp.br (IFSC)	

- Marilza Aparecida Rodrigues Tognetti - marilza@sc.usp.br (PUSP-SC)

- Regina C\'elia Vidal Medeiros - rcvm@icmc.usp.br (ICMC)

	O objetivo do presente trabalho \'e apresentar a vers\~ao 3.1 do Pacote USPSC, composto pela \textbf{Classe USPSC}, \textbf{Tutorial do Pacote USPSC para modelos de trabalhos de acad\^emicos em LaTeX - vers\~ao 3.1},  \textbf{Modelo para TCC em \LaTeX\ utilizando a classe USPSC} e o \textbf{Modelo para teses e disserta\c{c}\~oes em \LaTeX\ utilizando a classe USPSC}, concebidos em conformidade com a \textbf{ABNT NBR 14724} \cite{nbr14724}, as \textbf{Diretrizes para apresenta\c{c}\~ao de disserta\c{c}\~oes e teses da USP} \cite{aguia2020} e normas e padr\~oes estabelecidos pelas Unidades. 
	
	A expectativa \'e que o Pacote USPSC, mediante os modelos propostos, proporcione o aprimoramento da qualidade dos trabalhos acad\^emicos produzidos pelos alunos de gradua\c{c}\~ao e de p\'os-gradua\c{c}\~ao das referidas Unidades de Ensino e Pesquisa do Campus USP de S\~ao Carlos, garantindo a normaliza\c{c}\~ao e padroniza\c{c}\~ao estabelecidas.
	
	