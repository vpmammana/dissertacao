%% UTFPRCT-TEX, v1.0.6 wmeira on 2021/10/11
%% Copyright (C) 2020- by William H. T. Meira
%%
%% Modified version of project 'utfprpgtex' maintained 
%% by Luiz E. M. Lima
%%
%% Alterado por Victor Mammana
%%
%% This work may be distributed and/or modified under the
%% conditions of the LaTeX Project Public License, either version 1.3
%% of this license or (at your option) any later version.
%% The latest version of this license is in
%%   http://www.latex-project.org/lppl.txt
%% and version 1.3 or later is part of all distributions of LaTeX
%% version 2005/12/01 or later.
%%
%% This work has the LPPL maintenance status `maintained'.
%%
%% The Current Maintainer of this work is William H. T. Meira.
%%
%% This project consists mainly of files: 'utfprct.cls', 'utfprct.tex', 
%% 'utfprct-dados.tex' 
%% 
%% The 'abntex2-alf.bst' and 'abntex2-num.bst' files are slightly
%% modified versions of the bibtex styles from abntex2 (v.1.9.7)
%% package to suit NBR6023/2018 (not yet implemented there). 
%% Complementary, 'abntex2-alf-en.bst' and 'abntex2-num-en.bst' are
%% english versions of the respective bibtex styles.
%%
%% Contribute to improve this project (github repo):
%% https://github.com/wmeira/utfprct-tex

%%%%%%%%%%%%%%%%%%%%%%%%%%%%%%%%%%%%%%%%%%%%%%%
%%%%%%%%%%%%%%%%%%%%%%%%%%%%%%%%%%%%%%%%%%%%%%%
%% Tutorial do Documento de Dados 
%%%%%%%%%%%%%%%%%%%%%%%%%%%%%%%%%%%%%%%%%%%%%%%
%%
%% O 'utfprct-dados.tex' cont\'em todos as informa\c{c}\~oes do documento, 
%% metadados e outros valores importantes para o preenchimento dos 
%% elementos pr\'e-textuais: capa, folha de rosto, resumo, abstract.
%% 
%% N\~ao \'e necess\'ario preencher todos os campos, existem campos mais
%% espec\'{\i}ficos para o tipo de documento sendo elaborado. Quando TCC,
%% por exemplo, n\~ao \'e necess\'ario definir dados do programa de 
%% p\'os-gradua\c{c}\~ao. Todos os dados inseridos estar\~ao dispon\'{\i}veis para
%% inser\c{c}\~ao no tex usando o padr\~ao '\imprimir + nomedodadominusculo'. 
%% Por exemplo, para imprimir o tipo do documento ('\TipoDeDocumento'),
%% usar '\imprimirtipodedocumento'.
%%
%% \'E poss\'{\i}vel customizar a descri\c{c}\~ao do documento na folha de rosto. 
%% Exemplos de descri\c{c}\~ao s\~ao fornecidos para guiar a escrita do texto,
%% em que pode-se utilizar os dados j\'a definidos com o padr\~ao descrito.
%%
%% Existe a possibilidade de inserir dados de uma institui\c{c}\~ao de 
%% cotutela quando se aplicar. 
%%
%% Os dados da ficha catalogr\'afica s\~ao fornecidos pela biblioteca
%% e, na maioria das vezes, apenas anexa-se a folha digitalizada
%% na regi\~ao definida dos elementos pr\'e-textuais.
%%
%% A folha de aprova\c{c}\~ao \'e, na maioria das vezes, fornecido 
%% digitalmente pelo departamento ou pelo orientador ap\'os a defesa e 
%% dever\'a ser anexada ao documento na regi\~ao definida dos elementos 
%% pr\'e-textuais. Recomenda-se apenas anexar a folha de aprova\c{c}\~ao sem
%% precisar alterar os dados espec\'{\i}ficos aqui presentes, pois foram
%% criados originalmente para o template da folha de aprova\c{c}\~ao da
%% UTFPR-PG, presente no projeto base, e foram apenas mantidos.
%%
%%%%%%%%%%%%%%%%%%%%%%%%%%%%%%%%%%%%%%%%%%%%%%%
%%%%%%%%%%%%%%%%%%%%%%%%%%%%%%%%%%%%%%%%%%%%%%%

%%%%%%%%%%%%%%%%%%%%%%%%%%%%%%%%%%%%%%%%%%%%%%%
%% Informa\c{c}\~oes do Documento
%%%%%%%%%%%%%%%%%%%%%%%%%%%%%%%%%%%%%%%%%%%%%%%

%% Tipo de documento (op\c{c}\~oes: "Tese", "Disserta\c{c}\~ao", "Trabalho de Conclus\~ao de Curso"
\TipoDeDocumento{Disserta\c{c}\~ao}%copiar exatamente uma das op\c{c}\~oes 

%% [abstract] Document type: "Thesis", "Dissertation", "Bachelor Thesis"
\DocumentType{Dissertation}%

%% N\'{\i}vel de forma\c{c}\~ao: "Doutorado", "Mestrado", "Bacharelado"
\NivelDeFormacao{Mestrado}%

%% [abstract] Formation level: "Doctorate", "PhD", "Master's Degree", "Bachelor's Degree"
\FormationLevel{Msc.}%

%% T\'{\i}tulo ou grau pretendido: "Doutor", "Mestre" ou "Bacharel"
\TituloPretendido{Mestre}%

%%%%%%%%%%%%%%%%%%%%%%%%%
%% Por padr\~ao, o t\'{\i}tulo principal ser\'a o \TituloDoDocumento (portugu\^es)
%% Se a l\'{\i}ngua do documento for definida como ingl\^es, ser\'a o \DocumentTitle

%% T\'{\i}tulo do documento em PORTUGU\^ES (resumo)
\TituloDoDocumento{%%
HISTÓRIA E CARACTERIZAÇÃO DE 10 ANOS DO WASH, UM PROGRAMA HETERÁRQUICO DE APRENDIZAGEM STEAM %% : HISTÓRIA, MÉTODOS E RESULTADOS coloque se tiver subtitulo
}%

%% T\'{\i}tulo do trabalho em INGL\^ES (abstract)
\DocumentTitle{%%
HISTORY AND CHARACTERIZATION OF A DECADE OF WASH: A HETERARCHIAL STEAM EDUCATION PROGRAM
}%

%% T\'{\i}tulo em m\'ultiplas linhas na capa, folha de rosto e termo de aprova\c{c}\~ao
%% Use o comando \par para indicar a quebra de linha
%%
%% A CAPA apresentar\'a o t\'{\i}tulo no formato de m\'ultiplas linhas
%% para a respectiva l\'{\i}ngua do documento definida.
%%
%% Na folha de rosto, caso a l\'{\i}ngua do documento n\~ao seja portugu\^es, aparecer\'a o
%% \TituloEmMultiplasLinhasIngles seguido pela tradu\c{c}\~ao \TituloEmMultiplasLinhas
%%
\TituloEmMultiplasLinhas{%%
HISTÓRIA E CARACTERIZAÇÃO DE 10 ANOS DO WASH, UM PROGRAMA HETERÁRQUICO DE APRENDIZAGEM STEAM %%\par coloque se tiver subtitulo
%% HISTÓRIA, MÉTODOS E RESULTADOS coloque se tiver subtitulo
}%

\TituloEmMultiplasLinhasIngles{%%
HISTORY AND CHARACTERIZATION OF A DECADE OF WASH: A HETERARCHIAL STEAM EDUCATION PROGRAM %% \par coloque se tiver subtitulo
%%subtitle of this academic work
}%

%%%%%%%%%%%%%%%%%%%%%%%%%


%% Data da defesa
\Dia{10}%% Dia (opcional: usado na ficha catalogr\'afica apenas)
\MesPorExtenso{mar\c{c}o}%% m\^es por extenso (opcional: usado na ficha catalogr\'afica apenas)
\Ano{2023}%% Ano

%% Palavras-chave e keywords (m\'aximo 5)
\NumeroDePalavrasChave{5}%% N\'umero de palavras-chave 
\PalavraChaveA{Papert}%% Palavra-chave A
\PalavraChaveB{STEAM}%% Palavra-chave B
\PalavraChaveC{STEM}%% Palavra-chave C
\PalavraChaveD{WASH}%% Palavra-chave D
\PalavraChaveE{Educa\c{c}\~ao}%% Palavra-chave E

\NumeroDeKeywords{\imprimirnumerodepalavraschave}%% N\'umero de keywords (mesmo que palavras-chave)
\KeywordA{Papert}%% Keyword A
\KeywordB{STEAM}%% Keyword B
\KeywordC{STEM}%% Keyword C
\KeywordD{WASH}%% Keyword D
\KeywordE{Education}%% Keyword E

%%%%%%%%%%%%%%%%%%%%%%%%%%%%%%%%%%%%%%%%%%%%%%%
%% Informa\c{c}\~ao do Autor(a) ou Autores(as) (TCC)
%%%%%%%%%%%%%%%%%%%%%%%%%%%%%%%%%%%%%%%%%%%%%%%

%%% Autor(a)
%% Usado para cita\c{c}\~ao: "\SobrenomeDoAutor, PrenomeDoAutor" (ex: "Doe, John" ou "Doe, J.")
\NomeDoAutor{ELAINE DA SILVA TOZZI}%% Nome completo do(a) autor(a)
\SobrenomeDoAutor{TOZZI}%% \'Ultimo nome do(a) autor(a)
\PrenomeDoAutor{ELAINE DA SILVA}%% Nome do(a) autor(a) sem \'ultimo nome

%%% Autor(a) 2 (opcional)
%% *Considera apenas se "\TipoDeDocumento" == "Trabalho de Conclus\~ao de Curso"
\AtribuiAutorDois{false}%% Insere ou remove autor(a) 2: "true" ou "false"
\NomeDoAutorDois{Nome do(a) Autor(a) 2}%% Nome completo do(a) autor(a) 2
\SobrenomeDoAutorDois{\'Ultimo Nome}%% \'Ultimo nome do(a) autor(a) 2
\PrenomeDoAutorDois{Nome do(a) Autor(a) 2 Sem \'Ultimo}%% Nome do(a) autor(a) 2 sem \'ultimo nome

%%% Autor(a) 3 (opcional)
%% *Considera apenas se "\TipoDeDocumento" == "Trabalho de Conclus\~ao de Curso"
\AtribuiAutorTres{false}%% Insere ou remove autor(a) 3: "true" ou "false"
\NomeDoAutorTres{Nome do(a) Autor(a) 3}%% Nome completo do(a) autor(a) 3
\SobrenomeDoAutorTres{\'Ultimo Nome}%% \'Ultimo nome do(a) autor(a) 3
\PrenomeDoAutorTres{Nome do(a) Autor(a) 3 Sem \'Ultimo}%% Nome do(a) autor(a) 3 sem \'ultimo nome

%%%%%%%%%%%%%%%%%%%%%%%%%%%%%%%%%%%%%%%%%%%%%%%%%%
%% Informa\c{c}\~oes do Orientador(a) e Coorientador(a)
%%%%%%%%%%%%%%%%%%%%%%%%%%%%%%%%%%%%%%%%%%%%%%%%%%

%% Orientador(a)
%% Usado para cita\c{c}\~ao: "\SobrenomeDoOrientador, PrenomeDoOrientador" (ex: "Doe, John" ou "Doe, J.")
\AtribuicaoOrientador{Orientador(a)}%% Atribui\c{c}\~ao "Orientador(a)"
\TituloDoOrientador{Prof(a). Dr(a).}%% T\'{\i}tulo do(a) orientador(a)
\NomeDoOrientador{Prof. Dr. Paulo Sérgio de Camargo Filho}%% Nome completo do(a) orientador(a)
\SobrenomeDoOrienador{Filho}%% \'Ultimo nome do(a) orientador(a)
\PrenomeDoOrientador{Prof. Dr. Paulo Sérgio de Camargo}%% Nome do(a) orientador(a) sem \'ultimo nome

%% Coorientador(a) (opcional)
%% Usado para cita\c{c}\~ao: "\SobrenomeDoCoorientador, PrenomeDoCoorientador" (ex: "Doe, John" ou "Doe, J.")
\AtribuiCoorientador{true}%% Insere ou remove o(a) coorientador(a): "true" ou "false"
\AtribuicaoCoorientador{Coorientador(a)}%% Atribui\c{c}\~ao "Coorientador(a)"
\TituloDoCoorientador{Prof(a). Dr(a).}%% T\'{\i}tulo do(a) coorientador(a)
\NomeDoCoorientador{Dr. Victor Pellegrini Mammana}%% Nome completo do(a) coorientador(a)
\SobrenomeDoCoorienador{Mammana}%% \'Ultimo nome do(a) coorientador(a)
\PrenomeDoCoorientador{Dr. Victor Pellegrini}%% Nome do(a) coorientador(a) sem \'ultimo nome

%%%%%%%%%%%%%%%%%%%%%%%%%%%%%%%%%%%%%%%%%%%%%%%%%%
%% Informa\c{c}\~oes da Institui\c{c}\~ao
%%%%%%%%%%%%%%%%%%%%%%%%%%%%%%%%%%%%%%%%%%%%%%%%%%

%% Nome da institui\c{c}\~ao
\Instituicao{Universidade Tecnológica Federal do Paraná}

%% [abstract] Institution name (*nome sem traduzir \'e o recomendado para docs. da UTFPR)
\Institution{Universidade Tecnológica Federal do Paraná}

%% Sigla da Institui\c{c}\~ao
\SiglaInstituicao{UTFPR}

%% Nome da cidade (c\^ampus)
\Cidade{LONDRINA}

%% Diretoria: "Gradua\c{c}\~ao e Educa\c{c}\~ao Profissional" ou "Pesquisa e P\'os-Gradua\c{c}\~ao" (opcional)
\Diretoria{Pesquisa e P\'os-Gradua\c{c}\~ao}

%% Nome do departamento ou da coordena\c{c}\~ao (opcional: mais comum no Bacharelado: Departamento de Inform\'atica)
\Departamento{Programa de Pós-Graduação em Ensino de Ciências Humanas, Sociais e da Natureza - PPGEN}

%% Sigla do departamento (opcional, ex: DAINF, DAMEC, DAMAT...)
\SiglaDepartamento{PPGEN}

%% Nome do curso bachalerado ou p\'os-gradua\c{c}\~ao (PPG) (ex: "Engenharia de Computa\c{c}\~ao",  "Engenharia El{\'e}trica e Inform{\'a}tica Industrial") 
\Curso{Ensino de Ciências Humanas, Sociais e da Natureza}

%% [abstract] Course name
\Course{Teaching Social, Human and Natural Sciences}

%% Programa ou nome do curso completo (capa)
%% "Bachalerado em Engenharia de Computa\c{c}\~ao"
%% "Programa de Pós-Graduação em Ensino de Ciências Humanas, Sociais e da Natureza - PPGEN
\Programa{Programa de Pós-Graduação em Ensino de Ciências Humanas, Sociais e da Natureza - PPGEN}

%% Sigla do programa de p\'os-gradua\c{c}\~ao (opcional, ex: CPGEI, PPGCA)
\SiglaDoPPG{PPGEN}

%% Nome da \'area de concentra\c{c}\~ao
\AreaDeConcentracao{Educa\c{c}\~ao}

%%%%%%%%%%%%%%%%%%%%%%%%%%%%%%%%%%%%%%%%%%%%%%%%%%%%%%
%% Informa\c{c}\~oes de Cotutela (Duplo Grau) (opcional)
%%%%%%%%%%%%%%%%%%%%%%%%%%%%%%%%%%%%%%%%%%%%%%%%%%%%%%

%% Insere dados de cotutela: "true" ou "false"
\AtribuiCotutela{false}

%% Nome da institui\c{c}\~ao de cotutela
\InstituicaoCotutela{Universidade Da Cotutela}

%% [abstract] Institution name
\InstitutionCotutela{Double Degree University}

%% Sigla da institui\c{c}\~ao de cotutela
\SiglaInstituicaoCotutela{UC}

%% Nome do departamento ou da coordena\c{c}\~ao da inst. de cotutela (mais comum no Bacharelado: Departamento de Inform\'atica)
\DepartamentoCotutela{Nome do Departamento ou da Coordena\c{c}\~ao}

%% Sigla do departamento da inst. cotutela (ex: DAINF, DAMEC, DAMAT...)
\SiglaDepartamentoCotutela{DPT-EXT}

%% Nome do curso bachalerado ou p\'os-gradua\c{c}\~ao (PPG) na institui\c{c}\~ao de cotutela (ex: "Engenharia de Computa\c{c}\~ao",  "Engenharia El{\'e}trica e Inform{\'a}tica Industrial")
\CursoCotutela{Nome do @[curso]@}

%% [abstract] Course name 
\CourseCotutela{Second Degree Course}

%% Programa ou nome do curso completo na inst. cotutela (capa)
%% "Bachalerado em Engenharia de Computa\c{c}\~ao"
%% "Programa de Pós-Graduação em Ensino de Ciências Humanas, Sociais e da Natureza - PPGEN
\ProgramaCotutela{Programa de Doutoral em \imprimircursocotutela}

%% Sigla do programa externo de p\'os-gradua\c{c}\~ao
\SiglaDoPPGCotutela{PPG-EXT}

%% Nome da \'area de concentra\c{c}\~ao na institui\c{c}\~ao de cotutela
\AreaDeConcentracaoCotutela{Nome da \'Area de Concentra\c{c}\~ao}

%% N\'{\i}vel de forma\c{c}\~ao que ser\'a fornecido na refer\^encia do doc.
\NivelDeFormacaoResumo{Duplo doutorado}
\FormationLevelAbstract{Double PhD}

%% Informacoes do orientador(a) na institui\c{c}\~ao de cotutela

%% Orientador(a) da institui\c{c}\~ao de cotutela
%% Usado para cita\c{c}\~ao: "\SobrenomeDoOrientador, PrenomeDoOrientador" (ex: "Doe, John" ou "Doe, J.")
\AtribuicaoOrientadorCotutela{Orientador(a)}%% Atribui\c{c}\~ao "Orientador(a)"
\TituloDoOrientadorCotutela{Prof(a). Dr(a).}%% T\'{\i}tulo do(a) orientador(a)
\NomeDoOrientadorCotutela{Nome Completo do(a) Orientador(a)}%% Nome completo do(a) orientador(a)
\SobrenomeDoOrienadorCotutela{\'Ultimo Nome}%% \'Ultimo nome do(a) orientador(a)
\PrenomeDoOrientadorCotutela{Nome do(a) Orientador(a) Sem \'Ultimo}%% Nome do(a) orientador(a) sem \'ultimo nome

%% Coorientador(a) da institui\c{c}\~ao de cotutela
%% Usado para cita\c{c}\~ao: "\SobrenomeDoCoorientador, PrenomeDoCoorientador" (ex: "Doe, John" ou "Doe, J.")
\AtribuiCoorientadorCotutela{false}%% Insere ou remove o(a) coorientador(a) da cotutela: "true" ou "false"
\AtribuicaoCoorientadorCotutela{Coorientador(a)}%% Atribui\c{c}\~ao "Coorientador(a)"
\TituloDoCoorientadorCotutela{Prof(a). Dr(a).}%% T\'{\i}tulo do(a) coorientador(a)
\NomeDoCoorientadorCotutela{Nome Completo do(a) Coorientador(a)}%% Nome completo do(a) coorientador(a)
\SobrenomeDoCoorienadorCotutela{\'Ultimo Nome}%% \'Ultimo nome do(a) coorientador(a)
\PrenomeDoCoorientadorCotutela{Nome do(a) Coorientador(a) Sem \'Ultimo}%% Nome do(a) coorientador(a) sem \'ultimo nome

%%%%%%%%%%%%%%%%%%%%%%%%%%%%%%%%%%%%%%%%%%%%%%%%%%
%% Folha de Rosto
%%%%%%%%%%%%%%%%%%%%%%%%%%%%%%%%%%%%%%%%%%%%%%%%%%

%% Se desejar usar os dados inseridos, eles est\~ao dispon\'{\i}ves
%% usando o padr\~ao '\imprimir + nomedodadominusculo'. Por exemplo,
%% para imprimir o tipo do documento ('\TipoDeDocumento'), usar
%% '\imprimirtipodedocumento'

%% Descri\c{c}\~ao do documento na folha de rosto (exemplos)
\DescricaoDoDocumento{
\imprimirtipodedocumento\ apresentado(a) como requisito para obten\c{c}\~ao do t\'{\i}tulo(grau) de \imprimirtitulopretendido\ em \imprimircurso, do \imprimirppgoudepartamento, da \imprimirinstituicao\ (\imprimirsiglainstituicao).
}

% Exemplo: Mestrado
%\DescricaoDoDocumento{
%\imprimirtipodedocumento\ apresentada como requisito para obten\c{c}\~ao do grau de \imprimirtitulopretendido\ em \imprimircurso\ da \imprimirinstituicao\ (\imprimirsiglainstituicao).
%}

% Exemplo: Doutorado
%\DescricaoDoDocumento{
%\imprimirtipodedocumento\ apresentada como requisito para obten\c{c}\~ao do t\'{\i}tulo de \imprimirtitulopretendido\ em \imprimircurso\ da \imprimirinstituicao\ (\imprimirsiglainstituicao).
%}


%% Insere ou remove descri\c{c}\~ao da cotutela (extra) na folha de rosto: "true" ou "false". 
%% Se "true", a descri\c{c}\~ao do documento ser\'a colocada na folha de rosto, logo abaixo do orientador(a) e coorientador(a) da primeira inst. e depois o orientador(a) e coorientador(a) da inst. de cotutela. 
%% Se "false", os nomes do orientador(a) e coorientador(a) aparecer\~ao logo abaixo do orientador(a) da primeira institui\c{c}\~ao, sem uma descri\c{c}\~ao extra. Neste caso, recomenda-se revisar a "\DescricaoDoDocumento" para contemplar ambas as institui\c{c}\~oes.   
\AtribuiDescricaoCotutela{false}

%% Segunda Descricao da Inst. de Cotutela na folha de rosto (exemplos)
\DescricaoDoDocumentoCotutela{
\imprimirtipodedocumento\ apresentado(a) como requisito \`a obten\c{c}\~ao do t\'{\i}tulo de \imprimirtitulopretendido\ em \imprimircursocotutela, do \imprimirppgoudepartamentocotutela, da \imprimirinstituicaocotutela.  
%\imprimirtipodedocumento\ apresentada \`a Comiss\~ao de Acompanhamento de Tese do Programa Doutoral em \imprimircursocotutela\ do \imprimirinstituicaocotutela\ (\imprimirsiglainstituicaocotutela) como requisito \`a obten\c{c}\~ao de grau de \imprimirtitulopretendido\ na \'area de concentra\c{c}\~ao \imprimirareadeconcentracaocotutela.
}

%%%%%%%%%%%%%%%%%%%%%%%%%%%%%%%%%%%%%%%%%%%%%%%%%%
%% Ficha Catalogr\'afica* (opcional)
%%%%%%%%%%%%%%%%%%%%%%%%%%%%%%%%%%%%%%%%%%%%%%%%%%

%% *Pode ser usado como placeholder, por\'em para entrega deve-se inserir a ficha catologr\'afica digitalizado (PDF) pela biblioteca da UTFPR.

\NumeroDaPublicacao{00/\imprimirano}%% N\'umero da publica\c{c}\~ao - Fornecido pela biblioteca
\CDDOuCDU{CDD 000.00}%% Classifica\c{c}\~ao Decimal Dewey (CDD) ou Classifica\c{c}\~ao Decimal Universal (CDU) - Fornecida pela biblioteca

\TituloDaFichaCatalografica{%% T\'{\i}tulo da ficha catalogr\'afica
  Ficha catalogr\'afica elaborada pelo Departamento de Biblioteca da \par \imprimirinstituicao, C\^ampus \imprimircidade \par n.
  \imprimirnumerodapublicacao
}

%%%%%%%%%%%%%%%%%%%%%%%%%%%%%%%%%%%%%%%%%%%%%%%%%%
%% Folha de aprova\c{c}\~ao (Formato UTFPR-PG)* (opcional)
%%%%%%%%%%%%%%%%%%%%%%%%%%%%%%%%%%%%%%%%%%%%%%%%%%

%% *Pode ser usado como placeholder, por\'em para entrega final prefere-se a folha (termo) de aprova\c{c}\~ao digitalizado (PDF) fornecido pelo departamento ou orientador.

\NumeroDaTeseOuDissertacao{00/\imprimirano}%% N\'umero da Tese ou Disserta\c{c}\~ao - Fornecido pelo programa de p\'os-gradua\c{c}\~ao
\NumeroDaFichaCatalografica{A000}%% N\'umero da ficha catalogr\'afica - Fornecido pela biblioteca

\TituloDoResponsavelTCC{Prof(a). Dr(a).}%% T\'{\i}tulo do(a) respons\'avel pelos TCC
\NomeDoResponsavelTCC{Nome do(a) Respons\'avel}%% Nome completo do(a) respons\'avel pelos TCC
\AtribuicaoCoordenador{Coordenador(a)}%% Atribui\c{c}\~ao "Coordenador(a)" do curso
\TituloDoCoordenador{Prof(a). Dr(a).}%% T\'{\i}tulo do(a) coordenador(a) do curso
\NomeDoCoordenador{Nome do(a) Coordenador(a)}%% Nome completo do(a) coordenador(a) do curso

\TextoDeAprovacao{%% Texto de aprova\c{c}\~ao
  %% Exemplo de texto de aprova\c{c}\~ao para Tese ou Disserta\c{c}\~ao (descomente a pr\'oxima linha para utiliz\'a-lo):
  Esta \imprimirtipodedocumento\ foi apresentada \`as 00:00 de \imprimirdia\ de \imprimirmesporextenso\ de \imprimirano\ como requisito parcial para a obten\c{c}\~ao do t\'{\i}tulo de \imprimirtitulopretendido\ em \imprimircurso, na \'area de concentra\c{c}\~ao em \imprimirareadeconcentracao\ e na linha de pesquisa em (Nome da Linha de Pesquisa), do Programa de Pós-Graduação em Ensino de Ciências Humanas, Sociais e da Natureza - PPGEN. O(A) candidato(a) foi arguido(a) pela Banca Examinadora composta pelos professores abaixo citados. Ap\'os delibera\c{c}\~ao, a Banca Examinadora considerou o trabalho aprovado.
  %% Exemplo de texto de aprova\c{c}\~ao para Trabalho de Conclus\~ao de @[curso]@\'oxima linha para utiliz\'a-lo):
  % Este \imprimirtipodedocumento\ foi apresentado em \imprimirdia\ de \imprimirmesporextenso\ de \imprimirano\ como requisito parcial para a obten\c{c}\~ao do t\'{\i}tulo de \imprimirtitulopretendido\ em \imprimircurso. O(A) candidato(a) foi arguido(a) pela Banca Examinadora composta pelos professores abaixo assinados. Ap\'os delibera\c{c}\~ao, a Banca Examinadora considerou o trabalho aprovado.
}

\AvisoDeAprovacao{%% Aviso de aprova\c{c}\~ao
  %% Exemplo de aviso de aprova\c{c}\~ao para Tese ou Disserta\c{c}\~ao (descomente a pr\'oxima linha para utiliz\'a-lo):
  A Folha de Aprova\c{c}\~ao assinada encontra-se no \par Departamento de Registros Acad\^emicos da UTFPR -- C\^ampus \imprimircidade
  %% Exemplo de aviso de aprova\c{c}\~ao para Trabalho de Conclus\~ao de @[curso]@\'oxima linha para utiliz\'a-lo):
  % -- O Termo de Aprova\c{c}\~ao assinado encontra-se na Coordena\c{c}\~ao do @[curso]@--
}

%% Banca examinadora: 3 membros (Trabalho de Conclus\~ao de @[curso]@\c{c}\~ao); 5 a 7 membros (Tese)
\MembroAIgualOrientador{true}%% Insere ou remove o membro A igual ao(\`a) orientador(a): "true" ou "false"
\MembroA{Nome do Membro A}%% Nome completo do membro A - Presidente (autom\'atico se orientador(a))
\TituloDoMembroA{Prof(a). Dr(a).}%% T\'{\i}tulo do membro A - Presidente (autom\'atico se orientador(a))
\InstituicaoDoMembroA{Institui\c{c}\~ao do Membro A}%% Nome da institui\c{c}\~ao do membro A - Presidente (autom\'atico se orientador(a))
\MembroB{Nome do Membro B}%% Nome completo do membro B
\TituloDoMembroB{Prof(a). Dr(a).}%% T\'{\i}tulo do membro B
\InstituicaoDoMembroB{Institui\c{c}\~ao do Membro B}%% Nome da institui\c{c}\~ao do membro B
\MembroC{Nome do Membro C}%% Nome completo do membro C
\TituloDoMembroC{Prof(a). Dr(a).}%% T\'{\i}tulo do membro C
\InstituicaoDoMembroC{Institui\c{c}\~ao do Membro C}%% Nome da institui\c{c}\~ao do membro C
\MembroD{Nome do Membro D}%% Nome completo do membro D
\TituloDoMembroD{Prof(a). Dr(a).}%% T\'{\i}tulo do membro D
\InstituicaoDoMembroD{Institui\c{c}\~ao do Membro D}%% Nome da institui\c{c}\~ao do membro D
\MembroE{Nome do Membro E}%% Nome completo do membro E
\TituloDoMembroE{Prof(a). Dr(a).}%% T\'{\i}tulo do membro E
\InstituicaoDoMembroE{Institui\c{c}\~ao do Membro E}%% Nome da institui\c{c}\~ao do membro E
\AtribuiMembroF{false}%% Insere ou remove o Membro F: "true" ou "false"
\MembroF{Nome do Membro F}%% Nome completo do membro F
\TituloDoMembroF{Prof(a). Dr(a).}%% T\'{\i}tulo do membro F
\InstituicaoDoMembroF{Institui\c{c}\~ao do Membro F}%% Nome da institui\c{c}\~ao do membro F
\AtribuiMembroG{false}%% Insere ou remove o Membro G: "true" ou "false"
\MembroG{Nome do Membro G}%% Nome completo do membro G
\TituloDoMembroG{Prof(a). Dr(a).}%% T\'{\i}tulo do membro G
\InstituicaoDoMembroG{Institui\c{c}\~ao do Membro G}%% Nome da institui\c{c}\~ao do membro G
