%% USPSC-Cap2-Desenvolvimento.tex 

% ---
% Este cap\'{\i}tulo, utilizado por diferentes exemplos do abnTeX2, ilustra o uso de
% comandos do abnTeX2 e de LaTeX.
% ---

\chapter{Desenvolvimento}\label{cap_exemplos}
Este cap\'{\i}tulo \'e parte principal do trabalho acad\^emico e deve conter a exposi\c{c}\~ao ordenada e detalhada do assunto. Divide-se em se\c{c}\~oes e subse\c{c}\~oes, em conformidade com a abordagem do tema e do m\'etodo, abrangendo: revis\~ao bibliogr\'afica, materiais e m\'etodos, t\'ecnicas utilizadas, resultados obtidos e discuss\~ao.

Abaixo s\~ao apresentados minimamente exemplos tabelas, quadros, divis\~oes de documentos e outros itens. Consulte o \textbf{Tutorial do Pacote USPSC para modelos de trabalhos de acad\^emicos em LaTeX - vers\~ao 3.1} para demais informa\c{c}\~oes. 

\section{Resultados de comandos}\label{sec-divisoes}

% ---
\subsection{Tabelas e quadros}

O \textbf{Tutorial do Pacote USPSC para modelos de trabalhos de acad\^emicos em LaTeX - vers\~ao 3.1} apresenta orienta\c{c}\~oes completas e diversas formata\c{c}\~oes de tabelas, dentre elas a \autoref{tab-ibge}, que \'e um exemplo de tabela alinhada que pode ser longa ou curta, conforme padr\~ao do Instituto Brasileiro de Geografia e Estat\'{\i}stica (IBGE).

%\begin{table}[H]
\begin{table}[htb]
	\IBGEtab{%
		\caption{Frequ\^encia anual por categoria de usu\'arios}%
		\label{tab-ibge}
	}{%
		\begin{tabular}{ccc}
			\toprule
			Categoria de Usu\'arios & Frequ\^encia de Usu\'arios \\
			\midrule \midrule
			Gradua\c{c}\~ao & 72\% \\
			\midrule 
			P\'os-Gradua\c{c}\~ao & 15\% \\
			\midrule 
			Docente & 10\% \\
			\midrule 
			Outras & 3\% \\
			\bottomrule
		\end{tabular}%
	}{%
		\fonte{Elaborada pelos autores.}%
		\nota{Exemplo de uma nota.}%
		\nota[Anota\c{c}\~oes]{Uma anota\c{c}\~ao adicional, que pode ser seguida de v\'arias
			outras.}%
		
	}
\end{table}


A formata\c{c}\~ao do quadro \'e similar \`a tabela, mas deve ter suas laterais fechadas e conter as linhas horizontais.
\newpage

% o comando \newpage foi utilizado para for\c{c}ar a quebra de p\'agina

\begin{quadro}[htb]
	\caption{\label{quadro_modelo}N\'{\i}veis de investiga\c{c}\~ao}
	\begin{tabular}{|p{2.6cm}|p{6.0cm}|p{2.25cm}|p{3.40cm}|}
		\hline
		\textbf{N\'{\i}vel de Investiga\c{c}\~ao} & \textbf{Insumos}  & \textbf{Sistemas de Investiga\c{c}\~ao}  & \textbf{Produtos}  \\
		\hline
		Meta-n\'{\i}vel & Filosofia\index{filosofia} da Ci\^encia  & Epistemologia &
		Paradigma  \\
		\hline
		N\'{\i}vel do objeto & Paradigmas do metan\'{\i}vel e evid\^encias do n\'{\i}vel inferior &
		Ci\^encia  & Teorias e modelos \\
		\hline
		N\'{\i}vel inferior & Modelos e m\'etodos do n\'{\i}vel do objeto e problemas do n\'{\i}vel inferior & Pr\'atica & Solu\c{c}\~ao de problemas  \\
		\hline
	\end{tabular}
	\begin{flushleft}
		%\fonte{\citeonline{van1986}}
		Fonte: \citeonline{van1986}
	\end{flushleft}
\end{quadro} 


No \textbf{Tutorial do Pacote USPSC para modelos de trabalhos de acad\^emicos em LaTeX - vers\~ao 3.1} s\~ao apresentados mais exemplos de quadros.

% ---
\subsection{Figuras}\label{sec_figuras}
% ---
\index{figuras}Figuras podem ser criadas diretamente em \LaTeX,
como o exemplo da \autoref{fig_circulo}. \\ 

\begin{figure}[htb]
	\caption{\label{fig_circulo}A delimita\c{c}\~ao do espa\c{c}o}
	\begin{center}
		\setlength{\unitlength}{9cm}
		\begin{picture}(1,1)
		\put(0,0){\line(0,1){1}}
		\put(0,0){\line(1,0){1}}
		\put(0,0){\line(1,1){1}}
		\put(0,0){\line(1,2){.5}}
		\put(0,0){\line(1,3){.3333}}
		\put(0,0){\line(1,4){.25}}
		\put(0,0){\line(1,5){.2}}
		\put(0,0){\line(1,6){.1667}}
		\put(0,0){\line(2,1){1}}
		\put(0,0){\line(2,3){.6667}}
		\put(0,0){\line(2,5){.4}}
		\put(0,0){\line(3,1){1}}
		\put(0,0){\line(3,2){1}}
		\put(0,0){\line(3,4){.75}}
		\put(0,0){\line(3,5){.6}}
		\put(0,0){\line(4,1){1}}
		\put(0,0){\line(4,3){1}}
		\put(0,0){\line(4,5){.8}}
		\put(0,0){\line(5,1){1}}
		\put(0,0){\line(5,2){1}}
		\put(0,0){\line(5,3){1}}
		\put(0,0){\line(5,4){1}}
		\put(0,0){\line(5,6){.8333}}
		\put(0,0){\line(6,1){1}}
		\put(0,0){\line(6,5){1}}
		\end{picture}
	\end{center}
	\legend{Fonte: \citeonline{equipeabntex2}}
\end{figure}

Consulte o \textbf{Tutorial do Pacote USPSC para modelos de trabalhos de acad\^emicos em LaTeX - vers\~ao 3.1} para conhecer mais recursos referentes \`a figuras. 

% ---
\section{Divis\~oes do documento}\label{sec-divisoes-b}
Esta se\c{c}\~ao exemplifica o uso de divis\~oes de documentos em conformidade com a ABNT NBR 6024  \cite{nbr6024}.
% ---
% ---
\subsection{Divis\~oes do documento: subse\c{c}\~ao}\label{sec-divisoes-subsection}
% ---

Um exemplo de se\c{c}\~ao \'e a \autoref{sec-divisoes-b}. Esta \'e a \autoref{sec-divisoes-subsection}.

\subsubsection{Divis\~oes do documento: subsubse\c{c}\~ao}\label{sec-divisoes-subsubsection}

Isto \'e uma \texttt{subsubsection} do \LaTeX, mas \'e denominada de ``subse\c{c}\~ao'' porque no portugu\^es n\~ao temos a palavra ``subsubse\c{c}\~ao''.

\subsubsection{Divis\~oes do documento: subsubse\c{c}\~ao}

Isto \'e outra subsubse\c{c}\~ao.

\subsection{Divis\~oes do documento: subse\c{c}\~ao}\label{sec-exemplo-subsec}

Isto \'e uma subse\c{c}\~ao.

\subsubsection{Divis\~oes do documento: subsubse\c{c}\~ao}

Isto \'e mais uma subsubse\c{c}\~ao da \autoref{sec-exemplo-subsec}.


\subsubsubsection{Esta \'e uma subse\c{c}\~ao de quinto
n\'{\i}vel}\label{sec-exemplo-subsubsubsection}

Esta \'e uma se\c{c}\~ao de quinto n\'{\i}vel. Ela \'e produzida com o seguinte comando:

\begin{verbatim}
\subsubsubsection{Esta \'e uma subse\c{c}\~ao de quinto
n\'{\i}vel}\label{sec-exemplo-subsubsubsection}
\end{verbatim}

\subsubsubsection{Esta \'e outra subse\c{c}\~ao de quinto n\'{\i}vel}\label{sec-exemplo-subsubsubsection-outro}

Esta \'e outra se\c{c}\~ao de quinto n\'{\i}vel.


\paragraph{Este \'e um par\'agrafo numerado}\label{sec-exemplo-paragrafo}

Este \'e um exemplo de par\'agrafo nomeado. Ele \'e produzido com o comando de
par\'agrafo:

\begin{verbatim}
\paragraph{Este \'e um par\'agrafo nomeado}\label{sec-exemplo-paragrafo}
\end{verbatim}

A numera\c{c}\~ao entre par\'agrafos numerados e subsubsubse\c{c}\~oes s\~ao cont\'{\i}nuas.

\paragraph{Esta \'e outro par\'agrafo numerado}\label{sec-exemplo-paragrafo-outro}

Este \'e outro par\'agrafo nomeado.

% ---
\subsection{Este \'e um exemplo de nome de subse\c{c}\~ao longa que se aplica a se\c{c}\~oes e demais divis\~oes do documento. Ele deve estar alinhado \`a esquerda e a segunda e demais linhas devem iniciar logo abaixo da primeira palavra da primeira linha} 

Observe que o alinhamento do t\'{\i}tulo obedece esta regra tamb\'em no sum\'ario.
	






