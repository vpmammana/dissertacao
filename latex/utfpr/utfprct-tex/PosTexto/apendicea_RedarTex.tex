%%%% AP\^ENDICE A
%%
%% Texto ou documento elaborado pelo autor, a fim de complementar sua argumenta\c{c}\~ao, sem preju\'{\i}zo da unidade nuclear do trabalho.

%% T\'{\i}tulo e r\'otulo de ap\^endice (r\'otulos n\~ao devem conter caracteres especiais, acentuados ou cedilha)
\chapter{T\'{\i}tulo do Ap\^endice A com um Texto Muito Longo que Pode Ocupar Mais de uma Linha}\label{cap:apendicea}

Quando houver necessidade pode-se apresentar como ap\^endice documento(s) auxiliar(es) e/ou complementar(es) como: legisla\c{c}\~ao, estatutos, gr\'aficos, tabelas, etc. Os ap\^endices s\~ao enumerados com letras mai\'usculas: \autoref{cap:apendicea}, \autoref{cap:apendiceb}, etc.

No \latex\ ap\^endices s\~ao editados como cap\'{\i}tulos. O comando \verb|\appendix| faz com que todos os cap\'{\i}tulos seguintes sejam considerados ap\^endices.

Ap\^endices complementam o texto principal da tese com informa\c{c}\~oes para leitores com especial interesse no tema, devendo ser considerados leitura opcional, ou seja, o entendimento do texto principal da tese n\~ao deve exigir a leitura atenta dos ap\^endices.

Ap\^endices usualmente contemplam provas de teoremas, dedu\c{c}\~oes de f\'ormulas matem\'aticas, diagramas esquem\'aticos, gr\'aficos e trechos de c\'odigo. Quanto a este \'ultimo, c\'odigo extenso n\~ao deve fazer parte da tese, mesmo como ap\^endice. O ideal \'e disponibilizar o c\'odigo na Internet para os interessados em examin\'a-lo ou utiliz\'a-lo.

%% T\'{\i}tulo e r\'otulo de se\c{c}\~ao (r\'otulos n\~ao devem conter caracteres especiais, acentuados ou cedilha)
\section{T\'{\i}tulo da Se\c{c}\~ao Secund\'aria do Ap\^endice A}\label{sec:secaoapendicea}

Exemplo de se\c{c}\~ao secund\'aria em ap\^endice (\autoref{sec:secaoapendicea} do \autoref{cap:apendicea}).

%% T\'{\i}tulo e r\'otulo de se\c{c}\~ao (r\'otulos n\~ao devem conter caracteres especiais, acentuados ou cedilha)
\subsection{T\'{\i}tulo da Se\c{c}\~ao Terci\'aria do Ap\^endice A}\label{subsec:subsecaoapendicea}

Exemplo de se\c{c}\~ao terci\'aria em ap\^endice (\autoref{subsec:subsecaoapendicea} do \autoref{cap:apendicea}).

%% T\'{\i}tulo e r\'otulo de se\c{c}\~ao (r\'otulos n\~ao devem conter caracteres especiais, acentuados ou cedilha)
\subsubsection{T\'{\i}tulo da se\c{c}\~ao quatern\'aria do Ap\^endice A}\label{subsubsec:subsubsecaoapendicea}

Exemplo de se\c{c}\~ao quatern\'aria em ap\^endice (\autoref{subsubsec:subsubsecaoapendicea} do \autoref{cap:apendicea}).

%% T\'{\i}tulo e r\'otulo de se\c{c}\~ao (r\'otulos n\~ao devem conter caracteres especiais, acentuados ou cedilha)
\paragraph{T\'{\i}tulo da se\c{c}\~ao quin\'aria do Ap\^endice A}\label{para:paragraphapendicea}

Exemplo de se\c{c}\~ao quin\'aria em ap\^endice (\autoref{para:paragraphapendicea} do \autoref{cap:apendicea}).
