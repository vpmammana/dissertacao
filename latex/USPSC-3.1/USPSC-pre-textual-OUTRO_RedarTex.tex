%% USPSC-pre-textual-OUTRO.tex
%% Camandos para defini��o do tipo de documento (tese ou disserta��o), �rea de concentra��o, op��o, pre�mbulo, titula��o 
%% referentes ao Programa de P�s-Gradua��o o IQSC
\instituicao{PROGRAMA DE P\'OS-GRADUA\c{C}\~AO EM ENSINO DE CI\^ENCIAS HUMANAS, SOCIAIS E DA NATUREZA - PPGEN da UNIVERSIDADE TECNOL\'OGICA FEDERAL DO PARAN\'A}
\unidade{NOME DA UNIDADE USP}
\unidademin{PROGRAMA DE P\'OS-GRADUA\c{C}\~AO EM ENSINO DE CI\^ENCIAS HUMANAS, SOCIAIS E DA NATUREZA - PPGEN}
\universidademin{Universidade de S\~ao Paulo}

\notafolharosto{Vers\~ao original}
%Para vers�o original em ingl�s, comente do comando/declara��o 
%     acima(inclua % antes do comando acima) e tire a % do 
%     comando/declara��o abaixo no idioma do texto
%\notafolharosto{Original version} 
%Para vers�o corrigida, comente do comando/declara��o da 
%     vers�o original acima (inclua % antes do comando acima) 
%     e tire a % do comando/declara��o de um dos comandos 
%     abaixo em conformidade com o idioma do texto
%\notafolharosto{Vers\~ao corrigida \\(Vers\~ao original dispon\'ivel na Unidade que aloja o Programa)}
%\notafolharosto{Corrected version \\(Original version available on the Program Unit)}

% ---
% dados complementares para CAPA e FOLHA DE ROSTO
% ---
\universidade{UNIVERSIDADE TECNOL\'OGICA FEDERAL DO PARAN\'A}
\titulo{HIST\'ORIA E CARACTERIZA\c{C}\~AO DE 10 ANOS DO WASH, UM PROGRAMA HETER\'ARQUICO DE APRENDIZAGEM STEAM}
\titleabstract{HISTORY AND CHARACTERIZATION OF A DECADE OF WASH: A HETERARCHIAL STEAM EDUCATION PROGRAM}
\tituloresumo{HIST\'ORIA E CARACTERIZA\c{C}\~AO DE 10 ANOS DO WASH, UM PROGRAMA HETER\'ARQUICO DE APRENDIZAGEM STEAM}
\autor{ELAINE DA SILVA TOZZI}
\autorficha{TOZZI, ELAINE DA SILVA}
\autorabr{TOZZI, E.S.}

\cutter{S856m}
% Para gerar a ficha catalogr�fica sem o C�digo Cutter, basta 
% incluir uma % na linha acima e tirar a % da linha abaixo
%\cutter{ }

\local{LONDRINA 2023}
\data{2023}
% Quando for Orientador, basta incluir uma % antes do comando abaixo
\renewcommand{\orientadorname}{Orientador: Prof. Dr. Paulo S\'ergio de Camargo Filho Coorientador: Dr. Victor Pellegrini Mammana}
% Quando for Coorientadora, basta tirar a % utilizar o comando abaixo
%\newcommand{\coorientadorname}{Coorientador:}
\orientador{Prof(a). Dr(a). Prof. Dr. Paulo S\'ergio de Camargo Filho}
\orientadorcorpoficha{orientador(a) Prof. Dr. Paulo S\'ergio de Camargo Filho}
\orientadorficha{Camargo, Paulo S\'ergio}
%Se houver co-orientador, inclua % antes das duas linhas (antes dos comandos \orientadorcorpoficha e \orientadorficha) 
%          e tire a % antes dos 3 comandos abaixo
%\coorientador{Prof(a). Dr(a). Dr. Victor Pellegrini Mammana}
%\orientadorcorpoficha{orientador(a) Prof. Dr. Paulo S\'ergio de Camargo Filho}
%\orientadorficha{Camargo, Paulo S\'ergio}

\notaautorizacao{AUTORIZO A REPRODU\c{C}\~AO E DIVULGA\c{C}\~AO TOTAL OU PARCIAL DESTE TRABALHO, POR QUALQUER MEIO CONVENCIONAL OU ELETR\^ONICO PARA FINS DE ESTUDO E PESQUISA, DESDE QUE CITADA A FONTE.}
\notabib{Ficha catalogr\'afica elaborada pela Biblioteca da PROGRAMA DE P\'OS-GRADUA\c{C}\~AO EM ENSINO DE CI\^ENCIAS HUMANAS, SOCIAIS E DA NATUREZA - PPGEN, com os dados fornecidos pelo(a) autor(a)}

\newcommand{\programa}[1]{

% TCCOUTRO ==========================================================================
\ifthenelse{\equal{#1}{DOUTRO}}{
    \area{Nome da \'Area}
	\tipotrabalho{Dissert. (Mestrado)}
	\tipotrabalhoabs{Dissertation (Master)}
	%\opcao{Nome da Op��o}
    % O preambulo deve conter o tipo do trabalho, o objetivo, 
	% o nome da institui��o e a �rea de concentra��o 
	\preambulo{Disserta\c{c}\~ao apresentada ao PROGRAMA DE P\'OS-GRADUA\c{C}\~AO EM ENSINO DE CI\^ENCIAS HUMANAS, SOCIAIS E DA NATUREZA - PPGEN da UNIVERSIDADE TECNOL\'OGICA FEDERAL DO PARAN\'A, como parte dos requisitos para a obten\c{c}\~ao do t\'itulo de Doutor em YYYYYYYYYYY.}
	\notaficha{Tese (Doutorado - PROGRAMA DE P\'OS-GRADUA\c{C}\~AO EM ENSINO DE CI\^ENCIAS HUMANAS, SOCIAIS E DA NATUREZA - PPGEN}
    }{
% MOUTRO ===========================================================================
\ifthenelse{\equal{#1}{MOUTRO}}{
    \area{Nome da \'Area}
	\tipotrabalho{Disserta\c{c}\~ao (Mestrado)}
	\tipotrabalhoabs{Dissertation (Master)}
	%\opcao{Nome da Op��o}
    % O preambulo deve conter o tipo do trabalho, o objetivo, 
	% o nome da institui��o e a �rea de concentra��o 
	\preambulo{Disserta\c{c}\~ao apresentada ao PROGRAMA DE P\'OS-GRADUA\c{C}\~AO EM ENSINO DE CI\^ENCIAS HUMANAS, SOCIAIS E DA NATUREZA - PPGEN da UNIVERSIDADE TECNOL\'OGICA FEDERAL DO PARAN\'A, como parte dos requisitos para a obten\c{c}\~ao do t\'itulo de Mestre em YYYYYYYYYYY.}
	\notaficha{Disserta\c{c}\~ao (Mestrado - PROGRAMA DE P\'OS-GRADUA\c{C}\~AO EM ENSINO DE CI\^ENCIAS HUMANAS, SOCIAIS E DA NATUREZA - PPGEN}
    }{
% Outros
    \tipotrabalho{Disserta\c{c}\~ao/Tese (Mestrado/Doutorado)}
	\tipotrabalhoabs{Dissertation/Thesis (Master/Doctor)}
	\area{Nome da \'Area}
	\opcao{Nome da Op\c{c}\~ao}
	% O preambulo deve conter o tipo do trabalho, o objetivo, 
	% o nome da institui��o e a �rea de concentra��o 
	\preambulo{Disserta\c{c}\~ao/Disserta\c{c}\~ao apresentada ao PROGRAMA DE P\'OS-GRADUA\c{C}\~AO EM ENSINO DE CI\^ENCIAS HUMANAS, SOCIAIS E DA NATUREZA - PPGEN da UNIVERSIDADE TECNOL\'OGICA FEDERAL DO PARAN\'A, como parte dos requisitos para a obten\c{c}\~ao do t\'itulo de Mestre/Doutor em YYYYYYYYYYY.}
	\notaficha{Disserta\c{c}\~ao/Tese (Mestrado/Doutorado - PROGRAMA DE P\'OS-GRADUA\c{C}\~AO EM ENSINO DE CI\^ENCIAS HUMANAS, SOCIAIS E DA NATUREZA - PPGEN}
    }}}
				