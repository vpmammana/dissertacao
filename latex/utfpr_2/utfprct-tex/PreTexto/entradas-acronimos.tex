%%%% LISTA DE ABREVIATURAS, SIGLAS E ACR\^ONIMOS
%%
%% Rela\c{c}\~ao, em ordem alfab\'etica, das abreviaturas (representa\c{c}\~ao de uma palavra por meio de alguma(s) de sua(s) s\'{\i}laba(s) ou
%% letra(s)), siglas (conjunto de letras iniciais dos voc\'abulos e/ou n\'umeros que representa um determinado nome) e acr\^onimos
%% (conjunto de letras iniciais dos voc\'abulos e/ou n\'umeros que representa um determinado nome, formando uma palavra pronunci\'avel).
%%
%%
%% Este arquivo para defini\c{c}\~ao de abreviaturas, siglas e acr\^onimos \'e utilizado com a op\c{c}\~ao \incluirlistadeacronimos{glossaries}
%%
%% Vantagens do modo com "glossaries" em rela\c{c}\~ao ao modo "file":
%%   1) Ordena automaticamente a lista
%%	 2) Apenas os termos referenciados s\~ao colocados na lista


%% Como referenciar: 
%% \gls{lp} = Linear Programming (LP)  (First use)
%% \gls{lp} = LP (Next uses)
%% \glspl{lp} = LPs
%% \glsentrytext{lp} = Linear Programming    (recommended for chapter/section/....)
%% \glsentrylong{lp} = Linear Programming
%% \glsentryshort{lp} = LP

%% Para acr\^onimos tamb\'em funciona:  
%% \acrlong{lp} = Linear Programming
%% \acrshort{lp} = LP

%% Abreviaturas: \abreviatura{r\'otulo}{representa\c{c}\~ao}{defini\c{c}\~ao}

\abreviatura{art.}{art.}{Artigo}
\abreviatura{cap.}{cap.}{Cap\'{\i}tulo}
\abreviatura{sec.}{sec.}{Se\c{c}\~ao}

%% Siglas: \sigla{r\'otulo}{representa\c{c}\~ao}{defini\c{c}\~ao}

\sigla{abnt}{ABNT}{Associa\c{c}\~ao Brasileira de Normas T\'ecnicas}
\sigla{cnpq}{CNPq}{Conselho Nacional de Desenvolvimento Cient\'{\i}fico e Tecnol\'ogico}
\sigla{eps}{EPS}{\textit{Encapsulated PostScript}}
\sigla{pdf}{PDF}{Formato de Documento Port\'atil, do ingl\^es \textit{Portable Document Format}}
\sigla{ps}{PS}{\textit{PostScript}}
\sigla{utfpr}{UTFPR}{Universidade Tecnol\'ogica Federal do Paran\'a}

%% Acr\^onimos: \acronimo{r\'otulo}{representa\c{c}\~ao}{defini\c{c}\~ao}

\acronimo{gimp}{Gimp}{Programa de Manipula\c{c}\~ao de Imagem GNU, do ingl\^es \textit{GNU Image Manipulation Program}}
