%% USPSC-Agradecimentos.tex
\begin{agradecimentos}
Quero expressar minha gratid\~ao \`as crian\c{c}as e aos jovens; as que passaram pelo Programa WASH, as que est\~ao conosco e as que ser\~ao o futuro do Programa.
Em especial, agrade\c{c}o a uma “crian\c{c}a sempre viva”, ativa, presente, curiosa e que outrora usou o LOGO, gostou de fazer seu jogo e trouxe essa experi\^encia para o seu mundo adulto de cientista, professor, pesquisador, gestor, amigo e companheiro de luta, h\'a mais de uma d\'ecada. Refiro-me ao Dr. Victor Pellegrini Mammana, que ao vivenciar esse prazeroso experimento, quis legar a outras crian\c{c}as o \^extase das descobertas e comprovou que \'e poss\'{\i}vel somar esfor\c{c}os da sociedade civil, das unidades de pesquisa e de educa\c{c}\~ao para contribuir com os processos de aprendizagens em ci\^encia e tecnologia.
Sou, tamb\'em, grata ao meu orientador, Prof. Dr Paulo S\'ergio de Camargo Filho pela companhia e orienta\c{c}\~ao, ao Grupo de pesquisa STEM; e \`a banca de avalia\c{c}\~ao, composta pela Profa. Dra. Luciane Capelo e Prof. Dr Eduardo Damasceno.
N\~ao posso deixar de mencionar a generosidade e disposi\c{c}\~ao dos professores Drs. Ala\'{\i}de Pellegrini Mammana e Carlos Mammana, que me acompanharam durante esse percurso.
Agrade\c{c}o, carinhosamente, \`a Professora Dra. Afira Viana Ripper, por sua contribui\c{c}\~ao para a educa\c{c}\~ao cientifica no ensino fundamental; e por participar do v\'{\i}deo, que resgata essa trajet\'oria e foi parte integrante da minha pesquisa.
Algumas pessoas, tamb\'em, precisam ser destacadas, pois foram imprescind\'{\i}veis para o Programa WASH, desde as suas origens, e marcaram essa hist\'oria: Silvio Aparecido Spinella, Antonio Bezerra de Albuquerque, Claudio Romanelli, Alexandre Candido Paulo, Pedro Tourinho, Paulo Bufalo, Daniel Esp\'osito, Everbal de Castro, Luciano Rudnik, Fernando Accorsi, Clotilde Diogo, Ana Carolina de Deus Soares, Thatiane Verni Lopes de Ara\'ujo, Michel Morandi Alencar, Denise Vieira Pereira, Cleide Santos, Mariana Moura, Jacqueline Baumgratrz , Celso Pan, Sandra Lanza, Gisele Fink, Nadia Abiel, Andrea Victor, TC, Angel Luis, Andrea Saraiva, Rafael Gomes da Cruz, Fabio Couto, Delma Medeiros, Wil Namen, Wagner Rodrigo Silva, Glaucia Veloso, Rafael de Deus Soares, Fernanda Gon\c{c}alves, Rafael Proc\'opio, Alex Angelo, Angelo Benetti, Fabrini, Fernando Dallagnol, Fabiola Calixto, Cassia Oliveira, Chico Sim\~oes, Fabiana Kitagawa, Isabela Maria Vieira Pereira Rodrigues, Lucas Titon, Leila Bomfim, Juliana Moralles Louvison, Ana Paula Rodrigues, Jos\'e Leonardo de Oliveira, Jarandira, a Dita, Ant\^onio, o Tot\'o, Silvio Ant\^onio Damasceno, Roberta Santana, Tayssa Santana, Jacyara Rodrigues, Malu Alencar, Saulo Monteiro, Carlinhos Almeida, Am\'elia Naomi, Tob\'e, Marcela Moreira, Alexandre Motta, Alisson Ara\'ujo, S\'ergio Benassi, Cec\'{\i}lia Baranauskas, Celso Pansera, Marcelo Aguirre, Lucas Gabriel Batista da Silva, Leticia Mizael, Haissa Gabriela Silva, Celio Turino, Mirza Maria Pellicciotta, Marcela Moreira, Guida Calixto, Cec\'{\i}lia Baranauskas, Renan Inqu\'erito, Layla Xavier, Denise Renata Xavier, Valdirene Maria dos Santos, Fabiana Bonilha, Jana\'{\i}na Rocha, Fabiana Fernandes, Vitor Oliveira, Ev\^ania Rocha, Cidinha, Andrea Napolitano, M\'ario Sandro Rocha, Edison An\'{\i}cio Duarte, Maria Thereza Cyrino, Cibele Celestino, J\'ulio Lobo, J\'unior do MAAS, Michel Kusnir, Regina Thiene, M\'arcio Spinola, Profa. Danieli de Jacare\'{\i}, Aldo Parada, Rachel Trajber, D\'ebora Olivato, Marli Andr\'eia Abrah\~ao, F\'abio Staziak, Adilson, Rog\'erio Winter, Rufino, C\'assia Namen, Wagner Rom\~ao, Ant\^onio Pestana, Edeneziano, Gilson Schwartz, M\'ario Gobbo, Marina Godoy, Jos\'e Ripper, Sonelise Cizotto, Alexandra Archer, Marcelo Poletti e Felipe Can\'e.
A todas as gera\c{c}\~oes do WASH: as que passaram, as que compartilham conosco, nesse ano de 2023, os 10 anos do Programa: s\~ao colegas, bolsistas, educandos, educadores, cientistas, coordenadores, orientadores, Conselhos de classes, Sindicatos, gestores, pesquisadores, vereadores, comunidades, os entes federados, que acreditam na ci\^encia e no papel transformador da educa\c{c}\~ao.
Meus agradecimentos, tamb\'em. \`as institui\c{c}\~oes parceiras: Conselho Nacional de Desenvolvimento Cient\'{\i}fico e Tecnol\'ogico - CNPq , Funda\c{c}\~ao Arauc\'aria, WASH Paran\'a, Cia Bola de Meia, Legislativo Federal, por meio dos deputados: Ivan Valente, Alex e Luiza Canziani, Alexandre Padilha, Vicentinho, Carlos Zaratini, Orlando Silva e Alexandre Cury, que foram sens\'{\i}veis e valorizaram a educa\c{c}\~ao cientifica, atrav\'es do Programa WASH. Aos legislativos de Prado Ferreira e Dr. Camargo por fazerem o WASH leis municipais. N\~ao posso deixar de reconhecer a contribui\c{c}\~ao da AkiPosso , com o apoio dos colegas Kevin Martins, Adriana Tito, Priya Patel, Caroline Gardemann, Nelcina Tropardi e Daniela Napolitano. Por fim, agrade\c{c}o a minha fam\'{\i}lia, a minha filha, Agatha Abayomi Silva Sene, aos meus pais, Maria Imaculada de Oliveira e Silva e Joaquim Roberto da Silva, ao meu irm\~ao Eduardo Roberto da Silva in memoriam - presente! e ao Alessandro, que contribu\'{\i}ram para que as condi\c{c}\~oes necess\'arias para o desenvolvimento dessa pesquisa fossem as mais leves para a execu\c{c}\~ao do meu estudo.
% @[pontoinsercaoparagrafoagradecimento]@

\end{agradecimentos}
% ---