%% USPSC-TCC-pre-textual-OUTROS.tex
%% Camandos para defini��o do tipo de documento (tese ou disserta��o), �rea de concentra��o, op��o, pre�mbulo, titula��o 
%% referentes aos Programas de P�s-Gradua��o
\instituicao{Nome da Unidade USP, Universidade de S\~ao Paulo}
\unidade{NOME DA UNIDADE USP}
\unidademin{Nome da Unidade USP}
\universidademin{Universidade de S\~ao Paulo}

% A EESC n�o inclui a nota "Vers�o original", portanto o comando abaixo n�o tem a mensagem entre {}
\notafolharosto{ }
%Para a vers�o corrigida tire a % do comando/declara��o abaixo e inclua uma % antes do comando acima
%\notafolharosto{VERS\~AO CORRIGIDA}
% ---
% dados complementares para CAPA e FOLHA DE ROSTO
% ---
\universidade{UNIVERSIDADE DE S\~AO PAULO}
\titulo{@[titulo]@}
\titleabstract{@[tituloabstract]@}
\tituloresumo{@[titulo]@}
\autor{@[autor]@}
\autorficha{@[autorficha]@}
\autorabr{@[autorabr]@}

\cutter{S856m}
% Para gerar a ficha catalogr�fica sem o C�digo Cutter, basta 
% incluir uma % na linha acima e tirar a % da linha abaixo
%\cutter{ }

\local{S\~ao Carlos}
\data{2021}
% Quando for Orientador, basta incluir uma % antes do comando abaixo
\renewcommand{\orientadorname}{Orientadora:}
% Quando for Coorientadora, basta tirar a % do comando abaixo
%\newcommand{\coorientadorname}{Coorientador:}
\orientador{Prof(a). Dr(a). @[orientador]@}
%\orientadorcorpoficha{orientador(a) @[orientador]@}
%\orientadorficha{@[orientadorficha]@}
%Se houver co-orientador, inclua % antes das duas linhas (antes dos comandos \orientadorcorpoficha e \orientadorficha) 
%          e tire a % antes dos 3 comandos abaixo
\coorientador{Prof(a). Dr(a). @[coorientador]@}
\orientadorcorpoficha{orientador(a) @[orientador]@}
\orientadorficha{@[orientadorficha]@}

\notaautorizacao{AUTORIZO A REPRODU\c{C}\~AO E DIVULGA\c{C}\~AO TOTAL OU PARCIAL DESTE TRABALHO, POR QUALQUER MEIO CONVENCIONAL OU ELETR\^ONICO PARA FINS DE ESTUDO E PESQUISA, DESDE QUE CITADA A FONTE.}
\notabib{~  ~}

\newcommand{\programa}[1]{     	
% Outros
	\tipotrabalho{Monografia (Trabalho de Conclus\~ao de Curso)}
	\tipotrabalhoabs{Monograph (Conclusion Course Paper)}
	%\area{Nome da �rea}
	%\opcao{Nome da Op��o}
	% O preambulo deve conter o tipo do trabalho, o objetivo, 
	% o nome da institui��o, a �rea de concentra��o e op��o quando houver
	\preambulo{Monografia apresentada ao Curso de XXXXXXX, da Unidade NNNNN da Universidade de S\~ao Paulo, como parte dos requisitos para obten\c{c}\~ao do t\'itulo de XXXXXXX.}
	\notaficha{Monografia (Gradua\c{c}\~ao em XXXXXXX)}	
	}
