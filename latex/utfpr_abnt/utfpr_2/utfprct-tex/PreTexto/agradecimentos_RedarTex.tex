%%%% AGRADECIMENTOS
%%
%% Texto em que o autor faz agradecimentos dirigidos \`aqueles que contribu\'{\i}ram de maneira relevante \`a elabora\c{c}\~ao do trabalho.

\begin{agradecimentos}%% Ambiente agradecimentos
%\begin{agradecimentos}[Agradecimentos]%% (op\c{c}\~ao: t\'{\i}tulo do agradecimento)

Quero expressar minha gratidão às crianças e aos jovens; que passaram pelo WASH; que estão conosco e que serão o futuro do Programa.
Em especial, agradeço a uma “criança sempre viva”, ativa, presente, curiosa e que outrora usou o LOGO, gostou de fazer seu jogo e trouxe essa experiência para o seu mundo adulto de cientista, professor, pesquisador, gestor, amigo e companheiro de luta, há mais de uma década. Refiro-me ao Dr. Victor Pellegrini Mammana, que ao vivenciar esse prazeroso experimento, quis legar a outras crianças o êxtase das descobertas e comprovou que é possível somar esforços individuais, da sociedade civil, das unidades de pesquisa e de educação para contribuir com os processos de aprendizagens em ciência e tecnologia.
Sou, também, grata ao meu orientador, Prof. Dr Paulo Sérgio de Camargo Filho pela companhia e orientação, ao Grupo de Pesquisa STEM Education; e à banca de avaliação, Prof. Dra. Luciane Capelo e Prof. Dr Eduardo Damasceno.
Não posso deixar de mencionar a generosidade e disposição dos professores Doutores Alaíde Pellegrini Mammana e Carlos Mammana, que me fizeram companhia e forneceram preciosas informações para o trabalho de pesquisa.
Agradeço, carinhosamente, à Professora Dra. Afira Viana Ripper, por sua contribuição e pioneirismo para a educação cientifica a partir do ensino fundamental; e por participar do vídeo, que resgata essa trajetória e faz parte da minha pesquisa.
Algumas pessoas, também, precisam ser destacadas, pois foram imprescindíveis para a construção do Programa WASH, desde as suas origens, e marcaram essa história: Ana Carolina de Deus Soares, Adriane Pinheiro da Silva, Aldo Hurtado, Alex Ângelo, Alexandre Cândido Paulo, Alisson Alexandre de Araújo, Aloizio Mercadante, Amélia Naomi, Antônio Carlos dos Santos (o Totó), Antônio Pestana, Alexandre Motta, Ana Paula Rodrigues, Andréa Saraiva, Angel Luis, Antonio Bezerra de Albuquerque, Benedita Aparecida Rodrigues de Freitas, Carlinhos Almeida, Cássia Oliveira, Cecília Baranauskas, Célio Turino, Mirza Maria Pellicciotta, Celso Pansera, Celso Pan, Chico Simões, Cíntia Cinquini, Claudio Romanelli, Cleide Santos, Mariana Moura, Clotilde Diogo, Daniel Spózito, Eberval de Castro, Denise  Vieira Pereira, Denise Xavier, Dilma Rousseff,  Fabiana Kitagawa, Fábio Couto, Delma  Medeiros, Fernanda Gonçalves, Gisele Fink, Nádia Abiel, Gláucia Veloso, Guida Calixto,  Ingridy Janaina Alves, Haissa Gabriela Silva, Irma Passoni, Isabela Maria Vieira Pereira Rodrigues, Jader Gama, Jacqueline Baumgratrz, Jaciara Rodrigues dos Santos, Jandira Maria Rodrigues de Freitas, José Leonardo de Oliveira, Juliana Moralles Louvison, Juliana Rabelo, Kevin Martins, Layla Xavier, Leila Bomfim, Letícia Mizael, Lucas Gabriel Roberto da Silva, Lucas Titon, Luciano Rudinik, Fernando Accorsi, Magna Gonçalves, Malu Alencar, Marcela Moreira, Marcelo Aguirre, Marcelo Poletti, Maria Fernandes, Michel Morandi Alencar, Pedro Tourinho, Paula Ropelo, Rachel Trajber, Rafael de Deus Soares, Rafael Gomes da Cruz, Rafael Procópio, Renan Inquérito, Renata Lourenço, Roberta Santana,  Sandra Lanza, Saulo Monteiro, Sebastian Marques, Sergio Melo, Tayssa Santana,  Sérgio Benassi, Sílvio Antônio Damasceno, Sílvio Aparecido Spinella, TC ( Antonio Carlos), Thatiane Verni Lopes de Araújo, Toni Klaus, Valdirene Maria dos Santos, Vitor de Oliveira Pochmann, Wagner Rodrigo Silva, Wil Namen, dentre tantos outros.
A todas as gerações do WASH: as que passaram, as que compartilham conosco, nesse ano de 2023, os 10 anos do Programa: são colegas, bolsistas, educandos, educadores, cientistas, coordenadores, orientadores, Conselheiros de classes, Sindicatos, gestores, pesquisadores, vereadores, as pessoas e instituições, a grande rede do WASH que acreditam na ciência e no papel transformador da educação.
Meus agradecimentos, também. às instituições parceiras: Conselho Nacional de Desenvolvimento Científico e Tecnológico - CNPq, Fundação Araucária, WASH Paraná, Cia Bola de Meia, Legislativo Federal, por meio dos deputados: Ivan Valente, Alex e Luíza Canziani, Alexandre Padilha, Vicentinho, Carlos Zaratini, Orlando Silva e Eduardo Cury, que foram sensíveis e valorizaram a educação cientifica, através do Programa WASH. Aos legislativos de Prado Ferreira e  de Dr. Camargo por fazerem o WASH leis municipais. Não posso deixar de reconhecer a contribuição da AkiPosso , com o apoio dos colegas  Andrea Napolitano Mammana, Adriana Tito, Caroline Gardemann, Daniela Napolitano, Kevin Martins, Priya Patel e Nelcina Tropardi. Por fim, agradeço a minha filha, Agatha Abayomi Silva Sene, aos meus pais, Maria Imaculada de Oliveira Silva e Joaquim Roberto da Silva, ao meu irmão Eduardo Roberto da Silva in memoriam - presente! Alessandro Roberto da Silva, Márcia Daniela Silva Azzem que contribuíram para que as condições necessárias para o desenvolvimento dessa pesquisa fossem as mais leves para a execução do meu estudo.
Termino enfatizando os papeis especiais do vereador Paulo Búfalo, que está nesta caminhada conosco desde os primórdios do Programa, e da Dra Andréa Dias Victor, servidora do CNPq que permanece aceitando, com compromisso público, excelência administrativa e acadêmica, a carga de gestão do Programa WASH, representada por centenas de bolsistas.
% @[pontoinsercaoparagrafoagradecimento]@

\end{agradecimentos}
