%% USPSC-Abstract.tex
%\autor{Silva, M. J.}
\begin{resumo}[Abstract]
 \begin{otherlanguage*}{english}
	\begin{flushleft} 
		\setlength{\absparsep}{0pt} % ajusta o espa\c{c}amento dos par\'agrafos do resumo		
 		\SingleSpacing  		\imprimirautorabr~~\textbf{\imprimirtitleabstract}.	\imprimirdata.  \pageref{LastPage}p. 
		%Substitua p. por f. quando utilizar oneside em \documentclass
		%\pageref{LastPage}f.
		\imprimirtipotrabalhoabs~-~\imprimirinstituicao, \imprimirlocal, 	\imprimirdata. 
 	\end{flushleft}
	\OnehalfSpacing 
The Software and Hardware Workshop for Geeks, known as WASH, is Science, Technology, Engineering, Arts and Mathematics (STEAM) based effort that has been running since 2013 in dozens of Brazilian municipalities and with thousands of children attending. After years of practice, the main characteristics were grouped in the Reference Document published in 2018, attached to Federal Ordinance registered as CTI 178/2018. This research is divided into 2 axes: historiographical method (axis 1) and the use of structured queries applied to a database specially designed and developed to produce managerial indicators (axis 2). The work sought to compare, from the definitions of the Reference Document, "what WASH would like to have been" with "what WASH managed to be", the latter being a result from the overall analysis produced by this dissertation. To objectify this comparison, six hypotheses were formulated, based on the Reference Document, which at the end of the work were submitted to validation. The analysis of the successes and failures of this validation allowed producing a revision of the Reference Document, which is the main educational product of this dissertation, a mandatory requirement for obtaining the Master's degree. Added to this educational product is the interview with Prof. Afira Ripper, one of the elements used for the analysis in axis 1 and, also, a very rare testimony about the coming of Seymour Papert to Brazil at the end of the last century.


   \vspace{\onelineskip}
 
   \noindent 
   \textbf{Keywords}: LaTeX. USPSC class. Thesis. Dissertation. Conclusion course paper. 
 \end{otherlanguage*}
\end{resumo}
