%% USPSC-Resumo.tex
\setlength{\absparsep}{18pt} % ajusta o espa\c{c}amento dos par\'agrafos do resumo
\begin{resumo}
\begin{flushleft} 
\setlength{\absparsep}{0pt} % ajusta o espa\c{c}amento da refer\^encia
\SingleSpacing 
\imprimirautorabr~~\textbf{\imprimirtituloresumo}.\imprimirdata. \pageref{LastPage}p. 
%Substitua p. por f. quando utilizar oneside em \documentclass
%\pageref{LastPage}f.
\imprimirtipotrabalho~-~\imprimirinstituicao, \imprimirlocal, \imprimirdata. 
 \end{flushleft}
\OnehalfSpacing 
Neste trabalho, o Programa Workshop de Aficcionados em Software e Hardware (WASH) ser\'a caracterizado quanto \`a sua hist\'oria e resultados. O Programa WASH, voltado para educa\c{c}\~ao em Ci\^encia, Tecnologia, Engenharia, Artes e Matem\'atica (STEAM) vem sendo executado desde 2013, com presen\c{c}a em dezenas de munic\'{\i}pios brasileiros e milhares de crian\c{c}as atendidas. Ap\'os anos de pr\'atica, suas caracter\'{\i}sticas principais foram agrupadas em um termo de refer\^encia publicado em 2018, na forma de uma portaria (Portaria CTI 178/2018). Este trabalho \'e dividido em 2 eixos: m\'etodo historiogr\'afico (eixo 1) e o emprego de consultas estruturadas a uma base de dados especialmente desenvolvida para produzir os indicadores (eixo 2). A an\'alise dos resultados obtidos a partir do emprego destes 2 m\'etodos permitiu produzir uma revis\~ao no termo de refer\^encia, a qual \'e o principal produto tecnol\'ogico desta disserta\c{c}\~ao, quesito obrigat\'orio para a obten\c{c}\~ao do t\'{\i}tulo em Mestrado. S\~ao tamb\'em produtos tecnol\'ogicos desta disserta\c{c}\~ao, com a colabora\c{c}\~ao de outros pesquisadores, a Plataforma de Dados Platu\'osh, produ\c{c}\~ao audiovisual publicada em redes sociais, entre outros.
% @[pontoinsercaoparagraforesumo]@
 

 \textbf{Palavras-chave}: Papert, STEAM, STEM, WASH
\end{resumo}