%% USPSC-Apendice.tex
% ---
% Inicia os ap\^endices
% ---

\begin{apendicesenv}
% Imprime uma p\'agina indicando o in\'{\i}cio dos ap\^endices
\partapendices
\chapter{Ap\^endice(s)}
Elemento opcional, que consiste em texto ou documento elaborado pelo autor, a fim de complementar sua argumenta\c{c}\~ao, conforme a ABNT NBR 14724 \cite{nbr14724}.

Os ap\^endices devem ser identificados por letras mai\'usculas consecutivas, seguidas de h\'{\i}fen e pelos respectivos t\'{\i}tulos. Excepcionalmente, utilizam-se letras mai\'usculas dobradas na identifica\c{c}\~ao dos ap\^endices, quando esgotadas as 26 letras do alfabeto. A pagina\c{c}\~ao deve ser cont\'{\i}nua, dando seguimento ao texto principal. \cite{aguia2020}
% ----------------------------------------------------------
\chapter{Exemplo de tabela centralizada verticalmente e horizontalmente}
\index{tabelas}A \autoref{tab-centralizada} exemplifica como proceder para obter uma tabela centralizada verticalmente e horizontalmente.
% utilize \usepackage{array} no PREAMBULO (ver em USPSC-modelo.tex) obter uma tabela centralizada verticalmente e horizontalmente
\begin{table}[htb]
\ABNTEXfontereduzida
\caption[Exemplo de tabela centralizada verticalmente e horizontalmente]{Exemplo de tabela centralizada verticalmente e horizontalmente}
\label{tab-centralizada}

\begin{tabular}{ >{\centering\arraybackslash}m{6cm}  >{\centering\arraybackslash}m{6cm} }
\hline
 \centering \textbf{Coluna A} & \textbf{Coluna B}\\
\hline
  Coluna A, Linha 1 & Este \'e um texto bem maior para exemplificar como \'e centralizado verticalmente e horizontalmente na tabela. Segundo par\'agrafo para verificar como fica na tabela\\
  Quando o texto da coluna A, linha 2 \'e bem maior do que o das demais colunas  & Coluna B, linha 2\\
\hline
\end{tabular}
\begin{flushleft}
		Fonte: Elaborada pelos autores.\
\end{flushleft}
\end{table}

% ----------------------------------------------------------
\chapter{Exemplo de tabela com grade}
\index{tabelas}A \autoref{tab-grade} exemplifica a inclus\~ao de tra\c{c}os estruturadores de conte\'udo para melhor compreens\~ao do conte\'udo da tabela, em conformidade com as normas de apresenta\c{c}\~ao tabular do IBGE.
% utilize \usepackage{array} no PREAMBULO (ver em USPSC-modelo.tex) obter uma tabela centralizada verticalmente e horizontalmente
\begin{table}[htb]
\ABNTEXfontereduzida
\caption[Exemplo de tabelas com grade]{Exemplo de tabelas com grade}
\label{tab-grade}
\begin{tabular}{ >{\centering\arraybackslash}m{8cm} | >{\centering\arraybackslash}m{6cm} }
\hline
 \centering \textbf{Coluna A} & \textbf{Coluna B}\\
\hline
  A1 & B1\\
\hline
  A2 & B2\\
\hline
  A3 & B3\\
\hline
  A4 & B4\\
\hline
\end{tabular}
\begin{flushleft}
		Fonte: Elaborada pelos autores.\
\end{flushleft}
\end{table}


\end{apendicesenv}
% ---