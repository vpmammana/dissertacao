%% USPSC-Cap5-ConclusaoTutorial.tex
% ---
% Conclus\~ao
% ---
\chapter{Conclus\~ao}
% ---
% O comando abaixo insere par\'agrafos aleat\'orios s\'o para exemplificar
Apresentar as conclus\~oes correspondentes aos objetivos ou hip\'oteses propostos para o desenvolvimento do trabalho, podendo incluir  sugest\~oes para novas pesquisas.

O Grupo desenvolvedor do Pacote USPSC, atualmente na vers\~ao 3.1 composta pela \textbf{Classe USPSC}, pelo \textbf{Modelo para TCC em \LaTeX\ utilizando o Pacote USPSC} e pelo \textbf{Modelo para teses e disserta\c{c}\~oes em \LaTeX\ utilizando o Pacote USPSC}, acredita que esta ferramenta propiciar\'a o aprimoramento na qualidade dos trabalhos acad\^emicos produzidos pelos alunos de p\'os-gradua\c{c}\~ao das Unidades de Ensino e Pesquisa do Campus USP de S\~ao Carlos, garantindo a normaliza\c{c}\~ao e padroniza\c{c}\~ao estabelecidas.

A perspectiva \'e que em breve seja poss\'{\i}vel a customiza\c{c}\~ao da Classe USPSC em conformidade com as orienta\c{c}\~oes dadas em \url{https://github.com/abntex/abntex2/wiki/ComoCustomizar}.

A expectativa \'e que o Pacote USPSC passe a ser adotado por outras Unidades da USP e outras institui\c{c}\~oes interessadas, sendo que a facilidade de customiza\c{c}\~ao fatalmente contribuir\'a para tanto.

