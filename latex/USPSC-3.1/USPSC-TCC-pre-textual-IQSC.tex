%% USPSC-TCC-pre-textual-IQSC.tex
%% Camandos para defini��o do tipo de documento (tese ou disserta��o), �rea de concentra��o, op��o, pre�mbulo, titula��o 
%% referentes ao Programa de P�s-Gradua��o o IQSC
\instituicao{@[unidadefaculdade]@ da @[universidade]@}
\unidade{@[programaposmaiuscula]@}
\unidademin{@[unidadefaculdade]@}
\universidademin{Universidade de S\~ao Paulo}
% O IQSC n�o inclui a nota "Vers�o original", portanto o comando abaixo n�o tem a mensagem entre {}
\notafolharosto{ }
%Para a vers�o revisada tire a % do comando/declara��o abaixo e inclua uma % antes do comando acima

%\notafolharosto{Exemplar revisado \\O exemplar original encontra-se em acervo reservado na Biblioteca do IQSC-USP}

% ---
% dados complementares para CAPA e FOLHA DE ROSTO
% ---
\universidade{@[universidademaiuscula]@}
\titulo{@[titulo]@}
%\titulo{@[titulo]@}
\titleabstract{@[tituloabstract]@}
\tituloresumo{@[titulo]@}
\autor{@[autor]@}
\autorficha{@[autorficha]@}
\autorabr{@[autorabr]@}

\cutter{S856m}
% Para gerar a ficha catalogr�fica sem o C�digo Cutter, basta 
% incluir uma % na linha acima e tirar a % da linha abaixo
%\cutter{ }

\local{@[localidade]@}
\data{@[ano]@}
% Quando for Orientador, basta incluir uma % antes do comando abaixo
\renewcommand{\orientadorname}{Orientador: @[orientador]@ Coorientador: @[coorientador]@}
% Quando for Coorientadora, basta tirar a % utilizar o comando abaixo
%\newcommand{\coorientadorname}{Coorientador:}
\orientador{Prof(a). Dr(a). @[orientador]@}
\orientadorcorpoficha{orientador(a) @[orientador]@}
\orientadorficha{@[orientadorficha]@}
%Se houver co-orientador, inclua % antes das duas linhas (antes dos comandos \orientadorcorpoficha e \orientadorficha) 
%          e tire a % antes dos 3 comandos abaixo
%\coorientador{Prof(a). Dr(a). @[coorientador]@}
%\orientadorcorpoficha{orientador(a) @[orientador]@}
%\orientadorficha{@[orientadorficha]@}

\notaautorizacao{AUTORIZO A REPRODU\c{C}\~AO E DIVULGA\c{C}\~AO TOTAL OU PARCIAL DESTE TRABALHO, POR QUALQUER MEIO CONVENCIONAL OU ELETR\^ONICO PARA FINS DE ESTUDO E PESQUISA, DESDE QUE CITADA A FONTE.}
\notabib{Ficha catalogr\'afica elaborada pela Se\c{c}\~ao de Refer\^encia e Atendimento ao Usu\'ario do Servi\c{c}o de Biblioteca e Informa\c{c}\~ao Prof. Johannes R\"udiger Lechat, com os dados fornecidos pelo(a) autor(a)}

\newcommand{\programa}[1]{

% BQ ==========================================================================
\ifthenelse{\equal{#1}{BQ}}{
    %\area{Qu\'imica}
	\tipotrabalho{Monografia (Trabalho de Conclus\~ao de Curso)}
	\tipotrabalhoabs{Monograph (Conclusion Course Paper)}
	%\opcao{Nome da Op��o}
    % O preambulo deve conter o tipo do trabalho, o objetivo, 
	% o nome da institui��o e a �rea de concentra��o 
	\preambulo{Monografia apresentada ao @[unidadefaculdade]@, da Universidade de S\~ao Paulo, como parte dos requisitos para a obten\c{c}\~ao do t\'itulo de Bacharel em Qu\'imica.}
	\notaficha{Monografia (Gradua\c{c}\~ao em Qu\'imica)}
    }{
% REQ ===========================================================================
\ifthenelse{\equal{#1}{REQ}}{
    %\area{u\'imica}
	\tipotrabalho{Est\'agio}
	\tipotrabalhoabs{Internship}
	%\opcao{Nome da Op��o}
    % O preambulo deve conter o tipo do trabalho, o objetivo, 
	% o nome da institui��o e a �rea de concentra��o 
	\preambulo{Relat\'orio de est\'agio em  Qu\'imica apresentado ao @[unidadefaculdade]@, da Universidade de S\~ao Paulo, como parte dos requisitos para a obten\c{c}\~ao do t\'itulo de de Bacharel em Qu\'imica.}
	\notaficha{Relat\'orio de Est\'agio (Gradua\c{c}\~ao em Qu\'imica)}
    }{
% Outros 
	\tipotrabalho{Monografia (Trabalho de Conclus\~ao de Curso)}
	\tipotrabalhoabs{Monografia (Trabalho de Conclus\~ao de Curso)}
	%\area{Nome da \'Area}
	%\opcao{Nome da Op\c{c}\~ao}
	% O preambulo deve conter o tipo do trabalho, o objetivo, 
	% o nome da institui��o, a �rea de concentra��o e op��o quando houver
	\preambulo{Monografia/Relat\'orio de est\'agio apresentada(o) ...}
	\notaficha{Monografia/Relat\'orio de Est\'agio (Gradua\c{c}\~ao em Qu\'imica)}	
  
}}}
				