%% USPSC-ApendiceTutorial.tex
% ---
% Inicia os ap\^endices
% ---

\begin{apendicesenv}
% Imprime uma p\'agina indicando o in\'{\i}cio dos ap\^endices
\partapendices
\chapter{Ap\^endice(s)}
Elemento opcional, que consiste em texto ou documento elaborado pelo autor, a fim de complementar sua argumenta\c{c}\~ao, conforme a ABNT NBR 14724 \cite{nbr14724}.

Os ap\^endices devem ser identificados por letras mai\'usculas consecutivas, seguidas de h\'{\i}fen e pelos respectivos t\'{\i}tulos. Excepcionalmente, utilizam-se letras mai\'usculas dobradas na identifica\c{c}\~ao dos ap\^endices, quando esgotadas as 26 letras do alfabeto. A pagina\c{c}\~ao deve ser cont\'{\i}nua, dando seguimento ao texto principal. \cite{sibi2009}

\chapter{Siglas dos Programas de P\'os-Gradua\c{c}\~ao da EESC}
\index{quadros}O \autoref{quadro-eesc} relaciona as siglas estabelecidas para os programas de p\'os-gradua\c{c}\~ao da EESC.

\begin{quadro}[htb]
\ABNTEXfontereduzida
%\caption[Siglas dos Programas de P\'os-Gradua\c{c}\~ao da EESC]{Siglas dos Programas de P\'os-Gradua\c{c}\~ao da EESC]{Siglas dos Programas de P\'os-Gradua\c{c}\~ao da EESC}
\caption[Siglas dos Programas de P\'os-Gradua\c{c}\~ao da EESC]{Siglas dos Programas de P\'os-Gradua\c{c}\~ao da EESC} 
\label{quadro-eesc}
\begin{tabular}{|p{6.0cm}|p{4.5cm}|p{2.0cm}|p{1.75cm}|}
%\multicolumn{4}{c}%
%{{\tablename\ \thetable{} -- Siglas dos Programas de P\'os-Gradua\c{c}\~ao da EESC}} \\
\multicolumn{4}{r}{{(continua)}} \\ 
  \hline
   \textbf{PROGRAMA} & \textbf{\'AREA DE CONCENTRA\c{C}\~AO} & \textbf{T\'ITULO} & \textbf{SIGLA}  \\
    \hline
Programa de P\'os-Gradua\c{c}\~ao em Ci\^encias da Engenharia Ambiental & Ci\^encias da Engenharia Ambiental & Doutor(a) & DCEA \\
Programa de P\'os-Gradua\c{c}\~ao em Ci\^encias da Engenharia Ambiental & Ci\^encias da Engenharia Ambiental & Mestre & MCEA \\
Programa de P\'os-Gradua\c{c}\~ao em Ci\^encia e Engenharia de Materiais & Caracteriza\c{c}\~ao, Desenvolvimento e Aplica\c{c}\~ao de Materiais  & Doutor(a) & DCEM \\
Programa de P\'os-Gradua\c{c}\~ao em Ci\^encia e Engenharia de Materiais & Caracteriza\c{c}\~ao, Desenvolvimento e Aplica\c{c}\~ao de Materiais & Mestre & MCEM \\
Programa de P\'os-Gradua\c{c}\~ao em Engenharia Civil (Engenharia de Estruturas) & Estruturas & Doutor(a) & DEE \\
Programa de P\'os-Gradua\c{c}\~ao em Engenharia Civil (Engenharia de Estruturas) & Estruturas & Mestre & MEE \\
Programa de P\'os-Gradua\c{c}\~ao em Engenharia de Produ\c{c}\~ao & Economia, Organiza\c{c}\~oes e Gest\~ao do Conhecimento & Doutor(a) & DEPE \\
Programa de P\'os-Gradua\c{c}\~ao em Engenharia de Produ\c{c}\~ao & Economia, Organiza\c{c}\~oes e Gest\~ao do Conhecimento & Mestre & MEPE \\
Programa de P\'os-Gradua\c{c}\~ao em Engenharia de Produ\c{c}\~ao & Processos e Gest\~ao de Opera\c{c}\~oes & Doutor(a) & DEPP \\
Programa de P\'os-Gradua\c{c}\~ao em Engenharia de Produ\c{c}\~ao & Processos e Gest\~ao de Opera\c{c}\~oes & Mestre & MEPP \\
Programa de P\'os-Gradua\c{c}\~ao em Engenharia de Transportes & Infraestrutura de Transportes & Doutor(a) & DETI \\
Programa de P\'os-Gradua\c{c}\~ao em Engenharia de Transportes & Infraestrutura de Transportes & Mestre & METI \\
Programa de P\'os-Gradua\c{c}\~ao em Engenharia de Transportes & Planejamento e Opera\c{c}\~ao de Sistemas de Transporte & Doutor(a) & DETP \\
Programa de P\'os-Gradua\c{c}\~ao em Engenharia de Transportes & Planejamento e Opera\c{c}\~ao de Sistemas de Transporte & Mestre & METP \\


\end{tabular}
\end{quadro} 

% o comando \clearpage \'e necess\'ario para deixar o final da tabela o topo da p\'agina, sem ele o final da tabela \'e centralizado verticalmente na p\'agina 
\clearpage
\begin{quadro}[htb]
	\ABNTEXfontereduzida
\begin{tabular}{|p{6.0cm}|p{4.5cm}|p{2.0cm}|p{1.75cm}|}	
 \multicolumn{4}{c}%
	{{\quadroname\ \thequadro{} -- Siglas dos Programas de P\'os-Gradua\c{c}\~ao da EESC}} \\
	\multicolumn{4}{r}{{(continua\c{c}\~ao)}} \\
	 \hline
   \textbf{PROGRAMA} & \textbf{\'AREA DE CONCENTRA\c{C}\~AO} & \textbf{T\'ITULO} & \textbf{SIGLA}  \\
		 \hline
Programa de P\'os-Gradua\c{c}\~ao em Engenharia de Transportes & Transportes & Doutor(a) & DETT \\
Programa de P\'os-Gradua\c{c}\~ao em Engenharia de Transportes & Transportes & Mestre & METT \\
Programa de P\'os-Gradua\c{c}\~ao em Engenharia El\'etrica & Processamento de Sinais e Intrumenta\c{c}\~ao & Doutor(a) & DEEP \\
Programa de P\'os-Gradua\c{c}\~ao em Engenharia El\'etrica & Processamento de Sinais e Intrumenta\c{c}\~ao & Mestre & MEEP \\
Programa de P\'os-Gradua\c{c}\~ao em Engenharia El\'etrica & Sistemas Din\^amicos & Doutor(a) & DEED \\
Programa de P\'os-Gradua\c{c}\~ao em Engenharia El\'etrica & Sistemas Din\^amicos & Mestre & MEED \\
Programa de P\'os-Gradua\c{c}\~ao em Engenharia El\'etrica & Sistemas El\'etricos de Pot\^encia & Doutor(a) & DEEE \\
Programa de P\'os-Gradua\c{c}\~ao em Engenharia El\'etrica & Sistemas El\'etricos de Pot\^encia & Mestre & MEEE \\
Programa de P\'os-Gradua\c{c}\~ao em Engenharia El\'etrica & Telecomunica\c{c}\~oes & Doutor(a) & DEET \\
Programa de P\'os-Gradua\c{c}\~ao em Engenharia El\'etrica & Telecomunica\c{c}\~oes & Mestre & MEET \\
Programa de P\'os-Gradua\c{c}\~ao em Engenharia Hidr\'aulica e Saneamento & Hidr\'aulica e Saneamento & Doutor(a) & DEHS \\
Programa de P\'os-Gradua\c{c}\~ao em Engenharia Hidr\'aulica e Saneamento & Hidr\'aulica e Saneamento & Mestre & MEHS \\
Programa de P\'os-Gradua\c{c}\~ao em Engenharia Mec\^anica & Aerona\'utica & Doutor(a) & DEMA \\
Programa de P\'os-Gradua\c{c}\~ao em Engenharia Mec\^anica & Aerona\'utica & Mestre & MEMA \\
Programa de P\'os-Gradua\c{c}\~ao em Engenharia Mec\^anica & Din\^amica e Mecatr\^onica & Doutor(a) & DEMD \\
Programa de P\'os-Gradua\c{c}\~ao em Engenharia Mec\^anica & Din\^amica e Mecatr\^onica & Mestre & MEMD \\
Programa de P\'os-Gradua\c{c}\~ao em Engenharia Mec\^anica & Projeto, Materiais e Manufatura  & Doutor(a) & DEMF \\
Programa de P\'os-Gradua\c{c}\~ao em Engenharia Mec\^anica & Projeto, Materiais e Manufatura  & Mestre & MEMF \\
Programa de P\'os-Gradua\c{c}\~ao em Engenharia Mec\^anica & Termoci\^encias e Mec\^anica de Fluidos & Doutor(a) & DEMT \\
Programa de P\'os-Gradua\c{c}\~ao em Engenharia Mec\^anica & Termoci\^encias e Mec\^anica de Fluidos & Mestre & MEMT \\
Programa de P\'os-Gradua\c{c}\~ao em Geotecnia & Geotecnia & Doutor(a) & DGEO \\
Programa de P\'os-Gradua\c{c}\~ao em Geotecnia & Geotecnia & Mestre & MGEO \\
    
\end{tabular}
\end{quadro}

% o comando \clearpage \'e necess\'ario para deixar o final da tabela o topo da p\'agina, sem ele o final da tabela \'e centralizado verticalmente na p\'agina 
\clearpage
\begin{quadro}[htb]
	\ABNTEXfontereduzida
\begin{tabular}{|p{6.0cm}|p{4.5cm}|p{2.0cm}|p{1.75cm}|}
\multicolumn{4}{c}%
	{{\quadroname\ \thequadro{} -- Siglas dos Programas de P\'os-Gradua\c{c}\~ao da EESC}} \\
	\multicolumn{4}{r}{{(conclus\~ao)}} \\
\hline
\textbf{PROGRAMA} & \textbf{\'AREA DE CONCENTRA\c{C}\~AO} & \textbf{T\'ITULO} & \textbf{SIGLA}  \\
\hline    
Programa de P\'os-Gradua\c{c}\~ao Interunidades em Bioengenharia & Bioengenharia & Doutor(a) & DIUB \\
Programa de P\'os-Gradua\c{c}\~ao Interunidades em Bioengenharia & Bioengenharia & Mestre & MIUB \\
Programa de P\'os-Gradua\c{c}\~ao em Rede Nacional para Ensino das Ci\^encias Ambientais & Ensino de Ci\^encias Ambientais & Mestre & MRNECA \\    
    
    \hline
\end{tabular}
\begin{flushleft}
		Fonte: Elaborado pelos autores.\
\end{flushleft}
\end{quadro}

% ----------------------------------------------------------
\chapter{Siglas dos Programas de P\'os-Gradua\c{c}\~ao do IAU}
\index{quadros}O \autoref{quadro-iau} relaciona as siglas estabelecidas para os programas de p\'os-gradua\c{c}\~ao do IAU.
\begin{quadro}[htb]
\ABNTEXfontereduzida
\caption[Siglas dos Programas de P\'os-Gradua\c{c}\~ao do IAU]{Siglas dos Programas de P\'os-Gradua\c{c}\~ao do IAU}
\label{quadro-iau}
\begin{tabular}{|p{3.5cm}|p{3.5cm}|p{3.5cm}|p{1.5cm}|p{2.25cm}|}
  \hline
   \textbf{PROGRAMA} & \textbf{\'AREA DE CONCENTRA\c{C}\~AO} & \textbf{OP\c{C}\~AO} & \textbf{T\'ITULO} & \textbf{SIGLA}  \\
    \hline
Programa de P\'os-Gradua\c{c}\~ao em Arquitetura e Urbanismo & Arquitetura, Urbanismo e Tecnologia &  & Doutor(a) & DAUT\\
Programa de P\'os-Gradua\c{c}\~ao em Arquitetura e Urbanismo & Arquitetura, Urbanismo e Tecnologia &  & Mestre & MAUT\\
Programa de P\'os-Gradua\c{c}\~ao em Arquitetura e Urbanismo & Teoria e Hist\'oria da Arquitetura e do Urbanismo &  & Doutor(a) & DAUH\\
Programa de P\'os-Gradua\c{c}\~ao em Arquitetura e Urbanismo & Teoria e Hist\'oria da Arquitetura e do Urbanismo &  & Mestre & MAUH\\
    \hline

\end{tabular}
\begin{flushleft}
		Fonte: Elaborado pelos autores.\
\end{flushleft}
\end{quadro}

% ----------------------------------------------------------
\chapter{Siglas dos Programas de P\'os-Gradua\c{c}\~ao do ICMC}
\index{quadros}O \autoref{quadro-icmc} relaciona as siglas estabelecidas para os programas de p\'os-gradua\c{c}\~ao do ICMC.
\begin{quadro}[htb]
\ABNTEXfontereduzida
\caption[Siglas dos Programas de P\'os-Gradua\c{c}\~ao do ICMC]{Siglas dos Programas de P\'os-Gradua\c{c}\~ao do ICMC}
\label{quadro-icmc}
\begin{tabular}{|p{3.5cm}|p{3.5cm}|p{2.5cm}|p{2.5cm}|p{2.25cm}|}
  \multicolumn{5}{r}{{(continua)}} \\ 
  \hline
   \textbf{PROGRAMA} & \textbf{\'AREA DE CONCENTRA\c{C}\~AO} & \textbf{OP\c{C}\~AO} & \textbf{T\'ITULO} & \textbf{SIGLA}  \\
    \hline 
		Programa de P\'os-Gradua\c{c}\~ao em Ci\^encias de Computa\c{c}\~ao e Matem\'atica Computacional & Ci\^encias de Computa\c{c}\~ao e Matem\'atica Computacional	&   &	Doutor(a)	 & DCCp\\
		Graduate Program in Computer Science and Computational Mathematics & Computer Science and Computational Mathematics	&   &	Doctorate & DCCe\\
	    Programa de P\'os-Gradua\c{c}\~ao em Ci\^encias de Computa\c{c}\~ao e Matem\'atica Computacional & Ci\^encias de Computa\c{c}\~ao e Matem\'atica Computacional	&   &	Mestre	& MCCp\\
	    Graduate Program in Computer Science and Computational Mathematics & Computer Science and Computational Mathematics &  & Master & MCCe\\
	    Interinstitucional de P\'os-Gradua\c{c}\~ao em Estat\'{\i}stica & Estat\'{\i}stica &  & Doutor(a)	 & DESp\\
		Join Graduate Program in Statistics & Statistics &  & Doctorate & 	DESe\\
		Interinstitucional de P\'os-Gradua\c{c}\~ao em Estat\'{\i}stica & Estat\'{\i}stica &  & Mestre & MESp\\
		Join Graduate Program in Statistics & Statistics &  & Master & MESe\\
		Programa de P\'os-Gradua\c{c}\~ao em Matem\'atica  & Matem\'atica &   &	Doutor(a) & DMAp\\
		Graduate Program in Mathematics & Mathematics &   & Doctorate & DMAe\\
		Programa de P\'os-Gradua\c{c}\~ao em Matem\'atica  & Matem\'atica &   &	Mestre & MMAp\\

	\end{tabular}
\end{quadro}

% o comando \clearpage \'e necess\'ario para deixar o final da tabela o topo da p\'agina, sem ele o final da tabela \'e centralizado verticalmente na p\'agina 
\clearpage
\begin{quadro}[htb]
\ABNTEXfontereduzida
\begin{tabular}{|p{3.5cm}|p{3.5cm}|p{2.5cm}|p{2.5cm}|p{2.25cm}|}
	\multicolumn{5}{c}%
	{{\quadroname\ \thequadro{} -- Siglas dos Programas de P\'os-Gradua\c{c}\~ao do ICMC}} \\
	\multicolumn{5}{r}{{(conclus\~ao)}} \\
	\hline
   \textbf{PROGRAMA} & \textbf{\'AREA DE CONCENTRA\c{C}\~AO} & \textbf{OP\c{C}\~AO} & \textbf{T\'ITULO} & \textbf{SIGLA}  \\	
	 \hline
	 	Graduate Program in Mathematics & Mathematics &  & Master &	MMAe\\
		Programa de Mestrado Profissional em Matem\'atica & Matem\'atica &  & Mestre & MPMp\\
		Mathematics Professional Master's Program & Matem\'atica &  & Mestre & MPMe\\
		MBA em Ci\^encias de Dados & Ci\^encias de Dados &  & Especialista & MBACDp\\
		MBA in Data Science & Data Science &  & Specialist & MBACDe\\
		MBA em Intelig\^encia Artificial e Big Data & Intelig\^encia Artificial &  & Especialista & MBAIAp\\
		MBA in Artificial Intelligence and Big Data & Artificial Intelligence &  & Specialist & MBAIAe\\
		MBA em Seguran\c{c}a de Dados & Seguran\c{c}a de Dados &  & Especialista & MBASDp\\
		MBA in Data Security & Data Security &  & Specialist & MBASDe\\	
    \hline

\end{tabular}
\begin{flushleft}
		Fonte: Elaborado pelos autores.\
\end{flushleft}
\end{quadro}

% ----------------------------------------------------------
\chapter{Siglas dos Programas de P\'os-Gradua\c{c}\~ao do IFSC}
\index{quadros}O \autoref{quadro-fisi} relaciona as siglas estabelecidas para os programas de p\'os-gradua\c{c}\~ao do IFSC.
\begin{quadro}[htb] 
	\ABNTEXfontereduzida
	\caption[Siglas dos Programas de P\'os-Gradua\c{c}\~ao do IFSC]{Siglas dos Programas de P\'os-Gradua\c{c}\~ao do IFSC}
	\label{quadro-fisi}
	\begin{tabular}{|p{3.5cm}|p{3.5cm}|p{3.5cm}|p{1.5cm}|p{2.25cm}|}
	\multicolumn{5}{r}{{(continua)}} \\ 
    \hline
		\textbf{PROGRAMA} & \textbf{\'AREA DE CONCENTRA\c{C}\~AO} & \textbf{OP\c{C}\~AO} & \textbf{T\'ITULO} & \textbf{SIGLA}  \\
		\hline
		Programa de P\'os-Gradua\c{c}\~ao do Faculdade de Educa\c{c}\~ao& F\'{\i}sica Aplicada &  & Doutor(a) & DFAp\\
		Graduate Program in Physics & Applied Physics &  & Doctor & DFAe\\
		Programa de P\'os-Gradua\c{c}\~ao do Faculdade de Educa\c{c}\~ao& F\'{\i}sica Aplicada &  & Mestre & MFAp\\
		Graduate Program in Physics & Applied Physics &  & Master & MFAe\\
		Programa de P\'os-Gradua\c{c}\~ao do Faculdade de Educa\c{c}\~ao& F\'{\i}sica Aplicada & F\'{\i}sica Biomolecular & Doutor(a) & DFAFBp\\
		Graduate Program in Physics & Applied Physics & Biomolecular Physics & Doctor & DFAFBe\\
		Programa de P\'os-Gradua\c{c}\~ao do Faculdade de Educa\c{c}\~ao& F\'{\i}sica Aplicada & F\'{\i}sica Biomolecular & Mestre & MFAFBp\\
		Graduate Program in Physics & Applied Physics & Biomolecular Physics & Master & MFAFBe\\
		Programa de P\'os-Gradua\c{c}\~ao do Faculdade de Educa\c{c}\~ao& F\'{\i}sica Aplicada & F\'{\i}sica Computacional & Doutor(a) & DFAFCp\\
		Graduate Program in Physics & Applied Physics & Computational Physics & Doctor & DFAFCe\\		

	\end{tabular}
\end{quadro}

% o comando \clearpage \'e necess\'ario para deixar o final da tabela o topo da p\'agina, sem ele o final da tabela \'e centralizado verticalmente na p\'agina 
\clearpage
\begin{quadro}[htb]
	\ABNTEXfontereduzida
	\begin{tabular}{|p{3.5cm}|p{3.5cm}|p{3.5cm}|p{1.5cm}|p{2.25cm}|}
	\multicolumn{5}{c}%
	{{\quadroname\ \thequadro{} -- Siglas dos Programas de P\'os-Gradua\c{c}\~ao do IFSC}} \\
	\multicolumn{5}{r}{{(conclus\~ao)}} \\
	\hline
		\textbf{PROGRAMA} & \textbf{\'AREA DE CONCENTRA\c{C}\~AO} & \textbf{OP\c{C}\~AO} & \textbf{T\'ITULO} & \textbf{SIGLA}  \\	
		\hline
		Programa de P\'os-Gradua\c{c}\~ao do Faculdade de Educa\c{c}\~ao& F\'{\i}sica Aplicada & F\'{\i}sica Computacional & Mestre & MFAFCp\\
		Graduate Program in Physics & Applied Physics & Computational Physics & Master & MFAFCe\\		
		Programa de P\'os-Gradua\c{c}\~ao do Faculdade de Educa\c{c}\~ao& F\'{\i}sica B\'asica &  & Doutor(a) & DFBp\\			
		Graduate Program in Physics & Basic Physics &  & Doctor & DFBe\\
		Programa de P\'os-Gradua\c{c}\~ao do Faculdade de Educa\c{c}\~ao& F\'{\i}sica B\'asica &  & Mestre & MFBp\\
		Graduate Program in Physics & Basic Physics &  & Master & MFBe\\
		\hline
		
	\end{tabular}
	\begin{flushleft}
		Fonte: Elaborado pelos autores.\
	\end{flushleft}
\end{quadro}

% ----------------------------------------------------------
\chapter{Siglas dos Programas de P\'os-Gradua\c{c}\~ao do IFSC para ingressantes a partir de 2020}
\index{quadros}O \autoref{quadro-if} relaciona as siglas estabelecidas para os programas de p\'os-gradua\c{c}\~ao do IFSC para ingressantes a partir de 2020.
\begin{quadro}[htb] 
	\ABNTEXfontereduzida
	\caption[Siglas dos Programas de P\'os-Gradua\c{c}\~ao do IFSC para ingressantes a partir de 2020]{Siglas dos Programas de P\'os-Gradua\c{c}\~ao do IFSC para ingressantes a partir de 2020}
	\label{quadro-if}
	\begin{tabular}{|p{4.5cm}|p{4.0cm}|p{2.0cm}|p{1.5cm}|p{2.25cm}|}
		\hline
		\textbf{PROGRAMA} & \textbf{\'AREA DE CONCENTRA\c{C}\~AO} & \textbf{OP\c{C}\~AO} & \textbf{T\'ITULO} & \textbf{SIGLA}  \\
		\hline
		Programa de P\'os-Gradua\c{c}\~ao do Faculdade de Educa\c{c}\~ao& F\'{\i}sica Biomolecular &  & Doutor(a) & DFBMp\\
		Graduate Program in Physics & Biomolecular Physics &  & Doctor & DFBMe\\
		Programa de P\'os-Gradua\c{c}\~ao do Faculdade de Educa\c{c}\~ao& F\'{\i}sica Biomolecular &  & Mestre & MFBMp\\		
		Graduate Program in Physics & Biomolecular Physics &  & Master & MFBMe\\
		Programa de P\'os-Gradua\c{c}\~ao do Faculdade de Educa\c{c}\~ao& F\'{\i}sica Computacional &  & Doutor(a) & DFCp\\
		Graduate Program in Physics & Computational Physics &  & Doctor & DFCe\\
		Programa de P\'os-Gradua\c{c}\~ao do Faculdade de Educa\c{c}\~ao& F\'{\i}sica Computacional &  & Mestre & MFCp\\		
		Graduate Program in Physics & Computational Physics &  & Master & MFCe\\
		Programa de P\'os-Gradua\c{c}\~ao do Faculdade de Educa\c{c}\~ao& F\'{\i}sica Te\'orica e Experimental &  & Doutor(a) & DFTEp\\		
		Graduate Program in Physics & Theoretical and Experimental Physics &  & Doctor & DFTEe\\
		Programa de P\'os-Gradua\c{c}\~ao do Faculdade de Educa\c{c}\~ao& F\'{\i}sica Te\'orica e Experimental &  & Mestre & MFTEp\\
		Graduate Program in Physics & Theoretical and Experimental Physics &  & Master & MFTEe\\
		\hline
	\end{tabular}
	\begin{flushleft}
		Fonte: Elaborado pelos autores.\
	\end{flushleft}
\end{quadro}

% ----------------------------------------------------------
\chapter{Siglas dos Programas de P\'os-Gradua\c{c}\~ao do IQSC}
\index{quadros}O \autoref{quadro-iqsc} relaciona as siglas estabelecidas para os programas de p\'os-gradua\c{c}\~ao do IQSC.
\begin{quadro}[htb]
\ABNTEXfontereduzida
\caption[Siglas dos Programas de P\'os-Gradua\c{c}\~ao do IQSC]{Siglas dos Programas de P\'os-Gradua\c{c}\~ao do IQSC}
\label{quadro-iqsc}
\begin{tabular}{|p{3.5cm}|p{3.5cm}|p{3.5cm}|p{1.5cm}|p{2.25cm}|}
  \hline
   \textbf{PROGRAMA} & \textbf{\'AREA DE CONCENTRA\c{C}\~AO} & \textbf{OP\c{C}\~AO} & \textbf{T\'ITULO} & \textbf{SIGLA}  \\
    \hline
Programa de P\'os-Gradua\c{c}\~ao do Faculdade de Educa\c{c}\~ao& F\'{\i}sico-qu\'{\i}mica &  & Doutor(a) & DFQ\\
Programa de P\'os-Gradua\c{c}\~ao do Faculdade de Educa\c{c}\~ao& F\'{\i}sico-qu\'{\i}mica &  & Mestre & MFQ\\
Programa de P\'os-Gradua\c{c}\~ao do Faculdade de Educa\c{c}\~ao& Qu\'{\i}mica Anal\'{\i}tica e Inirg\^anica &  & Doutor(a) & DQAI\\
Programa de P\'os-Gradua\c{c}\~ao do Faculdade de Educa\c{c}\~ao& Qu\'{\i}mica Anal\'{\i}tica e Inirg\^anica &  & Mestre & MQAI\\
Programa de P\'os-Gradua\c{c}\~ao do Faculdade de Educa\c{c}\~ao& Qu\'{\i}mica Org\^anica e Biol\'ogica &  & Doutor(a) & DQOB\\
Programa de P\'os-Gradua\c{c}\~ao do Faculdade de Educa\c{c}\~ao& Qu\'{\i}mica Org\^anica e Biol\'ogica &  & Mestre & MQOB\\
\hline

\end{tabular}
\begin{flushleft}
		Fonte: Elaborado pelos autores.\
\end{flushleft}
\end{quadro}

% ----------------------------------------------------------
\chapter{Siglas dos Cursos de Gradua\c{c}\~ao da EESC}
\index{quadros}O \autoref{quadro-geesc} relaciona as siglas estabelecidas para os cursos de gradua\c{c}\~ao da EESC.
\begin{quadro}[htb]
	\ABNTEXfontereduzida
	\caption[Siglas dos Cursos de Gradua\c{c}\~ao da EESC]{Siglas dos Cursos de Gradua\c{c}\~ao da EESC}
	\label{quadro-geesc}
	\begin{tabular}{|p{6.5cm}|p{6.5cm}|p{1.75cm}|}
		\hline
		\textbf{CURSO} & \textbf{T\'ITULO} &  \textbf{SIGLA}  \\
		\hline
		Engenharia Ambiental & Engenheiro(a) Ambiental & EAMB\\
		Engenharia Aeron\'autica & Engenheiro(a) Aeron\'autico(a) & EAER\\
		Engenharia Civil & Engenheiro(a) Civil & ECIV\\
		Engenharia de Computa\c{c}\~ao & Engenheiro(a) de Computa\c{c}\~ao & ECOM\\
	    Engenharia El\'etrica com \^Enfase em Eletr\^onica & Engenheiro(a) Eletricista & EELT\\
	    Engenharia El\'etrica com \^Enfase em Sistemas de Energia e Automa\c{c}\~ao & Engenheiro(a) Eletricista & EELS\\
		Engenharia de Materiais e Manufatura & Engenheiro(a) de Materiais e de Manufatura & EMAT\\
		Engenharia Mec\^anica & Engenheiro(a) Mecatr\^onico(a) & EMET\\
		Engenharia Mec\^anica & Engenheiro(a) Mecatr\^onico(a) & EMET\\
		Engenharia de Produ\c{c}\~ao & Engenheiro(a) de Produ\c{c}\~ao & EPRO\\
		\hline
		
	\end{tabular}
	\begin{flushleft}
		Fonte: Elaborado pelos autores.\
	\end{flushleft}
\end{quadro}

% ----------------------------------------------------------
\chapter{Siglas dos Cursos de Gradua\c{c}\~ao do ICMC}
\index{quadros}O \autoref{quadro-gicmc} relaciona as siglas estabelecidas para os cursos de gradua\c{c}\~ao da ICMC.
\begin{quadro}[htb]
	\ABNTEXfontereduzida
	\caption[Siglas dos Cursos de Gradua\c{c}\~ao da ICMC]{Siglas dos Cursos de Gradua\c{c}\~ao da ICMC}
	\label{quadro-gicmc}
	\begin{tabular}{|p{6.5cm}|p{6.5cm}|p{1.75cm}|}
		\hline
		\textbf{CURSO} & \textbf{T\'ITULO} &  \textbf{SIGLA}  \\
		\hline
		Bachelor of Computer Science & Bachelor in Computer Science & BCCe\\
		Bacharelado em Ci\^encias da Computa\c{c}\~ao & Bacharel(a) em Ci\^encias da Computa\c{c}\~ao & BCCp\\
		Bachelor in Data Science & Bachelor in Data Science & BCDe\\
		Bacharelado em Ci\^encia de Dados & Bacharel(a) em Bacharelado em Ci\^encia de Dados & BCDp\\
		Bachelor in Statistics and Data Science & Bachelor in Statistics and Data Science & BECDe\\
		Bacharelado em Estat\'{\i}stica e Ci\^encia de Dados & Bacharel(a) em Estat\'{\i}stica e Ci\^encia de Dados & BECDp\\
		Bachelor of Mathematics & Bachelor in Mathematics & BMe\\
		Bacharelado em Matem\'atica & Bacharel(a) em Matem\'atica & BMp\\
		Bachelor of Applied Mathematics and Scientific Computing & Bachelor in Applied Mathematics and Scientific Computing & BMAe\\
		Bacharelado em Matem\'atica Aplicada e Computa\c{c}\~ao Cient\'{\i}fica & Bacharel(a) em Matem\'atica Aplicada e Computa\c{c}\~ao Cient\'{\i}fica & BMAp\\
		Bachelor of Information Systems & Bachelor in Information Systems & BSIe\\
		Bacharelado em Sistemas de Informa\c{c}\~ao & Bacharel(a) em Sistemas de Informa\c{c}\~ao & BSIp\\
		Bachelor in Statistics and Data Science (Project) & Bachelor in Statistics and Data Science & EBECDe\\
		Bacharelado em Estat\'{\i}stica e Ci\^encia de Dados (Projeto) & Bacharel(a) em Bacharelado em Estat\'{\i}stica e Ci\^encia de Dados & EBECDp\\
		Computer Engineering & Computer Engineer & ECe\\
		Engenharia de Computa\c{c}\~ao & Engenheiro(a) de Computa\c{c}\~ao & ECp\\
		Licenciate of  Mathematics & Licenciate in Mathematics & LMe\\
		Licenciatura em Matem\'atica & Licenciado(a) em Matem\'atica & LMp\\
	    \hline
		
	\end{tabular}
	\begin{flushleft}
		Fonte: Elaborado pelos autores.\
	\end{flushleft}
\end{quadro}

% ----------------------------------------------------------
\chapter{Siglas dos Cursos de Gradua\c{c}\~ao do IQSC}
\index{quadros}O \autoref{quadro-giqsc} relaciona as siglas estabelecidas para os cursos de gradua\c{c}\~ao da IQSC.
\begin{quadro}[htb]
	\ABNTEXfontereduzida
	\caption[Siglas dos Cursos de Gradua\c{c}\~ao da IQSC]{Siglas dos Cursos de Gradua\c{c}\~ao da IQSC}
	\label{quadro-giqsc}
	\begin{tabular}{|p{6.5cm}|p{6.5cm}|p{1.75cm}|}
		\hline
		\textbf{CURSO} & \textbf{T\'ITULO} &  \textbf{SIGLA}  \\
		\hline
		Bacharelado em Qu\'{\i}mica & Bacharel(a) em Qu\'{\i}mica & BQ\\
		Bacharelado em Qu\'{\i}mica (Relat\'orio de Est\'agio)  & Bacharel(a) em Qu\'{\i}mica & REQ\\
		\hline
		
	\end{tabular}
	\begin{flushleft}
		Fonte: Elaborado pelos autores.\
	\end{flushleft}
\end{quadro}

% ----------------------------------------------------------
\chapter{Exemplo de tabela centralizada verticalmente e horizontalmente}
\index{tabelas}A \autoref{tab-centralizada} exemplifica como proceder para obter uma tabela centralizada verticalmente e horizontalmente.
% utilize \usepackage{array} no PREAMBULO (ver em USPSC-modelo.tex) obter uma tabela centralizada verticalmente e horizontalmente
\begin{table}[htb]
\ABNTEXfontereduzida
\caption[Exemplo de tabela centralizada verticalmente e horizontalmente]{Exemplo de tabela centralizada verticalmente e horizontalmente}
\label{tab-centralizada}

\begin{tabular}{ >{\centering\arraybackslash}m{6cm}  >{\centering\arraybackslash}m{6cm} }
\hline
 \centering \textbf{Coluna A} & \textbf{Coluna B}\\
\hline
  Coluna A, Linha 1 & Este \'e um texto bem maior para exemplificar como \'e centralizado verticalmente e horizontalmente na tabela. Segundo par\'agrafo para verificar como fica na tabela\\
  Quando o texto da coluna A, linha 2 \'e bem maior do que o das demais colunas  & Coluna B, linha 2\\
\hline
\end{tabular}
\begin{flushleft}
		Fonte: Elaborada pelos autores.\
\end{flushleft}
\end{table}

% ----------------------------------------------------------
\chapter{Exemplo de tabela com grade}
\index{tabelas}A \autoref{tab-grade} exemplifica a inclus\~ao de tra\c{c}os estruturadores de conte\'udo para melhor compreens\~ao do conte\'udo da tabela, em conformidade com as normas de apresenta\c{c}\~ao tabular do IBGE.
% utilize \usepackage{array} no PREAMBULO (ver em USPSC-modelo.tex) obter uma tabela centralizada verticalmente e horizontalmente
\begin{table}[htb]
\ABNTEXfontereduzida
\caption[Exemplo de tabelas com grade]{Exemplo de tabelas com grade}
\label{tab-grade}
\begin{tabular}{ >{\centering\arraybackslash}m{8cm} | >{\centering\arraybackslash}m{6cm} }
\hline
 \centering \textbf{Coluna A} & \textbf{Coluna B}\\
\hline
  A1 & B1\\
\hline
  A2 & B2\\
\hline
  A3 & B3\\
\hline
  A4 & B4\\
\hline
\end{tabular}
\begin{flushleft}
		Fonte: Elaborada pelos autores.\
\end{flushleft}
\end{table}


\end{apendicesenv}
% ---