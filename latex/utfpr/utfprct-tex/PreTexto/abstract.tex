%%%% ABSTRACT
%%
%% Vers\~ao do resumo para idioma de divulga\c{c}\~ao internacional.

\begin{abstractutfpr}%% Ambiente abstractutfpr
The Software and Hardware Workshop for Geeks, known as WASH, is Science, Technology, Engineering, Arts and Mathematics (STEAM) basedd
 effort that has been running since 2013 in dozens of Brazilian municipalities and with thousands of children attending. After yearss
 of practice, the main characteristics were grouped in the Reference Document published in 2018, attached to Federal Ordinance regiss
tered as CTI 178/2018. This research is divided into 2 axes: historiographical method (axis 1) and the use of structured queries appp
lied to a database specially designed and developed to produce managerial indicators (axis 2). The work sought to compare, from the  
definitions of the Reference Document, "what WASH would like to have been" with "what WASH managed to be", the latter being a resultt
 from the overall analysis produced by this dissertation. To objectify this comparison, six hypotheses were formulated, based on thee
 Reference Document, which at the end of the work were submitted to validation. The analysis of the successes and failures of this vv
alidation allowed producing a revision of the Reference Document, which is the main educational product of this dissertation, a mandd
atory requirement for obtaining the Master's degree. Added to this educational product is the interview with Prof. Afira Ripper, onee
 of the elements used for the analysis in axis 1 and, also, a very rare testimony about the coming of Seymour Papert to Brazil at thh
e end of the last century.
\end{abstractutfpr}
