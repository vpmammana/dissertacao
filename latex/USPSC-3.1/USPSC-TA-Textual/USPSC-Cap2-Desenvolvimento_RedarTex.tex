%% USPSC-Cap2-Desenvolvimento.tex 

% ---
% Este capítulo, utilizado por diferentes exemplos do abnTeX2, ilustra o uso de
% comandos do abnTeX2 e de LaTeX.
% ---

\chapter{Desenvolvimento}\label{cap_exemplos}
Este capítulo \'e parte principal do trabalho acad\^emico e deve conter a exposição ordenada e detalhada do assunto. Divide-se em seções e subseções, em conformidade com a abordagem do tema e do m\'etodo, abrangendo: revisão bibliogr\'afica, materiais e m\'etodos, t\'ecnicas utilizadas, resultados obtidos e discussão.

Abaixo são apresentados minimamente exemplos tabelas, quadros, divisões de documentos e outros itens. Consulte o \textbf{Tutorial do Pacote USPSC para modelos de trabalhos de acad\^emicos em LaTeX - vers\~ao 3.1} para demais informações. 

\section{Resultados de comandos}\label{sec-divisoes}

% ---
\subsection{Tabelas e quadros}

O \textbf{Tutorial do Pacote USPSC para modelos de trabalhos de acad\^emicos em LaTeX - vers\~ao 3.1} apresenta orientações completas e diversas formatações de tabelas, dentre elas a \autoref{tab-ibge}, que \'e um exemplo de tabela alinhada que pode ser longa ou curta, conforme padrão do Instituto Brasileiro de Geografia e Estatística (IBGE).

%\begin{table}[H]
\begin{table}[htb]
	\IBGEtab{%
		\caption{Frequ\^encia anual por categoria de usu\'arios}%
		\label{tab-ibge}
	}{%
		\begin{tabular}{ccc}
			\toprule
			Categoria de Usu\'arios & Frequ\^encia de Usu\'arios \\
			\midrule \midrule
			Graduação & 72\% \\
			\midrule 
			Pós-Graduação & 15\% \\
			\midrule 
			Docente & 10\% \\
			\midrule 
			Outras & 3\% \\
			\bottomrule
		\end{tabular}%
	}{%
		\fonte{Elaborada pelos autores.}%
		\nota{Exemplo de uma nota.}%
		\nota[Anotações]{Uma anotação adicional, que pode ser seguida de v\'arias
			outras.}%
		
	}
\end{table}


A formatação do quadro \'e similar `a tabela, mas deve ter suas laterais fechadas e conter as linhas horizontais.
\newpage

% o comando \newpage foi utilizado para forçar a quebra de p\'agina

\begin{quadro}[htb]
	\caption{\label{quadro_modelo}Níveis de investigação}
	\begin{tabular}{|p{2.6cm}|p{6.0cm}|p{2.25cm}|p{3.40cm}|}
		\hline
		\textbf{Nível de Investigação} & \textbf{Insumos}  & \textbf{Sistemas de Investigação}  & \textbf{Produtos}  \\
		\hline
		Meta-nível & Filosofia\index{filosofia} da Ci\^encia  & Epistemologia &
		Paradigma  \\
		\hline
		Nível do objeto & Paradigmas do metanível e evid\^encias do nível inferior &
		Ci\^encia  & Teorias e modelos \\
		\hline
		Nível inferior & Modelos e m\'etodos do nível do objeto e problemas do nível inferior & Pr\'atica & Solução de problemas  \\
		\hline
	\end{tabular}
	\begin{flushleft}
		%\fonte{\citeonline{van1986}}
		Fonte: \citeonline{van1986}
	\end{flushleft}
\end{quadro} 


No \textbf{Tutorial do Pacote USPSC para modelos de trabalhos de acad\^emicos em LaTeX - vers\~ao 3.1} são apresentados mais exemplos de quadros.

% ---
\subsection{Figuras}\label{sec_figuras}
% ---
\index{figuras}Figuras podem ser criadas diretamente em \LaTeX,
como o exemplo da \autoref{fig_circulo}. \\ 

\begin{figure}[htb]
	\caption{\label{fig_circulo}A delimitação do espaço}
	\begin{center}
		\setlength{\unitlength}{9cm}
		\begin{picture}(1,1)
		\put(0,0){\line(0,1){1}}
		\put(0,0){\line(1,0){1}}
		\put(0,0){\line(1,1){1}}
		\put(0,0){\line(1,2){.5}}
		\put(0,0){\line(1,3){.3333}}
		\put(0,0){\line(1,4){.25}}
		\put(0,0){\line(1,5){.2}}
		\put(0,0){\line(1,6){.1667}}
		\put(0,0){\line(2,1){1}}
		\put(0,0){\line(2,3){.6667}}
		\put(0,0){\line(2,5){.4}}
		\put(0,0){\line(3,1){1}}
		\put(0,0){\line(3,2){1}}
		\put(0,0){\line(3,4){.75}}
		\put(0,0){\line(3,5){.6}}
		\put(0,0){\line(4,1){1}}
		\put(0,0){\line(4,3){1}}
		\put(0,0){\line(4,5){.8}}
		\put(0,0){\line(5,1){1}}
		\put(0,0){\line(5,2){1}}
		\put(0,0){\line(5,3){1}}
		\put(0,0){\line(5,4){1}}
		\put(0,0){\line(5,6){.8333}}
		\put(0,0){\line(6,1){1}}
		\put(0,0){\line(6,5){1}}
		\end{picture}
	\end{center}
	\legend{Fonte: \citeonline{equipeabntex2}}
\end{figure}

Consulte o \textbf{Tutorial do Pacote USPSC para modelos de trabalhos de acad\^emicos em LaTeX - vers\~ao 3.1} para conhecer mais recursos referentes `a figuras. 

% ---
\section{Divisões do documento}\label{sec-divisoes-b}
Esta seção exemplifica o uso de divisões de documentos em conformidade com a ABNT NBR 6024  \cite{nbr6024}.
% ---
% ---
\subsection{Divisões do documento: subseção}\label{sec-divisoes-subsection}
% ---

Um exemplo de seção \'e a \autoref{sec-divisoes-b}. Esta \'e a \autoref{sec-divisoes-subsection}.

\subsubsection{Divisões do documento: subsubseção}\label{sec-divisoes-subsubsection}

Isto \'e uma \texttt{subsubsection} do \LaTeX, mas \'e denominada de ``subseção'' porque no portugu\^es não temos a palavra ``subsubseção''.

\subsubsection{Divisões do documento: subsubseção}

Isto \'e outra subsubseção.

\subsection{Divisões do documento: subseção}\label{sec-exemplo-subsec}

Isto \'e uma subseção.

\subsubsection{Divisões do documento: subsubseção}

Isto \'e mais uma subsubseção da \autoref{sec-exemplo-subsec}.


\subsubsubsection{Esta \'e uma subseção de quinto
nível}\label{sec-exemplo-subsubsubsection}

Esta \'e uma seção de quinto nível. Ela \'e produzida com o seguinte comando:

\begin{verbatim}
\subsubsubsection{Esta \'e uma subseção de quinto
nível}\label{sec-exemplo-subsubsubsection}
\end{verbatim}

\subsubsubsection{Esta \'e outra subseção de quinto nível}\label{sec-exemplo-subsubsubsection-outro}

Esta \'e outra seção de quinto nível.


\paragraph{Este \'e um par\'agrafo numerado}\label{sec-exemplo-paragrafo}

Este \'e um exemplo de par\'agrafo nomeado. Ele \'e produzido com o comando de
par\'agrafo:

\begin{verbatim}
\paragraph{Este \'e um par\'agrafo nomeado}\label{sec-exemplo-paragrafo}
\end{verbatim}

A numeração entre par\'agrafos numerados e subsubsubseções são contínuas.

\paragraph{Esta \'e outro par\'agrafo numerado}\label{sec-exemplo-paragrafo-outro}

Este \'e outro par\'agrafo nomeado.

% ---
\subsection{Este \'e um exemplo de nome de subseção longa que se aplica a seções e demais divisões do documento. Ele deve estar alinhado `a esquerda e a segunda e demais linhas devem iniciar logo abaixo da primeira palavra da primeira linha} 

Observe que o alinhamento do título obedece esta regra tamb\'em no sum\'ario.
	






